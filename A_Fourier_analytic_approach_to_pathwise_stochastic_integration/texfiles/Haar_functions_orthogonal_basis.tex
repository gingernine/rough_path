%Haar関数系がL^2([0,1],leb_mes)において完全正規直行系となることの証明.
\section{Ciesielski同型}
	\begin{screen}
		\begin{dfn}[Haar関数系]
			Haar関数系$(H_{pm},\ p \in \N,\ 1 \leq m \leq 2^p)$は次で定義される:
			\begin{align}
				H_{pm}(t) \coloneqq
				\begin{cases}
					\sqrt{2^p}, & t \in \left[ \frac{m-1}{2^p},\frac{2m-1}{2^{p+1}} \right), \\
					-\sqrt{2^p}, & t \in \left[ \frac{2m-1}{2^{p+1}},\frac{m}{2^p} \right), \\
					0, & \mathrm{otherwise},
				\end{cases}
				\quad (t \in [0,1]).
			\end{align}
		\end{dfn}
	\end{screen}
	
	\begin{screen}
		\begin{thm}
			Haar関数系$(H_{pm},\ p \in \N,\ 1 \leq m \leq 2^p)$に
			$H_{00} = 1$を加えたものを$\mathbb{H}$とおくと,$\mathbb{H}$は
			$\Lp{2}{[0,1],\lambda}$における完全正規直交基底である.
			\label{thm:Haar_functions_orthogonal_basis}
		\end{thm}
	\end{screen}
	
	\begin{prf}
		$\Lp{2}{[0,1],\lambda}$のノルムと内積をそれぞれ$\Norm{\cdot}{},\inprod<\cdot,\cdot>$で表す.
		$\mathbb{H}$が正規直交系であることは積分により従うから,
		以下では$\mathbb{H}$の完全性を示す.つまり
		$f \in \Lp{2}{[0,1],\lambda}$が全ての$H_{pm} \in \mathbb{H}$に対して
		$\inprod<f,H_{pm}> = 0$を満たすなら
		$f = 0$が成り立つことを示す.
		\begin{description}
			\item[第一段]
				$p \geq 0$に対し$\mathfrak{I}_p \coloneqq \Set{ \left[ \frac{i-1}{2^{p+1}}, \frac{i}{2^{p+1}} \right)}{ i = 1,2,\cdots,2^{p+1} }$とおき,
				\begin{align}
					\mathfrak{B}_p \coloneqq \sgmalg{\mathfrak{I}_p}
				\end{align}
				によりフィルトレーション$\left( \mathfrak{B}_p \right)_p$を定めれば,$\borel{[0,1]} = \cup_p \mathfrak{B}_p$が成り立つ.実際,
				$\subset$の関係は$[0,1]$の開集合が$[(i-1)/2^{p+1},i/2^{p+1})$型の区間の可算和で表現できることにより従う.
				
			\item[第二段]
				任意の$p \in \N_0$に対して
				\begin{align}
					\cexp{f}{\mathfrak{B}_p} = \sum_{i=1}^{2^{p+1}} 2^{p+1} \Biggl( \int_{\left[ \frac{i-1}{2^{p+1}}, \frac{i}{2^{p+1}} \right)} f \dx\lambda
						\Biggr) \defunc_{\left[ \frac{i-1}{2^{p+1}}, \frac{i}{2^{p+1}} \right)}
					,\quad \mbox{$\lambda$-a.e.}
					\label{eq:thm_Haar_functions_orthogonal_basis_1}
 				\end{align}
 				が成り立つことを示す.実際,右辺を$g_p$と表せば
 				\begin{align}
 					\{\emptyset\} \cup \mathfrak{I}_p \subset
 					\Set{E \in \mathfrak{B}_p}{\int_E \cexp{f}{\mathfrak{B}_p} \dx\lambda = \int_E g_p \dx\lambda}
 					\label{eq:thm_Haar_functions_orthogonal_basis_2}
 				\end{align}
 				が従うから,Dynkin族定理により$\Set{E \in \mathfrak{B}_p}{\int_E \cexp{f}{\mathfrak{B}_p} \dx\lambda = \int_E g_p \dx\lambda} = \mathfrak{B}_p$となる.
 				そして$\cexp{f}{\mathfrak{B}_p},g_p$ともに$\mathfrak{B}_p$-可測であるから(\refeq{eq:thm_Haar_functions_orthogonal_basis_1})が得られる.
 				
			\item[第三段]
				$\inprod<f,H_{pm}> = 0\ (\forall H_{pm} \in \mathbb{H})$ならば
				$g_p(t) \longrightarrow 0\ (p \longrightarrow \infty,\ \forall t \in [0,1])$
				が出る.実際,
				\begin{align}
					0 = \inprod<f,H_{pm}> 
					= \sqrt{2^{p+1}} \int_{\left[ \frac{m-1}{2^{p+1}}, \frac{2m-1}{2^{p+2}} \right)} f \dx\lambda
						-  \sqrt{2^{p+1}} \int_{\left[ \frac{2m-1}{2^{p+2}}, \frac{m}{2^{p+1}} \right)} f \dx\lambda
					,\quad (\forall p,m)
				\end{align}
				が満たされているから
				\begin{align}
					\int_{\left[ \frac{m-1}{2^{p+1}}, \frac{m}{2^{p+1}} \right)} f \dx\lambda
					&= 2 \int_{\left[ \frac{2m-2}{2^{p+2}}, \frac{2m-1}{2^{p+2}} \right)} f \dx\lambda \\
					&= 2^2 \int_{\left[ \frac{4m-4}{2^{p+3}}, \frac{4m-3}{2^{p+3}} \right)} f \dx\lambda \\
					&= \cdots \\
					&= 2^k \int_{\left[ \frac{2^km-2^k}{2^{p+k+1}}, \frac{2^km-2^k+1}{2^{p+k+1}} \right)} f \dx\lambda \\
					&= \cdots
				\end{align}
				が従い,H\Ddot{o}lderの不等式より
				\begin{align}
					\left| \int_{\left[ \frac{m-1}{2^{p+1}}, \frac{m}{2^{p+1}} \right)} f \dx\lambda \right|
					= \left| 2^k \int_{\left[ \frac{2^km-2^k}{2^{p+k+1}}, \frac{2^km-2^k+1}{2^{p+k+1}} \right)} f \dx\lambda \right|
					\leq \frac{\Norm{f}{}}{2^{p+1}} 
				\end{align}
		\end{description}
	\end{prf}