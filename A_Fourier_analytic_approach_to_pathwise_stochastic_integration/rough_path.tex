\documentclass[11pt,a4paper]{jsreport}
%
\usepackage{amsmath,amssymb}
\usepackage{amsthm}
\usepackage{makeidx}
\usepackage{txfonts}
\usepackage{mathrsfs} %花文字
\usepackage{mathtools} %参照式のみ式番号表示
\usepackage{latexsym} %qed
\usepackage{ascmac}
\usepackage{color}
\usepackage{comment}

\newtheoremstyle{mystyle}% % Name
	{20pt}%                      % Space above
	{20pt}%                      % Space below
	{\rm}%           % Body font
	{}%                      % Indent amount
	{\gt}%             % Theorem head font
	{.}%                      % Punctuation after theorem head
	{10pt}%                     % Space after theorem head, ' ', or \newline
	{}%                      % Theorem head spec (can be left empty, meaning `normal')
\theoremstyle{mystyle}

\allowdisplaybreaks[1]

\newcommand{\bhline}[1]{\noalign {\hrule height #1}} %表の罫線を太くする.
\newcommand{\bvline}[1]{\vrule width #1} %表の罫線を太くする.
\newtheorem{Prop}{$Proposition.$}
\newtheorem{Proof}{$Proof.$}
\newcommand{\QED}{% %証明終了
	\relax\ifmmode
		\eqno{%
		\setlength{\fboxsep}{2pt}\setlength{\fboxrule}{0.3pt}
		\fcolorbox{black}{black}{\rule[2pt]{0pt}{1ex}}}
	\else
		\begingroup
		\setlength{\fboxsep}{2pt}\setlength{\fboxrule}{0.3pt}
		\hfill\fcolorbox{black}{black}{\rule[2pt]{0pt}{1ex}}
		\endgroup
	\fi}
\newtheorem{thm}{定理}[section]
\newtheorem{dfn}[thm]{定義}
\newtheorem{prp}[thm]{命題}
\newtheorem{lem}[thm]{補題}
\newtheorem*{prf}{証明}
\newtheorem{qst}{レポート問題}
\newtheorem*{bcs}{なぜならば}
\newtheorem{rem}[thm]{注意}
\newcommand{\defunc}{\mbox{1}\hspace{-0.25em}\mbox{l}} %定義関数
\newcommand{\wlim}{\mbox{w-}\lim}
\newcommand{\dx}{\ \operatorname{d}} %積分のd
\def\Set#1#2{\left\{\ #1\ \, ; \quad #2\ \right\}} %集合の書き方
\def\Box#1{$(\mbox{#1})$} %丸括弧つきコメント
\def\Hat#1{$\hat{\mathrm{#1}}$} %文中ハット
\def\Ddot#1{$\ddot{\mathrm{#1}}$} %文中ddot
\def\DEF{\overset{\mathrm{def}}{\Leftrightarrow}} %定義記号
\def\eqqcolon{=\mathrel{\mathop:}} %定義=:
\def\max#1#2{\operatorname*{max}_{#1} #2 } %最大
\def\min#1#2{\operatorname*{min}_{#1} #2 } %最小
\def\sin#1#2{\operatorname{sin}^{#2} #1} %sin
\def\cos#1#2{\operatorname{cos}^{#2} #1} %cos
\def\tan#1#2{\operatorname{tan}^{#2} #1} %tan
\def\inprod<#1>{\left\langle #1 \right\rangle} %内積
\def\sup#1#2{\operatorname*{sup}_{#1} #2 } %上限
\def\inf#1#2{\operatorname*{inf}_{#1} #2 } %下限
\def\Vector#1{\mbox{\boldmath $#1$}} %ベクトルを太字表示
\def\Norm#1#2{\left\|\, #1\, \right\|_{#2}} %ノルム
\def\Log#1{\operatorname{log} #1} %log
\def\Det#1{\operatorname{det} ( #1 )} %行列式
\def\Diag#1{\operatorname{diag} \left( #1 \right)} %行列の対角成分
\def\Tmat#1{#1^\mathrm{T}} %転置行列
\def\Exp#1{\operatorname{E} \left[ #1 \right]} %期待値
\def\Var#1{\operatorname{V} \left[ #1 \right]} %分散
\def\Cov#1#2{\operatorname{Cov} \left[ #1,\ #2 \right]} %共分散
\def\exp#1{e^{#1}} %指数関数
\def\N{\mathbb{N}} %自然数全体
\def\Q{\mathbb{Q}} %有理数全体
\def\R{\mathbb{R}} %実数全体
\def\C{\mathbb{C}} %複素数全体
\def\K{\mathbb{K}} %係数体K
\def\conj#1{\overline{#1}} %共役複素数
\def\Re#1{\mathfrak{Re}#1} %実部
\def\Im#1{\mathfrak{Im}#1} %虚部
\def\Conv#1{\mathrm{Conv}\left[\left\{\, #1 \, \right\}\right]} %凸包
\def\borel#1{\mathfrak{B}(#1)} %Borel集合族
\def\open#1{\mathfrak{O}(#1)} %位相空間 #1 の位相
\def\close#1{\mathfrak{A}(#1)} %%位相空間 #1 の閉集合系
\def\closure#1{\overline{#1}}
\def\equiv#1#2{\left[#1\right]_{#2}} %同値類
\def\rapid#1{\mathfrak{S}(#1)} %急減少空間
\def\c#1{C(#1)} %有界実連続関数
\def\cbound#1{C_{b} (#1)} %有界実連続関数
\def\semiLp#1#2{\mathscr{L}^{#1} \left(#2\right)} %ノルム空間L^p
\def\Lp#1#2{\operatorname{L}^{#1} \left(#2\right)} %ノルム空間L^p
\def\cinf#1{C^{\infty} (#1)} %無限回連続微分可能関数
\def\sgmalg#1{\sigma \left[#1\right]} %#1が生成するσ加法族
\def\ball#1#2{\operatorname{B} \left(#1\, ;\, #2 \right)} %開球
\def\prob#1{\operatorname{P} \left(#1\right)} %確率
\def\cprob#1#2{\operatorname{P} \left(\left\{ #1 \ \middle|\ #2 \right\}\right)} %条件付確率
\def\cexp#1#2{\operatorname{E} \left[ #1 \ \middle|\ #2 \right]} %条件付期待値
\def\tExp#1{\tilde{\operatorname{E}} \left[ #1 \right]} %拡張期待値
\def\tcexp#1#2{\tilde{\operatorname{E}} \left[ #1 \ \middle|\ #2 \right]} %拡張条件付期待値
%\renewcommand{\contentsname}{\bm Index}
%
\makeindex
%
\setlength{\textwidth}{\fullwidth}
\setlength{\textheight}{40\baselineskip}
\addtolength{\textheight}{\topskip}
\setlength{\voffset}{-0.2in}
\setlength{\topmargin}{0pt}
\setlength{\headheight}{0pt}
\setlength{\headsep}{0pt}
%
\title{論文勉強メモ\\A Fourier analytic approach to pathwise stochastic integration}
\author{基礎工学研究科システム創成専攻修士1年\\学籍番号29C17095\\百合川尚学}
\date{\today}

\begin{document}
%
%

\mathtoolsset{showonlyrefs = true}
\maketitle

\newpage
\tableofcontents

\chapter{Notation}
	$\N = \{1,2,\cdots\}$,\\
	$\lambda$:Lebesgue measure.
	
\chapter{Preliminaries}
	%Haar関数系がL^2([0,1],leb_mes)において完全正規直行系となることの証明.
\section{Ciesielski同型}
	\begin{screen}
		\begin{dfn}[Haar関数系]
			Haar関数系$(H_{pm},\ p \in \N,\ 1 \leq m \leq 2^p)$は次で定義される:
			\begin{align}
				H_{pm}(t) \coloneqq
				\begin{cases}
					\sqrt{2^p}, & t \in \left[ \frac{m-1}{2^p},\frac{2m-1}{2^{p+1}} \right), \\
					-\sqrt{2^p}, & t \in \left[ \frac{2m-1}{2^{p+1}},\frac{m}{2^p} \right), \\
					0, & \mathrm{otherwise},
				\end{cases}
				\quad (t \in [0,1]).
			\end{align}
		\end{dfn}
	\end{screen}
	
	\begin{screen}
		\begin{thm}
			Haar関数系$(H_{pm},\ p \in \N,\ 1 \leq m \leq 2^p)$に
			$H_{00} = 1$を加えたものを$\mathbb{H}$とおくと,$\mathbb{H}$は
			$\Lp{2}{[0,1],\lambda}$における完全正規直交基底である.
			\label{thm:Haar_functions_orthogonal_basis}
		\end{thm}
	\end{screen}
	
	\begin{prf}
		$\Lp{2}{[0,1],\lambda}$のノルムと内積をそれぞれ$\Norm{\cdot}{},\inprod<\cdot,\cdot>$で表す.
		$\mathbb{H}$が正規直交系であることは積分により従うから,
		以下では$\mathbb{H}$の完全性を示す.つまり
		$f \in \Lp{2}{[0,1],\lambda}$が全ての$H_{pm} \in \mathbb{H}$に対して
		$\inprod<f,H_{pm}> = 0$を満たすなら
		$f = 0$が成り立つことを示す.
		\begin{description}
			\item[第一段]
				$p \geq 0$に対し$\mathfrak{I}_p \coloneqq \Set{ \left[ \frac{i-1}{2^{p+1}}, \frac{i}{2^{p+1}} \right)}{ i = 1,2,\cdots,2^{p+1} }$とおき,
				\begin{align}
					\mathfrak{B}_p \coloneqq \sgmalg{\mathfrak{I}_p}
				\end{align}
				によりフィルトレーション$\left( \mathfrak{B}_p \right)_p$を定めれば,$\borel{[0,1]} = \cup_p \mathfrak{B}_p$が成り立つ.実際,
				$\subset$の関係は$[0,1]$の開集合が$[(i-1)/2^{p+1},i/2^{p+1})$型の区間の可算和で表現できることにより従う.
				
			\item[第二段]
				任意の$p \in \N_0$に対して
				\begin{align}
					\cexp{f}{\mathfrak{B}_p} = \sum_{i=1}^{2^{p+1}} 2^{p+1} \Biggl( \int_{\left[ \frac{i-1}{2^{p+1}}, \frac{i}{2^{p+1}} \right)} f \dx\lambda
						\Biggr) \defunc_{\left[ \frac{i-1}{2^{p+1}}, \frac{i}{2^{p+1}} \right)}
					,\quad \mbox{$\lambda$-a.e.}
					\label{eq:thm_Haar_functions_orthogonal_basis_1}
 				\end{align}
 				が成り立つことを示す.実際,右辺を$g_p$と表せば
 				\begin{align}
 					\{\emptyset\} \cup \mathfrak{I}_p \subset
 					\Set{E \in \mathfrak{B}_p}{\int_E \cexp{f}{\mathfrak{B}_p} \dx\lambda = \int_E g_p \dx\lambda}
 					\label{eq:thm_Haar_functions_orthogonal_basis_2}
 				\end{align}
 				が従うから,Dynkin族定理により$\Set{E \in \mathfrak{B}_p}{\int_E \cexp{f}{\mathfrak{B}_p} \dx\lambda = \int_E g_p \dx\lambda} = \mathfrak{B}_p$となる.
 				そして$\cexp{f}{\mathfrak{B}_p},g_p$ともに$\mathfrak{B}_p$-可測であるから(\refeq{eq:thm_Haar_functions_orthogonal_basis_1})が得られる.
 				
			\item[第三段]
				$\inprod<f,H_{pm}> = 0\ (\forall H_{pm} \in \mathbb{H})$ならば
				$g_p(t) \longrightarrow 0\ (p \longrightarrow \infty,\ \forall t \in [0,1])$
				が出る.実際,
				\begin{align}
					0 = \inprod<f,H_{pm}> 
					= \sqrt{2^{p+1}} \int_{\left[ \frac{m-1}{2^{p+1}}, \frac{2m-1}{2^{p+2}} \right)} f \dx\lambda
						-  \sqrt{2^{p+1}} \int_{\left[ \frac{2m-1}{2^{p+2}}, \frac{m}{2^{p+1}} \right)} f \dx\lambda
					,\quad (\forall p,m)
				\end{align}
				が満たされているから
				\begin{align}
					\int_{\left[ \frac{m-1}{2^{p+1}}, \frac{m}{2^{p+1}} \right)} f \dx\lambda
					&= 2 \int_{\left[ \frac{2m-2}{2^{p+2}}, \frac{2m-1}{2^{p+2}} \right)} f \dx\lambda \\
					&= 2^2 \int_{\left[ \frac{4m-4}{2^{p+3}}, \frac{4m-3}{2^{p+3}} \right)} f \dx\lambda \\
					&= \cdots \\
					&= 2^k \int_{\left[ \frac{2^km-2^k}{2^{p+k+1}}, \frac{2^km-2^k+1}{2^{p+k+1}} \right)} f \dx\lambda \\
					&= \cdots
				\end{align}
				が従い,H\Ddot{o}lderの不等式より
				\begin{align}
					\left| \int_{\left[ \frac{m-1}{2^{p+1}}, \frac{m}{2^{p+1}} \right)} f \dx\lambda \right|
					= \left| 2^k \int_{\left[ \frac{2^km-2^k}{2^{p+k+1}}, \frac{2^km-2^k+1}{2^{p+k+1}} \right)} f \dx\lambda \right|
					\leq \frac{\Norm{f}{}}{2^{p+1}} 
				\end{align}
		\end{description}
	\end{prf}

\end{document}