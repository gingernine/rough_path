\documentclass[a4j,10.5pt,oneside,openany]{jsbook}
%\documentclass[11pt,a4paper]{jsreport}
%
\usepackage{amsmath,amssymb}
\usepackage{amsthm}
\usepackage{makeidx}
\makeindex
\usepackage{txfonts}
\usepackage{mathrsfs} %花文字
\usepackage{mathtools} %参照式のみ式番号表示
\usepackage{latexsym} %qed
\usepackage{ascmac}
\usepackage{color}
\usepackage{comment}
\usepackage[matrix,arrow]{xy}

\newtheoremstyle{mystyle}% % Name
	{20pt}%                      % Space above
	{20pt}%                      % Space below
	{\rm}%           % Body font
	{}%                      % Indent amount
	{\gt}%             % Theorem head font
	{.}%                      % Punctuation after theorem head
	{10pt}%                     % Space after theorem head, ' ', or \newline
	{}%                      % Theorem head spec (can be left empty, meaning `normal')
\theoremstyle{mystyle}

\allowdisplaybreaks[1]

\newcommand{\bhline}[1]{\noalign {\hrule height #1}} %表の罫線を太くする.
\newcommand{\bvline}[1]{\vrule width #1} %表の罫線を太くする.
\newtheorem{Prop}{$Proposition.$}
\newtheorem{Proof}{$Proof.$}
\newcommand{\QED}{% %証明終了
	\relax\ifmmode
		\eqno{%
		\setlength{\fboxsep}{2pt}\setlength{\fboxrule}{0.3pt}
		\fcolorbox{black}{black}{\rule[2pt]{0pt}{1ex}}}
	\else
		\begingroup
		\setlength{\fboxsep}{2pt}\setlength{\fboxrule}{0.3pt}
		\hfill\fcolorbox{black}{black}{\rule[2pt]{0pt}{1ex}}
		\endgroup
	\fi}
\newtheorem{thm}{定理}[section]
\newtheorem{dfn}[thm]{定義}
\newtheorem{prp}[thm]{命題}
\newtheorem{lem}[thm]{補題}
\newtheorem{cor}[thm]{系}
\newtheorem*{prf}{証明}
\newtheorem{qst}{レポート問題}
\newtheorem*{bcs}{なぜならば}
\newtheorem{rem}[thm]{注意}
\newcommand{\defunc}{\mbox{1}\hspace{-0.25em}\mbox{l}} %定義関数
\newcommand{\wlim}{\mbox{w-}\lim}
\newcommand{\dx}{\ \operatorname{d}} %積分のd
\def\Set#1#2{\left\{\ #1\ \, ; \quad #2\ \right\}} %集合の書き方
\def\Box#1{$(\mbox{#1})$} %丸括弧つきコメント
\def\Hat#1{$\hat{\mathrm{#1}}$} %文中ハット
\def\Ddot#1{$\ddot{\mathrm{#1}}$} %文中ddot
\def\DEF{\overset{\mathrm{def}}{\Leftrightarrow}} %定義記号
\def\eqqcolon{=\mathrel{\mathop:}} %定義=:
\def\max#1#2{\operatorname*{max}_{#1} #2 } %最大
\def\min#1#2{\operatorname*{min}_{#1} #2 } %最小
\def\sin#1#2{\operatorname{sin}^{#2} #1} %sin
\def\cos#1#2{\operatorname{cos}^{#2} #1} %cos
\def\tan#1#2{\operatorname{tan}^{#2} #1} %tan
\def\inprod<#1>{\left\langle #1 \right\rangle} %内積
\def\sup#1#2{\operatorname*{sup}_{#1} #2 } %上限
\def\inf#1#2{\operatorname*{inf}_{#1} #2 } %下限
\def\Vector#1{\mbox{\boldmath $#1$}} %ベクトルを太字表示
\def\Norm#1#2{\left\|\, #1\, \right\|_{#2}} %ノルム
\def\crossnorm#1#2{\alpha_{#2}\left( #1 \right)} %クロスノルム
\def\compcrossnorm#1#2{\left|\, #1\, \right|_{#2}} %完備クロスノルム
\def\projectivenorm#1#2{\pi_{#2}\left( #1 \right)} %プロジェクティブノルム
\def\injectivenorm#1#2{\epsilon_{#2}\left( #1 \right)} %インジェクティブノルム
\def\Log#1{\operatorname{log} #1} %log
\def\Det#1{\operatorname{det} ( #1 )} %行列式
\def\Diag#1{\operatorname{diag} \left( #1 \right)} %行列の対角成分
\def\Tmat#1{#1^\mathrm{T}} %転置行列
\def\Exp#1{\operatorname{E} \left[ #1 \right]} %期待値
\def\Var#1{\operatorname{V} \left[ #1 \right]} %分散
\def\Cov#1#2{\operatorname{Cov} \left[ #1,\ #2 \right]} %共分散
\def\exp#1{e^{#1}} %指数関数
\def\N{\mathbb{N}} %自然数全体
\def\Z{\mathbb{Z}} %整数全体
\def\Q{\mathbb{Q}} %有理数全体
\def\R{\mathbb{R}} %実数全体
\def\C{\mathbb{C}} %複素数全体
\def\K{\mathbb{K}} %係数体K
\def\conj#1{\overline{#1}} %共役複素数
\def\Re#1{\mathfrak{Re}#1} %実部
\def\Im#1{\mathfrak{Im}#1} %虚部
\def\Conv#1{\mathrm{Conv}\left[\left\{\, #1 \, \right\}\right]} %凸包
\def\Span#1{\mathrm{Span}\left[ #1 \right]} %線型包
\def\borel#1{\mathfrak{B}(#1)} %Borel集合族
\def\open#1{\mathfrak{O}(#1)} %位相空間 #1 の位相
\def\close#1{\mathfrak{A}(#1)} %%位相空間 #1 の閉集合系
\def\closure#1{\overline{#1}}
%\def\equiv#1#2{\left[#1\right]_{#2}} %同値類
\def\rapid#1{\mathfrak{S}(#1)} %急減少空間
\def\c#1{C(#1)} %有界実連続関数
\def\cbound#1{C_{b} (#1)} %有界実連続関数
\def\Dim#1{\mathrm{dim}#1} %次元
\def\pmod#1{\mathrm{Mod}\ #1} %mod
\def\Map#1#2{\mathrm{Map} \left(#1,#2\right)} %写像全体
\def\Hom#1#2{\mathrm{Hom} \left(#1,#2\right)} %線形写像全体
\def\Ln#1#2#3{\mathrm{Hom}^{(#3)} \left(#1,#2\right)} %多重線形写像全体
\def\ContL#1#2{L \left(#1,#2\right)} %連続な線形写像全体
\def\ContLn#1#2#3{L^{(#3)} \left(#1,#2\right)} %連続な多重線形写像全体
\def\semiLp#1#2{\mathscr{L}^{#1} \left(#2\right)} %ノルム空間L^p
\def\Lp#1#2{\operatorname{L}^{#1} \left(#2\right)} %ノルム空間L^p
\def\cinf#1{C^{\infty} (#1)} %無限回連続微分可能関数
\def\sgmalg#1{\sigma \left[#1\right]} %#1が生成するσ加法族
\def\ball#1#2{\operatorname{B} \left(#1\, ;\, #2 \right)} %開球
\def\prob#1{\operatorname{P} \left(#1\right)} %確率
\def\cprob#1#2{\operatorname{P} \left(\left\{ #1 \ \middle|\ #2 \right\}\right)} %条件付確率
\def\cexp#1#2{\operatorname{E} \left[ #1 \ \middle|\ #2 \right]} %条件付期待値
\def\tExp#1{\tilde{\operatorname{E}} \left[ #1 \right]} %拡張期待値
\def\tcexp#1#2{\tilde{\operatorname{E}} \left[ #1 \ \middle|\ #2 \right]} %拡張条件付期待値

\setlength{\textwidth}{\fullwidth}
\setlength{\textheight}{40\baselineskip}
\addtolength{\textheight}{\topskip}

\title{ゼミ用ノート\\会田先生の資料``Rough path analysis:An Introduction''}
\author{基礎工学研究科システム創成専攻\\学籍番号29C17095\\百合川尚学}
\date{\today}

\begin{document}

\mathtoolsset{showonlyrefs = true}
\maketitle
\tableofcontents
\frontmatter
\mainmatter

\chapter{}
	\section{導入}
	以下,$x \in \R^d$について成分を込めて表現する場合は
	$x = (x^1,\cdots,x^d)$と書き,
	実$m \times d$行列$a$については$a=(a^i_j)_{1 \leq i \leq m,1 \leq j \leq d}$と表す.
	また$T > 0$を固定し$C^1 = C^1([0,T] \rightarrow \R^d)$とおく.
	ただし端点においては片側微分を考える.
	区間$[s,t] \subset [0,T]$の分割を
	$D = \{s = t_0 < t_1 < \cdots < t_N = t\}$で表現し
	$|D| \coloneqq \max{1 \leq i \leq N}{\left| t_i - t_{i-1} \right|}$とおく.
	また$[s,t]$の分割の全体を$\delta[s,t]$と書く.

	\begin{screen}
		\begin{thm}[Riemann-Stieltjes積分]
			$[s,t] \subset [0,T]$とし,$D \in \delta[s,t]$についてのみ考えるとき,
			任意の$x \in C^1,\ f \in C(\R^d,L(\R^d \rightarrow \R^m))$\footnotemark
			に対して
			次の極限が確定する:
			\begin{align}
				\lim_{|D| \to 0}
				\sum_{D} f(x_{s_{i-1}})(x_{t_i} - x_{t_{i-1}})
				\in \R^m.
			\end{align}
			ここで$s_{i-1}$は区間$[t_{i-1},t_i]$に属する任意の点である.
			極限は$s_{i-1}$の取り方にも依存しない.
		\end{thm}
	\end{screen}
	\footnotetext{
		極限の存在を保証する条件としては,$f$の有界性と微分可能性は必要ない.
	}
	\begin{prf}
		各$x^j$は$C^1$-級であるから,平均値の定理より
		$\sum_{D} f(x_{s_{i-1}})(x_{t_i} - x_{t_{i-1}})$
		の第$k$成分を
		\begin{align}
			\sum_{j=1}^{d} \sum_{D} f^k_j (x_{s_{i-1}})(x^j_{t_i} - x^j_{t_{i-1}})
			= \sum_{j=1}^{d} \sum_{D} f^k_j (x_{s_{i-1}}) \frac{d}{dt}x^j(\xi_{i-1,j})(t_i - t_{i-1}),
			\quad ({}^\exists \xi_{i-1,j} \in [t_{i-1},t_i])
		\end{align}
		と表現できる.各$j,k$について
		\begin{align}
			\lim_{|D| \to 0} \sum_{D} f^k_j (x_{s_{i-1}}) \frac{d}{dt}x^j(\xi_{i-1,j})(t_i - t_{i-1})
		\end{align}
		が確定すれば,第$k$成分の極限が確定し定理の主張を得る.
		いま,$t \longrightarrow f^k_j(x_t)$及び$t \longmapsto (d/dt)x^j_t$は([s,t]上一様)連続であるから,
		分割$D$による各区間$[t_{i-1},t_i]$において次の最大最小値が定まる:
		\begin{align}
			M_{i-1} \coloneqq \sup{t_{i-1} \leq t \leq t_i} f^k_j(x_t)\frac{d}{dt}x^j_t,
			\quad m_{i-1} \coloneqq \inf{t_{i-1} \leq t \leq t_i} f^k_j(x_t)\frac{d}{dt}x^j_t.
		\end{align}
		ここで
		\begin{align}
			S_D \coloneqq \sum_{D} M_{i-1}(t_i - t_{i-1}),
			\quad s_D \coloneqq \sum_{D} m_{i-1}(t_i - t_{i-1}),
			\quad \Sigma_D \coloneqq \sum_{D} f^k_j (x_{s_{i-1}}) \frac{d}{dt}x^j(\xi_{i-1})(t_i - t_{i-1})
		\end{align}
		とおいて
		\begin{align}
			S \coloneqq \inf{D \in \delta[s,t]}{S_D},
			\quad s \coloneqq \sup{D \in \delta[s,t]}{s_D}
		\end{align}
		を定めれば
		\begin{align}
			s_D \leq s \leq S \leq S_D,
			\quad s_D \leq \Sigma_D \leq S_D
		\end{align}
		が満たされる.実際,任意の$D_1,D_2 \in \delta[s,t]$に対して,
		分割の合併を$D_3$とすれば
		\begin{align}
			s_{D_1} \leq s_{D_3} \leq S_{D_3} \leq S_{D_2}
		\end{align}
		が成立し$s \leq S_D\ (\forall D \in \delta[s,t])$すなわち$s \leq S$が出る.
		一方で一様連続性から
		\begin{align}
			0 \leq S - s \leq S_D - s_D = \sum_D (M_{i-1} - m_{i-1})(t_i - t_{i-1})
			\longrightarrow 0
			\quad (|D| \longrightarrow 0)
		\end{align}
		が従い$s = S$を得る.以上より
		\begin{align}
			|S - \Sigma_D| \leq |S - S_D| + |S_D - \Sigma_D|
			\leq |S - S_D| + |S_D - s_D|
			\longrightarrow 0
			\quad (|D| \longrightarrow 0)
		\end{align}
		が成り立つ.
		\QED
\end{prf}

\begin{screen}
	\begin{dfn}
		$I_{0,T}(x)$の定義.
	\end{dfn}
\end{screen}

$C^1$において次でノルム$\Norm{\cdot}{C^1}$を定める:
\begin{align}
	\Norm{x}{\infty} &\coloneqq \sup{t \in [0,T]}{|x(t)|},
	\quad \Norm{x'}{\infty} \coloneqq \sup{t \in [0,T]}{|x'(t)|}, \\
	\Norm{x}{C^1} &\coloneqq
	\Norm{x}{\infty} + \Norm{x}{\infty}.
\end{align}

\begin{screen}
	\begin{thm}
		$x \longmapsto I_{0,T}(x)$は連続である.
	\end{thm}
\end{screen}

\begin{prf}
	$C^1$は次のノルムでBanach空間になる.ゆえに
	$x \longmapsto I_{0,T}(x)$は距離空間から距離空間への対応であり
	点列連続性と連続性が一致するから,
	$x_n \longrightarrow x$なら$I_{0,T}(x_n) \longrightarrow I_{0,T}(x)$
	が従うことを示せばよい.実際,$f,x$の連続性より
	\begin{align}
		\sum_{i=1}^{N} f(x^{(n)}_{s_{i-1}})(x^{(n)}_{t_i} - x^{(n)}_{t_{i-1}})
		\longrightarrow \sum_{i=1}^{N} f(x_{s_{i-1}})(x_{t_i} - x_{t_{i-1}})
	\end{align}
	が成り立つ.
\end{prf}

\begin{screen}
	\begin{dfn}[$p$-variation]
		$[0,T]$上の$\R^d$値関数$x$に対し$p$-variationノルムを次で定める:
		\begin{align}
			\Norm{x}{p}
			\coloneqq \left\{ \sup{D}{\sum_{i=1}^{N} 
				\left| x_{t_i} - x_{t_{i-1}} \right|^p }\right\}^{1/p}.
		\end{align}
		また線形空間$B_{p,T}(\R^d)$を
		\begin{align}
			B_{p,T}(\R^d)
			\coloneqq \Set{x:[0,T] \longrightarrow \R^d}{x_0=0,\ x:\mbox{continuous},\ \Norm{x}{p} < \infty}
		\end{align}
		により定める.
	\end{dfn}
\end{screen}

\begin{screen}
	\begin{thm}
		$\tilde{C}^1 \coloneqq \Set{x \in C^1}{x_0 = 0}$とおくと,
		$\tilde{C}^1 \subset B_{p,T}(\R^d)$が成り立つ.
	\end{thm}
\end{screen}

\begin{prf}
	$x \in \tilde{C}^1$に対して
	\begin{align}
		M \coloneqq \sum_{j=1}^{d} \sup{x \in [0,T]}{|x^j(t)|}
	\end{align}
	とおけば,$x'$の連続性より$M < \infty$が定まる.
	平均値の定理より,$|D| < 1$を満たす分割$D$に対して
	\begin{align}
		\left\{ \sum_{i=1}^{N} |x_{t_i} - x_{t_{i-1}}|^p \right\}^{1/p}
		\leq \left\{ \sum_{i=1}^{N} \Norm{x}{C^1}^p (t_i - t_{i-1})^p \right\}^{1/p}
		\leq M T < \infty
	\end{align}
	が成立し$\Norm{x}{p} \leq MT < \infty$が従う
	\footnote{
		$S_D \geq 0$ならば$(\sup{D}S_D)^{1/p} = \sup{D}S_D^{1/p}$が成り立つ.
	}
	.
\end{prf}

\begin{screen}
	\begin{thm}
		$B_{p,T}(\R^d)$はBanach空間である.
	\end{thm}
\end{screen}

\begin{prf}
	$(x^n)_{n=1}^{\infty} \subset B_{p,T}(\R^d)$をCauchy列とする.
	つまり任意の$\epsilon > 0$に対して或る$n_\epsilon \in \N$が存在し
	\begin{align}
		\Norm{x^n - x^m}{p}
		= \left\{ \sup{D}{\sum_{i=1}^{N} 
		\left| \left( x^n_{t_i} - x^m_{t_i} \right) 
		- \left(x^n_{t_{i-1}} - x^m_{t_{i-1}} \right) \right|^p }\right\}^{1/p} < \epsilon,
		\quad (n,m > n_\epsilon)
	\end{align}
	を満たす.いま,任意の$t \in [0,T]$に対して$[0,T]$の分割$\{0=t_0 \leq t \leq T\}$
	を考えれば
	\begin{align}
		|x^n_t - x^m_t| < \epsilon,
		\quad (n,m > n_\epsilon)
	\end{align}
	が得られ,実数の完備性より或る$x_t \in \R^d$が存在して
	\begin{align}
		|x^n_t - x_t| < \epsilon
		\quad (n > n_\epsilon)
	\end{align}
	を満たす.実際,もし或る$n > n_\epsilon$で$|x^n_t - x_t| \eqqcolon \alpha \geq \epsilon$
	が成り立つと,任意の$m > n_\epsilon$に対して
	\begin{align}
		|x^m_t - x_t| \geq |x^n_t - x_t| - |x^n_t - x^m_t| > \alpha - \epsilon
	\end{align}
	が従い$x^m_t \longrightarrow x_t$に反する.
	ゆえに収束は$t$に関して一様であり,$t \longmapsto x_t$は0出発かつ連続である.
	あとは$\Norm{x^n - x}{p} \longrightarrow 0\ (n \longrightarrow \infty)$であればよい.
\end{prf}

\begin{screen}
	\begin{thm}
		$C^1$空間において,$\Norm{}{C^1}$で定まる位相は
		$\Norm{}{p}$で定まる位相より強い.
		特に,$B_{p,T}(\R^d)$上で考える写像$x \longmapsto I_{p,T}(x)$は
		$\Norm{\cdot}{p}$の定める位相により連続である.
	\end{thm}
\end{screen}

\begin{prf}
	任意の$x \in \tilde{C}^1$に対して
	\begin{align}
		\Norm{x}{p} \leq T \Norm{x}{C^1}
	\end{align}
	を満たすことを証明する.実際,任意の分割$D$に対して
	\begin{align}
		\sum_{i=1}^{N} |x_{t_i} - x_{t_{i-1}}|
		\leq \sum_{i=1}^{N} \Norm{x}{C^1} (t_i - t_{i-1})
		= T \Norm{x}{C^1}
	\end{align}
	が成り立つ.
	\QED
\end{prf}
	\section{連続性定理}
	\begin{screen}
		\begin{dfn}[記号の定義]
			$x \in C^1,\ f \in C^2(\R^d,L(\R^d \rightarrow \R^m))$に対し次を定める.
			\begin{align}
				\Delta_T &\coloneqq \Set{(s,t)}{0 \leq s \leq t \leq T}, \\
				X^1 &:\Delta_T \longrightarrow \R^d\ \left( (s,t) \longmapsto X^1_{s,t} = x_t - x_s \right), \\
				X^2 &:\Delta_T \longrightarrow \R^d \otimes \R^d\ \left( (s,t) \longmapsto X^2_{s,t} = \int_s^t (x_u - x_s) \otimes dx_u \right), \\
				\tilde{I}_{s,t}(x) &\coloneqq f(x_s)X^1_{s,t} = f(x_s)(x_t - x_s), \\
				J_{s,t}(x) &\coloneqq f(x_s)X^1_{s,t} + (\nabla f)(x_s)X^2_{s,t}.
			\end{align}
		\end{dfn}
	\end{screen}
	
	以降,$a,b,c,d \in \R^d$に対して次の表現を使う:
	\begin{align}
		[a \otimes b]^i_j &= a^i b^j, \\
		\left[ (\nabla f)(x_s)X^2_{s,t} \right]^i &= \sum_{j,k=1}^d \partial_k f^i_j(x_s) \int_s^t \left(x^k_u - x^k_s \right)\ dx^j_u,\\
		\left[ (\nabla f)(x_s)(a \otimes b) \right]^i &= \sum_{j,k=1}^d \partial_k f^i_j(x_s) a^k b^j,\\
		\left[ (\nabla^2 f)(x_s)(a \otimes b \otimes c) \right]^i &= \sum_{j,k,v=1}^d \partial_v \partial_k f^i_j(x_s) a^v b^k c^j,\\
		\left[ (\nabla^3 f)(x_s)(a \otimes b \otimes c \otimes d) \right]^i &= \sum_{j,k,v,w=1}^d \partial_w \partial_v \partial_k f^i_j(x_s) a^w b^v c^k d^j.
	\end{align}
	
	\begin{screen}
		\begin{thm}\label{thm:Riemann_Stieltjes_approximation}
			$[s,t] \subset [0,T],\ x \in C^1,\ f \in C^2(\R^d,L(\R^d \rightarrow \R^m))$とする.$D \in \delta[s,t]$に対し
			\begin{align}
				\tilde{I}_{s,t}(x,D) \coloneqq \sum_D \tilde{I}_{t_{i-1},t_i}(x),
				\quad J_{s,t}(x,D) \coloneqq \sum_D J_{t_{i-1},t_i}(x)
			\end{align}
			を定めるとき,次が成立する:
			\begin{align}
				I_{s,t}(x) = \lim_{|D| \to 0} \tilde{I}_{s,t}(x,D)
				= \lim_{|D| \to 0} J_{s,t}(x,D).
			\end{align}
		\end{thm}
	\end{screen}
	
	\begin{prf}
		第一の等号は$I_{s,t}(x)$の定義によるから,第二の等号を証明する.まず,
		\begin{align}
			I_{s,t}(x)
			&= \int_s^t f(x_u)\ dx_u \\
			&= \int_s^t f(x_s) + f(x_u) - f(x_s)\ dx_u \\
			&= \int_s^t f(x_s)\ dx_u
				+ \int_s^t \int_0^1 (\nabla f)(x_s + \theta(x_u - x_s)) \left( X^1_{s,u} \otimes \dot{x}_u \right)\ d\theta\ du \\
			&= f(x_s)X^1_{s,t} + (\nabla f)(x_s) X^2_{s,t} \\
				&\quad+ \int_s^t \int_0^1 \left\{ (\nabla f)(x_s + \theta(x_u - x_s)) - (\nabla f)(x_s) \right\}\left( X^1_{s,u} \otimes \dot{x}_u \right)\ d\theta\ du \\
			&= J_{s,t}(x)
				+ \int_s^t 
				\int_0^1 \int_0^\theta (\nabla^2 f)(x_s + r(x_u - x_s))\left( X^1_{s,u} \otimes X^1_{s,u} \otimes \dot{x}_u \right)\ dr\ d\theta\ du
		\end{align}
		が成り立つ.$[0,T] \ni t \longmapsto x_t$の連続性より,
		最下段式中の$x_s + r(x_u - x_s)\ (0 \leq r \leq 1,\ s \leq u \leq t)$は或るコンパクト集合$K$に含まれ,
		$f$が$C^2$-級関数であるから
		\begin{align}
			M \coloneqq \sum_{i,j,k,v} \sup{x \in K}{\left|\partial_v \partial_k f_j^i(x) \right|}
		\end{align}
		として$M < \infty$を定めれば
		\begin{align}
			&\left| \int_s^t 
				\int_0^1 \int_0^\theta (\nabla^2 f)(x_s + r(x_u - x_s))\left( X^1_{s,u} \otimes X^1_{s,u} \otimes \dot{x}_u \right)\ dr\ d\theta\ du \right| \\
			&\qquad \leq \int_s^t 
				\int_0^1 \int_0^\theta \left| (\nabla^2 f)(x_s + r(x_u - x_s))\left( X^1_{s,u} \otimes X^1_{s,u} \otimes \dot{x}_u \right) \right|\ dr\ d\theta\ du \\
			&\qquad \leq M \int_s^t |X^1_{s,u}|^2 |\dot{x}_u|\ du \\
			&\qquad \leq M \Norm{x}{C^1}^3 \int_s^t (u - s)^2\ du
		\end{align}
		が出る.特に$D \in \delta[s,t]$に対して
		\begin{align}
			&\sum_D \int_{t_{i-1}}^{t_i} (u - t_{i-1})^2\ du
			\leq \sum_D |D| \int_{t_{i-1}}^{t_i} (u - t_{i-1})\ du \\
			&\qquad \leq \sum_D |D| \int_{t_{i-1}}^{t_i} (u - s)\ du
			\leq \frac{1}{2}(t-s)^2 |D|
			\longrightarrow 0 \quad (|D| \longrightarrow 0)
		\end{align}
		が成立するから,
		\begin{align}
			\left| I_{s,t}(x) - J_{s,t}(x,D) \right|
			\leq \sum_D \left| I_{t_{i-1},t_i}(x) - J_{t_{i-1},t_i}(x) \right| \longrightarrow 0 \quad (|D| \longrightarrow 0)
		\end{align}
		が従い定理の主張を得る.
		\QED
	\end{prf}
	
	\begin{screen}
		\begin{dfn}[control function]
			関数$\omega:\Delta_T \longrightarrow [0,\infty)$
			が連続かつ任意の$s \leq u \leq t$に対して
			\begin{align}
				\omega(s,u) + \omega(u,t) \leq \omega(s,t)
				\label{eq:control_function_subadditivity}
			\end{align}
			を満たすとき,$\omega$をcontrol functionと呼ぶ.
		\end{dfn}
	\end{screen}
	
	式(\refeq{eq:control_function_subadditivity})から$\omega(t,t)=0\ (\forall t \in [0,T])$が従う.
	つまりcontrol functionは''対角線上で0になる''.
	
	\begin{screen}
		\begin{dfn}[ノルム空間値写像の$p$-variation]
			$(V,\Norm{\cdot}{})$をノルム空間,$p \geq 1$とする.
			このとき連続写像$\psi:\Delta_T \longrightarrow V$に対する$p$-variationを
			\begin{align}
				\Norm{\psi}{p,[s,t]}
				\coloneqq \left\{ \sup{D \in \delta[s,t]}{ \sum_D \Norm{\psi_{t_{i-1},t_i}}{}^p} \right\}^{1/p},
				\quad ((s,t) \subset [0,T])
			\end{align}
			で定める.特に$\Norm{\cdot}{p,[0,T]}$を$\Norm{\cdot}{p}$と書く.
		\end{dfn}
	\end{screen}
	
	\begin{screen}
		\begin{thm}[$p$-variationが定めるcontrol function]
			$(V,\Norm{\cdot}{})$をノルム空間,$p \geq 1$とする.
			$\Norm{\psi}{p} < \infty$かつ$\psi_{t,t} = 0\ (\forall t \in [0,T])$を満たす連続写像$\psi:\Delta_T \longrightarrow V$に対して,
			\begin{align}
				\omega:\Delta_T \ni (s,t) \longmapsto \Norm{\psi}{p,[s,t]}^p
			\end{align}
			により定める$\omega$はcontrol functionである.
		\end{thm}
	\end{screen}
	
	\begin{prf}
		$\Norm{\psi}{p} < \infty$の仮定より$\omega$は$[0,\infty)$値であるから,
		以下では式(\refeq{eq:control_function_subadditivity})の成立と連続性を示す.
		\begin{description}
			\item[第一段]
				$\omega$が式(\refeq{eq:control_function_subadditivity})を満たすことを示す.実際,
				任意に$D_1 \in \delta[s,u],D_2 \in \delta[u,t]$を取れば
				\begin{align}
					\sum_{D_1}\Norm{\psi_{t_{i-1},t_i}}{}^p
					+ \sum_{D_2}\Norm{\psi_{t_{i-1},t_i}}{}^p
					= \sum_{D_1 \cup D_2}\Norm{\psi_{t_{i-1},t_i}}{}^p
					\leq \Norm{\psi}{p:[s,t]}^p
				\end{align}
				が成り立つ.左辺の$D_1,D_2$の取り方は独立であるから,それぞれに対し上限を取れば
				\begin{align}
					\Norm{\psi}{p:[s,u]}^p + \Norm{\psi}{p:[u,t]}^p
					\leq \Norm{\psi}{p:[s,t]}^p
				\end{align}
				が従う.
			\item[第二段]
				任意の$[s,t] \subset [0,T]$について
				\footnote{
					下段の二式については$s < t$と仮定する.また
					上段についても,$t=T$或は$s=0$の場合を除く.
				},
				\begin{align}
					\lim_{h \to +0} \omega(s,t+h) &= \inf{h>0}{\omega(s,t+h)},
					&\lim_{h \to +0} \omega(s-h,t) &= \inf{h>0}{\omega(s-h,t)}, \\
					\lim_{h \to +0} \omega(s,t-h) &= \sup{h>0}{\omega(s,t-h)},
					&\lim_{h \to +0} \omega(s+h,t) &= \sup{h>0}{\omega(s+h,t)}
				\end{align}
				が成立する.実際$\omega(s,t+h)$について見れば,これは下に有界かつ
				$h \to +0$に対し単調減少であるから極限が確定し下限に一致する.
				残りの三つも同様の理由で成立する.
				
			\item[第三段]
				任意の$s \in [0,T)$に対し,$(s,T] \ni t \longmapsto \omega(s,t)$
				の左連続性を示す.ここでは
				\begin{align}
					\tilde{\omega}(s,t) \coloneqq 
					\begin{cases}
						\lim_{h \to +0}\omega(s,t-h), & (s < t), \\
						0, & (s=t),
					\end{cases}
					\quad (\forall (s,t) \in \Delta_T)
				\end{align}
				で定める$\tilde{\omega}$が優加法性を持ち,かつ
				\begin{align}
					\Norm{\psi_{s,t}}{}^p \leq \tilde{\omega}(s,t),
					\quad (\forall (s,t) \in \Delta_T)
				\end{align}
				を満たすことを示す.
				実際これが示されれば,任意の$D \in \delta[s,t]$に対し
				\begin{align}
					\sum_D \Norm{\psi_{t_{i-1},t_i}}{}^p
					\leq \sum_D \tilde{\omega}(t_{i-1},t_i)
					\leq \tilde{\omega}(s,t)
				\end{align}
				が成立し$\omega(s,t) \leq \tilde{\omega}(s,t)$が従い,
				$\omega(s,t) \geq \omega(s,t-h)\ (\forall h > 0)$と併せて
				\begin{align}
					\omega(s,t) = \tilde{\omega}(s,t) = \lim_{h \to +0} \omega(s,t-h)
				\end{align}
				を得る.いま,任意に$s < u < t$を取れば,十分小さい$h_1,h_2 > 0$に対して
				\begin{align}	
					\omega(s,u-h_1) + \omega(u,t-h_2) \leq \omega(s,t-h_2)
				\end{align}
				が満たされ,$h_1 \longrightarrow +0,\ h_2 \longrightarrow +0$として
				\begin{align}
					\tilde{\omega}(s,u) + \tilde{\omega}(u,t) \leq \tilde{\omega}(s,t)
				\end{align}
				が成り立ち$\tilde{\omega}$は優加法性を持つ.また,もし
				或る$[u,v] \subset [0,T]$に対して
				\begin{align}
					\Norm{\psi_{u,v}}{}^p > \tilde{\omega}(u,v)
				\end{align}
				が成り立つと仮定すると
				\begin{align}
					\Norm{\psi_{u,v}}{}^p > \tilde{\omega}(u,v) \geq \omega(u,v-h) \geq \Norm{\psi_{u,v-h}}{}^p,
					\quad (\forall h > 0)
				\end{align}
				となる.一方$\psi$の連続性より
				$\Norm{\psi_{u,v-h}}{}^p \longrightarrow \Norm{\psi_{u,v}}{}^p\ (h \longrightarrow +0)$が従い矛盾が生じる.
				同様にして,任意の$t \in (0,T]$に対し$[0,t) \ni s \longmapsto \omega(s,t)$
				の右連続性も出る.
				
			\item[第二段]
				任意の$t \in [0,T)$に対して次を示す:
				\begin{align}
					\lim_{h \to +0} \omega(t,t+h) = \inf{h>0}{\omega(t,t+h)} = 0.
				\end{align}
				第一の等号は前段より従うから,第二の等号を背理法により証明する.いま
				\begin{align}
					\inf{h>0}{\omega(t,t+h)} \eqqcolon \delta > 0
					\label{eq:thm_continuity_of_norm_val_p_variation_2}
				\end{align}
				と仮定する.$\psi$の連続性より或る$h_1$が存在して
				\begin{align}
					\Norm{\psi_{t,t+h}}{}^p
					 = \Norm{\psi_{t,t+h} - \psi_{t,t}}{}^p
					 < \frac{\delta}{8},
					\quad (\forall h < h_1)
					\label{eq:thm_continuity_of_norm_val_p_variation_1}
				\end{align}
				が成立するから,任意に$h_0 < h_1$を取り固定する.
				一方で$\omega(t,t+h_0) \geq \delta$より
				\begin{align}
					\sum_{i=1}^{N} \Norm{\psi_{\tau_{i-1},\tau_i}}{}^p > \frac{7\delta}{8}
				\end{align}
				を満たす$D = \{t = \tau_0 < \tau_1 < \cdots, \tau_N = t+h_0\} \in \delta[t,t+h_0]$が存在し,
				(\refeq{eq:thm_continuity_of_norm_val_p_variation_1})と併せて
				\begin{align}
					\sum_{i=2}^{N} \Norm{\psi_{\tau_{i-1},\tau_i}}{}^p
					> \frac{7\delta}{8} - \Norm{\psi_{t,\tau_1}}{}^p
					>\frac{7\delta}{8} - \frac{\delta}{8}
					= \frac{3 \delta}{4}
				\end{align}
				を得る.また,$\omega(t,\tau_1) \geq \delta$より或る$D' \in \delta[t,\tau_1]$が存在して
				\begin{align}
					\sum_{D'} \Norm{\psi_{t_{i-1},t_i}}{}^p > \frac{3 \delta}{4}
				\end{align}
				を満たすから,$D' \cup \{\tau_1 < \cdots, \tau_N = t+h_0\} \in \delta[t,t+h_0]$より
				\begin{align}
					\omega(t,t+h_0) > \sum_{D'} \Norm{\psi_{t_{i-1},t_i}}{}^p + \sum_{i=2}^{N} \Norm{\psi_{\tau_{i-1},\tau_i}}{}^p
					> \frac{3\delta}{2}
				\end{align}
				が従うが,$h_0 < h_1$の任意性と単調減少性により
				\begin{align}
					\delta = \inf{h>0}{\omega(t,t+h)} = \inf{h_1>h>0}{\omega(t,t+h)} \geq \frac{3\delta}{2}
				\end{align}
				となり矛盾が生じる.
				同様にして
				\begin{align}
					\lim_{h \to +0} \omega(t-h,t) = 0,
					\quad (\forall t \in (0,T])
				\end{align}
				も成立する.
				
			\item[第三段]
				任意に$s \in [0,T)$を取り固定し,
				$[s,T) \ni t \longmapsto \omega(s,t)$が右連続であることを示す.
				\begin{align}
					\lim_{h \to +0} \omega(s,t+h) \leq \omega(s,t)
					\label{eq:thm_continuity_of_norm_val_p_variation_3}
				\end{align}
				を示せば,第二段より逆向きの不等号も従い右連続性を得る.
				任意に$h,\epsilon > 0$を取れば,
				\begin{align}
					\omega(s,t + h) - \epsilon
					\leq \sum_D \Norm{\psi_{t_{i-1},t_i}}{}^p
				\end{align}
				を満たす$D \in \delta[s,t+h]$が存在する.
				$D_1 \coloneqq [s,t] \cap D,\ D_2 \coloneqq D \backslash D_1$とおいて
				$D_2$の最小元を
				\begin{align}
					\omega(s,t + h) - \epsilon
					\leq \sum_{D_1} \Norm{\psi_{t_{i-1},t_i}}{}^p + \sum_{D_2} \Norm{\psi_{t_{i-1},t_i}}{}^p
					\leq \omega(s,t) + \omega(t,t+h)
				\end{align}
				が成り立つ.$h \longrightarrow +0$として
				\begin{align}
					\lim_{h \to +0} \omega(s,t+h) - \epsilon \leq \omega(s,t)
				\end{align}
				が従い,$\epsilon$の任意性より(\refeq{eq:thm_continuity_of_norm_val_p_variation_3})が出る.
				同様にして$s \longmapsto \omega(s,t)$の左連続性も成立する.
				
			\item[第四段]
				$\Delta_T \ni (s,t) \longmapsto \omega(s,t)$の連続性を示す.
		\end{description}
	\end{prf}
	
	\begin{screen}
		\begin{thm}[control functionの例]\label{thm:examples_of_control_functions}
			以下の関数$\omega:\Delta_T \longrightarrow [0,\infty)$はcontrol functionである.
			\begin{description}
				\item[(1)] $\omega \coloneqq \left( \omega_1^r + \omega_2^r \right)^{1/r},
					\quad (0 < r \leq 1,\ \omega_1,\omega_2:\mbox{control function}).$
				\item[(2)] $\omega:(s,t) \longmapsto \Norm{X^1}{p:[s,t]}^p,
					\quad (p \geq 1,\ x \in B_{p,T}(\R^d)).$
				\item[(3)] $\omega:(s,t) \longmapsto \Norm{X^2}{p:[s,t]}^p,
					\quad (p \geq 1,\ x \in C^1).$
			\end{description}
		\end{thm}
	\end{screen}
	
	行列$a = (a_j^i)$のノルムは$|a| = \sqrt{\sum_{i,j}|a_j^i|^2}$として考える.
	
	\begin{thm}\mbox{}
		\begin{description}
			\item[(1)]
			\item[(2)] 
				
			\item[(3)] 任意の$[s,t] \subset [0,T]$に対して
				$\Norm{X^2}{p:[s,t]}^p < \infty$を示せば,あとは
				上と同じ理由により定理の主張が得られる.
				実際,任意の分割$D = \{s=t_0 < \cdots < t_N = t\}$に対し
				\begin{align}
					\Norm{X^2_{t_{i-1},t_i}}{}
					&\leq \left| \int_{t_{i-1}}^{t_i} (x_u - x_{t_{i-1}}) \otimes \dot{x}_u\ du \right| \\
					&\leq \int_{t_{i-1}}^{t_i} \left| (x_u - x_{t_{i-1}}) \otimes \dot{x}_u \right|\ du \\
					&\leq \Norm{x}{C^1}^2 \left\{ \int_{t_{i-1}}^{t_i} (u-s)\ du \right\}^{1/p}
						\left\{ \int_{t_{i-1}}^{t_i} (u-s)\ du \right\}^{1-1/p} \\
					&\leq \Norm{x}{C^1}^2 \left\{ \int_{t_{i-1}}^{t_i} (u-s)\ du \right\}^{1/p}
						\left\{ \int_s^t (u-s)\ du \right\}^{1-1/p}
				\end{align}
				が成り立つから,
				\begin{align}
					&\sum_D \Norm{X^2_{t_{i-1},t_i}}{}^p
					\leq \sum_D \Norm{x}{C^1}^{2p} \left\{ \frac{1}{2}(t-s)^2 \right\}^{p-1}
						\int_{t_{i-1}}^{t_i} (u-s)\ du \\
					&\qquad= \Norm{x}{C^1}^{2p} \left\{ \frac{1}{2}(t-s)^2 \right\}^{p-1}
						\int_s^t (u-s)\ du
					= \Norm{x}{C^1}^{2p} \left\{ \frac{1}{2}(t-s)^2 \right\}^p
				\end{align}
				により$\Norm{X^2}{p:[s,t]}^p < \infty$が従う.
				\QED
		\end{description}
	\end{thm}
	
	\begin{prf}
		\begin{align}
			&\omega_1(s,u)\omega_2(s,u) + \omega_1(u,t)\omega_2(u,t) \\
			&= \left\{ \omega_1(s,u) + \omega_1(u,t) \right\}\omega_2(s,u) + \omega_1(u,t)\left\{\omega_2(u,t) - \omega_2(s,u)\right\} \\
			&\leq \omega_1(s,t)\omega_2(s,u) + \omega_1(u,t)\left\{\omega_2(u,t) - \omega_2(s,u)\right\} \\
			&\leq \omega_1(s,t)\omega_2(s,u) + \omega_1(u,t)\omega_2(u,t) \\
			&\leq \omega_1(s,t)\left\{ \omega_2(s,u) + \omega_2(u,t)\right\} \\
			&\leq \omega_1(s,t)\omega_2(s,t)
		\end{align}
	\end{prf}
	
	\begin{screen}
		\begin{lem}\label{lem:control_function_min}
			$\omega$を$\Delta_T$上のcontrol functionとする.
			$D = \{s = t_0 < t_1 < \cdots < t_N= t\}$について,$N \geq 2$の場合
			或る$1 \leq i \leq N-1$が存在して次を満たす:
			\begin{align}
				\omega(t_{i-1},t_{i+1})
				\leq \frac{2 \omega(s,t)}{N-1}.
				\label{eq:lem_control_function_min}
			\end{align}
		\end{lem}
	\end{screen}
	
	\begin{prf}
		\QED
	\end{prf}
	
	\begin{screen}
		\begin{thm}[$1 \leq p < 2$の場合の連続性定理]\label{thm:continuity_theorem_1}
			$1 \leq p < 2$とし,
			$x_0 = y_0$を満たす$x,y \in C^1$と$f \in C^2_b(\R^d,L(\R^d \rightarrow \R^m)),\ 0 < \epsilon, R < \infty$を任意に取る.
			このとき,
			\begin{align}
				\Norm{X^1}{p},\Norm{Y^1}{p} \leq R,
				\quad \Norm{X^1 - Y^1}{p} \leq \epsilon
			\end{align}
			なら,或る定数$C = C(p,R,f)$が存在し,任意の$0 \leq s \leq t \leq T$に対して次が成立する:
			\begin{align}
				\left| I_{s,t}(x) - I_{s,t}(y) \right| \leq \epsilon C.
			\end{align}
		\end{thm}
	\end{screen}
	
	\begin{screen}
		\begin{cor}[$p$-variationによる閉球上のLipschitz連続性]\label{cor:continuity_theorem_1}
			$1 \leq p < 2$とし,
			$x_0 = y_0$を満たす$x,y \in C^1$と$f \in C^2_b(\R^d,L(\R^d \rightarrow \R^m)),\ 0 < R < \infty,\ [s,t] \subset [0,T]$を任意に取る.
			このとき,
			\begin{align}
				\Norm{X^1}{p,[s,t]},\Norm{Y^1}{p,[s,t]} \leq R
			\end{align}
			なら,或る定数$C = C(p,R,f)$が存在して次を満たす:
			\begin{align}
				\left| I_{s,t}(x) - I_{s,t}(y) \right| \leq C\Norm{X^1 - Y^1}{p,[s,t]}.
			\end{align}
		\end{cor}
	\end{screen}
	
	\begin{prf}[系\ref{cor:continuity_theorem_1}]
		定理\ref{thm:continuity_theorem_1}において,
		$\epsilon = \Norm{X^1 - Y^1}{p}\ (x \neq y)$
		\footnote{
			$x=y$なら$\Norm{X^1 - Y^1}{p} = 0$かつ$I_{s,t}(x) = I_{s,t}(y)$が成り立つ.
		}
		として証明が通る.
		$[0,T]$で考察したことを$[s,t]$に置き換えれば系を得る.
		\QED
	\end{prf}
	
	\begin{prf}[定理\ref{thm:continuity_theorem_1}]
		$[s,t] \subset [0,T]$とする.
		\begin{description}
			\item[第一段]
				$\omega:\Delta_T \longrightarrow [0,\infty)$を
				\begin{align}
					\omega(\alpha,\beta) = \Norm{X^1}{p,[\alpha,\beta]}^p + \Norm{Y^1}{p,[\alpha,\beta]}^p + \epsilon^{-p} \Norm{X^1 - Y^1}{p,[\alpha,\beta]}^p,
					\quad ((\alpha,\beta) \in \Delta_T)
				\end{align}
				で定めれば,定理\ref{thm:examples_of_control_functions}により$1 \leq p$の下で$\omega$はcontrol functionである.
				
			\item[第二段]
				任意に$[s,t]$の分割
				$D = \{s=t_0 < \cdots < t_N=t\}\ (N \geq 2)$を取れば,
				補題\ref{lem:control_function_min}より(\refeq{eq:lem_control_function_min})を満たす$t_{(0)}$が存在する.
				ここで,$D_{-0} \coloneqq D,\ D_{-1} \coloneqq D \backslash \{t_{(0)}\}$と定める.
				$N \geq 3$ならば$D_{-1}$についても(\refeq{eq:lem_control_function_min})を満たす$t_{(1)}$が存在するから,
				$D_{-2} \coloneqq D_{-1} \backslash \{t_{(1)}\}$と定める.この操作を繰り返せば
				$t_{(k)},D_{-k}\ (k=0,1,\cdots,N-1)$が得られ,
				\begin{align}
					&\tilde{I}_{s,t}(x,D) - \tilde{I}_{s,t}(y,D) \\
					&\qquad = \sum_{k=0}^{N-2} \left[ \left\{ \tilde{I}_{s,t}(x,D_{-k}) - \tilde{I}_{s,t}(x,D_{-k-1}) \right\} - 
						\left\{ \tilde{I}_{s,t}(y,D_{-k}) - \tilde{I}_{s,t}(y,D_{-k-1}) \right\} \right] \label{eq:continuity_theorem_1_1}\\
					&\quad \qquad + \left\{ \tilde{I}_{s,t}(x) - \tilde{I}_{s,t}(y) \right\}	\label{eq:continuity_theorem_1_2}
				\end{align}
				と表現できる.
			
			\item[第三段]
				式(\refeq{eq:continuity_theorem_1_1})について,次を満たす定数$C_1$が存在することを示す:
				\begin{align}
					|(\refeq{eq:continuity_theorem_1_1})| \leq \epsilon C_1
					\label{eq:continuity_theorem_1_3}
				\end{align}
				見やすくするために$t_k = t_{(k)}$と書き直せば,
				\begin{align}
					&\left\{ \tilde{I}_{s,t}(x,D_{-k}) - \tilde{I}_{s,t}(x,D_{-k-1}) \right\} - 
						\left\{ \tilde{I}_{s,t}(y,D_{-k}) - \tilde{I}_{s,t}(y,D_{-k-1}) \right\} \\
					&=	\left\{ f(x_{t_k}) - f(x_{t_{k-1}}) \right\} X^1_{t_k,t_{k+1}}
						- \left\{ f(y_{t_k}) - f(y_{t_{k-1}}) \right\} Y^1_{t_k,t_{k+1}} \\
					&= \left\{ f(x_{t_k}) - f(x_{t_{k-1}}) \right\} \left( X^1_{t_k,t_{k+1}} - Y^1_{t_k,t_{k+1}} \right) \\
						&\quad +\left\{ f(x_{t_k}) - f(x_{t_{k-1}}) \right\} Y^1_{t_k,t_{k+1}} - \left\{ f(y_{t_k}) - f(y_{t_{k-1}}) \right\} Y^1_{t_k,t_{k+1}} \\
					&= \int_0^1 (\nabla f)( x_{t_{k-1}}+\theta ( x_{t_k}-x_{t_{k-1}} ))
						X^1_{t_{k-1},t_k} \otimes \left( X^1_{t_k,t_{k+1}} - Y^1_{t_k,t_{k+1}} \right)\ d\theta \\
						&\quad + \int_0^1 (\nabla f)( x_{t_{k-1}}+\theta ( x_{t_k}-x_{t_{k-1}} ))
						X^1_{t_{i_k-1},t_k} \otimes Y^1_{t_k,t_{i_k+1}}\ d\theta \\
						&\quad - \int_0^1 (\nabla f)( y_{t_{k-1}}+\theta ( y_{t_k}-y_{t_{k-1}} ))
						Y^1_{t_{i_k-1},t_k} \otimes Y^1_{t_k,t_{i_k+1}}\ d\theta \\
					&= \int_0^1 (\nabla f)( x_{t_{k-1}}+\theta ( x_{t_k}-x_{t_{k-1}} ))
						X^1_{t_{i_k-1},t_k} \otimes \left( X^1_{t_k,t_{i_k+1}} - Y^1_{t_k,t_{i_k+1}} \right)\ d\theta \\
						&\quad + \int_0^1 (\nabla f)( x_{t_{k-1}}+\theta ( x_{t_k}-x_{t_{k-1}} ))
						\left( X^1_{t_{k-1},t_k} - Y^1_{t_{k-1},t_k} \right) \otimes Y^1_{t_k,t_{k+1}}\ d\theta \\
						&\quad + \int_0^1 (\nabla f)( x_{t_{k-1}}+\theta ( x_{t_k}-x_{t_{k-1}} ))
						Y^1_{t_{k-1},t_k} \otimes Y^1_{t_k,t_{k+1}}\ d\theta \\
						&\quad - \int_0^1 (\nabla f)( y_{t_{k-1}}+\theta ( y_{t_k}-y_{t_{k-1}} ))
						Y^1_{t_{k-1},t_k} \otimes Y^1_{t_k,t_{k+1}}\ d\theta \\
					&= \int_0^1 (\nabla f)( x_{t_{k-1}}+\theta ( x_{t_k}-x_{t_{k-1}} ))
						X^1_{t_{k-1},t_k} \otimes \left( X^1_{t_k,t_{k+1}} - Y^1_{t_k,t_{k+1}} \right)\ d\theta \\
						&\quad + \int_0^1 (\nabla f)( x_{t_{k-1}}+\theta ( x_{t_k}-x_{t_{k-1}} ))
						\left( X^1_{t_{k-1},t_k} - Y^1_{t_{k-1},t_k} \right) \otimes Y^1_{t_k,t_{k+1}}\ d\theta \\
						&\quad + \int_0^1 \int_0^1 (\nabla^2 f)( y_{t_{k-1}}+\theta ( y_{t_k}-y_{t_{k-1}} + r(x_{t_{k-1}}+\theta ( x_{t_k}-x_{t_{k-1}}) - y_{t_{k-1}}-\theta ( y_{t_k}-y_{t_{k-1}}))) \\
						&\qquad\qquad \left( X^1_{0,t_{k-1}} - Y^1_{0,t_{k-1}} \right) \otimes Y^1_{t_{k-1},t_k} \otimes Y^1_{t_k,t_{k+1}}\ dr\ d\theta\ \footnotemark \\
						&\quad + \int_0^1 \int_0^1 (\nabla^2 f)( y_{t_{k-1}}+\theta ( y_{t_k}-y_{t_{k-1}} + r(x_{t_{k-1}}+\theta ( x_{t_k}-x_{t_{k-1}}) - y_{t_{k-1}}-\theta ( y_{t_k}-y_{t_{k-1}}))) \\
						&\qquad\qquad \theta \left( X^1_{t_{k-1},t_k} - Y^1_{t_{k-1},t_k} \right) \otimes Y^1_{t_{k-1},t_k} \otimes Y^1_{t_k,t_{k+1}}\ dr\ d\theta
				\end{align}
				\footnotetext{
					$x_0 = y_0$の仮定より$x_{t_{k-1}} - y_{t_{k-1}} = X^1_{0,t_{k-1}} - Y^1_{0,t_{k-1}}$が成り立つ.
				}
				が成り立つ.補題\ref{lem:control_function_min}より
				\begin{align}
					&\left| X^1_{t_{k-1},t_k} \right|, \left| Y^1_{t_{k-1},t_k} \right|,\left| X^1_{t_k,t_{k+1}} \right|, \left| Y^1_{t_k,t_{k+1}} \right|
					\leq \omega(t_{k-1},t_{k+1})^{1/p} \leq \left( \frac{2\omega(s,t)}{N-k-1} \right)^{1/p}, \\
					&\left| X^1_{t_{k-1},t_k} - Y^1_{t_{k-1},t_k} \right|,\left| X^1_{t_k,t_{k+1}} - Y^1_{t_k,t_{k+1}} \right|
					\leq \epsilon \omega(t_{k-1},t_{k+1})^{1/p} \leq \epsilon \left( \frac{2\omega(s,t)}{N-k-1} \right)^{1/p}
				\end{align}
				が満たされ,また
				\begin{align}
					\left| X^1_{0,t_{k-1}} - Y^1_{0,t_{k-1}} \right|
					\leq \epsilon \omega(0,t_{k-1})^{1/p}
					\leq \epsilon \omega(0,T)^{1/p}
					\leq \epsilon \left( 2 R^p + 1 \right)^{1/p}
				\end{align}
				でもあるから,
				\begin{align}
					M &\coloneqq \sum_{i,j} \sup{x \in \R^d}{|f^i_j(x)|} + \sum_{i,j,k} \sup{x \in \R^d}{|\partial_k f^i_j(x)|}
						+ \sum_{i,j,k,v} \sup{x \in \R^d}{|\partial_v \partial_k f^i_j(x)|}
					\label{eq:continuity_theorem_1_4}
				\end{align}
				と定めて
				\begin{align}
					&\left| \left\{ \tilde{I}_{s,t}(x,D_{-k}) - \tilde{I}_{s,t}(x,D_{-k-1}) \right\} - 
						\left\{ \tilde{I}_{s,t}(y,D_{-k}) - \tilde{I}_{s,t}(y,D_{-k-1}) \right\} \right| \\
					&\leq M \left|X^1_{t_{k-1},t_k}\right|\left| X^1_{t_k,t_{k+1}} - Y^1_{t_k,t_{k+1}} \right| \\
						&\quad + M \left| X^1_{t_{k-1},t_k} - Y^1_{t_{k-1},t_k} \right| \left| Y^1_{t_k,t_{k+1}} \right| \\
						&\quad + M \left| X^1_{0,t_{k-1}} - Y^1_{0,t_{k-1}} \right| \left| Y^1_{t_{k-1},t_k} \right|\left| Y^1_{t_k,t_{k+1}} \right| \\
						&\quad + M \left| X^1_{t_{k-1},t_k} - Y^1_{t_{k-1},t_k} \right| \left| Y^1_{t_{k-1},t_k} \right|\left| Y^1_{t_k,t_{k+1}} \right| \\
					&\leq \epsilon M \left[2 + 2 \left( 2 R^p + 1 \right)^{1/p} \right] \left( \frac{2\omega(s,t)}{N-k-1} \right)^{2/p} \\
					&\leq \epsilon M \left[2 + 2 \left( 2 R^p + 1 \right)^{1/p} \right] 2^{2/p} \left( 2 R^p + 1 \right)^{2/p} \left( \frac{1}{N-k-1} \right)^{2/p}
				\end{align}
				を得る.
				\begin{align}
					C_1' \coloneqq  M \left[2 + 2 \left( 2 R^p + 1 \right)^{1/p} \right] 2^{2/p} \left( 2 R^p + 1 \right)^{2/p}
				\end{align}
				とおけば
				\begin{align}
					|(\refeq{eq:continuity_theorem_1_1})| \leq
					\sum_{k=0}^{N-2} \epsilon C_1' \left( \frac{1}{N-k-1} \right)^{2/p}
					< \epsilon C_1' \zeta \biggl( \frac{2}{p} \biggr)
				\end{align}
				が成立し,$p < 2$より$\zeta(2/p) < \infty$であるから
				$C_1 \coloneqq C_1' \zeta(2/p)$とおいて(\refeq{eq:continuity_theorem_1_3})が従う.
			
			\item[第四段]
				$x_0 = y_0$の仮定により$x_s-y_s = X^1_{0,s} - Y^1_{0,s}$が成り立ち
				\begin{align}
					\left| \tilde{I}_{s,t}(x) - \tilde{I}_{s,t}(y) \right|
					&= \left| f(x_s)X^1_{s,t} - f(y_s)Y^1_{s,t} \right| \\
					&\leq \left| f(x_s)X^1_{s,t} - f(x_s)Y^1_{s,t} \right|
						+ \left| f(x_s)Y^1_{s,t} - f(y_s)Y^1_{s,t} \right| \\
					&\leq M \left| X^1_{s,t} - Y^1_{s,t} \right|
						+ \left| \int_0^1 (\nabla f)(y_s + \theta(x_s - y_s)) \left[ \left( X^1_{0,s} - Y^1_{0,s}\right) \otimes Y^1_{s,t} \right]\ d\theta \right| \\
					&\leq M \left| X^1_{s,t} - Y^1_{s,t} \right|
						+ M \left| X^1_{0,s} - Y^1_{0,s}\right| \left| Y^1_{s,t} \right| \\
					&\leq M \epsilon \omega(s,t)^{1/p} + M \epsilon \omega(0,s)^{1/p} \omega(s,t)^{1/p} \\
					&\leq \epsilon M\left[\left( 2 R^p + 1 \right)^{1/p} + \left( 2 R^p + 1 \right)^{2/p} \right]
				\end{align}
				が従う.ここで$C_2 \coloneqq M\left[\left( 2 R^p + 1 \right)^{1/p} + \left( 2 R^p + 1 \right)^{2/p} \right]$とおく.
				
			\item[第五段]
				第二段と第三段より,任意の$D \in \delta[s,t]$に対し
				\begin{align}
					\left| \tilde{I}_{s,t}(x,D) - \tilde{I}_{s,t}(y,D) \right|
					\leq \epsilon (C_1 + C_2)
				\end{align}
				が成立し.定理\ref{thm:Riemann_Stieltjes_approximation}により
				$|D| \longrightarrow 0$として
				\begin{align}
					\left| I_{s,t}(x) - I_{s,t}(y) \right|
					\leq \epsilon (C_1 + C_2)
				\end{align}
				が出る.
				\QED
		\end{description}
	\end{prf}
	
	\begin{screen}
		\begin{thm}[$2 \leq p < 3$の場合の連続性定理]\label{thm:continuity_theorem_2}
			$2 \leq p < 3$とし,
			$x_0 = y_0$を満たす$x,y \in C^1$と$f \in C^2_b(\R^d,L(\R^d \rightarrow \R^m)),\ 0 < \epsilon, R < \infty$を任意に取る.
			このとき,
			\begin{align}
				&\Norm{X^1}{p},\Norm{Y^1}{p},\Norm{X^2}{p/2},\Norm{Y^2}{p/2} \leq R < \infty,\\
				&\Norm{X^1 - Y^1}{p},\Norm{X^2 - Y^2}{p/2} \leq \epsilon
			\end{align}
			なら,或る定数$C = C(p,R,f)$が存在し,任意の$0 \leq s \leq t \leq T$に対して次が成立する:
			\begin{align}
				\left| I_{s,t}(x) - I_{s,t}(y) \right| \leq \epsilon C.
			\end{align}
		\end{thm}
	\end{screen}
	
	\begin{screen}
		\begin{cor}\label{cor:continuity_theorem_2}
			$1 \leq p < 2$とし,
			$x_0 = y_0$を満たす$x,y \in C^1$と$f \in C^2_b(\R^d,L(\R^d \rightarrow \R^m)),\ 0 < R < \infty,\ [s,t] \subset [0,T]$を任意に取る.
			このとき,
			\begin{align}
				\Norm{X^1}{p,[s,t]},\Norm{Y^1}{p,[s,t]}, \Norm{X^2}{p/2,[s,t]},\Norm{Y^2}{p/2,[s,t]} \leq R
			\end{align}
			なら,或る定数$C = C(p,R,f)$が存在して次を満たす:
			\begin{align}
				\left| I_{s,t}(x) - I_{s,t}(y) \right| \leq C\left( \Norm{X^1 - Y^1}{p,[s,t]}+\Norm{X^2 - Y^2}{p/2,[s,t]} \right).
			\end{align}
		\end{cor}
	\end{screen}
	
	\begin{prf}[系\ref{cor:continuity_theorem_2}]
		定理\ref{thm:continuity_theorem_2}において,
		$\epsilon = \Norm{X^1 - Y^1}{p} + \Norm{X^2 - Y^2}{p/2,[s,t]}\ (x \neq y)$
		として証明が通る.
		$[0,T]$で考察したことを$[s,t]$に置き換えれば系を得る.
		\QED
	\end{prf}
	
	\begin{prf}[定理\ref{cor:continuity_theorem_2}]
		$[s,t] \subset [0,T]$とする.
		\begin{description}
			\item[第一段]
				$\omega:\Delta_T \longrightarrow [0,\infty)$を
				\begin{align}
					\omega(\alpha,\beta) &= \Norm{X^1}{p,[\alpha,\beta]}^p + \Norm{Y^1}{p,[\alpha,\beta]}^p 
						+ \Norm{X^2}{p/2,[\alpha,\beta]}^{p/2} + \Norm{Y^2}{p/2,[\alpha,\beta]}^{p/2} \\
						&\quad + \epsilon^{-p} \Norm{X^1 - Y^1}{p,[\alpha,\beta]}^p +  \epsilon^{-p/2} \Norm{X^2 - Y^2}{p/2,[\alpha,\beta]}^{p/2},
					\quad ((\alpha,\beta) \in \Delta_T)
				\end{align}
				で定めれば,定理\ref{thm:examples_of_control_functions}により$2 \leq p$の下で$\omega$はcontrol functionである.
				
			\item[第二段]
				$D \in \delta[s,t]$に対し,
				定理\ref{thm:continuity_theorem_1}の証明と同様にして
				$t_{(k)},D_{-k}$を構成すれば
				\begin{align}
					&J_{s,t}(x,D) - J_{s,t}(y,D) \\
					&\qquad= \sum_{k=0}^{N-2} \left[ \left\{ J_{s,t}(x,D_{-k}) - J_{s,t}(x,D_{-k-1}) \right\} - 
						\left\{ J_{s,t}(y,D_{-k}) - J_{s,t}(y,D_{-k-1}) \right\} \right] \label{eq:continuity_theorem_2_1}\\
					&\quad\qquad + \left\{ J_{s,t}(x) - J_{s,t}(y) \right\}	\label{eq:continuity_theorem_2_2}
				\end{align}
				と表現できる.
			
			\item[第三段]
				$J_{s,t}(x,D_{-k}) - J_{s,t}(x,D_{-k-1})$を変形する.
				以降$t_k = t_{(k)}$と書き直せば
				\begin{align}
					&J_{s,t}(x,D_{-k}) - J_{s,t}(x,D_{-k-1}) \\
					&\qquad = J_{t_{k-1},t_k}(x) + J_{t_k,t_{k+1}}(x) - J_{t_{k-1},t_{k+1}}(x) \\
					&\qquad = f(x_{t_{k-1}}) X^1_{t_{k-1},t_k} + f(x_{t_k}) X^1_{t_k,t_{k+1}} - f(x_{t_{k-1}}) X^1_{t_{k-1},t_{k+1}} \\
						&\qquad \qquad + (\nabla f)(x_{t_{k-1}})X^2_{t_{k-1},t_k} + (\nabla f)(x_{t_k})X^2_{t_k,t_{k+1}} - (\nabla f)(x_{t_{k-1}})X^2_{t_{k-1},t_{k+1}} \\
					&\qquad = \left\{ f(x_{t_k}) - f(x_{t_{k-1}}) \right\} X^1_{t_k,t_{k+1}} \\
						&\qquad \qquad + (\nabla f)(x_{t_{k-1}})X^2_{t_{k-1},t_k} + (\nabla f)(x_{t_k})X^2_{t_k,t_{k+1}} - (\nabla f)(x_{t_{k-1}})X^2_{t_{k-1},t_{k+1}} \\
					&\qquad = \int_0^1 \left\{ (\nabla f)(x_{t_{k-1}} + \theta(x_{t_k} - x_{t_{k-1}})) - (\nabla f)(x_{t_{k-1}}) \right\} X^1_{t_{k-1},t_k} \otimes X^1_{t_k,t_{k+1}}\ d\theta \\
						&\qquad \qquad + (\nabla f)(x_{t_{k-1}})X^1_{t_{k-1},t_k} \otimes X^1_{t_k,t_{k+1}} \\
						&\qquad \qquad + (\nabla f)(x_{t_{k-1}})X^2_{t_{k-1},t_k} + (\nabla f)(x_{t_k})X^2_{t_k,t_{k+1}} - (\nabla f)(x_{t_{k-1}})X^2_{t_{k-1},t_{k+1}} \\
					&\qquad = \int_0^1 \int_0^\theta (\nabla f)(x_{t_{k-1}} + r(x_{t_k} - x_{t_{k-1}}))  X^1_{t_{k-1},t_k} \otimes X^1_{t_{k-1},t_k} \otimes X^1_{t_k,t_{k+1}}\ dr\ d\theta \\
						&\qquad \qquad + (\nabla f)(x_{t_{k-1}})\left( X^1_{t_{k-1},t_k} \otimes X^1_{t_k,t_{k+1}} + X^2_{t_{k-1},t_k} - X^2_{t_{k-1},t_{k+1}} \right) \\
						&\qquad \qquad + (\nabla f)(x_{t_k})X^2_{t_k,t_{k+1}} \\
					&\qquad = \int_0^1 \int_0^\theta (\nabla f)(x_{t_{k-1}} + r(x_{t_k} - x_{t_{k-1}}))  X^1_{t_{k-1},t_k} \otimes X^1_{t_{k-1},t_k} \otimes X^1_{t_k,t_{k+1}}\ dr\ d\theta \\
						&\qquad \qquad + \left\{ (\nabla f)(x_{t_k}) - (\nabla f)(x_{t_{k-1}}) \right\}X^2_{t_k,t_{k+1}} \\
					&\qquad = \int_0^1 \int_0^\theta (\nabla f)(x_{t_{k-1}} + r(x_{t_k} - x_{t_{k-1}}))  X^1_{t_{k-1},t_k} \otimes X^1_{t_{k-1},t_k} \otimes X^1_{t_k,t_{k+1}}\ dr\ d\theta \\
						&\qquad \qquad + \int_0^1 (\nabla^2 f)(x_{t_{k-1}} + \theta(x_{t_k} - x_{t_{k-1}})) X^1_{t_{k-1},t_k} \otimes X^2_{t_k,t_{k+1}}\ d\theta
				\end{align}
				を得る.
				
			\item[第四段]
				式(\refeq{eq:continuity_theorem_2_1})について,次を満たす定数$C_1$が存在することを示す:
				\begin{align}
					|(\refeq{eq:continuity_theorem_2_1})| \leq \epsilon C_1.
					\label{eq:continuity_theorem_2_3}
				\end{align}
				実際,前段の結果より
				\begin{align}
					&\left\{ J_{s,t}(x,D_{-k}) - J_{s,t}(x,D_{-k-1}) \right\} - 
						\left\{ J_{s,t}(y,D_{-k}) - J_{s,t}(y,D_{-k-1}) \right\} \\
					&=	\int_0^1 \int_0^\theta (\nabla^2 f)(x_{t_{k-1}}+r(x_{t_k}-x_{t_{k-1}})) X^1_{t_{k-1},t_k} \otimes X^1_{t_{k-1},t_k} \otimes X^1_{t_k,t_{k+1}}\ dr\ d\theta \\
						&\quad + \int_0^1 (\nabla^2 f)(x_{t_{k-1}}+\theta(x_{t_k}-x_{t_{k-1}}))X^1_{t_{k-1},t_k} \otimes X^2_{t_k,t_{k+1}}\ d\theta \\
						&\quad - \int_0^1 \int_0^\theta (\nabla^2 f)(y_{t_{k-1}}+r(y_{t_k}-y_{t_{k-1}})) Y^1_{t_{k-1},t_k} \otimes Y^1_{t_{k-1},t_k} \otimes Y^1_{t_k,t_{k+1}}\ dr\ d\theta \\
						&\quad - \int_0^1 (\nabla^2 f)(y_{t_{k-1}}+\theta(y_{t_k}-y_{t_{k-1}}))Y^1_{t_{k-1},t_k} \otimes Y^2_{t_k,t_{k+1}}\ d\theta \\
					&= \int_0^1 \int_0^\theta (\nabla^2 f)(x_{t_{k-1}}+r(x_{t_k}-x_{t_{k-1}})) X^1_{t_{k-1},t_k} \otimes X^1_{t_{k-1},t_k} \otimes \left(X^1_{t_k,t_{k+1}} - Y^1_{t_k,t_{k+1}}\right)\ dr\ d\theta \\
						&\quad + \int_0^1 \int_0^\theta (\nabla^2 f)(x_{t_{k-1}}+r(x_{t_k}-x_{t_{k-1}})) X^1_{t_{k-1},t_k} \otimes \left(X^1_{t_{k-1},t_k} - Y^1_{t_{k-1},t_k} \right) \otimes Y^1_{t_k,t_{k+1}}\ dr\ d\theta \\
						&\quad +  \int_0^1 \int_0^\theta \left\{ (\nabla^2 f)(x_{t_{k-1}}+r(x_{t_k}-x_{t_{k-1}})) - (\nabla^2 f)(y_{t_{k-1}}+r(y_{t_k}-y_{t_{k-1}})) \right\} \\
						&\qquad\qquad\qquad X^1_{t_{k-1},t_k} \otimes Y^1_{t_{k-1},t_k} \otimes Y^1_{t_k,t_{k+1}}\ dr\ d\theta \\
						&\quad + \int_0^1 \int_0^\theta (\nabla^2 f)(y_{t_{k-1}}+r(y_{t_k}-y_{t_{k-1}})) \left( X^1_{t_{k-1},t_k} - Y^1_{t_{k-1},t_k} \right) \otimes Y^1_{t_{k-1},t_k} \otimes Y^1_{t_k,t_{k+1}}\ dr\ d\theta \\
						&\quad + \int_0^1 (\nabla^2 f)(x_{t_{k-1}}+\theta(x_{t_k}-x_{t_{k-1}}))X^1_{t_{k-1},t_k} \otimes \left(X^2_{t_k,t_{k+1}} -  Y^2_{t_k,t_{k+1}}\right)\ d\theta \\
						&\quad + \int_0^1 \left\{ (\nabla^2 f) (x_{t_{k-1}}+\theta(x_{t_k}-x_{t_{k-1}})) - (\nabla^2 f)(y_{t_{k-1}}+\theta(y_{t_k}-y_{t_{k-1}})) \right\} \\
						&\qquad\qquad\qquad X^1_{t_{k-1},t_k} \otimes Y^2_{t_k,t_{k+1}}\ d\theta \\
						&\quad + \int_0^1 (\nabla^2 f)(y_{t_{k-1}}+\theta(y_{t_k}-y_{t_{k-1}})) \left( X^1_{t_{k-1},t_k} - Y^1_{t_{k-1},t_k} \right) \otimes Y^2_{t_k,t_{k+1}}\ d\theta \\
					&= \int_0^1 \int_0^\theta (\nabla^2 f)(x_{t_{k-1}}+r(x_{t_k}-x_{t_{k-1}})) X^1_{t_{k-1},t_k} \otimes X^1_{t_{k-1},t_k} \otimes \left(X^1_{t_k,t_{k+1}} - Y^1_{t_k,t_{k+1}}\right)\ dr\ d\theta \\
						&\quad + \int_0^1 \int_0^\theta (\nabla^2 f)(x_{t_{k-1}}+r(x_{t_k}-x_{t_{k-1}})) X^1_{t_{k-1},t_k} \otimes \left(X^1_{t_{k-1},t_k} - Y^1_{t_{k-1},t_k} \right) \otimes Y^1_{t_k,t_{k+1}}\ dr\ d\theta \\
						&\quad +  \int_0^1 \int_0^\theta \int_0^1 (\nabla^3 f)(y_{t_{k-1}}+r(y_{t_k}-y_{t_{k-1}}) + u(x_{t_{k-1}}+r(x_{t_k}-x_{t_{k-1}}) - y_{t_{k-1}}-r(y_{t_k}-y_{t_{k-1}}))) \\
						&\qquad\qquad\qquad \left\{ \left( X^1_{0,t_{k-1}} - Y^1_{0,t_{k-1}} \right) + r\left( X^1_{t_{k-1},t_k} - Y^1_{t_{k-1},t_k} \right) \right\} \otimes X^1_{t_{k-1},t_k} \otimes Y^1_{t_{k-1},t_k} \otimes Y^1_{t_k,t_{k+1}}\ du\ dr\ d\theta \\
						&\quad + \int_0^1 \int_0^\theta (\nabla^2 f)(y_{t_{k-1}}+r(y_{t_k}-y_{t_{k-1}})) \left( X^1_{t_{k-1},t_k} - Y^1_{t_{k-1},t_k} \right) \otimes Y^1_{t_{k-1},t_k} \otimes Y^1_{t_k,t_{k+1}}\ dr\ d\theta \\
						&\quad + \int_0^1 (\nabla^2 f)(x_{t_{k-1}}+\theta(x_{t_k}-x_{t_{k-1}}))X^1_{t_{k-1},t_k} \otimes \left(X^2_{t_k,t_{k+1}} -  Y^2_{t_k,t_{k+1}}\right)\ d\theta \\
						&\quad + \int_0^1 \int_0^1 (\nabla^3 f) (y_{t_{k-1}}+\theta(y_{t_k}-y_{t_{k-1}}) + r(x_{t_{k-1}}+\theta(x_{t_k}-x_{t_{k-1}})-y_{t_{k-1}}-\theta(y_{t_k}-y_{t_{k-1}}))) \\
						&\qquad\qquad\qquad \left\{ \left( X^1_{0,t_{k-1}} - Y^1_{0,t_{k-1}} \right) + \theta\left( X^1_{t_{k-1},t_k} - Y^1_{t_{k-1},t_k} \right) \right\} \otimes X^1_{t_{k-1},t_k} \otimes Y^2_{t_k,t_{k+1}}\ dr\ d\theta \\
						&\quad + \int_0^1 (\nabla^2 f)(y_{t_{k-1}}+\theta(y_{t_k}-y_{t_{k-1}})) \left( X^1_{t_{k-1},t_k} - Y^1_{t_{k-1},t_k} \right) \otimes Y^2_{t_k,t_{k+1}}\ d\theta
				\end{align}
				が成り立つから,
				\begin{align}
					M &\coloneqq \sum_{i,j} \sup{x \in \R^d}{|f^i_j(x)|} + \sum_{i,j,k} \sup{x \in \R^d}{|\partial_k f^i_j(x)|} \\
						&\qquad + \sum_{i,j,k,v} \sup{x \in \R^d}{|\partial_v \partial_k f^i_j(x)|}
						+ \sum_{i,j,k,v,w} \sup{x \in \R^d}{|\partial_w \partial_v \partial_k f^i_j(x)|}
					\label{eq:continuity_theorem_2_4}
				\end{align}
				とおいて
				\begin{align}
					&\left| \left\{ J_{s,t}(x,D_{-k}) - J_{s,t}(x,D_{-k-1}) \right\} - 
						\left\{ J_{s,t}(y,D_{-k}) - J_{s,t}(y,D_{-k-1}) \right\} \right| \\
					&\leq M \left|X^1_{t_{k-1},t_k}\right| \left|X^1_{t_{k-1},t_k}\right| \left| X^1_{t_k,t_{k+1}} - Y^1_{t_k,t_{k+1}} \right| \\
						&\quad + M \left| X^1_{t_{k-1},t_k} \right| \left| X^1_{t_{k-1},t_k} - Y^1_{t_{k-1},t_k} \right| \left| Y^1_{t_k,t_{k+1}} \right| \\
						&\quad + M \left| X^1_{0,t_{k-1}} - Y^1_{0,t_{k-1}} \right| \left| X^1_{t_{k-1},t_k} \right| \left| Y^1_{t_{k-1},t_k} \right| \left| Y^1_{t_k,t_{k+1}} \right| \\
						&\quad + M \left| X^1_{t_{k-1},t_k} - Y^1_{t_{k-1},t_k} \right| \left| X^1_{t_{k-1},t_k} \right| \left| Y^1_{t_{k-1},t_k} \right| \left| Y^1_{t_k,t_{k+1}} \right| \\
						&\quad + M \left| X^1_{t_{k-1},t_k} - Y^1_{t_{k-1},t_k} \right| \left| Y^1_{t_{k-1},t_k} \right| \left| Y^1_{t_k,t_{k+1}} \right| \\
						&\quad + M \left| X^1_{t_{k-1},t_k} \right| \left| X^2_{t_k,t_{k+1}} - Y^2_{t_k,t_{k+1}} \right| \\
						&\quad + M \left| X^1_{0,t_{k-1}} - Y^1_{0,t_{k-1}} \right| \left| X^1_{t_{k-1},t_k} \right| \left| Y^2_{t_k,t_{k+1}} \right| \\
						&\quad + M \left| X^1_{t_{k-1},t_k} - Y^1_{t_{k-1},t_k} \right| \left| X^1_{t_{k-1},t_k} \right| \left| Y^2_{t_k,t_{k+1}} \right| \\
						&\quad + M \left| X^1_{t_{k-1},t_k} - Y^1_{t_{k-1},t_k} \right| \left| Y^2_{t_k,t_{k+1}} \right| \\
					&\leq \epsilon M \left[5 + 2\omega(0,t_{k-1})^{1/p} + 2\omega(t_{k-1},t_k)^{1/p} \right] \left( \frac{2\omega(s,t)}{N-k-1} \right)^{3/p} \\
					&\leq \epsilon M \left[2 + 4\left( 2 R^p + 2R^{p/2} + 2 \right)^{1/p} \right] 2^{3/p} \left( 2 R^p + 2R^{p/2} + 2 \right)^{3/p} \left( \frac{1}{N-k-1} \right)^{3/p}
				\end{align}
				を得る.ここで
				\begin{align}
					C_1' \coloneqq  M \left[2 + 4\left( 2 R^p + 2R^{p/2} + 2 \right)^{1/p} \right] 2^{3/p} \left( 2 R^p + 2R^{p/2} + 2 \right)^{3/p}
				\end{align}
				と定めれば
				\begin{align}
					|(\refeq{eq:continuity_theorem_2_1})| \leq
					\sum_{k=0}^{N-2} \epsilon C_1' \left( \frac{1}{N-k-1} \right)^{3/p}
					< \epsilon C_1' \zeta \biggl( \frac{3}{p} \biggr)
				\end{align}
				が成立し,$p < 3$より$\zeta(3/p) < \infty$であるから
				$C_1 \coloneqq C_1' \zeta(3/p)$とおいて(\refeq{eq:continuity_theorem_2_3})が出る.
			
			\item[第五段]
				$x_0 = y_0$の仮定により
				\begin{align}
					&\left| J_{s,t}(x) - J_{s,t}(y) \right| \\
					&\leq \left| f(x_s)X^1_{s,t} - f(y_s)Y^1_{s,t} \right| + \left| (\nabla f)(x_s)X^2_{s,t} - (\nabla f)(y_s)Y^2_{s,t} \right| \\
					&\leq \left| f(x_s)X^1_{s,t} - f(x_s)Y^1_{s,t} \right|
						+ \left| f(x_s)Y^1_{s,t} - f(y_s)Y^1_{s,t} \right| \\
						&\qquad + \left| (\nabla f)(x_s)X^2_{s,t} - (\nabla f)(x_s)Y^2_{s,t} \right|
						+ \left| (\nabla f)(x_s)Y^2_{s,t} - (\nabla f)(y_s)Y^2_{s,t} \right| \\
					&\leq M \left| X^1_{s,t} - Y^1_{s,t} \right|
						+ \left| \int_0^1 (\nabla f)(y_s + \theta(x_s - y_s))(x_s - y_s) \otimes Y^1_{s,t}\ d\theta \right| \\
						&\qquad + M \left| X^2_{s,t} - Y^2_{s,t} \right|
						+ \left| \int_0^1 (\nabla^2 f)(y_s + \theta(x_s - y_s))(x_s - y_s) \otimes Y^2_{s,t}\ d\theta \right| \\
					&\leq M \left| X^1_{s,t} - Y^1_{s,t} \right|
						+ M \left| X^1_{0,s} - Y^1_{0,s}\right| \left| Y^1_{s,t} \right| \\
						&\qquad + M \left| X^2_{s,t} - Y^2_{s,t} \right| 
						+ M \left| X^1_{0,s} - Y^1_{0,s}\right| \left| Y^2_{s,t} \right| \\
					&\leq \epsilon M \omega(s,t)^{1/p} + \epsilon M \omega(0,s)^{1/p} \omega(s,t)^{1/p} \\
						&\quad + \epsilon M \omega(s,t)^{2/p} + \epsilon M \omega(0,s)^{1/p} \omega(s,t)^{2/p} \\
					&\leq \epsilon M \left[ \omega(0,T)^{1/p} + 2\omega(0,T)^{2/p} + \omega(0,T)^{3/p} \right] \\
					&\leq \epsilon M \left[ \left( 2 R^p + 2R^{p/2} + 2 \right)^{1/p}
						+ 2\left( 2 R^p + 2R^{p/2} + 2 \right)^{2/p}
						+\left( 2 R^p + 2R^{p/2} + 2 \right)^{3/p} \right]
				\end{align}
				が従う.ここで最下段の$\epsilon$の係数を$C_2$とおく.
				
			\item[第六段]
				以上より,任意の$D \in \delta[s,t]$に対し
				\begin{align}
					\left| J_{s,t}(x,D) - J_{s,t}(y,D) \right|
					\leq \epsilon (C_1 + C_2)
				\end{align}
				が成り立ち,定理\ref{thm:Riemann_Stieltjes_approximation}により
				$|D| \longrightarrow 0$として
				\begin{align}
					\left| I_{s,t}(x) - I_{s,t}(y) \right|
					\leq \epsilon (C_1 + C_2)
				\end{align}
				が出る.
				\QED
		\end{description}
	\end{prf}
	
	\begin{screen}
		\begin{cor}[パスが0出発なら$f$の有界性は要らない]
			定理\ref{thm:continuity_theorem_1}と定理\ref{thm:continuity_theorem_2}について,
			$x,y \in \tilde{C}^1$ならば
			$f \in C^2(\R^d,L(\R^d \rightarrow \R^m))$として主張が成り立つ.
		\end{cor}
	\end{screen}
	
	\begin{prf}
		$x_0 = 0$なら
		\begin{align}
			\Norm{X^1}{p} \leq R \quad \Rightarrow \quad |x_t| \leq R \quad (\forall t \in [0,T])
		\end{align}
		が成り立つから,式(\refeq{eq:continuity_theorem_1_4})と(\refeq{eq:continuity_theorem_2_4})において
		$\sup{x\in\R^d}{}$を$\sup{|x| \leq 9R}$に替えればよい.
		\QED
	\end{prf}
	%\section{Young 積分}
	\begin{screen}
		\begin{lem}\label{lem:lemma_for_Young_integral}
			$x \in C^1,\ f \in C^2(\R^d,L(\R^d \rightarrow \R^m))$とする.
			\begin{description}
				\item[(1)]
					$1 \leq p < 2$の場合,或るcontrol function $\omega$が存在して
					\begin{align}
						\left| X^1_{s,t} \right|
						\leq \omega(s,t)^{1/p},
						\quad (0 \leq \forall s \leq \forall t \leq T)
					\end{align}
					を満たすとき,ある定数$C=C(p,f)$があり
					\begin{align}
						\left|I_{s,t}(x) \right|
						\leq C\left( \omega(s,t)^{1/p} + \omega(s,t)^{2/p} \right).
					\end{align}
					が成立する.
				\item[(2)] $2 \leq p < 3$の場合,或るcontrol function $\omega$が存在して
					\begin{align}
						\left| X^1_{s,t} \right| \leq \omega(s,t)^{1/p},
						\quad \left| X^2_{s,t} \right| \leq \omega(s,t)^{2/p},
						\quad (0 \leq \forall s \leq \forall t \leq T)
					\end{align}
					を満たすとき,ある定数$C=C(p,f)$があり
					\begin{align}
						\left|I_{s,t}(x) \right|
						\leq C\left( \omega(s,t)^{1/p} + \omega(s,t)^{2/p} + \omega(s,t)^{3/p} \right).
					\end{align}
					が成立する.
			\end{description}
		\end{lem}
	\end{screen}
	
	\begin{prf}\mbox{}
		\begin{description}
			\item[(1)] $D = \{s=t_0 < \cdots < t_N = t\}\ (N \geq 2)$に対し,
				補題\refeq{lem:control_function_min}により存在する
				$i$を取り$D_{-1} \coloneqq D \backslash \{i\}$と書く.
				補題\refeq{lem:control_function_min}の添数を除く作業を続けて
				$D_{-k}\ (k=1,\cdots,N-1)$を構成する.
				\begin{align}
					M \coloneqq \max{
					\substack{t \in [0,T] \\ 1 \leq i \leq m \\ 1 \leq j,k \leq d}}
					{\left| \partial_k f^i_j(x_t) \right|},
					\quad 
					M' \coloneqq \max{t \in [0,T]}{\left| f(x_t) \right|}
				\end{align}
				とおけば$M,M' < \infty$であり,$\left| X^1_{t_i,t_{i+1}} \right| \leq \omega(t_i,t_{i+1})^{1/p} \leq \omega(t_{i-1},t_{i+1})^{1/p}$と補題\refeq{lem:control_function_min}により
				\begin{align}
					\left| \tilde{I}_{s,t}(x,D) - \tilde{I}_{s,t}(x,D_{-1}) \right|
					&= \left| \tilde{I}_{t_{i-1},t_i}(x) + \tilde{I}_{t_i,t_{i+1}}(x) - \tilde{I}_{t_{i-1},t_{i+1}}(x) \right| \\
					& \leq \left| \left\{ f(x_{t_i}) - f(x_{t_{i-1}}) \right\} X^1_{t_i,t_{i+1}} \right| \\
					&\leq \left| \left\{ \int_0^1 (\nabla f)(x_{t_{i-1}} + \theta (x_{t_i}-x_{t_{i-1}}))\ d\theta \right\} X^1_{t_{i-1},t_i} \otimes X^1_{t_i,t_{i+1}} \right| \\
					&\leq md^2 M \left| X^1_{t_i,t_{i+1}} \right|^2 \\
					&\leq md^2 M \left( \frac{2\omega(s,t)}{N-1} \right)^{2/p}
				\end{align}
				が成立する.同様に
				\begin{align}
					\left| \tilde{I}_{s,t}(x,D_{-k}) - \tilde{I}_{s,t}(x,D_{-k-1}) \right|
					\leq md^2 M \left( \frac{2\omega(s,t)}{N-k-1} \right)^{2/p},
					\quad(k=0,\cdots,N-2)
				\end{align}
				が成り立ち$(D_{-0} = D)$
				\begin{align}
					\left| \tilde{I}_{s,t}(x,D) - f(x_s)X^1_{s,t} \right|
					&\leq \sum_{k=0}^{N-2} \left| \tilde{I}_{s,t}(x,D_{-k}) - \tilde{I}_{s,t}(x,D_{-k-1}) \right| \\
					&\leq md^2 M (2\omega(s,t))^{2/p} \sum_{k=0}^{N-2} \left( \frac{1}{N-k-1} \right)^{2/p} \\
					&\leq md^2 M (2\omega(s,t))^{2/p} \zeta\biggl(\frac{2}{p}\biggr)
				\end{align}
				が従う.いま,仮定より$p < 2$であるから$\zeta(2/p) < \infty$であり,
				定理\ref{thm:Riemann_Stieltjes_approximation}より
				\begin{align}
					\left| I_{s,t}(x) \right|
					\leq M' \omega(s,t)^{1/p} + md^2 M (2\omega(s,t))^{2/p} \zeta\biggl(\frac{2}{p}\biggr)
				\end{align}
				を得る.
				
			\item[(2)]
				(1)と同様に$D_{-k}\ (k=1,\cdots,N-1)$を構成する.
				会田先生のノートの通りに
				\begin{align}
					J_{s,t}(x,D) - J_{s,t}(x,D_{-1})
					&= \left\{ \int_0^1 \int_0^\theta (\nabla^2 f)(x_{t_{i-1}} + \theta (x_{t_i}- x_{t_{i-1}}))\ dr\ d\theta \right\}
						X^1_{t_{i-1},t_i} \otimes X^1_{t_{i-1},t_i} \otimes X^1_{t_i,t_{i+1}} \\
						&\qquad + \left\{ \int_0^1 (\nabla^2 f)(x_{t_{i-1}} + \theta (x_{t_i}- x_{t_{i-1}}))\ d\theta \right\}
						X^1_{t_{i-1},t_i} \otimes X^2_{t_i,t_{i+1}}
				\end{align}
				を得る.ここで(1)の$M,M'$に加えて
				\begin{align}
					M'' \coloneqq \max{
					\substack{t \in [0,T] \\ 1 \leq i \leq m \\ 1 \leq j,k,v \leq d}}
					{\left| \partial_v \partial_k f^i_j(x_t) \right|}
				\end{align}
				とおけば
				\begin{align}
					\left| J_{s,t}(x,D_{-k}) - J_{s,t}(x,D_{-k-1}) \right|
					&\leq m d^2 M \left( \frac{2\omega(s,t)}{N-k-1} \right)^{3/p} + m d^2 M'' \left( \frac{2\omega(s,t)}{N-k-1} \right)^{1/p} \left( \frac{2\omega(s,t)}{N-k-1} \right)^{2/p} \\
					&\leq m d^2 (M + M'') \left( \frac{2\omega(s,t)}{N-k-1} \right)^{3/p}
				\end{align}
				が成立し,会田先生のノートの通りに
				\begin{align}
					\left| J_{s,t}(x,D) - \left( f(x_s)X^1_{s,t} + (\nabla f)(x_s)X^2_{s,t} \right) \right|
					\leq 2^{3/p} m d^2 (M+M'') \zeta\biggl(\frac{3}{p}\biggr) \omega(s,t)^{3/p}
				\end{align}
				が従い,$p < 3$の仮定より$\zeta(3/p) < \infty$である.(1)と同じく定理\ref{thm:Riemann_Stieltjes_approximation}より
				\begin{align}
					\left| I_{s,t}(x) \right|
					\leq M' \omega(s,t)^{1/p} + m d^2 M \omega(s,t)^{2/p} + 2^{3/p} m d^2 (M+M'') \zeta\biggl(\frac{3}{p}\biggr) \omega(s,t)^{3/p}
				\end{align}
				となる.
				\QED
		\end{description}
	\end{prf}
	
	\section{The notion of rough path}
	$(V,\Norm{\cdot}{})$を$\R$上のBanach空間とする$(V \neq \{0\})$.また
	$\otimes_a$により代数的テンソル積,或はその標準写像を表す.
	$k \geq 2$の場合,$k$重テンソル積$V^{\otimes_a k} = V \otimes_a \cdots \otimes_a V$に
	プロジェクティブノルム$\projectivenorm{\cdot}{k}$を導入し,
	その完備拡大を$(V^{\otimes k},\compcrossnorm{\cdot}{k})$と書く
	\footnote{
		$V$が有限次元なら$V^{\otimes_a k}$も有限次元であるから
		$V^{\otimes k} = V^{\otimes_a k},\ \compcrossnorm{\cdot}{k} = \projectivenorm{\cdot}{k}$
		でよい.しかし一般に$\left( V^{\otimes_a k},\projectivenorm{\cdot}{k} \right)$
		は完備ではない(\cite{key10} (p. 17), \cite{key9}).
	}.
	$k=0,1$に対しては$V^{\otimes 0} \coloneqq \R,\ V^{\otimes 1} \coloneqq V$とし,
	$\compcrossnorm{\cdot}{0} = \projectivenorm{\cdot}{0} \coloneqq \mbox{$\R$の絶対値}$,及び
	$\compcrossnorm{\cdot}{1} = \projectivenorm{\cdot}{1} \coloneqq \Norm{\cdot}{}$と定める.
	定理\ref{thm:tensor_product_with_scalar}
	と定理\ref{thm:associativity_of_tensor_products}
	により,任意の$0 \leq j \leq k$に対し
	$V^{\otimes_a k}$と$V^{\otimes_a j} \otimes_a V^{\otimes_a k-j}$は線型同型となる.
	この同型写像を
	\begin{align}
		F_{j,k}:V^{\otimes_a k} \longrightarrow V^{\otimes_a j} \otimes_a V^{\otimes_a k-j},
		\quad 0 \leq j \leq k
	\end{align}
	書けば,$F_{j,k}$は
	\begin{align}
		F_{0,k}(v) &= 1 \otimes_a v, && (\forall v \in V^{\otimes_a k}), \\
		F_{j,k}(v_1 \otimes_a \cdots \otimes_a v_k) 
			&= (v_1 \otimes_a \cdots \otimes_a v_{j}) \otimes_a (v_{j+1} \otimes_a \cdots \otimes_a v_k), 
			&& (\forall v_1 \otimes_a \cdots \otimes_a v_k \in V^{\otimes_a k},\ 1 \leq j \leq k-1), \\
		F_{k,k}(v) &= v \otimes_a 1, && (\forall v \in V^{\otimes_a k})
	\end{align}
	を満たす.また$V^{\otimes_a j} \otimes_a V^{\otimes_a k-j}$上にもプロジェクティブノルムを導入し,
	これを$\pi_{j,k}$と書く.
	\begin{screen}
		\begin{thm}
			このとき次式が成立する.特に,$F_{j,k},\ (0 \leq j \leq k)$は等長同型である.
			\begin{align}
				\pi_k \circ F^{-1}_{j,k} = \pi_{j,k}, \quad 0 \leq j \leq k.
			\end{align}
			
		\end{thm}
	\end{screen}
	
	\begin{prf}\mbox{}
		\begin{description}
			\item[第一段]
				$j=0$のとき,任意の$v \in V^{\otimes_a k}$に対し
				\begin{align}
					\projectivenorm{F_{0,k}(v)}{0,k}
					= \projectivenorm{1 \otimes_a v}{0,k}
					= \projectivenorm{1}{0}\projectivenorm{v}{k}
					= \projectivenorm{v}{k}
				\end{align}
				が成り立ち$\pi_k \circ F^{-1}_{0,k} = \pi_{0,k}$を得る.
				同様にして$\pi_k \circ F^{-1}_{k,k} = \pi_{k,k}$も出る.
				
			\item[第二段]
				$\pi_k \circ F^{-1}_{j,k} \leq \pi_{j,k},\ (1 \leq j \leq k-1)$が成り立つことを示す.
				$v \in V^{\otimes_a j} \otimes_a V^{\otimes_a k-j}$の分割
				\begin{align}
					v = \sum_{r} u^r \otimes_a v^r,
					\quad (u^r \in V^{\otimes_a j},\ v^r \in V^{\otimes_a k-j})
				\end{align}
				を任意に取り,一旦固定する.このとき$u^r,v^r$の任意の分割
				\begin{align}
					u^r = \sum_{n(r)} u^{n(r)}_1 \otimes_a \cdots \otimes_a u^{n(r)}_j, 
					\quad v^r = \sum_{m(r)} v^{m(r)}_{j+1} \otimes_a \cdots \otimes_a v^{m(r)}_{k},
					\quad (v^{n(r)}_i,v^{m(r)}_i \in V)
				\end{align}
				に対して
				\begin{align}
					\projectivenorm{F^{-1}_{j,k}(v)}{k}
					&\leq \sum_{r} \sum_{n(r),m(r)} \projectivenorm{u^{n(r)}_1 \otimes_a \cdots \otimes_a u^{n(r)}_j \otimes_a v^{m(r)}_{j+1} \otimes_a \cdots \otimes_a v^{m(r)}_{k}}{k} \\
					&= \sum_{r} \sum_{n(r),m(r)} \Norm{u^{n(r)}_1}{} \cdots \Norm{u^{n(r)}_j}{} \Norm{v^{m(r)}_{j+1}}{} \cdots \Norm{v^{m(r)}_k}{} \\
					&= \sum_{r} \left\{ \sum_{n(r)} \Norm{u^{n(r)}_1}{} \cdots \Norm{u^{n(r)}_j}{} \right\} \left\{ \sum_{m(r)} \Norm{v^{m(r)}_{j+1}}{} \cdots \Norm{v^{m(r)}_k}{} \right\} 
				\end{align}
				が成り立つから,分割の任意性と定理\ref{thm:expression_of_projective_norm}より
				\begin{align}
					\projectivenorm{F^{-1}_{j,k}(v)}{k} 
					\leq \sum_r \projectivenorm{u^r}{j}\projectivenorm{v^r}{k-j}
				\end{align}
				を得る.$v$の分割について下限を取れば,
				再び定理\ref{thm:expression_of_projective_norm}により
				\begin{align}
					\projectivenorm{F^{-1}_{j,k}(v)}{k} \leq \projectivenorm{v}{j,k}
				\end{align}
				が出る.
			
			\item[第三段]
				$\pi_k \circ F^{-1}_{j,k} \geq \pi_{j,k},\ (1 \leq j \leq k-1)$が成り立つことを示す.
				$v \in V^{\otimes_a k}$の任意の分割
				\begin{align}
					v = \sum_n v^n_1 \otimes_a \cdots \otimes_a v^n_k,
					\quad (v^n_i \in V,\ i=1,\cdots,k)
				\end{align}
				を取れば,
				\begin{align}
					\projectivenorm{F_{j,k}(v)}{j,k}
					&\leq \sum_n \projectivenorm{\left( v^n_1 \otimes_a \cdots \otimes_a v^n_j \right) \otimes_a \left( v^n_j \otimes_a \cdots \otimes_a v^n_k \right)}{j,k} \\
					&= \sum_n \projectivenorm{v^n_1 \otimes_a \cdots \otimes_a v^n_j}{j}
						\projectivenorm{v^n_j \otimes_a \cdots \otimes_a v^n_k}{k-j} \\
					&= \sum_n \Norm{v^n_1}{} \cdots \Norm{v^n_k}{}
				\end{align}
				が成立する.従って定理\ref{thm:expression_of_projective_norm}より
				\begin{align}
					\projectivenorm{F_{j,k}(v)}{j,k} \leq \projectivenorm{v}{k}
				\end{align}
				が得られる.
				\QED
		\end{description}
	\end{prf}
	
	
	$V^{\otimes_a i}$の$V^{\otimes i}$への等長埋め込みを$J_i$で表し
	($J_i$の終集合を$J_i V^{\otimes_a i}$と考える
	\footnote{
		$J_i$の終集合を$J_i V^{\otimes_a i}$と考えれば全単射であるから
		逆写像$J_i^{-1}$が存在する.
	}.$i=0,1$の場合$J_i$は恒等写像),
	\begin{align}
		J_j V^{\otimes_a j} \times J_{k-j} V^{\otimes_a k-j} \ni (u,v)
		& \longmapsto ( J_j^{-1}u,J_{k-j}^{-1}v ) &&\in V^{\otimes_a j} \times V^{\otimes_a k-j} \\
		& \longmapsto F_{j,k}^{-1} (J_j^{-1}u \otimes_a J_{k-j}^{-1}v) &&\in V^{\otimes_a k} \\
		& \longmapsto J_k F_{j,k}^{-1} (J_j^{-1}u \otimes_a J_{k-j}^{-1}v) &&\in V^{\otimes k}
		\label{bilinear_map_on_algebraic_Banach_tensor_products}
	\end{align}
	の対応関係により定まる写像$:J_j V^{\otimes_a j} \times J_{k-j} V^{\otimes_a k-j}
	\longrightarrow V^{\otimes k}$を$\varphi_{j,k}$と書けば,$\varphi_{j,k}$は有界双線型写像である.
	実際,$\otimes_a$の双線型性と埋め込み及び$F_{j,k}^{-1}$の線型性より
	$\varphi_{j,k}$の双線型性が従い,また
	\begin{align}
		\compcrossnorm{\varphi_{j,k}(u,v)}{k}
		&= \projectivenorm{F_{j,k}^{-1} (J_j^{-1}u \otimes_a J_{k-j}^{-1}v)}{k} \\
		&= \projectivenorm{J_j^{-1}u \otimes_a J_{k-j}^{-1}v}{j,k} \\
		&= \projectivenorm{J_j^{-1}u}{j} \projectivenorm{J_{k-j}^{-1}v}{k-j} \\
		&= \compcrossnorm{u}{j} \compcrossnorm{v}{k-j}
	\end{align}
	が任意の$(u,v) \in J_j V^{\otimes_a j} \times J_{k-j} V^{\otimes_a k-j}$に対して
	成り立つから$\Norm{\varphi_{j,k}}{\ContLn{J_j V^{\otimes_a j} \times J_{k-j} V^{\otimes_a k-j}}{V^{\otimes k}}{2}} = 1$を得る.
	従って,定理\ref{thm:expansion_of_multilinear_mapping}より
	$\varphi_{j,k}$は$V^{\otimes j} \times V^{\otimes k-j}$上の或るただ一つの双線型写像
	$\psi_{j,k}$にノルム保存拡張される.
	
	\begin{screen}
		\begin{thm}\label{thm:property_of_the_completion_of_the_projective_norm}
			$0 \leq j \leq k$とする.
			このとき,$\psi_{j,k}:V^{\otimes j} \times V^{\otimes k-j} \longrightarrow V^{\otimes k}$
			は次を満たす:
			\begin{align}
				\compcrossnorm{\psi_{j,k}(u,v)}{k} = \compcrossnorm{u}{j} \compcrossnorm{v}{k-j},
				\quad (\forall (u,v) \in V^{\otimes j} \times V^{\otimes k-j}).
			\end{align}
		\end{thm}
	\end{screen}
	
	\begin{prf}
		$(u,v)$に直積ノルムで収束する点列$(u_n,v_n) \in J_j V^{\otimes_a j} \times J_{k-j} V^{\otimes_a k-j}\ (n=1,2,\cdots)$を取れば
		\begin{align}
			\compcrossnorm{\varphi_{j,k}(u_n,v_n) - \psi_{j,k}(u,v)}{k} \longrightarrow 0,
			\quad (n \longrightarrow \infty)
		\end{align}
		が成り立つ.また
		\begin{align}
			\left| \compcrossnorm{u_n}{j} \compcrossnorm{v_n}{k-j} - 
			\compcrossnorm{u}{j} \compcrossnorm{v}{k-j} \right|
			&\leq \left| \compcrossnorm{u_n}{j} \compcrossnorm{v_n}{k-j} - 
			\compcrossnorm{u_n}{j} \compcrossnorm{v}{k-j} \right| 
			+ \left| \compcrossnorm{u_n}{j} \compcrossnorm{v}{k-j} - 
			\compcrossnorm{u}{j} \compcrossnorm{v}{k-j} \right| \\
			&\leq \compcrossnorm{u_n}{j} \compcrossnorm{v_n - v}{k-j} 
			+ \compcrossnorm{u_n - u}{j} \compcrossnorm{v}{k-j} \\
			&\longrightarrow 0, \quad (n \longrightarrow \infty)
		\end{align}
		も成立するから
		\begin{align}
			\left|\, \compcrossnorm{\psi_{j,k}(u,v)}{k} - \compcrossnorm{u}{j} \compcrossnorm{v}{k-j}\, \right|
			&\leq \compcrossnorm{\varphi_{j,k}(u_n,v_n) - \psi_{j,k}(u,v)}{k}
				+ \left| \compcrossnorm{u_n}{j} \compcrossnorm{v_n}{k-j} - 
			\compcrossnorm{u}{j} \compcrossnorm{v}{k-j} \right| \\
			& \longrightarrow 0, \quad (n \longrightarrow \infty)
		\end{align}
		が従い$\compcrossnorm{\psi_{j,k}(u,v)}{k} = \compcrossnorm{u}{j} \compcrossnorm{v}{k-j}$
		が得られる.
		\QED
	\end{prf}
	
	$T(V) \coloneqq \bigoplus_{k=0}^{\infty} V^{\otimes k}$
	とおく.また上で定めた双線型写像$\psi_{j,k}$を$\otimes_{j,k}$と書き直す:
	\begin{align}
		u \otimes_{j,k} v = \psi_{j,k}(u,v),
		\quad (\forall (u,v) \in V^{\otimes j} \times V^{\otimes k-j},\ 0 \leq j \leq k,\ k \geq 0).
		\label{eq:def_of_otimes_for_completion_V_tensor_k}
	\end{align}
	このとき,任意の$a=(a_k)_{k=0}^{\infty},\ b=(b_k)_{k=0}^{\infty} \in T(V)$に対し
	\begin{align}
		c_k \coloneqq \sum_{j=0}^{k} a_j \otimes_{j,k} b_{k-j},
		\quad (k=0,1,2,\cdots)
	\end{align}
	で$c_k \in V^{\otimes k}$を定めれば,
	有限個の$k$を除いて$c_k = 0$となり$c = (c_k)_{k=0}^{\infty} \in T(V)$が満たされる.
	これで定まる二項算法を
	\begin{align}
		c = a \otimes b
	\end{align}
	と書けば,次の二つの主張により$T(V)$は$\otimes$を乗法として代数となる.
	
	\begin{screen}
		\begin{thm}[$\otimes$は$T(V)$の乗法となる]\label{thm:otimes_is_a_multiplication}
			$\otimes$は$T(V)$において結合則を満たす双線型写像である.
		\end{thm}
	\end{screen}
	
	\begin{prf} $\otimes$の双線型性は各$\otimes_{j,k}$の双線型性より従う.
		以下,$\otimes$の結合性を示す.
		\begin{description}
			\item[第一段]
				任意の$a,b,c \in T(V)$及び$0 \leq r \leq j \leq k$に対して
				\begin{align}
					\left( a_r \otimes_{r,j} b_{j-r} \right) \otimes_{j,k} c_{k-j}
					= a_r \otimes_{r,k} \left( b_{j-r} \otimes_{j-r,k-r} c_{k-j} \right)
					\label{eq:thm_otimes_is_a_multiplication_1}
				\end{align}
				が成立すれば$\otimes$は結合的である.実際,上式が満たされていれば次が従う:
				\begin{align}
					&((a \otimes b) \otimes c)_k
					= \sum_{j=0}^k (a \otimes b)_j \otimes_{j,k} c_{k-j} \\
					&= \sum_{j=0}^k \sum_{r=0}^j \left( a_r \otimes_{r,j} b_{j-r} \right) \otimes_{j,k} c_{k-j}
					= \sum_{j=0}^k \sum_{r=0}^j a_r \otimes_{r,k} \left( b_{j-r} \otimes_{j-r,k-r} c_{k-j} \right)
					= \sum_{r=0}^k \sum_{j=r}^k a_r \otimes_{r,k} \left( b_{j-r} \otimes_{j-r,k-r} c_{k-j} \right) \\
					&= \sum_{r=0}^k a_r \otimes_{r,k} (b \otimes c)_{k-r} \\
					&= (a \otimes (b \otimes c))_k,
					\quad (\forall k = 0,1,2,\cdots).
				\end{align}
			
			\item[第二段]
				$a_r \in J_r V^{\otimes_a r},\ b_{j-r} \in J_{j-r} V^{\otimes_a j-r},
				\ c_{k-j} \in J_{k-j} V^{\otimes_a k-j}$として
				(\refeq{eq:thm_otimes_is_a_multiplication_1})を示す.
				この場合,(\refeq{bilinear_map_on_algebraic_Banach_tensor_products})の対応関係により
				\begin{align}
					\left( a_r \otimes_{r,j} b_{j-r} \right) \otimes_{j,k} c_{k-j}
					&= J_k F_{j,k}^{-1}\left( F_{r,j}^{-1} \left( J_r^{-1}a_r \otimes_a J_{j-r}^{-1} b_{j-r} \right) \otimes_a J_{k-j}^{-1} c_{k-j} \right), \\
					a_r \otimes_{r,k} \left( b_{j-r} \otimes_{j-r,k-r} c_{k-j} \right)
					&= J_k F_{r,k}^{-1}\left( J_r^{-1}a_r \otimes_a F_{j-r,k-j}^{-1} \left( J_{j-r}^{-1} b_{j-r} \otimes_a J_{k-j}^{-1} c_{k-j} \right) \right)
				\end{align}
				となる.ここで
				\begin{align}
					J_r^{-1} a_r = \sum_{i_1 = 1}^{I_1} v_1^{i_1} \otimes_a \cdots \otimes_a v_r^{i_1},
					\quad J_{j-r}^{-1} b_{j-r} = \sum_{i_2 = 1}^{I_2} v_{r+1}^{i_2} \otimes_a \cdots \otimes_a v_j^{i_2},
					\quad J_{k-j}^{-1} c_{k-j} = \sum_{i_3 = 1}^{I_3} v_{j+1}^{i_3} \otimes_a \cdots \otimes_a v_k^{i_3}
				\end{align}
				と表現できるから
				\begin{align}
					&F_{j,k}^{-1}\left( F_{r,j}^{-1} \left( J_r^{-1}a_r \otimes_a J_{j-r}^{-1} b_{j-r} \right) \otimes_a J_{k-j}^{-1} c_{k-j} \right) \\
					&\qquad = \sum_{i_1=1}^{I_1}\sum_{i_2=1}^{I_2}\sum_{i_3=1}^{I_3} 
						v_1^{i_1} \otimes_a \cdots \otimes_a v_r^{i_1}
						\otimes_a v_{r+1}^{i_2} \otimes_a \cdots \otimes_a v_j^{i_2}
						\otimes_a v_{j+1}^{i_3} \otimes_a \cdots \otimes_a v_k^{i_3} \\
					&\qquad = 	F_{r,k}^{-1}\left( J_r^{-1}a_r \otimes_a F_{j-r,k-j}^{-1} \left( J_{j-r}^{-1} b_{j-r} \otimes_a J_{k-j}^{-1} c_{k-j} \right) \right)
				\end{align}
				が成立し,$J_k$の単射性から(\refeq{eq:thm_otimes_is_a_multiplication_1})が得られる.
			
			\item[第三段]
				一般に$a_r \in V^{\otimes r},\ b_{j-r} \in V^{\otimes j-r},
				\ c_{k-j} \in V^{\otimes k-j}$の場合,
				\begin{align}
					\compcrossnorm{a_r^n - a_r}{r} \longrightarrow 0,
					\quad \compcrossnorm{b_{j-r}^n - b_{j-r}}{j-r} \longrightarrow 0,
					\quad \compcrossnorm{c_{k-j}^n - c_{k-j}}{k-j} \longrightarrow 0,
					\quad (n \longrightarrow \infty)
				\end{align}
				を満たす点列$\left\{ a_r^n \right\}_{n=1}^\infty \subset J_r V^{\otimes_a r},\ 
				\left\{ b_{j-r}^n \right\}_{n=1}^\infty \subset J_{j-r} V^{\otimes_a j-r},\ 
				\left\{ c_{k-j}^n \right\}_{n=1}^\infty \subset J_{k-j} V^{\otimes_a k-j}$
				を取れば,前段の結果より
				\begin{align}
					&\compcrossnorm{\left( a_r \otimes_{r,j} b_{j-r} \right) \otimes_{j,k} c_{k-j}
					- a_r \otimes_{r,k} \left( b_{j-r} \otimes_{j-r,k-r} c_{k-j} \right)}{k} \\
					&\qquad \leq \compcrossnorm{\left( a_r \otimes_{r,j} b_{j-r} \right) \otimes_{j,k} c_{k-j} - \left( a^n_r \otimes_{r,j} b^n_{j-r} \right) \otimes_{j,k} c^n_{k-j}}{k} \\
					&\qquad\qquad + \compcrossnorm{a_r \otimes_{r,k} \left( b_{j-r} \otimes_{j-r,k-r} c_{k-j} \right) - a^n_r \otimes_{r,k} \left( b^n_{j-r} \otimes_{j-r,k-r} c^n_{k-j} \right)}{k} \\
					&\qquad \leq \compcrossnorm{a_r \otimes_{r,j} b_{j-r}}{j} \compcrossnorm{c_{k-j} - c^n_{k-j}}{k-j} 
						+ \compcrossnorm{a_r \otimes_{r,j} b_{j-r} - a^n_r \otimes_{r,j} b^n_{j-r}}{j} \compcrossnorm{c^n_{k-j}}{k-j} \\
					&\qquad\qquad + \compcrossnorm{a_r}{r} \compcrossnorm{b_{j-r} \otimes_{j-r,k-r} c_{k-j} - b^n_{j-r} \otimes_{j-r,k-r} c^n_{k-j}}{k-r}
						+ \compcrossnorm{a_r - a_r^n}{r} \compcrossnorm{b^n_{j-r} \otimes_{j-r,k-r} c^n_{k-j}}{k-r} \\
					&\qquad \leq \compcrossnorm{a_r}{r} \compcrossnorm{b_{j-r}}{j-r} \compcrossnorm{c_{k-j} - c^n_{k-j}}{k-j} 
						+ \left\{ \compcrossnorm{a_r-a^n_r}{r} \compcrossnorm{b_{j-r}}{j-r} 
						+ \compcrossnorm{a^n_r}{r} \compcrossnorm{b_{j-r} - b^n_{j-r}}{j-r} \right\} \compcrossnorm{c^n_{k-j}}{k-j} \\
					&\qquad\qquad + \compcrossnorm{a_r}{r} 
						\left\{ \compcrossnorm{b_{j-r} - b^n_{j-r}}{j-r} \compcrossnorm{c_{k-j}}{k-j} 
						+ \compcrossnorm{b^n_{j-r}}{j-r} \compcrossnorm{c_{k-j}-c^n_{k-j}}{k-j} \right\}
						+ \compcrossnorm{a_r - a_r^n}{r} \compcrossnorm{b^n_{j-r}}{j-r} \compcrossnorm{c^n_{k-j}}{k-j} \\
					&\qquad \longrightarrow 0
					\quad (n \longrightarrow \infty) 
				\end{align}
				が従い(\refeq{eq:thm_otimes_is_a_multiplication_1})が出る.
				\QED
		\end{description}	
	\end{prf}
	
	\begin{screen}
		\begin{thm}[$T(V)$の単位元]
			任意の$k \geq 0$,および任意の$\alpha \in \R$と$v \in V^{\otimes k}$に対し
			\begin{align}
				\alpha \otimes_{0,k} v = v \otimes_{k,k} \alpha = \alpha v
				\label{eq:thm_identity_element_of_T_V}
			\end{align}
			が成立する.特に,$e_0 \coloneqq 1,\ e_k \coloneqq 0\ (k \geq 0)$
			として定める$e=(e_k)_{k=1}^\infty \in T(V)$は$\otimes$に関する単位元である:
			\begin{align}
				\mbox{i.e.}\quad
				e \otimes a = a \otimes e = a,
				\quad (\forall a \in T(V))
			\end{align}
		\end{thm}
	\end{screen}
	
	\begin{prf}
		$v \in J_k V^{\otimes_a k}$の場合は
		(\refeq{bilinear_map_on_algebraic_Banach_tensor_products})
		の対応関係により
		\begin{align}
			\alpha \otimes_{0,k} v
			= J_k F_{0,k}^{-1}\left( \alpha \otimes_a J_k^{-1}v \right)
			= J_k \left( \alpha J_k^{-1}v \right)
			= J_k J_k^{-1} \alpha v
			= \alpha v
		\end{align}
		が成立する.一般の$v \in V^{\otimes k}$に対しては,
		$\lim_{n \to \infty} \compcrossnorm{v-v_n}{k} = 0$を満たす
		$\{v_n\}_{n=1}^\infty \subset J_k V^{\otimes_a k}$を取れば
		\begin{align}
			\compcrossnorm{\alpha v - \alpha \otimes_{0,k} v}{k}
			\leq \compcrossnorm{\alpha v - \alpha v_n}{k}
				+ \compcrossnorm{\alpha \otimes_{0,k} v_n - \alpha \otimes_{0,k} v}{k}
			= 2|\alpha| \compcrossnorm{v-v_n}{k}
			\longrightarrow 0
			\quad (n \longrightarrow \infty)
		\end{align}
		が成立し$\alpha v = \alpha \otimes_{0,k} v$が得られる.
		同様にして$\alpha v = v \otimes_{k,k} \alpha$も成り立つ.
		従って,任意の$a \in T(V)$に対し
		\begin{align}
			(e \otimes a)_k
			&= \sum_{j=0}^k e_j \otimes_{j,k} a_{k-j}
			= 1 \otimes_{0,k} a_k \\
			&= a_k
			= a_k \otimes_{k,k} 1
			= \sum_{j=0}^k a_j \otimes_{j,k} e_{k-j}
			= (a \otimes e)_k,
			\quad (k=0,1,2,\cdots)
		\end{align}
		が満たされ$e \otimes a = a \otimes e = a$が出る.
		\QED
	\end{prf}
	
	$n \geq 0$に対して
	\begin{align}
		T^{(n)}(V) \coloneqq \bigoplus_{k=0}^{n} V^{\otimes k}
	\end{align}
	とおき,$T(V)$の場合と同様に乗法$\otimes$を
	\begin{align}
		a \otimes b,\quad
		\Biggl((a \otimes b)_k \coloneqq \sum_{j=0}^{k} a_j \otimes_{j,k} b_{k-j},\ k=1,\cdots,n \Biggr),
		\quad a,b \in T^{(n)}(V)
	\end{align}
	により定め,次の直積ノルムでノルム位相を導入する:
	\begin{align}
		\compcrossnorm{a}{} \coloneqq \sum_{k=0}^n \compcrossnorm{a_k}{k},
		\quad (a = (a_k)_{k=0}^n \in T^{(n)}(V)).
		\label{eq:def_norm_on_truncated_tensor_algebra}
	\end{align}
	また,写像$X:\Delta_T \longrightarrow T^{(n)}(V)$に対して
	$X_{s,t} = (X^0_{s,t},\cdots,X^n_{s,t}),\ ((s,t) \in \Delta_T)$と表記し
	\begin{align}
		C_0 \left(\Delta_T,T^{(n)}(V) \right)
		\coloneqq \Set{X:\Delta_T \longrightarrow T^{(n)}(V)}{\mbox{continuous},\ X^0 \equiv 1}
	\end{align}
	を定める.
	
	\begin{screen}
		\begin{dfn}[有限$p$-変動]
			$p \geq 1$とする.$X:\Delta_T \longrightarrow T^{(n)}(V)$に対して
			或るコントロール関数$\omega$が存在して
			\begin{align}
				\compcrossnorm{X^i_{s,t}}{i} \leq \omega(s,t)^{i/p},
				\quad (\forall i=1,\cdots,n,\ \forall (s,t) \in \Delta_T)
				\label{eq:def_finite_p_variation}
			\end{align}
			を満たすとき,$X$は有限$p$-変動(finite $p$-variation)であるという.\footnotemark
		\end{dfn}
	\end{screen}
	\footnotetext{
		$X:\Delta_T \longrightarrow T^{(n)}(V)$が有限$p$-変動であることと
		$\Norm{X}{p}$が有限であることは一致しない.
		実際,後述のシグネチャー$X=(X^0,\cdots,X^n)$
		は有限$p$-変動であるが,その定義より$X^0 \equiv 1$が満たされているから
		\begin{align}
			\mbox{$D$の分割小区間の数}
			= \sum_D \compcrossnorm{X^0_{t_{i-1},t_i}}{0}^p
			\leq \sum_D \compcrossnorm{X_{t_{i-1},t_i}}{}^p
			\leq \Norm{X}{p}^p,
			\quad (\forall D \in \delta[0,T])
		\end{align}
		が成り立ち$\Norm{X}{p} = \infty$となる.
	}
	
	\begin{screen}
		\begin{dfn}[有限総$p$-変動]
			$p \geq 1$とする.$X \in C_0 \left(\Delta_T,T^{(n)}(V) \right)$
			が有限総$p$-変動(finite total $p$-variation)とは
			\begin{align}
				\Norm{X^i}{p/i} < \infty,
				\quad \forall i=1,\cdots,n
			\end{align}
			が満たされることをいう.また次の線型空間を定める:
			\begin{align}
				C_{0,p} \left(\Delta_T,T^{(n)}(V) \right)
				\coloneqq \Set{X \in C_0 \left(\Delta_T,T^{(n)}(V) \right)}{\mbox{$X$ has finite total $p$-variation}}	.
			\end{align}
		\end{dfn}
	\end{screen}
	
	\begin{screen}
		\begin{dfn}[乗法的汎関数]
			次の関係式(Chen's identity)を満たす$X \in C_0 \left(\Delta_T,T^{(n)}(V) \right)$
			を$n$次の乗法的汎関数(multiplicative functional of degree $n$)と呼ぶ:
			\begin{align}
				X_{s,u} \otimes X_{u,t} = X_{s,t},
				\quad (\forall 0 \leq s \leq u \leq t \leq T).
			\end{align}
		\end{dfn}
	\end{screen}
	
	\begin{screen}
		\begin{lem}\label{lem:multiplicative_functional_vanishes_on_diagonal}
			$X:\Delta_T \longrightarrow T^{(n)}(V)$が$X^0 \equiv 1$かつ
			Chen's identity を満たせば$X^k_{t,t} = 0,
			\ (0 \leq \forall t \leq T,\ 1 \leq \forall k \leq n)$.
		\end{lem}
	\end{screen}
	
	\begin{prf}
		任意に$t \in [0,T]$を取る.
		$X^k_{t,t} = \sum_{j=0}^{k} X^j_{t,t} \otimes_{j,k} X^{k-j}_{t,t}$
		と式(\refeq{eq:thm_identity_element_of_T_V})より,先ず
		\begin{align}
			X^1_{t,t} = X^0_{t,t} \otimes_{0,1} X^1_{t,t} + X^1_{t,t} \otimes_{1,1} X^0_{t,t}
			= X^1_{t,t} + X^1_{t,t}
		\end{align}
		が成り立ち$X^1_{t,t} = 0$が従う.同様に
		\begin{align}
			X^2_{t,t} = X^0_{t,t} \otimes_{0,2} X^2_{t,t} + X^1_{t,t} \otimes_{1,2} X^1_{t,t}
				+ X^2_{t,t} \otimes_{2,2} X^0_{t,t}
			= X^2_{t,t} + X^2_{t,t}
		\end{align}
		より$X^2_{t,t} = 0$が成立し,帰納的に$X^k_{t,t} = 0\ (1 \leq k \leq n)$が出る.
		\QED
	\end{prf}
	
	\begin{screen}
		\begin{thm}\label{thm:fin_p_var_and_fin_ttl_p_var_is_equiv_for_multiplicative}
			$n$次乗法的汎関数については
			有限$p$-変動であることと有限総$p$-変動であることは同値である($p \geq 1$).
		\end{thm}
	\end{screen}
	
	\begin{prf}
		$X \in C_0 \left(\Delta_T,T^{(n)}(V) \right)$を$n$次乗法的とする.
		$X$が有限総$p$-変動ならば,補題\ref{lem:multiplicative_functional_vanishes_on_diagonal}と
		定理\ref{thm:control_function_defined_by_p_variation}により
		\begin{align}
			\omega(s,t) \coloneqq \sum_{i=1}^n \Norm{X^i}{p/i,[s,t]}^{p/i},
			\quad ((s,t) \in \Delta_T)
		\end{align}
		で定める$\omega$はコントロール関数となる.このとき
		\begin{align}
			\compcrossnorm{X^i_{s,t}}{i}
			\leq \Norm{X^i}{p/i,[s,t]}
			\leq \omega(s,t)^{i/p},
			\quad (\forall i = 1,\cdots,n,\ \forall (s,t) \in \Delta_T)
		\end{align}
		が成り立つから$X$は有限$p$-変動である.逆に$X$が有限$p$-変動なら,
		(\refeq{eq:def_finite_p_variation})を満たす$\omega$に対し
		\begin{align}
			\sum_D \compcrossnorm{X^i_{t_{i-1},t_i}}{i}^{p/i}
			\leq \sum_D \omega(t_{i-1},t_i)
			\leq \omega(0,T),
			\quad (\forall D \in \delta[0,T],\ \forall i=1,\cdots,n)
		\end{align}
		が満たされ$\Norm{X^i}{p/i} < \infty,\ i=1,\cdots,n$が従うので$X$は有限総$p$-変動である.
		\QED
	\end{prf}
	
	実際に乗法的汎関数を構成する.有界変動な連続写像$x:[0,T] \longrightarrow V$に対して
	\begin{align}
		X^1_{s,t} \coloneqq x_t - x_s,
		\quad (\forall (s,t) \in \Delta_T)
	\end{align}
	とおけば,$X^1:\Delta_T \longrightarrow V$は連続かつ$\Norm{X^1}{1} < \infty$を満たす.
	このとき,
	\begin{screen}
		\begin{lem}\label{lem:def_integration_of_continuous_mapping_by_X_1}
			任意の$(s,t) \in \Delta_T$と
			連続写像$Y:\Delta_T \longrightarrow V^{\otimes k},\ (k \geq 1)$
			に対して次の積分が$V^{\otimes k+1}$で確定する:
			\begin{align}
				\int_s^t Y_{s,u} \otimes d x_u
				\coloneqq \lim_{|D| \to 0} \sum_{D} Y_{s,u_{i-1}} \otimes_{k,k+1} X^1_{u_{i-1},u_i}
				\label{eq:def_integration_of_continuous_mapping_by_X_1}
			\end{align}
		\end{lem}
	\end{screen}
	
	\begin{prf}
		$D=\{s=u_0 < \cdots <u_n= t\},\ D'=\{s=v_0 < \cdots <v_m= t\} \in \delta[s,t]$
		を任意に取り,共通細分を$D''=\{s=w_0 < \cdots < w_r = t\}$と表して
		\begin{align}
			\begin{cases}
				\tilde{Y}_{s,w_\ell} \coloneqq Y_{s,u_i}, & (u_i \leq w_\ell < u_{i+1}), \\
				\hat{Y}_{s,w_\ell} \coloneqq Y_{s,v_j}, & (v_j \leq w_\ell < v_{j+1}),
			\end{cases}
			\quad (\ell=0,1,\cdots,r)
		\end{align}
		で$\tilde{Y},\hat{Y}$を定めれば,
		定理\ref{thm:property_of_the_completion_of_the_projective_norm}より
		\begin{align}
			&\compcrossnorm{ \sum_D Y_{s,u_{i-1}} \otimes_{k,k+1} X^1_{u_{i-1},u_i} - 
				\sum_{D'} Y_{s,v_{j-1}} \otimes_{k,k+1} X^1_{v_{j-1},v_j}}{k+1} \\
			&\quad\leq \compcrossnorm{ \sum_D Y_{s,u_{i-1}} \otimes_{k,k+1} X^1_{u_{i-1},u_i} - 
				\sum_{D''} Y_{s,w_{\ell-1}} \otimes_{k,k+1} X^1_{w_{\ell-1},w_\ell}}{k+1}
				+ \compcrossnorm{ \sum_{D'} Y_{s,v_{j-1}} \otimes_{k,k+1} X^1_{v_{j-1},v_j} - 
				\sum_{D''} Y_{s,w_{\ell-1}} \otimes_{k,k+1} X^1_{w_{\ell-1},w_\ell}}{k+1} \\
			&\quad= \compcrossnorm{ \sum_{D''} \tilde{Y}_{s,w_{\ell-1}} \otimes_{k,k+1} X^1_{w_{\ell-1},w_\ell} - 
				\sum_{D''} Y_{s,w_{\ell-1}} \otimes_{k,k+1} X^1_{w_{\ell-1},w_\ell}}{k+1} 
				+ \compcrossnorm{ \sum_{D''} \hat{Y}_{s,w_{\ell-1}} \otimes_{k,k+1} X^1_{w_{\ell-1},w_\ell} - 
				\sum_{D''} Y_{s,w_{\ell-1}} \otimes_{k,k+1} X^1_{w_{\ell-1},w_\ell}}{k+1} \\
			&\quad= \compcrossnorm{ \sum_{D''} \left(\tilde{Y}_{s,w_{\ell-1}} - Y_{s,w_{\ell-1}} \right)
		 		\otimes_{k,k+1} X^1_{w_{\ell-1},w_\ell}}{k+1}
				+ \compcrossnorm{ \sum_{D''} \left( \hat{Y}_{s,w_{\ell-1}} - Y_{s,w_{\ell-1}} \right)
				\otimes_{k,k+1} X^1_{w_{\ell-1},w_\ell}}{k+1} \\
			&\quad\leq \sum_{D''} \compcrossnorm{\tilde{Y}_{s,w_{\ell-1}} - Y_{s,w_{\ell-1}}}{k} \compcrossnorm{X^1_{w_{\ell-1},w_\ell}}{1}
				+ \sum_{D''} \compcrossnorm{\hat{Y}_{s,w_{\ell-1}} - Y_{s,w_{\ell-1}}}{k} \compcrossnorm{X^1_{w_{\ell-1},w_\ell}}{1} \\
			&\quad\leq \max{k}{\compcrossnorm{\tilde{Y}_{s,w_{\ell-1}} - Y_{s,w_{\ell-1}}}{k}} 
				\Norm{X^1}{1,[s,t]} + \max{k}{\compcrossnorm{\hat{Y}_{s,w_{\ell-1}} - Y_{s,w_{\ell-1}}}{k}} \Norm{X^1}{1,[s,t]}
		\end{align}
		が成立する.いま,$[s,t] \ni u \longmapsto Y_{s,u}$は一様連続であるから,
		$|D|,|D'|\longrightarrow 0$として右辺は0に収束する.
		従って$|D_n| \longrightarrow 0\ (n \longrightarrow \infty)$を満たす細分列$D_n \in \delta[s,t]$を
		取れば,$\left(\sum_{D_n} Y_{s,u_{i-1}} \otimes_{k,k+1} X^1_{u_{i-1},u_i} \right)_{n=1}^{\infty}$
		は$V^{\otimes k+1}$のCauchy列となり$V^{\otimes k+1}$で収束する.別の細分列
		$\tilde{D}_m \in \delta[s,t],\ (|\tilde{D}_m| \longrightarrow 0)$を取っても
		\begin{align}
			\compcrossnorm{ \sum_{D_n} Y_{s,u_{i-1}} \otimes_{k,k+1} X^1_{u_{i-1},u_i} - 
				\sum_{\tilde{D}_m} Y_{s,v_{j-1}} \otimes_{k,k+1} X^1_{v_{j-1},v_j}}{k+1}
			\longrightarrow 0,
			\quad (n,m \longrightarrow \infty)
		\end{align}
		が成り立つから,極限は細分列に依らず確定する.従って補題の主張が得られる.
		\QED
	\end{prf}
	
	\begin{screen}
		\begin{lem}\label{lem:signature_of_path}
			(\refeq{eq:def_integration_of_continuous_mapping_by_X_1})の積分を
			\begin{align}
				Z_{s,t} \coloneqq \int_s^t Y_{s,u} \otimes d x_u,
				\quad (\forall (s,t) \in \Delta_T)
				\label{eq:signature_of_path_1}
			\end{align}
			とおけば,$Z:\Delta_T \longrightarrow V^{\otimes k+1}$は連続かつ有界変動である.
		\end{lem}
	\end{screen}
	
	\begin{prf}\mbox{}
		\begin{description}
			\item[第一段]
				$Z$が有界変動であることを示す.いま,任意に$(s,t) \in \Delta_T\ (s < t)$
				\footnote{
					$s=t$なら,$X^1_{s,t} = 0$より$Z_{s,t} = 0$が成り立つ.
				}を取る.
				\begin{align}
					M \coloneqq \sup{(x,y) \in \Delta_T}{\compcrossnorm{Y_{x,y}}{k}}
				\end{align}
				とおけば$Y$の連続性より$M < \infty$となり,
				任意の$\epsilon > 0$に対し或る$D \in \delta[s,t]$が存在して
				\begin{align}
					\compcrossnorm{Z_{s,t}}{k+1}
					\leq \epsilon + \compcrossnorm{\sum_{D} Y_{s,u_{i-1}} \otimes_{k,k+1} X^1_{u_{i-1},u_i}}{k+1}
					\leq \epsilon + \sum_D \compcrossnorm{Y_{s,u_{i-1}}}{k}
					\compcrossnorm{X^1_{t_{i-1},t_i}}{1} 
					\leq \epsilon + M \Norm{X^1}{1,[s,t]}
				\end{align}
				が成立する.$\epsilon > 0$と$(s,t)$の任意性より
				\begin{align}
					\compcrossnorm{Z_{s,t}}{k+1} \leq M \Norm{X^1}{1,[s,t]},
					\quad (\forall (s,t) \in \Delta_T)
				\end{align}
				が従い$\Norm{Z}{1} \leq M \Norm{X^1}{1}$\ (1-変動ノルム)を得る.
				
			\item[第二段]
				点$(s,s)\ (\forall s \in [0,T])$において$Z$が連続であること示す.
				実際,定理\ref{thm:control_function_defined_by_p_variation}より
				\begin{align}
					\Delta_T \ni (s,t) \longmapsto \Norm{X^1}{1,[s,t]}
					\label{map:lem_signature_of_path_1}
				\end{align}
				はコントロール関数であるから,
				\begin{align}
					&\compcrossnorm{Z_{t,u} - Z_{s,s}}{k+1}
					= \compcrossnorm{Z_{t,u}}{k+1}
					\leq M \Norm{X^1}{1,[t,u]}
					\longrightarrow 0 \quad ((t,u) \longrightarrow (s,s))
				\end{align}
				が成立し$Z$の$(s,s)$における連続性を得る.
			
			\item[第三段]
				$s < t$を満たす点$(s,t) \in \Delta_T$において$Z$が連続であること示す.
				いま,任意に$\epsilon > 0$を取れば,
				\begin{description}
					\item[(i)] (\refeq{map:lem_signature_of_path_1})がコントロール関数であるから,
						或る$\eta_1 > 0$が存在して$|s-a|,|t-b| < \eta_1$ならば
						\begin{align}
							\Norm{X^1}{1,[s \wedge a,s \vee a]} < \epsilon, 
							\quad \Norm{X^1}{1,[t \wedge b,t \vee b]} < \epsilon
						\end{align}
						が満たされる.
						
					\item[(ii)] 或る$\eta_2 > 0$が存在して$|s-a|,|t-b| < \eta_2$ならば
						$[s,t] \cap [a,b] \neq \emptyset$が満たされる.
						
					\item[(iii)] $Y$は$\Delta_T$上一様連続であるから,
						或る$\eta_3 > 0$が存在して$|s-a| < \eta_3$なら
						\begin{align}
							\sup{}{\Set{\compcrossnorm{Y_{s,u} - Y_{a,u}}{k}}{(s \vee a) \leq u \leq T}} < \epsilon
						\end{align}
						が満たされる.
						
					\item[(iv)] 補題\ref{lem:def_integration_of_continuous_mapping_by_X_1}
						より或る$\eta_4 > 0$が存在して$|D_1|,|D_2| < \eta_4,\ 
						(D_1 \in \delta[s,t],\ D_2 \in \delta[a,b])$なら
						\begin{align}
							\compcrossnorm{Z_{s,t} - \sum_{D_1} Y_{s,u_{i-1}} \otimes_{k,k+1} X^1_{u_{i-1},u_i}}{k+1} < \epsilon,
					\quad \compcrossnorm{Z_{a,b} - \sum_{D_2} Y_{a,v_{j-1}} \otimes_{k,k+1} X^1_{v_{j-1},v_j}}{k+1} < \epsilon
						\end{align}
						が満たされる.
				\end{description}
				ここで$\eta \coloneqq \eta_1 \wedge \eta_2 \wedge \eta_3$として,
				$|s-a|,|t-b| < \eta,\ |D_1|,|D_2| < \eta_4$を満たす$(a,b),D_1,D_2$を取り
				\begin{align}
					\Omega_3 &\coloneqq (D_1 \cup D_2) \cap [s,t] \cap [a,b], \\
					\Omega_1 &\coloneqq D_1 \cup \Omega_3,
						\quad \Omega_1^< \coloneqq \Omega_1 \cap [0,a],
						\quad \Omega_1^> \coloneqq \Omega_1 \cap [b,T], \\
					\Omega_2 &\coloneqq D_2 \cup \Omega_3,
						\quad \Omega_2^< \coloneqq \Omega_2 \cap [0,s],
						\quad \Omega_2^> \coloneqq \Omega_2 \cap [t,T]
				\end{align}
				とおけば,$\Omega_1 = \Omega_1^< \cup \Omega_3 \cup \Omega_1^>,\ 
				\Omega_2 = \Omega_2^< \cup \Omega_3 \cup \Omega_2^>$と分割できる.
				このとき(i)(ii)(iii)が満たされるから
				\begin{align}
					&\compcrossnorm{\sum_{\Omega_1} Y_{s,u_{i-1}} \otimes_{k,k+1} X^1_{u_{i-1},u_i}
						- \sum_{\Omega_2} Y_{a,v_{j-1}} \otimes_{k,k+1} X^1_{v_{j-1},v_j}}{k+1} \\
					&\quad\leq \compcrossnorm{\sum_{\Omega_3} Y_{s,u_{i-1}} \otimes_{k,k+1} X^1_{u_{i-1},u_i}
						- \sum_{\Omega_3} Y_{a,u_{i-1}} \otimes_{k,k+1} X^1_{u_{i-1},u_i}}{k+1} \\
					&\quad\qquad + \compcrossnorm{\sum_{\Omega_1^<} Y_{s,u_{i-1}} \otimes_{k,k+1} X^1_{u_{i-1},u_i}}{k+1} 
					+ \compcrossnorm{\sum_{\Omega_1^>} Y_{s,u_{i-1}} \otimes_{k,k+1} X^1_{u_{i-1},u_i}}{k+1} \\
					&\quad\qquad + \compcrossnorm{\sum_{\Omega_2^<} Y_{a,u_{i-1}} \otimes_{k,k+1} X^1_{u_{i-1},u_i}}{k+1} 
					+ \compcrossnorm{\sum_{\Omega_2^>} Y_{a,v_{j-1}} \otimes_{k,k+1} X^1_{v_{j-1},v_j}}{k+1} \\
					&\quad\leq \sum_{\Omega_3} \compcrossnorm{Y_{s,u_{i-1}} - Y_{a,u_{i-1}}}{k}
					\compcrossnorm{X^1_{u_{i-1},u_i}}{1}
					+ \sum_{\Omega_1^<} \compcrossnorm{Y_{s,u_{i-1}}}{k}
						\compcrossnorm{X^1_{u_{i-1},u_i}}{1}
					+ \sum_{\Omega_1^>} \compcrossnorm{Y_{s,u_{i-1}}}{k}
						\compcrossnorm{X^1_{u_{i-1},u_i}}{1} \\
					&\quad\qquad + \sum_{\Omega_2^<} \compcrossnorm{Y_{a,v_{j-1}}}{k}
						\compcrossnorm{X^1_{v_{j-1},v_i}}{1} 
					+ \sum_{\Omega_2^>} \compcrossnorm{Y_{a,v_{j-1}}}{k}
						\compcrossnorm{X^1_{v_{j-1},v_i}}{1} \\
					&\quad\leq \sup{u \in [s \vee a, T]}{\compcrossnorm{Y_{s,u} - Y_{a,u}}{k}}
						\Norm{X^1}{1}
						+ M \left( \Norm{X^1}{1,[s,s \vee a]}
						+ \Norm{X^1}{1,[a,s \vee a]} 
						+ \Norm{X^1}{1,[t,t \vee b]} 
						+ \Norm{X^1}{1,[b,t \vee b]} \right) \\
					&\quad< \left( \Norm{X^1}{1} + 4 M \right) \epsilon 
				\end{align}
				が成立し,(iv)と併せれば
				\begin{align}
					\compcrossnorm{Z_{s,t} - Z_{a,b}}{k+1} 
					< \left( \Norm{X^1}{1} + 4 M + 2\right)\epsilon,
					\quad (|s-a|,|t-b| < \eta)
				\end{align}
				が従い$Z$の$(s,t)$における連続性が得られる.
				\QED
		\end{description}	
	\end{prf}
	
	\begin{screen}
		\begin{dfn}[パスのシグネチャー]
			有界変動な連続写像$x:[0,T] \longrightarrow V$に対して
			$X^1_{s,t} \coloneqq x_t-x_s,\ (\forall (s,t) \in \Delta_T)$とおけば,
			補題\ref{lem:def_integration_of_continuous_mapping_by_X_1}と
			補題\ref{lem:signature_of_path}により逐次的に
			次を構成することができる:
			\begin{align}
				V^{\otimes i+1} \ni X^{i+1}_{s,t} \coloneqq \int_s^t X^i_{s,u} \otimes dx_u,
				\quad (\forall (s,t) \in \Delta_T,\ i=1,2,\cdots)
			\end{align}
			ここで
			$S(x)_{[s,t]} \coloneqq (1,X^1_{s,t},X^2_{s,t},\cdots)$
			とおき,特に$S(x)_{[0,T]}$をパス$x$のシグネチャー(signature of path $x$)と呼ぶ.
		\end{dfn}
	\end{screen}
	
	\begin{screen}
		\begin{thm}[逐次積分により定まる乗法的汎関数]
			任意の$n \geq 1$に対し,$S(x)_{[s,t]}$の最初の
			$n+1$個\footnotemark
			の元の族を$S_n(x)_{[s,t]} = (X^0_{s,t},X^1_{s,t},\cdots,X^n_{s,t}),\ 
			(\forall (s,t) \in \Delta_T)$
			と書けば,$S_n(x)$は$n$次乗法的である.
		\end{thm}
	\end{screen}
	
	\begin{prf}
		補題\refeq{lem:signature_of_path}より
		$S_n(x) \in C_0 \left(\Delta_T,T^{(n)}(V) \right)$が従うから,以下では
		任意の$k \geq 0$と$0 \leq s \leq u \leq t \leq T$に対して
		\begin{align}
			X^k_{s,t} = \sum_{j=0}^k X^j_{s,u} \otimes_{j,k} X^{k-j}_{u,t}.
		\end{align}
		が成り立つことを示す.
	\end{prf}
	
	\begin{screen}
		\begin{dfn}[$p$-ラフパス]
			$p \geq 1$とし,$p$を超えない最大の整数を$[p]$で表す.
			有限$p$-変動を持つ$[p]$次乗法的汎関数を
			$p$-ラフパス($p$-rough path)と呼び,その全体を$\Omega_p(V)$と書く:
			\begin{align}
				\Omega_p(V) 
				= \Set{X \in C_0\left( \Delta_T,T^{([p])}(V) \right)}{
					\mbox{$[p]$次乗法的,有限$p$-変動.}}.
			\end{align}
		\end{dfn}
	\end{screen}
	
	\begin{screen}
		\begin{thm}\label{thm:p_rough_path_complete_dist}
			$\Omega_p(V)$は次で定める距離により完備距離空間となる:
			\begin{align}
				d_p(X,Y) \coloneqq \max{1 \leq i \leq [p]}{\Norm{X^i - Y^i}{p/i}}.
			\end{align}
		\end{thm}
	\end{screen}
	
	$X \in \Omega_p(V)$は$X^0 \equiv 1$を満たすから,
	$\max{1 \leq i \leq [p]}{\Norm{\cdot}{p/i}}$は$\Omega_p(V)$においてノルムとはならない.
	
	\begin{prf}完備性を示す.
		\begin{description}
			\item[第一段] 極限を構成する.いま,
			任意の$X = (X^0,X^1,\cdots,X^{[p]}) \in \Omega_p(V)$に対して
			\begin{align}
				X^i \in B_{p/i,T}\left( V^{\otimes i} \right),\footnotemark
				\quad (\forall i=1,\cdots,[p]) 
			\end{align}
			\footnotetext{
				$B_{p/i,T}\left( V^{\otimes i} \right)$の定義は
				(p. \pageref{def:Banach_space_of_continuous_mapping_on_V})
				の(\refeq{def:Banach_space_of_continuous_mapping_on_V}).
			}
			が満たされるから,定理\ref{thm:B_p_T_Banach_space_2}より$\Omega_p(V)$の任意のCauchy列
			$\left( X^{(k)} = (X^{(k),0},\cdots,X^{(k),[p]}) \right)_{k=1}^{\infty}$
			に対して
			\begin{align}
				\Norm{X^{(k),i} - X^i}{p/i} \longrightarrow 0
				\quad (k \longrightarrow \infty,\ \forall i=1,\cdots,[p])
				\label{eq:thm_p_rough_path_complete_dist_2}
			\end{align}
			を満たす$X^i \in B_{p/i,T}\left( V^{\otimes i} \right)$が存在する.
			ここで$X:\Delta_T \longrightarrow T^{([p])}(V)$を
			\begin{align}
				X_{s,t} \coloneqq (1,X^1_{s,t},\cdots,X^n_{s,t}),
				\quad (\forall (s,t) \in \Delta_T)
			\end{align}
			により定めれば,(\refeq{eq:thm_p_rough_path_complete_dist_2})及び
			$X^i,\ i=1,\cdots,n$の連続性より
			$X \in C_{0,p} \left(\Delta_T,T^{([p])}(V) \right)$となる.
		
		\item[第二段] $X$が Chen's identity を満たすことを示す.
			これが示されれば,前段の結果と
			定理\ref{thm:fin_p_var_and_fin_ttl_p_var_is_equiv_for_multiplicative}
			より$X$は有限$p$-変動となり$X \in \Omega_p(V)$が従う.各$1 \leq i \leq [p]$に対して
			次が成立すればよい:
			\begin{align}
				X^i_{s,t} = \sum_{j=0}^i X^j_{s,u} \otimes_{j,i} X^{i-j}_{u,t},
				\quad (\forall 0 \leq s \leq u \leq t \leq T).
				\label{eq:thm_p_rough_path_complete_dist_1}
			\end{align}
			実際,(\refeq{eq:thm_p_rough_path_complete_dist_2})より
			\begin{align}
				\compcrossnorm{X^{(k),i}_{s,t} - X^i_{s,t}}{i}
				\leq \Norm{X^{(k),i} - X^i}{p/i} \longrightarrow 0,
				\quad (k \longrightarrow \infty,\ \forall 0 \leq s \leq t \leq T)
			\end{align}
			が成り立ち,かつ定理\ref{thm:property_of_the_completion_of_the_projective_norm}より
			\begin{align}
				\compcrossnorm{X^j_{s,u} \otimes_{j,i} X^{i-j}_{u,t}}{i}
				= \compcrossnorm{X^j_{s,u}}{j} \compcrossnorm{X^{i-j}_{u,t}}{i-j},
				\quad (\forall 0 \leq j \leq i)
			\end{align}
			が満たされるから,任意の$0 \leq s \leq t \leq T$に対して
			\begin{align}
				\compcrossnorm{X^i_{s,t} - \sum_{j=0}^i X^j_{s,u} \otimes_{j,i} X^{i-j}_{u,t}}{i}
				&\leq \compcrossnorm{X^i_{s,t} - X^{(k),i}_{s,t}}{i}
					+ \compcrossnorm{\sum_{j=0}^i X^j_{s,u} \otimes_{j,i} X^{i-j}_{u,t} 
					- \sum_{j=0}^i X^{(k),j}_{s,u} \otimes_{j,i} X^{(k),i-j}_{u,t}}{i} \\
				&\leq \compcrossnorm{X^i_{s,t} - X^{(k),i}_{s,t}}{i}
					+ \sum_{j=0}^i \compcrossnorm{X^j_{s,u} - X^{(k),j}_{s,u}}{j}
						\compcrossnorm{X^{i-j}_{u,t}}{i-j} \\
					&\qquad + \sum_{j=0}^i \compcrossnorm{X^{(k),j}_{s,u}}{j}
						\compcrossnorm{X^{i-j}_{u,t} - X^{(k),i-j}_{u,t}}{i-j} \\
				&\longrightarrow 0,
				\quad (k \longrightarrow \infty)
			\end{align}
			が従い(\refeq{eq:thm_p_rough_path_complete_dist_1})を得る.
			(\refeq{eq:thm_p_rough_path_complete_dist_2})より
			$d_p(X^{(k)},X) \longrightarrow 0\ (k \longrightarrow \infty)$が成り立ち定理の主張が得られる.
			\QED
	\end{description}
\end{prf}

\appendix
\chapter{テンソル積・クロスノルム}
	以下,零元のみの線型空間は考えない.すなわち以下で扱う全ての線型空間には基底が存在する.
	$E,E_i,F$を体$\K$上の線形空間とするとき,
	$\Hom{E}{F}$で$E$から$F$への$\K$-線型写像の全体を表す.
	また$\Ln{E_1 \times \cdots \times E_n}{F}{n}$
	で$E_1 \times \cdots \times E_n$から$F$への$\K$-$n$重線型写像の全体を表す.
	$X$をノルム空間と考えるときは,
	特に指定しなければノルムを$\Norm{\cdot}{X}$と書いてノルム位相を導入する.
	$X$に何らかのノルム$\Norm{}{}$が定まっているとき,$(X,\Norm{}{})$
	の位相的双対空間を$(X,\Norm{}{})^*$と書く.
	ノルム空間の族$(X_i)_{i=1}^{n}$の
	直和$\bigoplus_{i=1}^n X_i$には直積ノルム$\Norm{\cdot}{X_1} + \cdots + \Norm{\cdot}{X_n}$
	により位相を導入する.
	
	\section{多重線型写像}
	
	\begin{screen}
		\begin{thm}[多重線型写像の一意拡張]\label{thm:expansion_of_multilinear_mapping}
			$(X_i)_{i=1}^{n}$をノルム空間,$Z$をBanach空間,
			$Y_i$を$X_i$の稠密な部分空間とする$(i=1,\cdots,n)$.
			このとき,有界$n$重線型写像
			$b:\bigoplus_{i=1}^n Y_i \longrightarrow Z$は
			$(X_i)_{i=1}^{n}$上の$Z$値$n$重線型写像$\tilde{b}$に一意に拡張され,
			$b$と$\tilde{b}$の作用素ノルムは一致する.
		\end{thm}
	\end{screen}
	
	\begin{prf}
		$\bigoplus_{i=1}^n Y_i$は$\bigoplus_{i=1}^n X_i$で稠密であるから,
		任意の$x = (x_1,\cdots,x_n) \in \bigoplus_{i=1}^n X_i$に対して
		\begin{align}
			\Norm{x - x^k}{\bigoplus_{i=1}^n X_i}
			= \sum_{i=1}^n \Norm{x_i - x^k_i}{X_i} \longrightarrow 0,
			\quad (k \longrightarrow \infty)
		\end{align}
		を満たす点列$x^k = (x^k_1,\cdots,x^k_n) \in \bigoplus_{i=1}^n Y_i\ (k=1,2,\cdots)$
		が存在する.
		\begin{align}
			M_i \coloneqq \sup{k \geq 1}{\Norm{x^k_i}{X_i}},
			\quad (i=1,\cdots,n)
		\end{align}
		とおけば,$M_i < \infty\ (i=1,\cdots,n)$より
		\begin{align}
			\Norm{b(x^k) - b(x^\ell)}{Z}
			&= \left\|\, b(x^k_1,x^k_2,\cdots,x^k_n) - b(x^\ell_1,x^k_2,\cdots,x^k_n) \right. \\
				&\quad + b(x^\ell_1,x^k_2,\cdots,x^k_n) - b(x^\ell_1,x^\ell_2,\cdots,x^k_n) \\
				&\quad \cdots \\
				&\quad \left. + b(x^\ell_1,\cdots,x^\ell_{n-1},x^k_n) - b(x^\ell_1,\cdots,x^\ell_n)\,  \right\|_Z \\
			&\leq \Norm{b}{\ContLn{\bigoplus_{i=1}^n Y_i}{Z}{n}} \Norm{x^k_1 - x^\ell_1}{X_1} M_2 \cdots M_n \\
			&\quad + \Norm{b}{\ContLn{\bigoplus_{i=1}^n Y_i}{Z}{n}} M_1 \Norm{x^k_2 - x^\ell_2}{X_2} M_3 \cdots M_n \\
			&\quad \cdots \\
			&\quad + \Norm{b}{\ContLn{\bigoplus_{i=1}^n Y_i}{Z}{n}} M_1 \cdots M_{n-1} \Norm{x^k_n - x^\ell_n}{X_n} \\
			&\longrightarrow 0,
			\quad (k,\ell \longrightarrow \infty)
		\end{align}
		が成り立ち,$Z$の完備性より$\lim_{k \to \infty}b(x^k)$が存在する.
		別の収束列
		$\bigoplus_{i=1}^n Y_i \ni y^m \longrightarrow x$を取れば
		\begin{align}
			\Norm{x^k_i - y^m_i}{X_i} \leq \Norm{x^k_i - x_i}{X_i} + \Norm{x_i - y^m_i}{X_i}
			\longrightarrow 0,
			\quad (k,m \longrightarrow \infty,\ i=1,\cdots,n)
		\end{align}
		より$\Norm{b(x^k) - b(y^m)}{Z} \longrightarrow 0\ (k,m \longrightarrow \infty)$が従い
		\begin{align}
			\lim_{k \to \infty} b(x^k) = \lim_{m \to \infty} b(y^m)
		\end{align}
		が得られ,これにより
		写像$\tilde{b}:x \longmapsto \lim_{k \to \infty} b(x^k)$が定まる.
		この$\tilde{b}$は$b$の拡張であり,有界かつ$n$重線型性を持つ.先ず$n$重線型性を示す.
		$x = (x_1,x_2,\cdots,x_n)$と$y = (y_1,x_2,\cdots,x_n)$に対し
		\begin{align}
			\Norm{x - x^k}{\bigoplus_{i=1}^n X_i} 
			\longrightarrow 0,
			\quad \Norm{y - y^k}{\bigoplus_{i=1}^n X_i} 
			\longrightarrow 0
		\end{align}
		を満たす点列$(x^k)_{k=1}^\infty,(y^k)_{m=1}^\infty \subset \bigoplus_{i=1}^n Y_i$を取れば
		\begin{align}
			&\Norm{\tilde{b}(\alpha x_1 + \beta y_1,x_2,\cdots,x_n) 
				- \alpha \tilde{b}(x_1,\cdots,x_n)
				- \beta \tilde{b}(y_1,\cdots,x_n)}{Z} \\
			&\leq \Norm{\tilde{b}(\alpha x_1 + \beta y_1,x_2,\cdots,x_n)
				- b(\alpha x^k_1 + \beta y^k_1,x^k_2,\cdots,x^k_n)}{Z} \\
				&\quad + |\alpha| \Norm{\tilde{b}(x_1,\cdots,x_n)
				- b(x^k_1,\cdots,x^k_n)}{Z} \\
				&\quad + |\beta| \Norm{\tilde{b}(y_1,\cdots,x_n)
				- b(y^k_1,\cdots,x^k_n)}{Z} \\
			&\longrightarrow 0,
			\quad (k \longrightarrow \infty)
		\end{align}
		が成り立ち,$\tilde{b}$の第一成分に関する線型性を得る.他の成分も同じである.
		また任意の$x \in \bigoplus_{i=1}^\infty X_i$に対して
		収束列$(x^k)_{k=1}^\infty \subset \bigoplus_{i=1}^n Y_i$を取れば,
		任意の$\epsilon > 0$に対し或る$k$が存在して
		\begin{align}
			\Norm{\tilde{b}(x)}{Z} \leq \Norm{b(x^k)}{Z} + \epsilon
		\end{align}
		かつ
		\begin{align}
			\Norm{x^k_i}{X_i} \leq \Norm{x_i}{X_i} + \epsilon,
			\quad (i=1,\cdots,n)
		\end{align}
		が満たされ
		\begin{align}
			\Norm{\tilde{b}(x)}{Z} \leq \Norm{b(x^k)}{Z} + \epsilon
			\leq \Norm{b}{\ContLn{\bigoplus_{i=1}^n Y_i}{Z}{n}} \prod_{i=1}^n \left( \Norm{x_i}{X_i} + \epsilon \right) + \epsilon
		\end{align}
		が従う.$x$及び$\epsilon$の任意性より$\Norm{\tilde{b}}{\ContLn{\bigoplus_{i=1}^n X_i}{Z}{n}}
		\leq \Norm{b}{\ContLn{\bigoplus_{i=1}^n Y_i}{Z}{n}}$が成り立ち,
		$\tilde{b}$は$b$の拡張だから
		\begin{align}
			\Norm{\tilde{b}}{\ContLn{\bigoplus_{i=1}^n X_i}{Z}{n}}
			= \Norm{b}{\ContLn{\bigoplus_{i=1}^n Y_i}{Z}{n}}
		\end{align}
		が出る.拡張の一意性は$\bigoplus_{i=1}^n Y_i$の稠密性と$\tilde{b}$の連続性による.
		\QED
	\end{prf}
	\section{ノルム空間の完備拡大}
	$\K$を$\R$或は$\C$と考える.
	\begin{screen}
		\begin{thm}[完備拡大の存在定理]
			$(X,\Norm{\cdot}{X})$を$\K$上のノルム空間とするとき,次を満たす
			Banach空間$(Y,\Norm{\cdot}{Y})$と線型等長写像$J:X \longrightarrow Y$が
			存在する:
			\begin{description}
				\item[(e1)] $JX$は$Y$において稠密である.
				\item[(e2)] 別のBanach空間$(Z,\Norm{\cdot}{Z})$と線型等長$K:X \longrightarrow Z$
					が存在して(e1)を満たすとき,$F \circ J = K$を満たす線型同型
					$F:Y \longrightarrow Z$が存在する.
					\begin{align}
						\xymatrix{
							Y \ar@{-}[r] & F \ar[r] & Z \\
							& X \ar@{^{(}->}[lu]^-J \ar@{}[u]|\circlearrowleft \ar@{^{(}->}[ru]_-K & 
						}
					\end{align}
				\item[(e3)] $X$が内積空間なら$Y$はHilbert空間である.
			\end{description}
		\end{thm}
	\end{screen}
	
	\begin{prf}\mbox{}
		\begin{description}
			\item[第一段]
				$X$のCauchy列の全体を
				$Cauchy(X)$で表す.任意の$(x_n),(y_n) \in Cauchy(X)$に対し
				\begin{align}
					\left| \Norm{x_n - y_n}{X} - \Norm{x_m - y_m}{X} \right|
					\leq \Norm{x_n - x_m}{X} + \Norm{y_n - y_m}{X}
				\end{align}
				が成り立つから,$\left( \Norm{x_n - y_n}{X} \right)_{n=1}^\infty$は
				$\R$のCauchy列をなして収束し,
				\begin{align}
					(x_n)\ R\ (y_n)
					\DEF \lim_{n \to \infty} \Norm{x_n - y_n}{X} = 0
				\end{align}
				により$Cauchy(X)$に同値関係$R$が定まる.
				$Cauchy(X)$は,線型演算を
				\begin{align}
					(x_n) + (y_n) \coloneqq (x_n + y_n),
					\quad \alpha (x_n) \coloneqq (\alpha x_n)
				\end{align}
				で定義すれば線型空間となるから,これを$R$で割った次の商
				\begin{align}
					Y \coloneqq Cauchy(X) / R
				\end{align}
				は線型空間である.
				$(x_n)$の$R$に関する同値類を$[(x_n)]$と表すとき,
				\begin{align}
					\Norm{[(x_n)]}{Y} \coloneqq \lim_{n \to \infty} \Norm{x_n}{X}
				\end{align}
				はwell-definedであり$Y$においてノルムとなる.
			
			\item[第二段]
				任意の$x \in X$に対し$x_n = x\ (\forall n \geq 1)$を満たす
				$(x_n)$を$\zeta_x$と書けば,
				\begin{align}
					J: X \ni x \longmapsto [\zeta_x] \in Y
				\end{align}
				により等長線型が定まる.実際,
				\begin{align}
					\Norm{J (\alpha u + \beta v) - \alpha J u - \beta J v}{Y}
					= \Norm{[\zeta_{\alpha u + \beta v}] - \alpha [\zeta_u] - \beta [\zeta_v]}{Y}
					= \Norm{[\zeta_{\alpha u + \beta v} - \alpha \zeta_u - \beta \zeta_v]}{Y}
					= 0
				\end{align}
				により線型性が得られ,
				\begin{align}
					\Norm{J x}{X} = \lim_{n \to \infty} \Norm{x_n}{X}
					= \Norm{x}{X},
					\quad (\forall x \in X)
				\end{align}
				により等長性を得る.
			
			\item[第三段]
				任意の$y \in Y$に対し,その代表を$(x_n)$とすれば,
				\begin{align}
					\Norm{y - [\zeta_{x_m}]}{Y}
					= \lim_{n \to \infty} \Norm{x_n - x_m}{X}
					\longrightarrow 0
					\quad (m \longrightarrow \infty)
				\end{align}
				が成り立つから$J X$は$Y$で稠密である.
				
			\item[第四段]
				$(Y,\Norm{\cdot}{Y})$の完備性を示す.$(y_n)$を$Y$の
				Cauchy列とすれば,
				\begin{align}
					\Norm{y_n - J x_n}{Y} < \frac{1}{n},
					\quad (\forall n = 1,2,\cdots)
				\end{align}
				を満たす$(x_n) \subset X$が存在する.
				\begin{align}
					\Norm{x_n - x_m}{X}
					= \Norm{J x_n - J x_m}{Y}
					< \frac{1}{n} + \Norm{y_n - y_m}{Y} + \frac{1}{m}
					\longrightarrow 0
					\quad (n,m \longrightarrow \infty)
				\end{align} 
				より$(x_n) \in Cauchy(X)$が従い,$y \coloneqq [(x_n)]$とおけば
				\begin{align}
					\Norm{y - y_m}{Y}
					\leq \Norm{y - J x_m}{Y} + \Norm{J x_m - y_m}{Y}
					< \lim_{n \to \infty} \Norm{x_n - x_m}{X} + \frac{1}{m}
					\longrightarrow 0
					\quad (m \longrightarrow \infty)
				\end{align}
				が成り立ち$Y$の完備性が出る.
		\end{description}
	\end{prf}
	\section{テンソル積}
	$n \geq 2$として,体$\K$上の線形空間の族$(E_i)_{i=1}^n$に対して
	テンソル積を定義する.
	\begin{align}
		\Lambda\biggl( \bigoplus_{i=1}^n E_i \biggr)
		= \Set{b:\bigoplus_{i=1}^n E_i \longrightarrow \K}{\mbox{有限個の$e \in \bigoplus_{i=1}^n E_i$を除いて$b(e)=0$.}}
	\end{align}
	により$\K$-線形空間$\Lambda\biggl( \bigoplus_{i=1}^n E_i \biggr)$を定める.また$e=(e_1,\cdots,e_n) \in \bigoplus_{i=1}^n E_i$に対する定義関数を
	\begin{align}
		\defunc_{e_1,\cdots,e_n} (x) = 
		\begin{cases}
			1, & x = e, \\
			0, & x \neq e
		\end{cases}
	\end{align}
	で表す.$\Lambda\biggl( \bigoplus_{i=1}^n E_i \biggr)$の線型部分空間を
	\begin{align}
		&\Lambda_0\biggl( \bigoplus_{i=1}^n E_i \biggr) \\
		&\coloneqq
		\Span{\Set{ \substack{\defunc_{e_1,\cdots,e_i + e_i',\cdots,e_n}
			-\defunc_{e_1,\cdots,e_i,\cdots,e_n}
			-\defunc_{e_1,\cdots,e_i',\cdots,e_n},\\
			\defunc_{e_1,\cdots,\lambda e_i,\cdots,e_n}
			-\lambda\defunc_{e_1,\cdots,e_i,\cdots,e_n}}}{e_i,e_i' \in E_i,
			\lambda \in \K,
			1 \leq i \leq n}}
	\end{align}
	により定め,
	$b \in \Lambda\biggl( \bigoplus_{i=1}^n E_i \biggr)$の$\Lambda_0\biggl( \bigoplus_{i=1}^n E_i \biggr)$に関する同値類を$[b]$と書く.そして
	\begin{align}
		E_1 \otimes \cdots \otimes E_n = \bigotimes_{i=1}^n E_i 
		\coloneqq \Lambda\biggl( \bigoplus_{i=1}^n E_i \biggr)
		\left/ \Lambda_0\biggl( \bigoplus_{i=1}^n E_i \biggr) \right.
	\end{align}
	で定める商空間を$(E_i)_{i=1}^n$のテンソル積と定義する.
	また$(e_1,\cdots,e_n) \in \bigoplus_{i=1}^n E_i$に対し
	\begin{align}
		e_1 \otimes \cdots \otimes e_n \coloneqq \left[ \defunc_{e_1,\cdots,e_n} \right]
	\end{align}
	により定める$\otimes:\bigoplus_{i=1}^n E_i \longrightarrow \bigotimes_{i=1}^n E_i$を
	テンソル積の標準写像と呼ぶ.
	
	\begin{screen}
		\begin{thm}[標準写像の多重線型性]\label{thm:tensor_product_is_bilinear}
			$(E_i)_{i=1}^n$を$\K$-線形空間の族とするとき,
			\begin{align}
				\otimes : \bigoplus_{i=1}^n E_i \ni (e_1,\cdots,e_n) \longmapsto e_1 \otimes \cdots \otimes e_n \in \bigotimes_{i=1}^n E_i
			\end{align}
			は$n$重線型写像である.また次が成り立つ:
			\begin{align}
				\bigotimes_{i=1}^n E_i = \Span{\Set{e_1 \otimes \cdots \otimes e_n}{(e_1,\cdots,e_n) \in \bigoplus_{i=1}^n E_i}}.
				\label{eq:thm_tensor_product_is_bilinear}
			\end{align}
		\end{thm}
	\end{screen}
	
	\begin{prf}
		任意の$1 \leq i \leq n,\ e_1 \in E_1,\cdots,e_n \in E_n,
		\ e_i,e_i' \in E_i,\ \lambda \in \K$に対して
		\begin{align}
			e_1 \otimes \cdots \otimes (e_i + e_i') \otimes \cdots \otimes e_n 
			&= \left[ \defunc_{e_1,\cdots,e_i + e_i',\cdots,e_n} \right] \\
			&= \left[ \defunc_{e_1,\cdots,e_i,\cdots,e_n} 
				+ \defunc_{e_1,\cdots,e_i',\cdots,e_n} \right] \\
			&= \left[ \defunc_{e_1,\cdots,e_i,\cdots,e_n} \right]
				+ \left[ \defunc_{e_1,\cdots,e_i',\cdots,e_n} \right] \\
			&=e_1 \otimes \cdots \otimes e_i \otimes \cdots \otimes e_n 
			+ e_1 \otimes \cdots \otimes e_i' \otimes \cdots \otimes e_n, \\
			e_1 \otimes \cdots \otimes (\lambda e_i) \otimes \cdots \otimes e_n 
			&= \left[ \defunc_{e_1,\cdots,\lambda e_i,\cdots,e_n} \right] \\
			&= \left[ \lambda \defunc_{e_1,\cdots,e_i,\cdots,e_n} \right] \\
			&= \lambda \left[ \defunc_{e_1,\cdots,e_i,\cdots,e_n} \right] \\
			&= \lambda (e_1 \otimes \cdots \otimes e_i \otimes \cdots \otimes e_n) 
		\end{align}
		が成立するから$\otimes$は$n$重線型である.また
		任意に$u = [b] \in E \otimes F$を取れば
		\begin{align}
			b = \sum_{j=1}^m k_j \defunc_{e^j_i,\cdots,e^j_n},
			\quad \left( k_j = b(e^j_i,\cdots,e^j_n),\ j=1,\cdots,m \right)
		\end{align}
		と表せるから,
		\begin{align}
			u = \left[ \sum_{j=1}^m k_j \defunc_{e^j_i,\cdots,e^j_n} \right]
			= \left[ \sum_{j=1}^m \defunc_{k_j e^j_i,\cdots,e^j_n} \right]
			= \sum_{j=1}^m (k_j e^j_1) \otimes \cdots \otimes e^j_n
		\end{align}
		が従い(\refeq{eq:thm_tensor_product_is_bilinear})を得る.
		\QED
	\end{prf}
	
	\begin{screen}
		\begin{thm}[$\cdots \otimes 0 \otimes \cdots$は零ベクトル]
		\label{thm:when_tensor_product_zero}
			$(E_i)_{i=1}^n$を$\K$-線形空間の族とし,
			テンソル積$\bigotimes_{i=1}^n E_i$を定める.
			このとき,或る$i$で$e_i = 0$なら
			$e_1 \otimes \cdots \otimes e_n = 0$が成り立つ.
		\end{thm}
	\end{screen}
	
	\begin{prf}
		$e_i = 0$のとき,$\lambda = 0$とすれば
		\begin{align}
			e_1 \otimes \cdots \otimes e_n
			= \left[ \defunc_{e_1,\cdots,0,\cdots,e_n} \right]
			= \left[ \defunc_{e_1,\cdots,\lambda e_i,\cdots,e_n} - \lambda \defunc_{e_1,\cdots,e_i,\cdots,e_n}\right]
			= 0
		\end{align}
		が成立する.
		\QED
	\end{prf}
	
	\begin{screen}
		\begin{thm}[普遍性(universality of tensor products)]
		\label{thm:universality_of_tensor_product}
			$(E_i)_{i=1}^n$を$\K$-線形空間の族とする.このとき
			任意の$\K$-線型空間$V$に対して,$T \in \Hom{\bigotimes_{i=1}^n E_i}{V}$ならば
			$T \circ \otimes \in \Ln{\bigoplus_{i=1}^n E_i}{V}{n}$
			が満たされ,これで定める次の対応$\Phi_V$は線型同型である:
			\begin{align}
				\begin{array}{ccc}
					\Phi_V:\Hom{\bigotimes_{i=1}^n E_i}{V} & \longrightarrow & \Ln{\bigoplus_{i=1}^n E_i}{V}{n} \\
					\rotatebox{90}{$\in$} & & \rotatebox{90}{$\in$} \\
					T & \longmapsto & T \circ \otimes 
					\label{eq:thm_universality_of_tensor_product}
				\end{array}
			\end{align}
			\begin{align}
				\xymatrix{
					&\bigoplus_{i=1}^n E_i \ar[d]_-\otimes \ar[rd]^-{\Phi(T)} & \\
					&\bigotimes_{i=1}^n E_i \ar[r]^-T & V \ar@{}[lu]<2ex>|\circlearrowright
				}
			\end{align}
			また$\K$-線型空間$U_0$と$n$重線型写像$\iota:\bigoplus_{i=1}^n E_i \longrightarrow U_0$が,
			任意の$\K$-線型空間$V$に対し
			\begin{description}
				\item[$(\otimes)_1$] $U_0$は$\iota$の像で生成される.
				\item[$(\otimes)_2$] 任意の$\delta \in \Ln{\bigoplus_{i=1}^n E_i}{V}{n}$に対して
					$\delta = \tau \circ \iota$を満たす$\tau \in \Hom{U_0}{V}$が存在する.
			\end{description}
			を満たすなら,(\refeq{eq:thm_universality_of_tensor_product})において
			$V = U_0$とするとき$T = \Phi_{U_0}^{-1}(\iota):
			\bigotimes_{i=1}^n E_i \longrightarrow U_0$は線型同型である.
		\end{thm}
	\end{screen}
	後半の主張により,$(E_i)_i$のテンソル積を別の方法で導入しても,
	商空間を用いて導入した$\bigotimes_i E_i$と線型同型に結ばれる.
	このとき,別の方法で導入したテンソル積及び標準写像を$\bigotimes\tilde{ }_i E_i,\ \tilde{\otimes}$と表せば,
	或る線型同型$T:\bigotimes_i E_i \longrightarrow \bigotimes\tilde{ }_i E_i$がただ一つ存在して
	\begin{align}
		T(e_1 \otimes \cdots \otimes e_n) = e_1 \tilde{\otimes} \cdots \tilde{\otimes} e_n 
	\end{align}
	を満たす.特に任意の並べ替え$\varphi:\{1,\cdots,n\} \longrightarrow \{1,\cdots,n\}$に対し
	\begin{align}
		\begin{array}{ccc}
		\bigotimes_{i=1}^{n} E_i & \cong & \bigotimes_{i=1}^{n} E_{\varphi(i)} \\
		\rotatebox{90}{$\in$} & & \rotatebox{90}{$\in$} \\
		e_1 \otimes \cdots \otimes e_n & \longleftrightarrow & e_{\varphi(1)} \otimes \cdots \otimes e_{\varphi(n)}
		\end{array}
	\end{align}
	が成立する.
	
	\begin{prf}\mbox{}
		\begin{description}
			\item[第一段]
				$T \in \Hom{\bigotimes_{i=1}^n E_i}{V}$の線型性と
				$\otimes$の$n$重線型性より
				$T \circ \otimes$は$n$重線型である.
				
			\item[第二段]
				$\Phi_V(T_1) = \Phi_V(T_2)$ならば
				$T_1$と$T_2$は$\Set{e_1 \otimes \cdots \otimes e_n}{(e_1,\cdots,e_n) \in \bigoplus_{i=1}^n E_i}$の上で一致する.
				(\refeq{eq:thm_tensor_product_is_bilinear})より
				$T_1 = T_2$が成立し$\Phi_V$の単射性が従う.
			
			\item[第三段]
				次の二段で$\Phi_V$の全射性を示す.まず,$\varphi \in \Hom{\Lambda(\bigoplus_{i=1}^n E_i)}{V}$に対し
				\begin{align}
					g: \bigoplus_{i=1}^n E_i \ni (e_1,\cdots,e_n) \longmapsto \varphi(\defunc_{e_1,\cdots,e_n}) \in V
				\end{align}
				を対応させる次の写像が全単射であることを示す:
				\begin{align}
					\begin{array}{ccc}
						F:\Hom{\Lambda(\bigoplus_{i=1}^n E_i)}{V} & \longrightarrow & \Map{\bigoplus_{i=1}^n E_i}{V} \\
						\rotatebox{90}{$\in$} & & \rotatebox{90}{$\in$} \\
						\varphi & \longmapsto & g
					\end{array}
				\end{align}
				$F(\varphi_1) = F(\varphi_2)$のとき,
				任意の$e \in \bigoplus_{i=1}^n E_i$に対して
				$\varphi_1(\defunc_{e_1,\cdots,e_n}) = \varphi_2(\defunc_{e_1,\cdots,e_n})$が成り立ち,
				\begin{align}
					\Lambda\biggl( \bigoplus_{i=1}^n E_i \biggr) 
					= \Span{\Set{\defunc_{e_1,\cdots,e_n}}{(e_1,\cdots,e_n) \in \bigoplus_{i=1}^n E_i}}
				\end{align}
				より$\varphi_1 = \varphi_2$が従い$F$の単射性が得られる.また
				$g \in \Map{\bigoplus_{i=1}^n E_i}{V}$に対して
				\begin{align}
					\varphi(0) &\coloneqq 0, \\
					\varphi(a) &\coloneqq \sum_{\substack{e \in \bigoplus_{i=1}^n E_i \\ a(e) \neq 0}} a(e) g(e),
					\quad \biggl(\forall a \in \Lambda\biggl(\bigoplus_{i=1}^n E_i\biggr),\ a \neq 0\biggr)
				\end{align}
				により$\varphi$を定めれば,$\varphi \in \Hom{\Lambda(\bigoplus_{i=1}^n E_i)}{V}$より
				\footnote{
					$\left( \Set{e}{a(e) \neq 0} \cup \Set{e}{a'(e) \neq 0} \right) 
					\cap \Set{e}{(a+a')(e) \neq 0} = \Set{e}{(a+a')(e) \neq 0}$より
					\begin{align}
						\varphi(a) + \varphi(a')
						&= \sum_{a(e) \neq 0} a(e) g(e) + \sum_{a'(e) \neq 0} a'(e) g(e)
						= \sum_{\substack{a(e) \neq 0 \\ (a + a')(e) = 0}} a(e) g(e)
						+ \sum_{\substack{a(e) \neq 0 \\ (a + a')(e) \neq 0}} a(e) g(e)
						+ \sum_{\substack{a'(e) \neq 0 \\ (a + a')(e) = 0}} a'(e) g(e)
						+ \sum_{\substack{a'(e) \neq 0 \\ (a + a')(e) \neq 0}} a'(e) g(e) \\
						&= \sum_{\substack{a(e) \neq 0 \\ (a + a')(e) \neq 0}} a(e) g(e)
						+ \sum_{\substack{a'(e) \neq 0 \\ (a + a')(e) \neq 0}} a'(e) g(e) 
						= \sum_{\substack{(a + a')(e) \neq 0}} (a + a')(e) g(e) 
						= \varphi(a + a')
					\end{align}
					$\varphi$の加法性を得る.スカラ倍は$\varphi(\beta a) = \sum_{(\beta a)(e) \neq 0} (\beta a)(e)g(e) = \beta \sum_{a(e) \neq 0} a(e)g(e) = \beta \varphi(a)\ (\beta \neq 0)$及び$\varphi(0) = 0$より従う.
				}
				$F$の全射性が出る.
				
			\item[第四段]
				任意に$b \in \Ln{\bigoplus_{i=1}^n E_i}{V}{n}$を取り
				$h \coloneqq F^{-1}(b)$とおけば,$h$の線型性より
				\begin{align}
					&b(e_1,\cdots,e_i+e_i',\cdots,e_n) - b(e_1,\cdots,e_i,\cdots,e_n) - b(e_1,\cdots,e_i',\cdots,e_n) \\
					&\qquad = h(\defunc_{e_1,\cdots,e_i+e_i',\cdots,e_n} - \defunc_{e_1,\cdots,e_i,\cdots,e_n} - \defunc_{e_1,\cdots,e_i',\cdots,e_n}), \\
					&b(e_1,\cdots,\lambda e_i,\cdots,e_n) - \lambda b(e_1,\cdots,e_i,\cdots,e_n) \\
					&\qquad = h(\defunc_{e_1,\cdots,\lambda e_i,\cdots,e_n} - \lambda \defunc_{e_1,\cdots,e_i,\cdots,e_n})
				\end{align}
				が成り立ち,$b$の$n$重線型性により$h$は$\Lambda_0(\bigoplus_{i=1}^n E_i)$上で0である.
				従って
				\begin{align}
					T([b]) \coloneqq h(b),
					\quad (b \in \Lambda(\bigoplus_{i=1}^n E_i))
				\end{align}
				で定める$T$はwell-definedであり,$T \in \Hom{\bigotimes_{i=1}^n E_i}{V}$かつ
				\begin{align}	
					b(e_1,\cdots,e_n) = h(\defunc_{e_1,\cdots,e_n}) = (T \circ \otimes) (e_1,\cdots,e_n),
					\quad (\forall (e_1,\cdots,e_n) \in \bigoplus_{i=1}^n E_i)
				\end{align}
				が満たされ$\Phi_V$の全射性が得られる.
				
			\item[第五段]
				$(\otimes)_1,(\otimes)_2$の下で
				$\Hom{U_0}{\bigotimes_{i=1}^n E_i} \ni \tau \longmapsto \tau \circ \iota \in \Ln{\bigoplus_{i=1}^n E_i}{\bigotimes_{i=1}^n E_i}{n}$は全単射であるから,
				$\tau \circ \iota = \otimes$を満たす$\tau \in \Hom{U_0}{\bigotimes_{i=1}^n E_i}$がただ一つ存在する.
				同様にして$\iota = T \circ \otimes$を満たす
				$T \in \Hom{\bigotimes_{i=1}^n E_i}{U_0}$がただ一つ存在し,併せれば
				\begin{align}
					\otimes = \tau \circ \iota = (\tau \circ T) \circ \otimes,
					\quad \iota = T \circ \otimes = (T \circ \tau) \circ \iota
				\end{align}
				が成り立つ.$T \longmapsto T \circ \otimes,\ \tau \longmapsto \tau \circ \iota$
				が一対一であるから,$\tau \circ T,\ T \circ \tau$はそれぞれ恒等写像に一致して
				$T^{-1} = \tau$が従う.すなわち$T$は$\bigotimes_{i=1}^n E_i$から$U_0$への
				線型同型である.
				\QED
		\end{description}
	\end{prf}
	
	\begin{screen}
		\begin{thm}[スカラーとのテンソル積]\label{thm:tensor_product_with_scalar}
			$E$を$\K$-線型空間とするとき,
			$\K \otimes E$と$E$は
			$f(\alpha \otimes e) = \alpha e$を満たす
			線型写像$f:\K \otimes E \longmapsto E$により
			同型となる.同様に$E \otimes \K$と$E$は
			$g(e \otimes \alpha) = \alpha e$を満たす
			線型写像$g$により
			同型となる.
		\end{thm}
	\end{screen}
	
	\begin{prf}
		スカラ倍$\iota:(\alpha, e) \longmapsto \alpha e$は双線型である.
		また定理\ref{thm:universality_of_tensor_product}の
		$(\otimes)_1,(\otimes)_2$について,
		\begin{align}
			E = \Span{\Set{\alpha e}{\alpha \in \K,\ e \in E}}
		\end{align}
		より$(\otimes)_1$が従い,かつ
		任意の双線型写像$\delta:\K \times E \longrightarrow V$に対し
		\begin{align}
			\tau(e) \coloneqq \delta(1,e),
			\quad (\forall e \in E)
		\end{align}
		で線型写像$\tau:E \longrightarrow V$を定めれば,
		\begin{align}
			\tau \circ \iota (\alpha,e) 
			= \tau(\alpha e) 
			= \delta(1,\alpha e)
			= \alpha \delta(1,e)
			= \delta (\alpha ,e)
		\end{align}
		となり$(\otimes)_2$が満たされる.従って,定理\ref{thm:universality_of_tensor_product}より
		$f \circ \otimes = \iota$を満たす線型同型$f:\K \otimes E \longrightarrow E$が存在して
		\begin{align}
			f(\alpha \otimes e) = \iota(\alpha,e) = \alpha e,
			\quad (\forall \alpha \in \K,\ e \in E)
		\end{align}
		が成立する.
		\QED
	\end{prf}
	
	\begin{screen}
		\begin{dfn}[線型写像のテンソル積]
				$(E_i)_{i=1}^n$と$(F_i)_{i=1}^n$を$\K$-線型空間の族とする.
				$f_i:E_i \longrightarrow F_i\ (i=1,\cdots,n)$が線型写像であるとき,
				\begin{align}
					b: \bigoplus_{i=1}^n E_i \ni (e_1,\cdots,e_n)
					\longmapsto f_1(e_1)\otimes \cdots \otimes f_n(e_n)
					\in \bigotimes_{i=1}^n F_i
				\end{align}
				により定める$b$は$n$重線型であり,定理\ref{thm:universality_of_tensor_product}
				より$b = g \circ \otimes$を満たす
				$g \in \Hom{\bigotimes_{i=1}^{n} E_i}{\bigotimes_{i=1}^{n} F_i}$がただ一つ存在する.
				$g$を$f_1 \otimes \cdots \otimes f_n$と表記して線型写像のテンソル積と定義する.特に,
				\begin{align}
					f_1 \otimes \cdots \otimes f_n(e_1 \otimes \cdots \otimes e_n)
					= f_1(e_1)\otimes \cdots \otimes f_n(e_n),
					\quad (\forall (e_1,\cdots,e_n) \in \bigoplus_{i=1}^n E_i)
				\end{align}
				が成り立つ.
				%$F_i = \K\ (i=1,\cdots,n)$の場合は$\otimes$を$\K$の乗法と考える$(\bigotimes_{i=1}^n F_i = \K)$.
		\end{dfn}
	\end{screen}
	
	\begin{screen}
		\begin{thm}[零写像のテンソル積は零写像]
		\label{thm:tensor_product_contains_zero_mapping_is_zero}
			$\K$-線型空間の族$(E_i)_{i=1}^n$と$(F_i)_{i=1}^n$と
			線型写像$f_i:E_i \longrightarrow F_i\ (i=1,\cdots,n)$について,
			或る$f_i$が零写像なら
			$f_1 \otimes \cdots \otimes f_n = 0$となる.
		\end{thm}
	\end{screen}
	
	\begin{prf}
		$f_i = 0$とすると,定理\ref{thm:when_tensor_product_zero}より
		$f_1 \otimes \cdots \otimes f_n$は
		$\Set{e_1 \otimes \cdots \otimes e_n}{e_i \in E_i}$上で0となる.
		この空間は$\bigotimes_{i=1}^{n} E_i$を生成するから
		$f_1 \otimes \cdots \otimes f_n = 0$が従う.
		\QED
	\end{prf}
	
	\begin{screen}
		\begin{thm}[テンソル積の基底]
			$(E_i)_{i=1}^n$を$\K$-線型空間の族とし,$E_i$の基底を
			$\left\{ u^i_{\lambda_i} \right\}_{\lambda_i \in \Lambda_i}$
			とする$(i=1,\cdots,n)$.このとき$\left\{ u^1_{\lambda_1} \otimes 
			\cdots \otimes u^n_{\lambda_n} \right\}_{\lambda_1,\cdots,\lambda_n}$
			は$\bigotimes_{i=1}^n E_i$の基底となる.
		\end{thm}
	\end{screen}
	
	\begin{prf}\mbox{}
		\begin{description}
			\item[第一段]
				任意の$e_1 \otimes \cdots \otimes e_n \in \bigotimes_{i=1}^n E_i$は
				$\left\{ u^1_{\lambda_1} \otimes \cdots \otimes u^n_{\lambda_n} 
				\right\}_{\lambda_1,\cdots,\lambda_n}$
				の線型結合で表現されるから,式(\refeq{eq:thm_tensor_product_is_bilinear})より
				\begin{align}
					\bigotimes_{i=1}^n E_i = \Span{\Set{u^1_{\lambda_1} \otimes 
				\cdots \otimes u^n_{\lambda_n}}{\lambda_i \in \Lambda_i,\ i=1,\cdots,n}}
				\end{align}
				が成立する.
				
			\item[第二段]
				$\left\{ u^1_{\lambda_1} \otimes 
				\cdots \otimes u^n_{\lambda_n} \right\}_{\lambda_1,\cdots,\lambda_n}$
				の一次独立性を示す.
				$\left\{ u^i_{\lambda_i} \right\}_{\lambda_i \in \Lambda_i}$に対する
				双対基底を$\left\{ f^i_{\lambda_i} \right\}_{\lambda_i \in \Lambda_i}$
				と書けば,各$f^i_{\lambda_i}$は
				\begin{align}
					f^i_{\lambda_i}(u^i_\lambda) =
					\begin{cases}
						1, & (\lambda = \lambda_i), \\
						0, & (\lambda \neq \lambda_i),
					\end{cases}
					\quad \forall \lambda \in \Lambda_i
				\end{align}
				を満たし,双対基底により構成する写像のテンソル積
				$f^1_{\lambda_1} \otimes \cdots \otimes f^n_{\lambda_n}$について
				\begin{align}
					f^1_{\lambda_1} \otimes \cdots \otimes f^n_{\lambda_n}(u^1_{\nu_1} \otimes 
					\cdots \otimes u^n_{\nu_n}) = 
					\begin{cases}
						1, & (\nu_1,\cdots,\nu_n) = (\lambda_1,\cdots,\lambda_n), \\
						0, & (\nu_1,\cdots,\nu_n) \neq (\lambda_1,\cdots,\lambda_n),
					\end{cases}
					\quad \forall (\nu_1,\cdots,\nu_n) \in \prod_{i=1}^n \Lambda_i
				\end{align}
				が成立する.従って$u^1_{\lambda_1} \otimes \cdots \otimes u^n_{\lambda_n}$
				は全て零ではなく,かつ
				\begin{align}
					0 = \sum_{j=1}^k \alpha_j \left(u^1_{\lambda^{(j)}_1} \otimes 
					\cdots \otimes u^n_{\lambda^{(j)}_n} \right),
					\quad (\alpha_j \in \K,\ j=1,\cdots,k)
				\end{align}
				を満たすような任意の線型結合に対し(ただし$i \neq j$なら$(\lambda_1^{(i)},\cdots,\lambda_n^{(i)}) \neq (\lambda_1^{(j)},\cdots,\lambda_n^{(j)})$)
				\begin{align}
					\alpha_j = f^1_{\lambda^{(j)}_1} \otimes \cdots \otimes f^n_{\lambda^{(j)}_n}
					\biggl( \sum_{j=1}^k \alpha_j \left(u^1_{\lambda^{(j)}_1} \otimes 
					\cdots \otimes u^n_{\lambda^{(j)}_n} \right) \biggr)
					= 0,
					\quad (j=1,\cdots,k)
				\end{align}
				が従い$\left\{ u^1_{\lambda_1} \otimes 
				\cdots \otimes u^n_{\lambda_n} \right\}_{\lambda_1,\cdots,\lambda_n}$
				の一次独立性を得る.
				\QED
		\end{description}
	\end{prf}
	
	\begin{screen}
		\begin{thm}[結合律]
		\label{thm:associativity_of_tensor_products}
			$(E_i)_{i=1}^n$を$\K$-線型空間の族とし,
			$k \in \{ 1,\cdots,n-1 \}$を任意に取る.このとき,次の対応関係を満たす
			$F$は線型同型である:
			\begin{align}
				\begin{array}{ccc}
					F:\bigotimes_{i=1}^n E_i & \longrightarrow & \biggl( \bigotimes_{i=1}^k E_i \biggr) \bigotimes \biggl( \bigotimes_{i=k+1}^n E_i \biggr) \\
					\rotatebox{90}{$\in$} & & \rotatebox{90}{$\in$} \\
					e_1 \otimes \cdots \otimes e_n & \longmapsto & (e_1 \otimes \cdots \otimes e_k) \otimes (e_{k+1} \otimes \cdots \otimes e_n)
				\end{array}
			\end{align}
		\end{thm}
	\end{screen}
	
	\begin{prf}\mbox{}
		\begin{description}
			\item[第一段]
				$n$重線型写像$f:\bigoplus_{i=1}^n E_i \longrightarrow \biggl( \bigotimes_{i=1}^k E_i \biggr) \bigotimes \biggl( \bigotimes_{i=k+1}^n E_i \biggr)$を
				\begin{align}
					f(e_1,\cdots,e_n) = (e_1 \otimes \cdots \otimes e_k) \otimes 
					(e_{k+1} \otimes \cdots \otimes e_n),
					\quad (\forall (e_1,\cdots,e_n) \in \bigoplus_{i=1}^n E_i)
				\end{align}
				により定めれば,定理\ref{thm:universality_of_tensor_product}より
				\begin{align}
					F(e_1 \otimes \cdots \otimes e_n)
					= (e_1 \otimes \cdots \otimes e_k) \otimes 
					(e_{k+1} \otimes \cdots \otimes e_n),
					\quad (\forall (e_1,\cdots,e_n) \in \bigoplus_{i=1}^n E_i)
				\end{align}
				を満たす線型写像$F:\bigotimes_{i=1}^n E_i \longrightarrow \biggl( \bigotimes_{i=1}^k E_i \biggr) \bigotimes \biggl( \bigotimes_{i=k+1}^n E_i \biggr)$
				が存在する:
				\begin{align}
					\xymatrix{
						&\bigoplus_{i=1}^n E_i \ar[d]_-\otimes \ar[rd]^-f & \\
						&\bigotimes_{i=1}^n E_i \ar[r]^-F & \biggl( \bigotimes_{i=1}^k E_i \biggr) \bigotimes \biggl( \bigotimes_{i=k+1}^n E_i \biggr)
					}
				\end{align}
				以降は$F$の逆写像を構成し$F$が全単射であることを示す.
				
			\item[第二段]
				$u_{k+1} \in E_{k+1},\cdots,u_n \in E_n$を固定し
				\begin{align}
					\Phi_{u_{k+1},\cdots,u_n}(e_1,\cdots,e_n)
					\coloneqq e_1 \otimes \cdots \otimes e_k \otimes u_{k+1} \otimes \cdots \otimes u_n
				\end{align}
				によって$k$重線型$\Phi_{u_{k+1},\cdots,u_n}:\bigoplus_{i=1}^k E_i \longrightarrow 
				\bigotimes_{i=1}^n E_i$を定めれば,定理\ref{thm:universality_of_tensor_product}より
				\begin{align}
					G_{u_{k+1},\cdots,u_n}(e_1 \otimes \cdots \otimes e_k)
					= e_1 \otimes \cdots \otimes e_k \otimes u_{k+1} \otimes \cdots \otimes u_n
				\end{align}
				を満たす線型写像$G_{u_{k+1},\cdots,u_n}:\bigotimes_{i=1}^k E_i \longrightarrow 
				\bigotimes_{i=1}^n E_i$が存在する.
				\begin{align}
					\xymatrix{
						&\bigoplus_{i=1}^k E_i \ar[d]_-\otimes \ar[rd]^-{\Phi_{u_{k+1},\cdots,u_n}} & \\
						&\bigotimes_{i=1}^k E_i \ar[r]^-{G_{u_{k+1},\cdots,u_n}} & \bigotimes_{i=1}^n E_i
					}
				\end{align}
			
			\item[第三段]
				任意の$v \in \bigotimes_{i=1}^k E_i$に対して
				\begin{align}
					\Psi_v: \bigoplus_{i=k+1}^n E_i
					\ni (u_{k+1},\cdots,u_n)
					\longmapsto G_{u_{k+1},\cdots,u_n}(v)
				\end{align}
				を定めれば,$\Psi_v$は$n-k$重線型であるから,定理\ref{thm:universality_of_tensor_product}より
				\begin{align}
					H_v(u_{k+1} \otimes \cdots \otimes u_n)
					= \Psi_v(u_{k+1},\cdots,u_n)
				\end{align}
				を満たす線型写像$H_v:\bigotimes_{i=k+1}^n E_i \longrightarrow 
				\bigotimes_{i=1}^n E_i$が存在する.
				\begin{align}
					\xymatrix{
						&\bigoplus_{i=k+1}^n E_i \ar[d]_-\otimes \ar[rd]^-{\Psi_v} & \\
						&\bigotimes_{i=k+1}^n E_i \ar[r]^-{H_v} & \bigotimes_{i=1}^n E_i
					}
				\end{align}
				いま,$v \longmapsto \Psi_v$は線型であり,かつ
				$\Psi_v$と$H_v$は一対一対応であるから
				$v \longmapsto H_v$の線型性が従う.
				
			\item[第四段]
				$H_v$の線型性と$v \longmapsto H_v$の線型性より
				\begin{align}
					\Gamma:\biggl( \bigotimes_{i=1}^k E_i \biggr) \times \biggl( \bigotimes_{i=k+1}^n E_i \biggr) \ni (v,w) \longmapsto H_v(w)
				\end{align}
				により定める$\Gamma$は
				\begin{align}
					\Gamma(e_1 \otimes \cdots \otimes e_k, e_{k+1} \otimes \cdots \otimes e_n) 
					&= H_{e_1 \otimes \cdots \otimes e_k}\left(e_{k+1} \otimes \cdots \otimes e_n \right) \\
					&= \Psi_{e_1 \otimes \cdots \otimes e_k}\left(e_{k+1},\cdots,e_n \right) \\
					&= G_{e_{k+1},\cdots,e_n}\left(e_1 \otimes \cdots \otimes e_k \right) \\
					&= \Phi_{e_{k+1},\cdots,e_n}\left(e_1, \cdots, e_k \right) \\
					&= e_1 \otimes \cdots \otimes e_n
					\label{eq:thm_associativity_of_tensor_products}
				\end{align}
				を満たす双線型であり,定理\ref{thm:universality_of_tensor_product}より
				\begin{align}
					\xymatrix{
						&\biggl( \bigotimes_{i=1}^k E_i \biggr) \times \biggl( \bigotimes_{i=k+1}^n E_i \biggr) \ar[d]_-\otimes \ar[rd]^-{\Gamma} & \\
						&\biggl( \bigotimes_{i=1}^k E_i \biggr) \bigotimes \biggl( \bigotimes_{i=k+1}^n E_i \biggr) \ar[r]^-{G} & \bigotimes_{i=1}^n E_i
					}
				\end{align}
				を可換にする線型写像$G$が存在する.この$G$は$F$の逆写像である.実際,
				(\refeq{eq:thm_associativity_of_tensor_products})より
				\begin{align}
					F \circ G \left( (e_1 \otimes \cdots \otimes e_k) \otimes (e_{k+1} \otimes \cdots \otimes e_n) \right)
					&= F \left(\Gamma(e_1 \otimes \cdots \otimes e_k, e_{k+1} \otimes \cdots \otimes e_n) \right) \\
					&= F (e_1 \otimes \cdots \otimes e_n) \\
					&= (e_1 \otimes \cdots \otimes e_k) \otimes (e_{k+1} \otimes \cdots \otimes e_n)
				\end{align}
				かつ
				\begin{align}
					G \circ F \left( e_1 \otimes \cdots \otimes e_n \right)
					&= G \left( (e_1 \otimes \cdots \otimes e_k) \otimes (e_{k+1} \otimes \cdots \otimes e_n) \right) \\
					&= \Gamma\left(e_1 \otimes \cdots \otimes e_k,e_{k+1} \otimes \cdots \otimes e_n \right) \\
					&= e_1 \otimes \cdots \otimes e_n
				\end{align}
				が満たされ$F^{-1} = G$が従う.
				\QED
		\end{description}
	\end{prf}
	\section{クロスノルム}
	$\K = \R$または$\K=\C$と考える.以下では$n (\geq 2)$個のBanach空間で構成する
	テンソル積におけるクロスノルムを考察する.
	\begin{screen}
		\begin{dfn}[クロスノルム]\label{def:cross_norm}
			$\K$-Banach空間の族$(X_i)_{i=1}^n$のテンソル積$\bigotimes_{i=1}^n X_i$において
			\begin{align}
				\alpha(x_1 \otimes \cdots \otimes x_n) &\leq \Norm{x_1}{X_1} \Norm{x_2}{X_2} \cdots \Norm{x_n}{X_n}, && (x_i \in X_i), \label{eq:cross_norm_def_1}\\
				\sup{\substack{v \in \bigotimes_{i=1}^n X_i \\ v \neq 0}}{\left| x_1^* \otimes \cdots \otimes x_n^* (v) \right|} &\leq \Norm{x_1^*}{X_1^*} \Norm{x_2^*}{X_2^*} \cdots \Norm{x_n^*}{X_n^*}\alpha(v),
				&& (x_i^* \in X_i^*) \label{eq:cross_norm_def_2}
			\end{align}
			を満たすようなノルム$\alpha:\bigotimes_{i=1}^n X_i \longrightarrow [0,\infty)$を
			クロスノルム(cross norm)と呼ぶ.
		\end{dfn}
	\end{screen}
	
	\begin{screen}
		\begin{thm}
			$\K$-Banach空間の族$(X_i)_{i=1}^{n}$に対するテンソル積
			$\bigotimes_{i=1}^n X_i$上のクロスノルム$\alpha$は
			次を満たす:
			\begin{align}
				&\alpha(x_1 \otimes \cdots \otimes x_n) 
					= \Norm{x_1}{X_1} \cdots \Norm{x_n}{X_n}, && (x_i \in X_i,\ i=1,\cdots,n), \\
				&\Norm{x_1^* \otimes \cdots \otimes x_n^*}{(\bigotimes_{i=1}^n X_i, \alpha)^*} 
					= \Norm{x_1^*}{X_1^*} \cdots \Norm{x_n^*}{X_n^*},
				&& (x_i^* \in X_i^*,\ i=1,\cdots,n).
			\end{align}
		\end{thm}
	\end{screen}
	
	\begin{prf}
		先ず,Hahn-Banachの定理と式(\refeq{eq:cross_norm_def_2})より
		\begin{align}
			\Norm{x_1}{X_1} \cdots \Norm{x_n}{X_n} 
			&= \sup{\Norm{x_1^*}{X_1^*} \leq 1}{\left| \inprod<x_1,x_1^*> \right|} 
				\cdots \sup{\Norm{x_n^*}{X_n^*} \leq 1}{\left| \inprod<x_n,x_n^*> \right|} \\
			&= \sup{\substack{\Norm{x_i^*}{X_i^*} \leq 1 \\ i=1,\cdots,n}}{\left| x_1^* \otimes \cdots \otimes x_n^* (x_1 \otimes \cdots \otimes x_n) \right|} \\
			&\leq \sup{\substack{\Norm{x_i^*}{X_i^*} \leq 1 \\ i=1,\cdots,n}}{\Norm{x_1^*}{X_1^*} \cdots \Norm{x_n^*}{X_n^*}}\alpha(x_1 \otimes \cdots \otimes x_n) \\
			&= \alpha(x_1 \otimes \cdots \otimes x_n)
		\end{align}
		が成り立ち定理の主張の第一式を得る.またこの結果より
		\begin{align}
			\Norm{x_1^*}{X_1^*} \cdots \Norm{x_n^*}{X_n^*} 
			&= \sup{\Norm{x_1}{X_1} \leq 1}{\left| \inprod<x_1,x_1^*> \right|}
				\cdots \sup{\Norm{x_n}{X_n} \leq 1}{\left| \inprod<x_n,x_n^*> \right|} \\
			&= \sup{\substack{\Norm{x_i}{X_i} \leq 1 \\ i=1,\cdots,n}}{\left| x_1^* \otimes \cdots \otimes x_n^* (x_1 \otimes \cdots \otimes x_n) \right|} \\
			&\leq \sup{\alpha(x_1 \otimes \cdots \otimes x_n) \leq 1}{\left| x_1^* \otimes \cdots \otimes x_n^* (x_1 \otimes \cdots \otimes x_n) \right|} \\
			&\leq \sup{\alpha(v) \leq 1}{\left| x_1^* \otimes \cdots \otimes x_n^* (v) \right|} \\
			&= \Norm{x_1^* \otimes \cdots \otimes x_n^*}{(\bigotimes_{i=1}^n X_i, \alpha)^*}
		\end{align}
		が成立し主張の第二式も得られる.
		\QED
	\end{prf}
	
	以下,実際クロスノルムが存在することを示す.
	\begin{screen}
		\begin{dfn}[インジェクティブノルム]
			$\K$-Banach空間の族$(X_i)_{i=1}^n$に対し
			\begin{align}
				\epsilon(v) \coloneqq
				\sup{\substack{\Norm{x_i^*}{X_i^*}\leq 1 \\ i=1,\cdots,n}}{\left| x_1^* \otimes \cdots \otimes x_n^* (v) \right|},
				\quad (v \in \bigotimes_{i=1}^n X_i)
			\end{align}
			により定める$\epsilon$をインジェクティブノルム(injective norm)と呼ぶ.
		\end{dfn}
	\end{screen}
	
	\begin{screen}
		\begin{thm}[インジェクティブノルムは最小のクロスノルム]
		\label{thm:injective_norm_is_the_minimum_cross_norm}
			$\K$-Banach空間の族$(X_i)_{i=1}^n$のテンソル積$\bigotimes_{i=1}^{n} X_i$において,
			インジェクティブノルムは最小のクロスノルムである.
		\end{thm}
	\end{screen}
	
	\begin{prf}\mbox{}
		\begin{description}
			\item[第一段]
				$\epsilon$が$\bigotimes_{i=1}^{n} X_i$上のノルムであることを示す.
				劣加法性と同次性は$x_1^* \otimes \cdots \otimes x_n^*$の線型性より従う.
				$v = 0 \Leftrightarrow \epsilon(v) = 0$については,
				$v = 0$なら任意の$x_1^* \otimes \cdots \otimes x_n^*$について
				$x_1^* \otimes \cdots \otimes x_n^* (v) = 0$
				が成り立ち$\epsilon(v) = 0$が出る.
				逆に$v \neq 0$とするとき,定理\ref{thm:tensor_product_is_bilinear}より
				\begin{align}
					v = \sum_{j=1}^{m} x^j_1 \otimes \cdots \otimes x^j_n,
					\quad (x^j_i \in X_i,\ j=1,\cdots,m,\ i=1,\cdots,n)
				\end{align}
				と表現できるが,定理\ref{thm:when_tensor_product_zero}より
				$x^1_i \neq 0\ (i=1,\cdots,n)$と仮定できる.
				$x^1_1$について,
				もし全ての$2 \leq j \leq m$に対し$x^j_1 = x^1_1$が満たされているなら,
				$\hat{x}_1^* \in X_1^*$を
				\begin{align}
					\inprod<x^1_1, \hat{x}_1^*> = \Norm{x^1_1}{X_1},
					\quad \Norm{\hat{x}_1^*}{X^*_1} = 1
				\end{align}
				を満たすように選ぶ(Hahn-Banachの定理).$x^j_1 \neq x^1_1$を満たす$j$がある場合,
				\begin{align}
					L_1 \coloneqq \Span{\Set{x^j_1}{2 \leq j \leq m,\ x^1_1 \neq x^j_1}}
				\end{align}
				により閉部分空間を定めれば$x^1_1$と$L_1$との距離$d_1$は正であり,Hahn-Banachの定理より
				\begin{align}
					\inprod<x_1,\hat{x}_1^*> = 0\ (\forall x_1 \in L_1),
					\quad \inprod<x^1_1,\hat{x}_1^*> = d_1 > 0,
					\quad \Norm{\hat{x}_1^*}{X_1^*} = 1
				\end{align}
				を満たす$\hat{x}_1^* \in X_1^*$を取ることができる.
				同様に$\hat{x}_i^* \in X_i^*\ (i=2,\cdots,n)$を選べば
				\begin{align}
					\hat{x}_1^* \otimes \cdots \otimes \hat{x}_n^*(x^j_1 \otimes \cdots \otimes x^j_n) =
					\begin{cases}
						\hat{x}_1^* \otimes \cdots \otimes \hat{x}_n^*(x^1_1 \otimes \cdots \otimes x^1_n), & (x^j_i = x^1_i,\ i=1,\cdots,n), \\
						0, & (\mbox{o.w.}),
					\end{cases}
				\end{align}
				$(j=2,\cdots,m)$が満たされるから
				\begin{align}
					0 < \hat{x}_1^* \otimes \cdots \otimes \hat{x}_n^*(x^1_1 \otimes \cdots \otimes x^1_n) 
					\leq \left| \hat{x}_1^* \otimes \cdots \otimes \hat{x}_n^* (v) \right|
					\leq \epsilon(v)
					\label{eq:thm_injective_norm_is_the_minimum_cross_norm_1}
				\end{align}
				が成立し,対偶により$\epsilon(v) = 0 \Rightarrow v = 0$が従う.
				
			\item[第二段]
				$\epsilon$がクロスノルムであることを示す.
				先ずHahn-Banachの定理より
				\begin{align}
					\epsilon(x_1 \otimes \cdots \otimes x_n) 
					&= \sup{\substack{\Norm{x_i^*}{X_i^*}\leq 1 \\ i=1,\cdots,n}}{\left| x_1^* \otimes \cdots \otimes x_n^* (x_1 \otimes \cdots \otimes x_n) \right|} \\
					&= \sup{\Norm{x_1^*}{X_1^*} \leq 1}{\left| \inprod<x_1,x_1^*> \right|}
					\cdots \sup{\Norm{x_n^*}{X_n^*} \leq 1}{\left| \inprod<x_n,x_n^*> \right|} \\
					&= \Norm{x_1}{X_1} \cdots \Norm{x_n}{X_n},
					\quad (\forall x_i \in X_i,\ i=1,\cdots,n)
				\end{align}
				が成り立つ.また0でない$x_i^* \in X_i^*,\ (i=1,\cdots,n)$に対しては
				\begin{align}
					\left| x_1^* \otimes \cdots \otimes x_n^* (v) \right|
					&\leq \Norm{x_1^*}{X_1^*} \cdots \Norm{x_n^*}{X_n^*} \left[ \frac{x_1^*}{\Norm{x_1^*}{X_1^*}} \otimes \cdots \otimes \frac{x_n^*}{\Norm{x_n^*}{X_n^*}} \right] (v) \\
					&\leq \Norm{x_1^*}{X_1^*} \cdots \Norm{x_n^*}{X_n^*} \epsilon(v)
				\end{align}
				が成立し,或る$i$で$x_i^*$が零写像のときは
				定理\ref{thm:tensor_product_contains_zero_mapping_is_zero}より
				$x_1^* \otimes \cdots \otimes x_n^* = 0$が満たされ,
				\begin{align}
					\Norm{x_1^* \otimes \cdots \otimes x_n^*}{(\bigotimes_{i=1}^n X_i,\epsilon)} \leq \Norm{x_1^*}{X_1^*} \cdots \Norm{x_n^*}{X_n^*}
				\end{align}
				を得る.
			
			\item[第三段]
				$\epsilon$が最小のクロスノルムであることを示す.$\alpha$を任意のクロスノルムとすれば
				\begin{align}
					\left| x_1^* \otimes \cdots \otimes x_n^* (v) \right| 
					\leq \Norm{x_1^*}{X_1^*} \cdots \Norm{x_n^*}{X_n^*} \alpha(v),
					\quad (\forall v \in \bigotimes_{i=1}^n X_i)
				\end{align}
				が成り立つから,特に$\Norm{x_i^*}{X_i^*} \leq 1,\ (i=1,\cdots,n)$の範囲でsupを取れば
				\begin{align}
					\epsilon(v) \leq \alpha(v),
					\quad (\forall v \in \bigotimes_{i=1}^n X_i)
				\end{align}
				が従い$\epsilon$の最小性が出る.
				\QED
		\end{description}
	\end{prf}
	
	\begin{screen}
		\begin{dfn}[プロジェクティブノルム]
			$\K$-Banach空間の族$(X_i)_{i=1}^n$に対し,定理
			\ref{thm:universality_of_tensor_product}により
			\begin{align}
				\begin{array}{ccc}
					\Phi:\Hom{\bigotimes_{i=1}^n X_i}{\K} & \longrightarrow & \Ln{\bigoplus_{i=1}^n X_i}{\K}{n} \\
					\rotatebox{90}{$\in$} & & \rotatebox{90}{$\in$} \\
					T & \longmapsto & T \circ \otimes
					\label{eq:dfn_projective_norm_isomorphism}
				\end{array}
			\end{align}
			により線型同型$\Phi$が定まる.これを用いて
			\begin{align}
				\pi(v) \coloneqq
				\sup{\substack{b \in \ContLn{\bigoplus_{i=1}^n X_i}{\K}{n} \\ \Norm{b}{\ContLn{\bigoplus_{i=1}^n X_i}{\K}{n}} \leq 1}}{\left| \Phi^{-1}(b)(v) \right|},
				\quad (v \in \bigotimes_{i=1}^n X_i)
			\end{align}
			により定める$\pi$をプロジェクティブノルム(projective norm)と呼ぶ.
		\end{dfn}
	\end{screen}
	
	\begin{screen}
		\begin{thm}[プロジェクティブノルムは最大のクロスノルム]
		\label{thm:projective_norm_is_maximum_cross_norm}
			$\K$-Banach空間の族$(X_i)_{i=1}^n$のテンソル積$\bigotimes_{i=1}^n X_i$上に
			プロジェクティブノルム$\pi$を導入する.このとき式(\refeq{eq:cross_norm_def_1})を満たす任意のセミノルム
			$p$に対し$p \leq \pi$が成立する.特に$\pi$は最大のクロスノルムである.
		\end{thm}
	\end{screen}
	
	\begin{prf}\mbox{}
		\begin{description}
			\item[第一段]
				$\pi$がノルムであることを示す.$v \neq 0$とすれば,
				定理\ref{thm:injective_norm_is_the_minimum_cross_norm}
				の証明と同様にして
				\begin{align}
					0 < \left| \hat{x}_1^* \otimes \cdots \otimes \hat{x}_n^* (v) \right|,
					\quad \Norm{\hat{x}^*_i}{X^*_i} = 1,
					\quad (i=1,\cdots,n)
				\end{align}
				を満たす$\hat{x}^*_i \in X^*_i\ (i=1,\cdots,n)$
				が存在する.
				\begin{align}
					b(x_1,\cdots,x_n) 
					\coloneqq \inprod<x_1,\hat{x}_1^*> \cdots \inprod<x_n,\hat{x}_n^*>,
					\quad (x_i \in X_i,\ i=1,\cdots,n)
				\end{align}
				により$n$重線型写像$b$を定めれば,$\Norm{b}{\ContLn{\bigoplus_{i=1}^n X_i}{\K}{n}}
				\leq \Norm{x_1^*}{X_1^*} \cdots \Norm{x_n^*}{X_n^*} = 1$かつ
				\begin{align}
					0 < \left| \hat{x}_1^* \otimes \cdots \otimes \hat{x}_n^* (v) \right| 
					= |\Phi^{-1}(b)(v)| \leq \pi(v)
				\end{align}
				が成立する.$\pi(0) = 0$と
				劣加法性及び同次性は$\Phi^{-1}(b)$の線型性より従う.
			
			\item[第二段]
				$\pi$がクロスノルムであることを示す.
				先ず,任意の$x_i \in X_i,\ (i=1,\cdots,n)$に対して
				\begin{align}
					\pi(x_1 \otimes \cdots \otimes x_n) 
					&= \sup{\substack{b \in \ContLn{\bigoplus_{i=1}^n X_i}{\K}{n} \\ \Norm{b}{\ContLn{\bigoplus_{i=1}^n X_i}{\K}{n}} \leq 1}}{\left| \Phi^{-1}(b)(x_1 \otimes \cdots \otimes x_n) \right|} \\
					&\leq \sup{\substack{b \in \ContLn{\bigoplus_{i=1}^n X_i}{\K}{n} \\ \Norm{b}{\ContLn{\bigoplus_{i=1}^n X_i}{\K}{n}} \leq 1}}{\Norm{b}{\ContLn{\bigoplus_{i=1}^n X_i}{\K}{n}}\Norm{x_1}{X_1} \cdots \Norm{x_n}{X_n}} \\
					&= \Norm{x_1}{X_1} \cdots \Norm{x_n}{X_n}
				\end{align}
				が成立する.また0でない$x_i^* \in X_i^*,\ (i=1,\cdots,n)$に対し
				\begin{align}
					b(x_1,\cdots,x_n) 
					\coloneqq \frac{x_1^*}{\Norm{x_1^*}{X_1^*}}(x_1) \cdots \frac{x_n^*}{\Norm{x_n^*}{X_n^*}}(x_n),
					\quad (x_i \in X_i,\ i=1,\cdots,n)
				\end{align}
				により$\Norm{b}{\ContLn{\bigoplus_{i=1}^n X_i}{\K}{n}} \leq 1$
				を満たす有界$n$重線型$b$を定めれば,
				$\pi$の定義より
				\begin{align}
					\left| \Phi^{-1}(b)(v) \right| \leq \pi(v),
					\quad (\forall v \in \bigotimes_{i=1}^n X_i)
				\end{align}
				が成り立つ.一方で写像のテンソル積の定義より
				\begin{align}
					\Phi^{-1}(b) 
					= \frac{x_1^*}{\Norm{x_1^*}{X_1^*}} 
						\otimes \cdots \otimes \frac{x_n^*}{\Norm{x_n^*}{X_n^*}}
					= \frac{1}{\Norm{x_1^*}{X_1^*} \cdots \Norm{x_n^*}{X_n^*}} 
						x_1^* \otimes \cdots \otimes x_n^*
				\end{align}
				が満たされるから
				\begin{align}
					\left| x_1^* \otimes \cdots \otimes x_n^*(v) \right| 
						\leq \Norm{x_1^*}{X_1^*} \cdots \Norm{x_n^*}{X_n^*} \pi(v),
					\quad (\forall v \in \bigotimes_{i=1}^n X_i)
				\end{align}
				が従う.定理\ref{thm:tensor_product_contains_zero_mapping_is_zero}より
				$x_1^* \otimes \cdots \otimes x_n^* = 0$なら
				どれか一つは$x_i^* = 0$であるから
				\begin{align}
					\Norm{x_1^* \otimes \cdots \otimes x_n^*}{(\bigotimes_{i=1}^n X_i,\pi)^*} 
					\leq \Norm{x_1^*}{X_1^*} \cdots \Norm{x_n^*}{X_n^*}
				\end{align}
				が得られる.
				
			\item[第三段]
				$p$を(\refeq{eq:cross_norm_def_1})を満たすセミノルムとし,
				$v \in \bigotimes_{i=1}^n X_i$を任意に取れば
				\begin{align}
					p(v) = \phi_v(v),
					\quad |\phi_v(u)| \leq p(u)
					\quad (\forall u \in \bigotimes_{i=1}^n X_i)
				\end{align}
				を満たす$\phi_v \in (\bigotimes_{i=1}^n X_i,\pi)^*$が存在する(Hahn-Banachの定理).
				\begin{align}
					\left| (\phi_v \circ \otimes)(x_1,\cdots,x_n) \right|
					&= \left|\phi_v(x_1 \otimes \cdots \otimes x_n)\right| \\ 
					&\leq p(x_1 \otimes \cdots \otimes x_n) \\
					&\leq \Norm{x_1}{X_1} \cdots \Norm{x_n}{X_n},
					\quad (\forall x_i \in X_i,\ i=1,\cdots,n)
				\end{align}
				が成り立つから$\Norm{\phi_v \circ \otimes}{\ContLn{\bigoplus_{i=1}^n X_i}{\K}{n}} \leq 1$が従い,
				$\pi$の定義より
				\begin{align}
					p(v) = \phi_v(v) = \Phi^{-1}(\phi_v \circ \otimes)(v)
					\leq \pi(v)
				\end{align}
				が得られる.
				\QED
		\end{description}
	\end{prf}
	
	\begin{screen}
		\begin{thm}[プロジェクティブノルムの表現]\label{thm:expression_of_projective_norm}
			$\K$-Banach空間の族$(X_i)_{i=1}^n$のテンソル積$\bigotimes_{i=1}^n X_i$
			にプロジェクトノルム$\pi$を導入する.このとき次が成り立つ:
			\begin{align}
				\pi(v) = \inf{}{\Set{\sum_{i=1}^n \Norm{x_1}{X_1} \cdots \Norm{x_n}{X_n}}{
					v = \sum_{j=1}^m x^j_1 \otimes \cdots \otimes x^j_n}}.
			\end{align}
		\end{thm}
	\end{screen}
	
	\begin{prf}\mbox{}
		\begin{description}
			\item[第一段]
				$x_1 \otimes \cdots \otimes x_n$上のセミノルム$\lambda$を次で定める:
				\begin{align}
					\lambda(v) \coloneqq \inf{}{\Set{\sum_{i=1}^n \Norm{x_1}{X_1} \cdots \Norm{x_n}{X_n}}{
					v = \sum_{j=1}^m x^j_1 \otimes \cdots \otimes x^j_n}},
					\quad (\forall v \in \bigotimes_{i=1}^n X_i).
				\end{align}
				このとき$\lambda$が式(\refeq{eq:cross_norm_def_1})かつ
				$\lambda \geq \pi$を満たせば,
				定理\ref{thm:projective_norm_is_maximum_cross_norm}より$\lambda = \pi$が従う.
				
			\item[第二段]
				$\lambda$がセミノルムであることを示す.実際,任意に
				$u,v \in \bigotimes_{i=1}^n X_i$を取り,
				\begin{align}
					u = \sum_{j=1}^{m} x^j_1 \otimes \cdots \otimes x^j_n,
					\quad v = \sum_{k=1}^r a^k_1 \otimes \cdots \otimes a^k_n
				\end{align}
				を一つの表現とすれば,$\lambda$の定め方より
				\begin{align}
					\lambda(u+v) \leq \sum_{j=1}^{m} x^j_1 \otimes \cdots \otimes x^j_n 
						+ \sum_{k=1}^r a^k_1 \otimes \cdots \otimes a^k_n
				\end{align}
				が成り立つ.右辺を移項して
				\begin{align}
					\lambda(u+v) - \sum_{k=1}^r a^k_1 \otimes \cdots \otimes a^k_n 
					\leq \lambda(u) 
					\leq \sum_{j=1}^{m} x^j_1 \otimes \cdots \otimes x^j_n
				\end{align}
				かつ
				\begin{align}
					\lambda(u+v) - \lambda(u) \leq \lambda(v) \leq \sum_{k=1}^r a^k_1 \otimes \cdots \otimes a^k_n 
				\end{align}
				が従い$\lambda$の劣加法性を得る.
				また任意の$0 \neq \alpha \in \K,\ v \in \bigotimes_{i=1}^n X_i$に対し
				\begin{align}
					v = \sum_{j=1}^{m} x^j_1 \otimes \cdots \otimes x^j_n
				\end{align}
				を一つの分割とすれば
				\begin{align}
					\alpha v = \sum_{j=1}^{m} \left( \alpha x^j_1 \right) \otimes \cdots \otimes x^j_n
				\end{align}
				は$\alpha v$の一つの分割となるから
				\begin{align}
					\lambda(\alpha v) \leq \sum_{j=1}^m \Norm{\alpha x^j_1}{X_1} \cdots \Norm{x^j_n}{X_n}
					= |\alpha| \sum_{j=1}^m \Norm{x^j_1}{X_1} \cdots \Norm{x^j_n}{X_n}
				\end{align}
				が成立し,$v$の分割について下限を取れば$\lambda(\alpha v) \leq |\alpha| \lambda(v)$
				が従う.逆に
				\begin{align}
					\alpha v = \sum_{k=1}^r a^k_1 \otimes \cdots \otimes a^k_n
				\end{align}
				に対しては
				\begin{align}
					\lambda(v) \leq \sum_{k=1}^r \Norm{\frac{1}{\alpha} a^k_1}{X_1} \cdots \Norm{a^k_n}{X_n}
					= \frac{1}{|\alpha|} \sum_{k=1}^r \Norm{a^k_1}{X_1} \cdots \Norm{a^k_n}{X_n}
				\end{align}
				が成り立ち$|\alpha| \lambda(v) \leq \lambda(\alpha v)$
				が従う.$v=0$なら$v = 0 \otimes \cdots \otimes 0$より$\lambda(v) = 0$が満たされ
				\begin{align}
					\lambda(\alpha v) = |\alpha| \lambda(v),
					\quad (\forall \alpha \in \K,\ v \in \bigotimes_{i=1}^n X_i)
				\end{align}
				が得られる.
				
			\item[第三段]
				$\lambda$が式(\refeq{eq:cross_norm_def_1})を満たすことを示す.実際$\lambda$の定め方より
				\begin{align}
					\lambda(x_1 \otimes \cdots \otimes x_n) 
					\leq \Norm{x_1}{X_1} \cdots \Norm{x_n}{X_n},
					\quad (\forall x_i \in X_i,\ i=1,\cdots,n)
				\end{align}
				が成り立つ.
				
			\item[第四段]
				$\lambda \geq \pi$を示す.いま,任意に$v \in \bigotimes_{i=1}^n X_i$を取り,
				次の分割を持つとする:
				\begin{align}
					v = \sum_{j=1}^{m} x^j_1 \otimes \cdots \otimes x^j_n.
				\end{align}
				$\Norm{b}{\ContLn{\bigoplus_{i=1}^n X_i}{\K}{n}} \leq 1$を満たす
				$b \in \ContLn{\bigoplus_{i=1}^n X_i}{\K}{n}$と
				式(\refeq{eq:dfn_projective_norm_isomorphism})の
				$\Phi$に対し
				\begin{align}
					&\left| \Phi^{-1}(b)(v) \right|
					\leq \sum_{j=1}^{m} \left| \Phi^{-1}(b)(x^j_1 \otimes \cdots \otimes x^j_n) \right| \\
					&\qquad = \sum_{j=1}^{m} \left| b(x^j_1,\cdots,x^j_n) \right|
					\leq \sum_{j=1}^{m} \Norm{x^j_1}{X_1} \cdots \Norm{x^j_n}{X_n}
				\end{align}
				が成り立つから,$b$に無関係に
				\begin{align}
					\left| \Phi^{-1}(b)(v) \right| \leq \lambda(v)
				\end{align}
				が満たされ
				\begin{align}
					\pi(v) = \sup{\substack{b \in \ContLn{\bigoplus_{i=1}^n X_i}{\K}{n} \\ \Norm{b}{\ContLn{\bigoplus_{i=1}^n X_i}{\K}{n}} \leq 1}}{\left| \Phi^{-1}(b)(v) \right|}
					\leq \lambda(v)
				\end{align}
				が従う.
				\QED
		\end{description}
	\end{prf}
	
	\begin{screen}
		\begin{thm}
			$\bigotimes_{i=1}^n X_i$を$\K$-Banach空間の族$(X_i)_{i=1}^n$のテンソル積とする.このとき
			$\bigotimes_{i=1}^n X_i$上の任意のノルム$\alpha$に対し次が成立する:
			\begin{align}
				\mbox{$\alpha$がクロスノルム}
				\quad \Leftrightarrow \quad 
				\epsilon \leq \alpha \leq \pi.
			\end{align}
		\end{thm}
	\end{screen}
	
	\begin{prf}
		$(\Rightarrow)$はすでに示したから
		$(\Leftarrow)$を示す.実際,任意の$x_i \in X_i,\ (i=1,\cdots,n)$に対して
		\begin{align}
			\alpha(x_1 \otimes \cdots \otimes x_n) 
			\leq \pi(x_1 \otimes \cdots \otimes x_n) 
			\leq \Norm{x_1}{X_1} \cdots \Norm{x_n}{X_n}
		\end{align}
		が成立し,また任意の$x_i^* \in X_i^*,\ (i=1,\cdots,n)$に対して
		\begin{align}
			\left| x_1^* \otimes \cdots \otimes x_n^*(v) \right|
			&\leq \Norm{x_1^* \otimes \cdots \otimes x_n^*}{(\bigotimes_{i=1}^n X_i, \epsilon)^*} \epsilon(v) \\
			&\leq \Norm{x_1^*}{X_1^*} \cdots \Norm{x_n^*}{X_n^*} \alpha(v),
			\quad (\forall v \in \bigotimes_{i=1}^n X_i)
		\end{align}
		が満たされ$\Norm{x_1^* \otimes \cdots \otimes x_n^*}{(\bigotimes_{i=1}^n X_i, \alpha)^*} 
		\leq \Norm{x_1^*}{X_1^*} \cdots \Norm{x_n^*}{X_n^*}$が得られる.
		\QED
	\end{prf}
	\section{テンソル積の内積}
	[参考:\cite{key8}(pp. 1-24), \cite{key7}] 
	$(H_i)_{i=1}^n$を$\R$-Hilbert空間の族として,$\bigotimes_{i=1}^n H_i$に内積を導入する.
	$H_i$の内積を$\inprod<\cdot,\cdot>_{H_i}$と書く.
	\begin{description}
		\item[第一段] 
			任意に$y_i \in H_i,\ i=1,\cdots,n$を取り
			\begin{align}
				\Phi_{y_1,\cdots,y_n}:
				\bigoplus_{i=1}^n H_i \ni (x_1,\cdots,x_n)
				\longmapsto \inprod<x_1,y_1>_{H_1} \cdots \inprod<x_n,y_n>_{H_n}
			\end{align}
			とおけば,$\Phi_{y_1,\cdots,y_n}$は$n$重線型であるから
			或る$\Psi_{y_1,\cdots,y_n} \in \Hom{\bigotimes_{i=1}^n H_i}{\R}$がただ一つ存在して
			\begin{align}
				\Psi_{y_1,\cdots,y_n} \circ \otimes = \Phi_{y_1,\cdots,y_n}
			\end{align}
			を満たす.
			
		\item[第二段]
			任意の$u \in \bigotimes_{i=1}^n H_i$に対し
			$\bigoplus_{i=1}^n H_i \ni (y_1,\cdots,y_n) \longmapsto \Psi_{y_1,\cdots,y_n}(u)$
			は$n$重線型である.実際,
			\begin{align}
				u = \sum_{j=1}^k x^j_1 \otimes \cdots \otimes x^j_n
			\end{align}
			と表せるとき,任意の$\alpha,\beta \in \R$と$y_i,z_i \in H_i$に対して
			\begin{align}
				\Psi_{y_1,\cdots,\alpha y_i + \beta z_i,\cdots,y_n}(u)
				&= \sum_{j=1}^k \Psi_{y_1,\cdots,\alpha y_i + \beta z_i,\cdots,y_n}(x^j_1 \otimes \cdots \otimes x^j_n) \\
				&= \sum_{j=1}^k \inprod<x^j_1,y_1>_{H_1} \cdots \inprod<x^j_i,\alpha y_i + \beta z_i>_{H_i} \cdots \inprod<x^j_n,y_n>_{H_n} \\
				&= \alpha \sum_{j=1}^k \inprod<x^j_1,y_1>_{H_1} \cdots \inprod<x^j_i,y_i>_{H_i} \cdots \inprod<x^j_n,y_n>_{H_n} \\
				&\quad + \beta \sum_{j=1}^k \inprod<x^j_1,y_1>_{H_1} \cdots \inprod<x^j_i,z_i>_{H_i} \cdots \inprod<x^j_n,y_n>_{H_n} \\
				&= \alpha \sum_{j=1}^k \Psi_{y_1,\cdots,y_i,\cdots,y_n}(x^j_1 \otimes \cdots \otimes x^j_n)
				+ \beta \sum_{j=1}^k \Psi_{y_1,\cdots,z_i,\cdots,y_n}(x^j_1 \otimes \cdots \otimes x^j_n) \\
				&= \alpha \Psi_{y_1,\cdots,y_i,\cdots,y_n}(u)
				+ \beta \Psi_{y_1,\cdots,z_i,\cdots,y_n}(u)
			\end{align}
			が成立する.従って或る$F_u \in \Hom{\bigotimes_{i=1}^n H_i}{\R}$がただ一つ存在して
			\begin{align}
				F_u(y_1 \otimes \cdots \otimes y_n) = \Psi_{y_1,\cdots,y_n}(u),
				\quad (\forall y_1 \otimes \cdots \otimes y_n \in \bigotimes_{i=1}^n H_i)
			\end{align}
			を満たす.
			
		\item[第三段]
			任意の$v \in \bigotimes_{i=1}^n H_i$に対し
			$\bigotimes_{i=1}^n H_i \ni u \longmapsto F_u(v)$は線型性を持つ.実際,
			\begin{align}
				v = \sum_{j=1}^k x^j_1 \otimes \cdots \otimes x^j_n
			\end{align}
			と表せるとき,任意の$\alpha,\beta \in \R$と$u,w \in \bigotimes_{i=1}^n H_i$に対して
			\begin{align}
				F_{\alpha u + \beta w}(v)
				&= \sum_{j=1}^k F_{\alpha u + \beta w}(x^j_1 \otimes \cdots \otimes x^j_n) \\
				&= \sum_{j=1}^k \Psi_{x^j_1, \cdots, x^j_n}(\alpha u + \beta w) \\
				&= \alpha \sum_{j=1}^k \Psi_{x^j_1, \cdots, x^j_n}(u)
					+ \beta \sum_{j=1}^k \Psi_{x^j_1, \cdots, x^j_n}(w) \\
				&= \alpha F_{u}(v) + \beta F_{w}(v)
			\end{align}
			が成立する.従って
			\begin{align}
				s(u,v) \coloneqq F_u(v),
				\quad (\forall u,v \in \bigotimes_{i=1}^n H_i)
				\label{eq:def_inner_product_of_tensor_product}
			\end{align}
			により定める$s:\bigotimes_{i=1}^n H_i \times \bigotimes_{i=1}^n H_i \longrightarrow \R$
			は双線型形式である.
	\end{description}
	
	\begin{screen}
		\begin{thm}
			式(\refeq{eq:def_inner_product_of_tensor_product})で定める$s$は
			$\bigotimes_{i=1}^n H_i$の内積となる.また任意の
			$x_1 \otimes \cdots \otimes x_n,\ y_1 \otimes \cdots \otimes y_n 
			\in \bigotimes_{i=1}^n H_i$に対し次を満たす:
			\begin{align}
				s(x_1 \otimes \cdots \otimes x_n,y_1 \otimes \cdots \otimes y_n)
				= \inprod<x_1,y_1>_{H_1} \cdots \inprod<x_n,y_n>_{H_n}.
			\end{align}
		\end{thm}
	\end{screen}
	
	\begin{prf} $s$は双線型性を持つ写像として定めたから,後は対称性と正定値性及び$s(u,u)=0 \Leftrightarrow u=0$
		を示せばよい.
		\begin{description}
			\item[第一段]
				任意の$u \in \bigotimes_{i=1}^n H_i$に対し$s(u,u) \geq 0$が成り立つことを示す.
				
			\item[第一段]
				$s(u,u)=0 \Leftrightarrow u=0,\ (u \in \bigotimes_{i=1}^n H_i)$が成り立つことを示す.
				定理\ref{thm:basis_of_tensor_product}より,
				基底$\left\{ e^i_{\lambda_i} \right\}_{\lambda_i \in \Lambda_i} \subset H_i,\ (i=1,\cdots,n)$
				に対し$\left\{ e^1_{\lambda_1} \otimes \cdots \otimes e^n_{\lambda_n} \right\}_{\lambda_1,\cdots,\lambda_n}$
				は$\bigotimes_{i=1}^n H_i$の基底となるから,任意の
				$u \in \bigotimes_{i=1}^n H_i$は
				\begin{align}
					u = \sum_{j=1}^k \alpha_j \left( e^1_{\lambda^{(j)}_1} \otimes \cdots \otimes e^n_{\lambda^{(j)}_n} \right)
					= \sum_{j=1}^k e^1_{\lambda^{(j)}_1} \otimes \cdots \otimes \left( \alpha_j e^n_{\lambda^{(j)}_n} \right),
					\quad (\alpha_j \neq 0,\ j=1,\cdots,k)
				\end{align}
				と表現できる.
			
			\item[第三段]
				$s$の対称性を示す.
		\end{description}
	\end{prf}
	
	\begin{screen}
		\begin{dfn}[テンソル積上の内積]
			式(\refeq{eq:def_inner_product_of_tensor_product})で定めた双線型形式$s$を
			$\inprod<\cdot,\cdot>$と書き直して$\bigotimes_{i=1}^n H_i$の内積とする.
			また$\sigma(u) \coloneqq \sqrt{\inprod<u,u>},\ (u \in \bigotimes_{i=1}^n H_i)$
			によりノルム$\sigma$を導入する.
		\end{dfn}
	\end{screen}
	
	\begin{screen}
		\begin{thm}[$\sigma$はクロスノルム]
		\end{thm}
	\end{screen}
	
	\begin{prf}
		\begin{description}
			\item[第二段]
				\begin{align}
					(x^*_1 \otimes \cdots \otimes x^*_n)(x)
					&= \sum_{j=1}^{k} (x^*_1 \otimes \cdots \otimes x^*_n)(x^j_1 \otimes \cdots \otimes x^j_n)
					= \sum_{j=1}^{k} x^*_1(x^j_1) \cdots x^*_n(x^j_n) \\
					&= \sum_{j=1}^{k} \inprod<x^j_1,a_1>_{H_1} \cdots \inprod<x^j_n,a_n>_{H_n}
					= \sum_{j=1}^{k} \inprod<x^j_1 \otimes \cdots \otimes x^j_n,a_1 \otimes \cdots \otimes a_n>
					= \inprod<x,a_1 \otimes \cdots \otimes a_n> \\
					&\leq \sigma(x) \sigma(a_1 \otimes \cdots \otimes a_n)
					= \sigma(x) \Norm{a_1}{H_1} \cdots \Norm{a_n}{H_n} \\
					&= \sigma(x) \Norm{x^*_1}{H^*_1} \cdots \Norm{x^*_n}{H^*_n}
				\end{align}
		\end{description}
	\end{prf}
	
	\begin{screen}
		\begin{thm}[$H_i$が有限次元なら$\sigma$と$\pi$は同値]
		\end{thm}
	\end{screen}

\begin{thebibliography}{数字}
  \bibitem{key1} S. Aida, Rough path analysis: an introduction.
  \bibitem{key2} T. Lyons and Z. Qian, System control and rough paths, Oxford science publications, 2002.
  \bibitem{key3} K. Friz and Nicholas B. Victor, Multidimensional stochastic processes as rough path: theory and applications, 2009.
  \bibitem{key4} M. Sugiura and M. Yokonuma, ジョルダン標準形・テンソル代数, 岩波基礎数学選書, 1990.
  \bibitem{key5} Y. Hirai and K. Matsuura, 自主ゼミ: Normal Approximations with Malliavin Calculus: From Stein's Method to Universality 用ノート, 2015. 
  \bibitem{key7} Y. Hirai, 関数解析ノート:ノルム空間上の有界双線形写像, 2017.
  \bibitem{key6} K. Matsuzaka, 集合・位相入門, 岩波書店, 2016.
\end{thebibliography}

\printindex
\end{document}