\documentclass[11pt,a4paper]{jsreport}
%
\usepackage{amsmath,amssymb}
\usepackage{amsthm}
\usepackage{makeidx}
\usepackage{txfonts}
\usepackage{mathrsfs} %花文字
\usepackage{mathtools} %参照式のみ式番号表示
\usepackage{latexsym} %qed
\usepackage{ascmac}
\usepackage{color}
\usepackage{comment}

\newtheoremstyle{mystyle}% % Name
	{20pt}%                      % Space above
	{20pt}%                      % Space below
	{\rm}%           % Body font
	{}%                      % Indent amount
	{\gt}%             % Theorem head font
	{.}%                      % Punctuation after theorem head
	{10pt}%                     % Space after theorem head, ' ', or \newline
	{}%                      % Theorem head spec (can be left empty, meaning `normal')
\theoremstyle{mystyle}

\allowdisplaybreaks[1]

\newcommand{\bhline}[1]{\noalign {\hrule height #1}} %表の罫線を太くする.
\newcommand{\bvline}[1]{\vrule width #1} %表の罫線を太くする.
\newtheorem{Prop}{$Proposition.$}
\newtheorem{Proof}{$Proof.$}
\newcommand{\QED}{% %証明終了
	\relax\ifmmode
		\eqno{%
		\setlength{\fboxsep}{2pt}\setlength{\fboxrule}{0.3pt}
		\fcolorbox{black}{black}{\rule[2pt]{0pt}{1ex}}}
	\else
		\begingroup
		\setlength{\fboxsep}{2pt}\setlength{\fboxrule}{0.3pt}
		\hfill\fcolorbox{black}{black}{\rule[2pt]{0pt}{1ex}}
		\endgroup
	\fi}
\newtheorem{thm}{定理}[section]
\newtheorem{dfn}[thm]{定義}
\newtheorem{prp}[thm]{命題}
\newtheorem{lem}[thm]{補題}
\newtheorem{cor}[thm]{系}
\newtheorem*{prf}{証明}
\newtheorem{qst}{レポート問題}
\newtheorem*{bcs}{なぜならば}
\newtheorem{rem}[thm]{注意}
\newcommand{\defunc}{\mbox{1}\hspace{-0.25em}\mbox{l}} %定義関数
\newcommand{\wlim}{\mbox{w-}\lim}
\newcommand{\dx}{\ \operatorname{d}} %積分のd
\def\Set#1#2{\left\{\ #1\ \, ; \quad #2\ \right\}} %集合の書き方
\def\Box#1{$(\mbox{#1})$} %丸括弧つきコメント
\def\Hat#1{$\hat{\mathrm{#1}}$} %文中ハット
\def\Ddot#1{$\ddot{\mathrm{#1}}$} %文中ddot
\def\DEF{\overset{\mathrm{def}}{\Leftrightarrow}} %定義記号
\def\eqqcolon{=\mathrel{\mathop:}} %定義=:
\def\max#1#2{\operatorname*{max}_{#1} #2 } %最大
\def\min#1#2{\operatorname*{min}_{#1} #2 } %最小
\def\sin#1#2{\operatorname{sin}^{#2} #1} %sin
\def\cos#1#2{\operatorname{cos}^{#2} #1} %cos
\def\tan#1#2{\operatorname{tan}^{#2} #1} %tan
\def\inprod<#1>{\left\langle #1 \right\rangle} %内積
\def\sup#1#2{\operatorname*{sup}_{#1} #2 } %上限
\def\inf#1#2{\operatorname*{inf}_{#1} #2 } %下限
\def\Vector#1{\mbox{\boldmath $#1$}} %ベクトルを太字表示
\def\Norm#1#2{\left\|\, #1\, \right\|_{#2}} %ノルム
\def\Log#1{\operatorname{log} #1} %log
\def\Det#1{\operatorname{det} ( #1 )} %行列式
\def\Diag#1{\operatorname{diag} \left( #1 \right)} %行列の対角成分
\def\Tmat#1{#1^\mathrm{T}} %転置行列
\def\Exp#1{\operatorname{E} \left[ #1 \right]} %期待値
\def\Var#1{\operatorname{V} \left[ #1 \right]} %分散
\def\Cov#1#2{\operatorname{Cov} \left[ #1,\ #2 \right]} %共分散
\def\exp#1{e^{#1}} %指数関数
\def\N{\mathbb{N}} %自然数全体
\def\Z{\mathbb{Z}} %整数全体
\def\Q{\mathbb{Q}} %有理数全体
\def\R{\mathbb{R}} %実数全体
\def\C{\mathbb{C}} %複素数全体
\def\K{\mathbb{K}} %係数体K
\def\conj#1{\overline{#1}} %共役複素数
\def\Re#1{\mathfrak{Re}#1} %実部
\def\Im#1{\mathfrak{Im}#1} %虚部
\def\Conv#1{\mathrm{Conv}\left[\left\{\, #1 \, \right\}\right]} %凸包
\def\borel#1{\mathfrak{B}(#1)} %Borel集合族
\def\open#1{\mathfrak{O}(#1)} %位相空間 #1 の位相
\def\close#1{\mathfrak{A}(#1)} %%位相空間 #1 の閉集合系
\def\closure#1{\overline{#1}}
%\def\equiv#1#2{\left[#1\right]_{#2}} %同値類
\def\rapid#1{\mathfrak{S}(#1)} %急減少空間
\def\c#1{C(#1)} %有界実連続関数
\def\cbound#1{C_{b} (#1)} %有界実連続関数
\def\semiLp#1#2{\mathscr{L}^{#1} \left(#2\right)} %ノルム空間L^p
\def\Lp#1#2{\operatorname{L}^{#1} \left(#2\right)} %ノルム空間L^p
\def\cinf#1{C^{\infty} (#1)} %無限回連続微分可能関数
\def\sgmalg#1{\sigma \left[#1\right]} %#1が生成するσ加法族
\def\ball#1#2{\operatorname{B} \left(#1\, ;\, #2 \right)} %開球
\def\prob#1{\operatorname{P} \left(#1\right)} %確率
\def\cprob#1#2{\operatorname{P} \left(\left\{ #1 \ \middle|\ #2 \right\}\right)} %条件付確率
\def\cexp#1#2{\operatorname{E} \left[ #1 \ \middle|\ #2 \right]} %条件付期待値
\def\tExp#1{\tilde{\operatorname{E}} \left[ #1 \right]} %拡張期待値
\def\tcexp#1#2{\tilde{\operatorname{E}} \left[ #1 \ \middle|\ #2 \right]} %拡張条件付期待値
%\renewcommand{\contentsname}{\bm Index}
%
\makeindex
%
\setlength{\textwidth}{\fullwidth}
\setlength{\textheight}{40\baselineskip}
\addtolength{\textheight}{\topskip}
\setlength{\voffset}{-0.2in}
\setlength{\topmargin}{0pt}
\setlength{\headheight}{0pt}
\setlength{\headsep}{0pt}
%
\title{ゼミ用ノート\\会田先生の資料''Rough path analysis:An Introduction''}
\author{基礎工学研究科システム創成専攻\\学籍番号29C17095\\百合川尚学}
\date{\today}

\begin{document}
%
%

\mathtoolsset{showonlyrefs = true}
\maketitle

\newpage
\tableofcontents

	\section{導入}
	以下,$x \in \R^d$について成分を込めて表現する場合は
	$x = (x^1,\cdots,x^d)$と書き,
	実$m \times d$行列$a$については$a=(a^i_j)_{1 \leq i \leq m,1 \leq j \leq d}$と表す.
	また$T > 0$を固定し$C^1 = C^1([0,T] \rightarrow \R^d)$とおく.
	ただし端点においては片側微分を考える.
	区間$[s,t] \subset [0,T]$の分割を
	$D = \{s = t_0 < t_1 < \cdots < t_N = t\}$で表現し
	$|D| \coloneqq \max{1 \leq i \leq N}{\left| t_i - t_{i-1} \right|}$とおく.
	また$[s,t]$の分割の全体を$\delta[s,t]$と書く.

	\begin{screen}
		\begin{thm}[Riemann-Stieltjes積分]
			$[s,t] \subset [0,T]$とし,$D \in \delta[s,t]$についてのみ考えるとき,
			任意の$x \in C^1,\ f \in C(\R^d,L(\R^d \rightarrow \R^m))$\footnotemark
			に対して
			次の極限が確定する:
			\begin{align}
				\lim_{|D| \to 0}
				\sum_{D} f(x_{s_{i-1}})(x_{t_i} - x_{t_{i-1}})
				\in \R^m.
			\end{align}
			ここで$s_{i-1}$は区間$[t_{i-1},t_i]$に属する任意の点である.
			極限は$s_{i-1}$の取り方にも依存しない.
		\end{thm}
	\end{screen}
	\footnotetext{
		極限の存在を保証する条件としては,$f$の有界性と微分可能性は必要ない.
	}
	\begin{prf}
		各$x^j$は$C^1$-級であるから,平均値の定理より
		$\sum_{D} f(x_{s_{i-1}})(x_{t_i} - x_{t_{i-1}})$
		の第$k$成分を
		\begin{align}
			\sum_{j=1}^{d} \sum_{D} f^k_j (x_{s_{i-1}})(x^j_{t_i} - x^j_{t_{i-1}})
			= \sum_{j=1}^{d} \sum_{D} f^k_j (x_{s_{i-1}}) \frac{d}{dt}x^j(\xi_{i-1,j})(t_i - t_{i-1}),
			\quad ({}^\exists \xi_{i-1,j} \in [t_{i-1},t_i])
		\end{align}
		と表現できる.各$j,k$について
		\begin{align}
			\lim_{|D| \to 0} \sum_{D} f^k_j (x_{s_{i-1}}) \frac{d}{dt}x^j(\xi_{i-1,j})(t_i - t_{i-1})
		\end{align}
		が確定すれば,第$k$成分の極限が確定し定理の主張を得る.
		いま,$t \longrightarrow f^k_j(x_t)$及び$t \longmapsto (d/dt)x^j_t$は([s,t]上一様)連続であるから,
		分割$D$による各区間$[t_{i-1},t_i]$において次の最大最小値が定まる:
		\begin{align}
			M_{i-1} \coloneqq \sup{t_{i-1} \leq t \leq t_i} f^k_j(x_t)\frac{d}{dt}x^j_t,
			\quad m_{i-1} \coloneqq \inf{t_{i-1} \leq t \leq t_i} f^k_j(x_t)\frac{d}{dt}x^j_t.
		\end{align}
		ここで
		\begin{align}
			S_D \coloneqq \sum_{D} M_{i-1}(t_i - t_{i-1}),
			\quad s_D \coloneqq \sum_{D} m_{i-1}(t_i - t_{i-1}),
			\quad \Sigma_D \coloneqq \sum_{D} f^k_j (x_{s_{i-1}}) \frac{d}{dt}x^j(\xi_{i-1})(t_i - t_{i-1})
		\end{align}
		とおいて
		\begin{align}
			S \coloneqq \inf{D \in \delta[s,t]}{S_D},
			\quad s \coloneqq \sup{D \in \delta[s,t]}{s_D}
		\end{align}
		を定めれば
		\begin{align}
			s_D \leq s \leq S \leq S_D,
			\quad s_D \leq \Sigma_D \leq S_D
		\end{align}
		が満たされる.実際,任意の$D_1,D_2 \in \delta[s,t]$に対して,
		分割の合併を$D_3$とすれば
		\begin{align}
			s_{D_1} \leq s_{D_3} \leq S_{D_3} \leq S_{D_2}
		\end{align}
		が成立し$s \leq S_D\ (\forall D \in \delta[s,t])$すなわち$s \leq S$が出る.
		一方で一様連続性から
		\begin{align}
			0 \leq S - s \leq S_D - s_D = \sum_D (M_{i-1} - m_{i-1})(t_i - t_{i-1})
			\longrightarrow 0
			\quad (|D| \longrightarrow 0)
		\end{align}
		が従い$s = S$を得る.以上より
		\begin{align}
			|S - \Sigma_D| \leq |S - S_D| + |S_D - \Sigma_D|
			\leq |S - S_D| + |S_D - s_D|
			\longrightarrow 0
			\quad (|D| \longrightarrow 0)
		\end{align}
		が成り立つ.
		\QED
\end{prf}

\begin{screen}
	\begin{dfn}
		$I_{0,T}(x)$の定義.
	\end{dfn}
\end{screen}

$C^1$において次でノルム$\Norm{\cdot}{C^1}$を定める:
\begin{align}
	\Norm{x}{\infty} &\coloneqq \sup{t \in [0,T]}{|x(t)|},
	\quad \Norm{x'}{\infty} \coloneqq \sup{t \in [0,T]}{|x'(t)|}, \\
	\Norm{x}{C^1} &\coloneqq
	\Norm{x}{\infty} + \Norm{x}{\infty}.
\end{align}

\begin{screen}
	\begin{thm}
		$x \longmapsto I_{0,T}(x)$は連続である.
	\end{thm}
\end{screen}

\begin{prf}
	$C^1$は次のノルムでBanach空間になる.ゆえに
	$x \longmapsto I_{0,T}(x)$は距離空間から距離空間への対応であり
	点列連続性と連続性が一致するから,
	$x_n \longrightarrow x$なら$I_{0,T}(x_n) \longrightarrow I_{0,T}(x)$
	が従うことを示せばよい.実際,$f,x$の連続性より
	\begin{align}
		\sum_{i=1}^{N} f(x^{(n)}_{s_{i-1}})(x^{(n)}_{t_i} - x^{(n)}_{t_{i-1}})
		\longrightarrow \sum_{i=1}^{N} f(x_{s_{i-1}})(x_{t_i} - x_{t_{i-1}})
	\end{align}
	が成り立つ.
\end{prf}

\begin{screen}
	\begin{dfn}[$p$-variation]
		$[0,T]$上の$\R^d$値関数$x$に対し$p$-variationノルムを次で定める:
		\begin{align}
			\Norm{x}{p}
			\coloneqq \left\{ \sup{D}{\sum_{i=1}^{N} 
				\left| x_{t_i} - x_{t_{i-1}} \right|^p }\right\}^{1/p}.
		\end{align}
		また線形空間$B_{p,T}(\R^d)$を
		\begin{align}
			B_{p,T}(\R^d)
			\coloneqq \Set{x:[0,T] \longrightarrow \R^d}{x_0=0,\ x:\mbox{continuous},\ \Norm{x}{p} < \infty}
		\end{align}
		により定める.
	\end{dfn}
\end{screen}

\begin{screen}
	\begin{thm}
		$\tilde{C}^1 \coloneqq \Set{x \in C^1}{x_0 = 0}$とおくと,
		$\tilde{C}^1 \subset B_{p,T}(\R^d)$が成り立つ.
	\end{thm}
\end{screen}

\begin{prf}
	$x \in \tilde{C}^1$に対して
	\begin{align}
		M \coloneqq \sum_{j=1}^{d} \sup{x \in [0,T]}{|x^j(t)|}
	\end{align}
	とおけば,$x'$の連続性より$M < \infty$が定まる.
	平均値の定理より,$|D| < 1$を満たす分割$D$に対して
	\begin{align}
		\left\{ \sum_{i=1}^{N} |x_{t_i} - x_{t_{i-1}}|^p \right\}^{1/p}
		\leq \left\{ \sum_{i=1}^{N} \Norm{x}{C^1}^p (t_i - t_{i-1})^p \right\}^{1/p}
		\leq M T < \infty
	\end{align}
	が成立し$\Norm{x}{p} \leq MT < \infty$が従う
	\footnote{
		$S_D \geq 0$ならば$(\sup{D}S_D)^{1/p} = \sup{D}S_D^{1/p}$が成り立つ.
	}
	.
\end{prf}

\begin{screen}
	\begin{thm}
		$B_{p,T}(\R^d)$はBanach空間である.
	\end{thm}
\end{screen}

\begin{prf}
	$(x^n)_{n=1}^{\infty} \subset B_{p,T}(\R^d)$をCauchy列とする.
	つまり任意の$\epsilon > 0$に対して或る$n_\epsilon \in \N$が存在し
	\begin{align}
		\Norm{x^n - x^m}{p}
		= \left\{ \sup{D}{\sum_{i=1}^{N} 
		\left| \left( x^n_{t_i} - x^m_{t_i} \right) 
		- \left(x^n_{t_{i-1}} - x^m_{t_{i-1}} \right) \right|^p }\right\}^{1/p} < \epsilon,
		\quad (n,m > n_\epsilon)
	\end{align}
	を満たす.いま,任意の$t \in [0,T]$に対して$[0,T]$の分割$\{0=t_0 \leq t \leq T\}$
	を考えれば
	\begin{align}
		|x^n_t - x^m_t| < \epsilon,
		\quad (n,m > n_\epsilon)
	\end{align}
	が得られ,実数の完備性より或る$x_t \in \R^d$が存在して
	\begin{align}
		|x^n_t - x_t| < \epsilon
		\quad (n > n_\epsilon)
	\end{align}
	を満たす.実際,もし或る$n > n_\epsilon$で$|x^n_t - x_t| \eqqcolon \alpha \geq \epsilon$
	が成り立つと,任意の$m > n_\epsilon$に対して
	\begin{align}
		|x^m_t - x_t| \geq |x^n_t - x_t| - |x^n_t - x^m_t| > \alpha - \epsilon
	\end{align}
	が従い$x^m_t \longrightarrow x_t$に反する.
	ゆえに収束は$t$に関して一様であり,$t \longmapsto x_t$は0出発かつ連続である.
	あとは$\Norm{x^n - x}{p} \longrightarrow 0\ (n \longrightarrow \infty)$であればよい.
\end{prf}

\begin{screen}
	\begin{thm}
		$C^1$空間において,$\Norm{}{C^1}$で定まる位相は
		$\Norm{}{p}$で定まる位相より強い.
		特に,$B_{p,T}(\R^d)$上で考える写像$x \longmapsto I_{p,T}(x)$は
		$\Norm{\cdot}{p}$の定める位相により連続である.
	\end{thm}
\end{screen}

\begin{prf}
	任意の$x \in \tilde{C}^1$に対して
	\begin{align}
		\Norm{x}{p} \leq T \Norm{x}{C^1}
	\end{align}
	を満たすことを証明する.実際,任意の分割$D$に対して
	\begin{align}
		\sum_{i=1}^{N} |x_{t_i} - x_{t_{i-1}}|
		\leq \sum_{i=1}^{N} \Norm{x}{C^1} (t_i - t_{i-1})
		= T \Norm{x}{C^1}
	\end{align}
	が成り立つ.
	\QED
\end{prf}
	$C^1$パス$[0,T] \longrightarrow \R^d$に対し$f(x) = (f_j^i(x))$の積分を
\begin{align}
	I_{s,t} (x)
	\coloneqq \int_s^t f(x_u)\ dx_u
	= \left( \sum_{j=1}^d \int_s^t f_j^i(x_u)\ dx^j_u \right)_i
\end{align}
により定める.
このとき次が成り立つ
\begin{align}
	I_{s,t}(x)_i
	&= \sum_{j=1}^d \int_s^t f_j^i(x_u)\ dx_u \\
	&= \sum_{j=1}^d \int_s^t f_j^i(x_s) + f_j^i(x_u) - f_j^i(x_s)\ dx_u \\
	&= \sum_{j=1}^d \int_s^t f_j^i(x_s) 
		+ \sum_{k=1}^d  \left\{ \int_0^1 \partial_k f_j^i(x_s + \theta(x_u - x_s))\ d\theta \right\} (x^k_u - x^k_s)\ dx_u \\
	&= \sum_{j=1}^d \int_s^t f_j^i(x_s) + \sum_{k=1}^d  \left\{ \int_0^1 \partial_k f_j^i(x_s)\ d\theta \right\} (x^k_u - x^k_s)\ dx_u \\
		&\quad + \sum_{j=1}^d \int_s^t \sum_{k=1}^d  \left\{ \int_0^1 \partial_k f_j^i(x_s + \theta(x_u - x_s)) - \partial_k f_j^i(x_s)\ d\theta \right\} (x^k_u - x^k_s)\ dx_u \\
\end{align}
	\section{連続性定理}
	\begin{screen}
		\begin{dfn}[記号の定義]
			$x \in C^1,\ f \in C^2(\R^d,L(\R^d \rightarrow \R^m))$に対し次を定める.
			\begin{align}
				\Delta_T &\coloneqq \Set{(s,t)}{0 \leq s \leq t \leq T}, \\
				X^1 &:\Delta_T \longrightarrow \R^d\ \left( (s,t) \longmapsto X^1_{s,t} = x_t - x_s \right), \\
				X^2 &:\Delta_T \longrightarrow \R^d \otimes \R^d\ \left( (s,t) \longmapsto X^2_{s,t} = \int_s^t (x_u - x_s) \otimes dx_u \right), \\
				\tilde{I}_{s,t}(x) &\coloneqq f(x_s)X^1_{s,t} = f(x_s)(x_t - x_s), \\
				J_{s,t}(x) &\coloneqq f(x_s)X^1_{s,t} + (\nabla f)(x_s)X^2_{s,t}.
			\end{align}
		\end{dfn}
	\end{screen}
	
	以降,$a,b,c,d \in \R^d$に対して次の表現を使う:
	\begin{align}
		[a \otimes b]^i_j &= a^i b^j, \\
		\left[ (\nabla f)(x_s)X^2_{s,t} \right]^i &= \sum_{j,k=1}^d \partial_k f^i_j(x_s) \int_s^t \left(x^k_u - x^k_s \right)\ dx^j_u,\\
		\left[ (\nabla f)(x_s)(a \otimes b) \right]^i &= \sum_{j,k=1}^d \partial_k f^i_j(x_s) a^k b^j,\\
		\left[ (\nabla^2 f)(x_s)(a \otimes b \otimes c) \right]^i &= \sum_{j,k,v=1}^d \partial_v \partial_k f^i_j(x_s) a^v b^k c^j,\\
		\left[ (\nabla^3 f)(x_s)(a \otimes b \otimes c \otimes d) \right]^i &= \sum_{j,k,v,w=1}^d \partial_w \partial_v \partial_k f^i_j(x_s) a^w b^v c^k d^j.
	\end{align}
	
	\begin{screen}
		\begin{thm}\label{thm:Riemann_Stieltjes_approximation}
			$[s,t] \subset [0,T],\ x \in C^1,\ f \in C^2(\R^d,L(\R^d \rightarrow \R^m))$とする.$D \in \delta[s,t]$に対し
			\begin{align}
				\tilde{I}_{s,t}(x,D) \coloneqq \sum_D \tilde{I}_{t_{i-1},t_i}(x),
				\quad J_{s,t}(x,D) \coloneqq \sum_D J_{t_{i-1},t_i}(x)
			\end{align}
			を定めるとき,次が成立する:
			\begin{align}
				I_{s,t}(x) = \lim_{|D| \to 0} \tilde{I}_{s,t}(x,D)
				= \lim_{|D| \to 0} J_{s,t}(x,D).
			\end{align}
		\end{thm}
	\end{screen}
	
	\begin{prf}
		第一の等号は$I_{s,t}(x)$の定義によるから,第二の等号を証明する.まず,
		\begin{align}
			I_{s,t}(x)
			&= \int_s^t f(x_u)\ dx_u \\
			&= \int_s^t f(x_s) + f(x_u) - f(x_s)\ dx_u \\
			&= \int_s^t f(x_s)\ dx_u
				+ \int_s^t \int_0^1 (\nabla f)(x_s + \theta(x_u - x_s)) \left( X^1_{s,u} \otimes \dot{x}_u \right)\ d\theta\ du \\
			&= f(x_s)X^1_{s,t} + (\nabla f)(x_s) X^2_{s,t} \\
				&\quad+ \int_s^t \int_0^1 \left\{ (\nabla f)(x_s + \theta(x_u - x_s)) - (\nabla f)(x_s) \right\}\left( X^1_{s,u} \otimes \dot{x}_u \right)\ d\theta\ du \\
			&= J_{s,t}(x)
				+ \int_s^t 
				\int_0^1 \int_0^\theta (\nabla^2 f)(x_s + r(x_u - x_s))\left( X^1_{s,u} \otimes X^1_{s,u} \otimes \dot{x}_u \right)\ dr\ d\theta\ du
		\end{align}
		が成り立つ.$[0,T] \ni t \longmapsto x_t$の連続性より,
		最下段式中の$x_s + r(x_u - x_s)\ (0 \leq r \leq 1,\ s \leq u \leq t)$は或るコンパクト集合$K$に含まれ,
		$f$が$C^2$-級関数であるから
		\begin{align}
			M \coloneqq \sum_{i,j,k,v} \sup{x \in K}{\left|\partial_v \partial_k f_j^i(x) \right|}
		\end{align}
		として$M < \infty$を定めれば
		\begin{align}
			&\left| \int_s^t 
				\int_0^1 \int_0^\theta (\nabla^2 f)(x_s + r(x_u - x_s))\left( X^1_{s,u} \otimes X^1_{s,u} \otimes \dot{x}_u \right)\ dr\ d\theta\ du \right| \\
			&\qquad \leq \int_s^t 
				\int_0^1 \int_0^\theta \left| (\nabla^2 f)(x_s + r(x_u - x_s))\left( X^1_{s,u} \otimes X^1_{s,u} \otimes \dot{x}_u \right) \right|\ dr\ d\theta\ du \\
			&\qquad \leq M \int_s^t |X^1_{s,u}|^2 |\dot{x}_u|\ du \\
			&\qquad \leq M \Norm{x}{C^1}^3 \int_s^t (u - s)^2\ du
		\end{align}
		が出る.特に$D \in \delta[s,t]$に対して
		\begin{align}
			&\sum_D \int_{t_{i-1}}^{t_i} (u - t_{i-1})^2\ du
			\leq \sum_D |D| \int_{t_{i-1}}^{t_i} (u - t_{i-1})\ du \\
			&\qquad \leq \sum_D |D| \int_{t_{i-1}}^{t_i} (u - s)\ du
			\leq \frac{1}{2}(t-s)^2 |D|
			\longrightarrow 0 \quad (|D| \longrightarrow 0)
		\end{align}
		が成立するから,
		\begin{align}
			\left| I_{s,t}(x) - J_{s,t}(x,D) \right|
			\leq \sum_D \left| I_{t_{i-1},t_i}(x) - J_{t_{i-1},t_i}(x) \right| \longrightarrow 0 \quad (|D| \longrightarrow 0)
		\end{align}
		が従い定理の主張を得る.
		\QED
	\end{prf}
	
	\begin{screen}
		\begin{dfn}[control function]
			関数$\omega:\Delta_T \longrightarrow [0,\infty)$
			が連続かつ任意の$s \leq u \leq t$に対して
			\begin{align}
				\omega(s,u) + \omega(u,t) \leq \omega(s,t)
				\label{eq:control_function_subadditivity}
			\end{align}
			を満たすとき,$\omega$をcontrol functionと呼ぶ.
		\end{dfn}
	\end{screen}
	
	式(\refeq{eq:control_function_subadditivity})から$\omega(t,t)=0\ (\forall t \in [0,T])$が従う.
	つまりcontrol functionは''対角線上で0になる''.
	
	\begin{screen}
		\begin{dfn}[ノルム空間値写像の$p$-variation]
			$(V,\Norm{\cdot}{})$をノルム空間,$p \geq 1$とする.
			このとき連続写像$\psi:\Delta_T \longrightarrow V$に対する$p$-variationを
			\begin{align}
				\Norm{\psi}{p,[s,t]}
				\coloneqq \left\{ \sup{D \in \delta[s,t]}{ \sum_D \Norm{\psi_{t_{i-1},t_i}}{}^p} \right\}^{1/p},
				\quad ((s,t) \subset [0,T])
			\end{align}
			で定める.特に$\Norm{\cdot}{p,[0,T]}$を$\Norm{\cdot}{p}$と書く.
		\end{dfn}
	\end{screen}
	
	\begin{screen}
		\begin{thm}[$p$-variationが定めるcontrol function]
			$(V,\Norm{\cdot}{})$をノルム空間,$p \geq 1$とする.
			$\Norm{\psi}{p} < \infty$かつ$\psi_{t,t} = 0\ (\forall t \in [0,T])$を満たす連続写像$\psi:\Delta_T \longrightarrow V$に対して,
			\begin{align}
				\omega:\Delta_T \ni (s,t) \longmapsto \Norm{\psi}{p,[s,t]}^p
			\end{align}
			により定める$\omega$はcontrol functionである.
		\end{thm}
	\end{screen}
	
	\begin{prf}
		$\Norm{\psi}{p} < \infty$の仮定より$\omega$は$[0,\infty)$値であるから,
		以下では式(\refeq{eq:control_function_subadditivity})の成立と連続性を示す.
		\begin{description}
			\item[第一段]
				$\omega$が式(\refeq{eq:control_function_subadditivity})を満たすことを示す.実際,
				任意に$D_1 \in \delta[s,u],D_2 \in \delta[u,t]$を取れば
				\begin{align}
					\sum_{D_1}\Norm{\psi_{t_{i-1},t_i}}{}^p
					+ \sum_{D_2}\Norm{\psi_{t_{i-1},t_i}}{}^p
					= \sum_{D_1 \cup D_2}\Norm{\psi_{t_{i-1},t_i}}{}^p
					\leq \Norm{\psi}{p:[s,t]}^p
				\end{align}
				が成り立つ.左辺の$D_1,D_2$の取り方は独立であるから,それぞれに対し上限を取れば
				\begin{align}
					\Norm{\psi}{p:[s,u]}^p + \Norm{\psi}{p:[u,t]}^p
					\leq \Norm{\psi}{p:[s,t]}^p
				\end{align}
				が従う.
			\item[第二段]
				任意の$[s,t] \subset [0,T]$について
				\footnote{
					下段の二式については$s < t$と仮定する.また
					上段についても,$t=T$或は$s=0$の場合を除く.
				},
				\begin{align}
					\lim_{h \to +0} \omega(s,t+h) &= \inf{h>0}{\omega(s,t+h)},
					&\lim_{h \to +0} \omega(s-h,t) &= \inf{h>0}{\omega(s-h,t)}, \\
					\lim_{h \to +0} \omega(s,t-h) &= \sup{h>0}{\omega(s,t-h)},
					&\lim_{h \to +0} \omega(s+h,t) &= \sup{h>0}{\omega(s+h,t)}
				\end{align}
				が成立する.実際$\omega(s,t+h)$について見れば,これは下に有界かつ
				$h \to +0$に対し単調減少であるから極限が確定し下限に一致する.
				残りの三つも同様の理由で成立する.
				
			\item[第三段]
				任意の$s \in [0,T)$に対し,$(s,T] \ni t \longmapsto \omega(s,t)$
				の左連続性を示す.ここでは
				\begin{align}
					\tilde{\omega}(s,t) \coloneqq 
					\begin{cases}
						\lim_{h \to +0}\omega(s,t-h), & (s < t), \\
						0, & (s=t),
					\end{cases}
					\quad (\forall (s,t) \in \Delta_T)
				\end{align}
				で定める$\tilde{\omega}$が優加法性を持ち,かつ
				\begin{align}
					\Norm{\psi_{s,t}}{}^p \leq \tilde{\omega}(s,t),
					\quad (\forall (s,t) \in \Delta_T)
				\end{align}
				を満たすことを示す.
				実際これが示されれば,任意の$D \in \delta[s,t]$に対し
				\begin{align}
					\sum_D \Norm{\psi_{t_{i-1},t_i}}{}^p
					\leq \sum_D \tilde{\omega}(t_{i-1},t_i)
					\leq \tilde{\omega}(s,t)
				\end{align}
				が成立し$\omega(s,t) \leq \tilde{\omega}(s,t)$が従い,
				$\omega(s,t) \geq \omega(s,t-h)\ (\forall h > 0)$と併せて
				\begin{align}
					\omega(s,t) = \tilde{\omega}(s,t) = \lim_{h \to +0} \omega(s,t-h)
				\end{align}
				を得る.いま,任意に$s < u < t$を取れば,十分小さい$h_1,h_2 > 0$に対して
				\begin{align}	
					\omega(s,u-h_1) + \omega(u,t-h_2) \leq \omega(s,t-h_2)
				\end{align}
				が満たされ,$h_1 \longrightarrow +0,\ h_2 \longrightarrow +0$として
				\begin{align}
					\tilde{\omega}(s,u) + \tilde{\omega}(u,t) \leq \tilde{\omega}(s,t)
				\end{align}
				が成り立ち$\tilde{\omega}$は優加法性を持つ.また,もし
				或る$[u,v] \subset [0,T]$に対して
				\begin{align}
					\Norm{\psi_{u,v}}{}^p > \tilde{\omega}(u,v)
				\end{align}
				が成り立つと仮定すると
				\begin{align}
					\Norm{\psi_{u,v}}{}^p > \tilde{\omega}(u,v) \geq \omega(u,v-h) \geq \Norm{\psi_{u,v-h}}{}^p,
					\quad (\forall h > 0)
				\end{align}
				となる.一方$\psi$の連続性より
				$\Norm{\psi_{u,v-h}}{}^p \longrightarrow \Norm{\psi_{u,v}}{}^p\ (h \longrightarrow +0)$が従い矛盾が生じる.
				同様にして,任意の$t \in (0,T]$に対し$[0,t) \ni s \longmapsto \omega(s,t)$
				の右連続性も出る.
				
			\item[第二段]
				任意の$t \in [0,T)$に対して次を示す:
				\begin{align}
					\lim_{h \to +0} \omega(t,t+h) = \inf{h>0}{\omega(t,t+h)} = 0.
				\end{align}
				第一の等号は前段より従うから,第二の等号を背理法により証明する.いま
				\begin{align}
					\inf{h>0}{\omega(t,t+h)} \eqqcolon \delta > 0
					\label{eq:thm_continuity_of_norm_val_p_variation_2}
				\end{align}
				と仮定する.$\psi$の連続性より或る$h_1$が存在して
				\begin{align}
					\Norm{\psi_{t,t+h}}{}^p
					 = \Norm{\psi_{t,t+h} - \psi_{t,t}}{}^p
					 < \frac{\delta}{8},
					\quad (\forall h < h_1)
					\label{eq:thm_continuity_of_norm_val_p_variation_1}
				\end{align}
				が成立するから,任意に$h_0 < h_1$を取り固定する.
				一方で$\omega(t,t+h_0) \geq \delta$より
				\begin{align}
					\sum_{i=1}^{N} \Norm{\psi_{\tau_{i-1},\tau_i}}{}^p > \frac{7\delta}{8}
				\end{align}
				を満たす$D = \{t = \tau_0 < \tau_1 < \cdots, \tau_N = t+h_0\} \in \delta[t,t+h_0]$が存在し,
				(\refeq{eq:thm_continuity_of_norm_val_p_variation_1})と併せて
				\begin{align}
					\sum_{i=2}^{N} \Norm{\psi_{\tau_{i-1},\tau_i}}{}^p
					> \frac{7\delta}{8} - \Norm{\psi_{t,\tau_1}}{}^p
					>\frac{7\delta}{8} - \frac{\delta}{8}
					= \frac{3 \delta}{4}
				\end{align}
				を得る.また,$\omega(t,\tau_1) \geq \delta$より或る$D' \in \delta[t,\tau_1]$が存在して
				\begin{align}
					\sum_{D'} \Norm{\psi_{t_{i-1},t_i}}{}^p > \frac{3 \delta}{4}
				\end{align}
				を満たすから,$D' \cup \{\tau_1 < \cdots, \tau_N = t+h_0\} \in \delta[t,t+h_0]$より
				\begin{align}
					\omega(t,t+h_0) > \sum_{D'} \Norm{\psi_{t_{i-1},t_i}}{}^p + \sum_{i=2}^{N} \Norm{\psi_{\tau_{i-1},\tau_i}}{}^p
					> \frac{3\delta}{2}
				\end{align}
				が従うが,$h_0 < h_1$の任意性と単調減少性により
				\begin{align}
					\delta = \inf{h>0}{\omega(t,t+h)} = \inf{h_1>h>0}{\omega(t,t+h)} \geq \frac{3\delta}{2}
				\end{align}
				となり矛盾が生じる.
				同様にして
				\begin{align}
					\lim_{h \to +0} \omega(t-h,t) = 0,
					\quad (\forall t \in (0,T])
				\end{align}
				も成立する.
				
			\item[第三段]
				任意に$s \in [0,T)$を取り固定し,
				$[s,T) \ni t \longmapsto \omega(s,t)$が右連続であることを示す.
				\begin{align}
					\lim_{h \to +0} \omega(s,t+h) \leq \omega(s,t)
					\label{eq:thm_continuity_of_norm_val_p_variation_3}
				\end{align}
				を示せば,第二段より逆向きの不等号も従い右連続性を得る.
				任意に$h,\epsilon > 0$を取れば,
				\begin{align}
					\omega(s,t + h) - \epsilon
					\leq \sum_D \Norm{\psi_{t_{i-1},t_i}}{}^p
				\end{align}
				を満たす$D \in \delta[s,t+h]$が存在する.
				$D_1 \coloneqq [s,t] \cap D,\ D_2 \coloneqq D \backslash D_1$とおいて
				$D_2$の最小元を
				\begin{align}
					\omega(s,t + h) - \epsilon
					\leq \sum_{D_1} \Norm{\psi_{t_{i-1},t_i}}{}^p + \sum_{D_2} \Norm{\psi_{t_{i-1},t_i}}{}^p
					\leq \omega(s,t) + \omega(t,t+h)
				\end{align}
				が成り立つ.$h \longrightarrow +0$として
				\begin{align}
					\lim_{h \to +0} \omega(s,t+h) - \epsilon \leq \omega(s,t)
				\end{align}
				が従い,$\epsilon$の任意性より(\refeq{eq:thm_continuity_of_norm_val_p_variation_3})が出る.
				同様にして$s \longmapsto \omega(s,t)$の左連続性も成立する.
				
			\item[第四段]
				$\Delta_T \ni (s,t) \longmapsto \omega(s,t)$の連続性を示す.
		\end{description}
	\end{prf}
	
	\begin{screen}
		\begin{thm}[control functionの例]\label{thm:examples_of_control_functions}
			以下の関数$\omega:\Delta_T \longrightarrow [0,\infty)$はcontrol functionである.
			\begin{description}
				\item[(1)] $\omega \coloneqq \left( \omega_1^r + \omega_2^r \right)^{1/r},
					\quad (0 < r \leq 1,\ \omega_1,\omega_2:\mbox{control function}).$
				\item[(2)] $\omega:(s,t) \longmapsto \Norm{X^1}{p:[s,t]}^p,
					\quad (p \geq 1,\ x \in B_{p,T}(\R^d)).$
				\item[(3)] $\omega:(s,t) \longmapsto \Norm{X^2}{p:[s,t]}^p,
					\quad (p \geq 1,\ x \in C^1).$
			\end{description}
		\end{thm}
	\end{screen}
	
	行列$a = (a_j^i)$のノルムは$|a| = \sqrt{\sum_{i,j}|a_j^i|^2}$として考える.
	
	\begin{thm}\mbox{}
		\begin{description}
			\item[(1)]
			\item[(2)] 
				
			\item[(3)] 任意の$[s,t] \subset [0,T]$に対して
				$\Norm{X^2}{p:[s,t]}^p < \infty$を示せば,あとは
				上と同じ理由により定理の主張が得られる.
				実際,任意の分割$D = \{s=t_0 < \cdots < t_N = t\}$に対し
				\begin{align}
					\Norm{X^2_{t_{i-1},t_i}}{}
					&\leq \left| \int_{t_{i-1}}^{t_i} (x_u - x_{t_{i-1}}) \otimes \dot{x}_u\ du \right| \\
					&\leq \int_{t_{i-1}}^{t_i} \left| (x_u - x_{t_{i-1}}) \otimes \dot{x}_u \right|\ du \\
					&\leq \Norm{x}{C^1}^2 \left\{ \int_{t_{i-1}}^{t_i} (u-s)\ du \right\}^{1/p}
						\left\{ \int_{t_{i-1}}^{t_i} (u-s)\ du \right\}^{1-1/p} \\
					&\leq \Norm{x}{C^1}^2 \left\{ \int_{t_{i-1}}^{t_i} (u-s)\ du \right\}^{1/p}
						\left\{ \int_s^t (u-s)\ du \right\}^{1-1/p}
				\end{align}
				が成り立つから,
				\begin{align}
					&\sum_D \Norm{X^2_{t_{i-1},t_i}}{}^p
					\leq \sum_D \Norm{x}{C^1}^{2p} \left\{ \frac{1}{2}(t-s)^2 \right\}^{p-1}
						\int_{t_{i-1}}^{t_i} (u-s)\ du \\
					&\qquad= \Norm{x}{C^1}^{2p} \left\{ \frac{1}{2}(t-s)^2 \right\}^{p-1}
						\int_s^t (u-s)\ du
					= \Norm{x}{C^1}^{2p} \left\{ \frac{1}{2}(t-s)^2 \right\}^p
				\end{align}
				により$\Norm{X^2}{p:[s,t]}^p < \infty$が従う.
				\QED
		\end{description}
	\end{thm}
	
	\begin{prf}
		\begin{align}
			&\omega_1(s,u)\omega_2(s,u) + \omega_1(u,t)\omega_2(u,t) \\
			&= \left\{ \omega_1(s,u) + \omega_1(u,t) \right\}\omega_2(s,u) + \omega_1(u,t)\left\{\omega_2(u,t) - \omega_2(s,u)\right\} \\
			&\leq \omega_1(s,t)\omega_2(s,u) + \omega_1(u,t)\left\{\omega_2(u,t) - \omega_2(s,u)\right\} \\
			&\leq \omega_1(s,t)\omega_2(s,u) + \omega_1(u,t)\omega_2(u,t) \\
			&\leq \omega_1(s,t)\left\{ \omega_2(s,u) + \omega_2(u,t)\right\} \\
			&\leq \omega_1(s,t)\omega_2(s,t)
		\end{align}
	\end{prf}
	
	\begin{screen}
		\begin{lem}\label{lem:control_function_min}
			$\omega$を$\Delta_T$上のcontrol functionとする.
			$D = \{s = t_0 < t_1 < \cdots < t_N= t\}$について,$N \geq 2$の場合
			或る$1 \leq i \leq N-1$が存在して次を満たす:
			\begin{align}
				\omega(t_{i-1},t_{i+1})
				\leq \frac{2 \omega(s,t)}{N-1}.
				\label{eq:lem_control_function_min}
			\end{align}
		\end{lem}
	\end{screen}
	
	\begin{prf}
		\QED
	\end{prf}
	
	\begin{screen}
		\begin{thm}[$1 \leq p < 2$の場合の連続性定理]\label{thm:continuity_theorem_1}
			$1 \leq p < 2$とし,
			$x_0 = y_0$を満たす$x,y \in C^1$と$f \in C^2_b(\R^d,L(\R^d \rightarrow \R^m)),\ 0 < \epsilon, R < \infty$を任意に取る.
			このとき,
			\begin{align}
				\Norm{X^1}{p},\Norm{Y^1}{p} \leq R,
				\quad \Norm{X^1 - Y^1}{p} \leq \epsilon
			\end{align}
			なら,或る定数$C = C(p,R,f)$が存在し,任意の$0 \leq s \leq t \leq T$に対して次が成立する:
			\begin{align}
				\left| I_{s,t}(x) - I_{s,t}(y) \right| \leq \epsilon C.
			\end{align}
		\end{thm}
	\end{screen}
	
	\begin{screen}
		\begin{cor}[$p$-variationによる閉球上のLipschitz連続性]\label{cor:continuity_theorem_1}
			$1 \leq p < 2$とし,
			$x_0 = y_0$を満たす$x,y \in C^1$と$f \in C^2_b(\R^d,L(\R^d \rightarrow \R^m)),\ 0 < R < \infty,\ [s,t] \subset [0,T]$を任意に取る.
			このとき,
			\begin{align}
				\Norm{X^1}{p,[s,t]},\Norm{Y^1}{p,[s,t]} \leq R
			\end{align}
			なら,或る定数$C = C(p,R,f)$が存在して次を満たす:
			\begin{align}
				\left| I_{s,t}(x) - I_{s,t}(y) \right| \leq C\Norm{X^1 - Y^1}{p,[s,t]}.
			\end{align}
		\end{cor}
	\end{screen}
	
	\begin{prf}[系\ref{cor:continuity_theorem_1}]
		定理\ref{thm:continuity_theorem_1}において,
		$\epsilon = \Norm{X^1 - Y^1}{p}\ (x \neq y)$
		\footnote{
			$x=y$なら$\Norm{X^1 - Y^1}{p} = 0$かつ$I_{s,t}(x) = I_{s,t}(y)$が成り立つ.
		}
		として証明が通る.
		$[0,T]$で考察したことを$[s,t]$に置き換えれば系を得る.
		\QED
	\end{prf}
	
	\begin{prf}[定理\ref{thm:continuity_theorem_1}]
		$[s,t] \subset [0,T]$とする.
		\begin{description}
			\item[第一段]
				$\omega:\Delta_T \longrightarrow [0,\infty)$を
				\begin{align}
					\omega(\alpha,\beta) = \Norm{X^1}{p,[\alpha,\beta]}^p + \Norm{Y^1}{p,[\alpha,\beta]}^p + \epsilon^{-p} \Norm{X^1 - Y^1}{p,[\alpha,\beta]}^p,
					\quad ((\alpha,\beta) \in \Delta_T)
				\end{align}
				で定めれば,定理\ref{thm:examples_of_control_functions}により$1 \leq p$の下で$\omega$はcontrol functionである.
				
			\item[第二段]
				任意に$[s,t]$の分割
				$D = \{s=t_0 < \cdots < t_N=t\}\ (N \geq 2)$を取れば,
				補題\ref{lem:control_function_min}より(\refeq{eq:lem_control_function_min})を満たす$t_{(0)}$が存在する.
				ここで,$D_{-0} \coloneqq D,\ D_{-1} \coloneqq D \backslash \{t_{(0)}\}$と定める.
				$N \geq 3$ならば$D_{-1}$についても(\refeq{eq:lem_control_function_min})を満たす$t_{(1)}$が存在するから,
				$D_{-2} \coloneqq D_{-1} \backslash \{t_{(1)}\}$と定める.この操作を繰り返せば
				$t_{(k)},D_{-k}\ (k=0,1,\cdots,N-1)$が得られ,
				\begin{align}
					&\tilde{I}_{s,t}(x,D) - \tilde{I}_{s,t}(y,D) \\
					&\qquad = \sum_{k=0}^{N-2} \left[ \left\{ \tilde{I}_{s,t}(x,D_{-k}) - \tilde{I}_{s,t}(x,D_{-k-1}) \right\} - 
						\left\{ \tilde{I}_{s,t}(y,D_{-k}) - \tilde{I}_{s,t}(y,D_{-k-1}) \right\} \right] \label{eq:continuity_theorem_1_1}\\
					&\quad \qquad + \left\{ \tilde{I}_{s,t}(x) - \tilde{I}_{s,t}(y) \right\}	\label{eq:continuity_theorem_1_2}
				\end{align}
				と表現できる.
			
			\item[第三段]
				式(\refeq{eq:continuity_theorem_1_1})について,次を満たす定数$C_1$が存在することを示す:
				\begin{align}
					|(\refeq{eq:continuity_theorem_1_1})| \leq \epsilon C_1
					\label{eq:continuity_theorem_1_3}
				\end{align}
				見やすくするために$t_k = t_{(k)}$と書き直せば,
				\begin{align}
					&\left\{ \tilde{I}_{s,t}(x,D_{-k}) - \tilde{I}_{s,t}(x,D_{-k-1}) \right\} - 
						\left\{ \tilde{I}_{s,t}(y,D_{-k}) - \tilde{I}_{s,t}(y,D_{-k-1}) \right\} \\
					&=	\left\{ f(x_{t_k}) - f(x_{t_{k-1}}) \right\} X^1_{t_k,t_{k+1}}
						- \left\{ f(y_{t_k}) - f(y_{t_{k-1}}) \right\} Y^1_{t_k,t_{k+1}} \\
					&= \left\{ f(x_{t_k}) - f(x_{t_{k-1}}) \right\} \left( X^1_{t_k,t_{k+1}} - Y^1_{t_k,t_{k+1}} \right) \\
						&\quad +\left\{ f(x_{t_k}) - f(x_{t_{k-1}}) \right\} Y^1_{t_k,t_{k+1}} - \left\{ f(y_{t_k}) - f(y_{t_{k-1}}) \right\} Y^1_{t_k,t_{k+1}} \\
					&= \int_0^1 (\nabla f)( x_{t_{k-1}}+\theta ( x_{t_k}-x_{t_{k-1}} ))
						X^1_{t_{k-1},t_k} \otimes \left( X^1_{t_k,t_{k+1}} - Y^1_{t_k,t_{k+1}} \right)\ d\theta \\
						&\quad + \int_0^1 (\nabla f)( x_{t_{k-1}}+\theta ( x_{t_k}-x_{t_{k-1}} ))
						X^1_{t_{i_k-1},t_k} \otimes Y^1_{t_k,t_{i_k+1}}\ d\theta \\
						&\quad - \int_0^1 (\nabla f)( y_{t_{k-1}}+\theta ( y_{t_k}-y_{t_{k-1}} ))
						Y^1_{t_{i_k-1},t_k} \otimes Y^1_{t_k,t_{i_k+1}}\ d\theta \\
					&= \int_0^1 (\nabla f)( x_{t_{k-1}}+\theta ( x_{t_k}-x_{t_{k-1}} ))
						X^1_{t_{i_k-1},t_k} \otimes \left( X^1_{t_k,t_{i_k+1}} - Y^1_{t_k,t_{i_k+1}} \right)\ d\theta \\
						&\quad + \int_0^1 (\nabla f)( x_{t_{k-1}}+\theta ( x_{t_k}-x_{t_{k-1}} ))
						\left( X^1_{t_{k-1},t_k} - Y^1_{t_{k-1},t_k} \right) \otimes Y^1_{t_k,t_{k+1}}\ d\theta \\
						&\quad + \int_0^1 (\nabla f)( x_{t_{k-1}}+\theta ( x_{t_k}-x_{t_{k-1}} ))
						Y^1_{t_{k-1},t_k} \otimes Y^1_{t_k,t_{k+1}}\ d\theta \\
						&\quad - \int_0^1 (\nabla f)( y_{t_{k-1}}+\theta ( y_{t_k}-y_{t_{k-1}} ))
						Y^1_{t_{k-1},t_k} \otimes Y^1_{t_k,t_{k+1}}\ d\theta \\
					&= \int_0^1 (\nabla f)( x_{t_{k-1}}+\theta ( x_{t_k}-x_{t_{k-1}} ))
						X^1_{t_{k-1},t_k} \otimes \left( X^1_{t_k,t_{k+1}} - Y^1_{t_k,t_{k+1}} \right)\ d\theta \\
						&\quad + \int_0^1 (\nabla f)( x_{t_{k-1}}+\theta ( x_{t_k}-x_{t_{k-1}} ))
						\left( X^1_{t_{k-1},t_k} - Y^1_{t_{k-1},t_k} \right) \otimes Y^1_{t_k,t_{k+1}}\ d\theta \\
						&\quad + \int_0^1 \int_0^1 (\nabla^2 f)( y_{t_{k-1}}+\theta ( y_{t_k}-y_{t_{k-1}} + r(x_{t_{k-1}}+\theta ( x_{t_k}-x_{t_{k-1}}) - y_{t_{k-1}}-\theta ( y_{t_k}-y_{t_{k-1}}))) \\
						&\qquad\qquad \left( X^1_{0,t_{k-1}} - Y^1_{0,t_{k-1}} \right) \otimes Y^1_{t_{k-1},t_k} \otimes Y^1_{t_k,t_{k+1}}\ dr\ d\theta\ \footnotemark \\
						&\quad + \int_0^1 \int_0^1 (\nabla^2 f)( y_{t_{k-1}}+\theta ( y_{t_k}-y_{t_{k-1}} + r(x_{t_{k-1}}+\theta ( x_{t_k}-x_{t_{k-1}}) - y_{t_{k-1}}-\theta ( y_{t_k}-y_{t_{k-1}}))) \\
						&\qquad\qquad \theta \left( X^1_{t_{k-1},t_k} - Y^1_{t_{k-1},t_k} \right) \otimes Y^1_{t_{k-1},t_k} \otimes Y^1_{t_k,t_{k+1}}\ dr\ d\theta
				\end{align}
				\footnotetext{
					$x_0 = y_0$の仮定より$x_{t_{k-1}} - y_{t_{k-1}} = X^1_{0,t_{k-1}} - Y^1_{0,t_{k-1}}$が成り立つ.
				}
				が成り立つ.補題\ref{lem:control_function_min}より
				\begin{align}
					&\left| X^1_{t_{k-1},t_k} \right|, \left| Y^1_{t_{k-1},t_k} \right|,\left| X^1_{t_k,t_{k+1}} \right|, \left| Y^1_{t_k,t_{k+1}} \right|
					\leq \omega(t_{k-1},t_{k+1})^{1/p} \leq \left( \frac{2\omega(s,t)}{N-k-1} \right)^{1/p}, \\
					&\left| X^1_{t_{k-1},t_k} - Y^1_{t_{k-1},t_k} \right|,\left| X^1_{t_k,t_{k+1}} - Y^1_{t_k,t_{k+1}} \right|
					\leq \epsilon \omega(t_{k-1},t_{k+1})^{1/p} \leq \epsilon \left( \frac{2\omega(s,t)}{N-k-1} \right)^{1/p}
				\end{align}
				が満たされ,また
				\begin{align}
					\left| X^1_{0,t_{k-1}} - Y^1_{0,t_{k-1}} \right|
					\leq \epsilon \omega(0,t_{k-1})^{1/p}
					\leq \epsilon \omega(0,T)^{1/p}
					\leq \epsilon \left( 2 R^p + 1 \right)^{1/p}
				\end{align}
				でもあるから,
				\begin{align}
					M &\coloneqq \sum_{i,j} \sup{x \in \R^d}{|f^i_j(x)|} + \sum_{i,j,k} \sup{x \in \R^d}{|\partial_k f^i_j(x)|}
						+ \sum_{i,j,k,v} \sup{x \in \R^d}{|\partial_v \partial_k f^i_j(x)|}
					\label{eq:continuity_theorem_1_4}
				\end{align}
				と定めて
				\begin{align}
					&\left| \left\{ \tilde{I}_{s,t}(x,D_{-k}) - \tilde{I}_{s,t}(x,D_{-k-1}) \right\} - 
						\left\{ \tilde{I}_{s,t}(y,D_{-k}) - \tilde{I}_{s,t}(y,D_{-k-1}) \right\} \right| \\
					&\leq M \left|X^1_{t_{k-1},t_k}\right|\left| X^1_{t_k,t_{k+1}} - Y^1_{t_k,t_{k+1}} \right| \\
						&\quad + M \left| X^1_{t_{k-1},t_k} - Y^1_{t_{k-1},t_k} \right| \left| Y^1_{t_k,t_{k+1}} \right| \\
						&\quad + M \left| X^1_{0,t_{k-1}} - Y^1_{0,t_{k-1}} \right| \left| Y^1_{t_{k-1},t_k} \right|\left| Y^1_{t_k,t_{k+1}} \right| \\
						&\quad + M \left| X^1_{t_{k-1},t_k} - Y^1_{t_{k-1},t_k} \right| \left| Y^1_{t_{k-1},t_k} \right|\left| Y^1_{t_k,t_{k+1}} \right| \\
					&\leq \epsilon M \left[2 + 2 \left( 2 R^p + 1 \right)^{1/p} \right] \left( \frac{2\omega(s,t)}{N-k-1} \right)^{2/p} \\
					&\leq \epsilon M \left[2 + 2 \left( 2 R^p + 1 \right)^{1/p} \right] 2^{2/p} \left( 2 R^p + 1 \right)^{2/p} \left( \frac{1}{N-k-1} \right)^{2/p}
				\end{align}
				を得る.
				\begin{align}
					C_1' \coloneqq  M \left[2 + 2 \left( 2 R^p + 1 \right)^{1/p} \right] 2^{2/p} \left( 2 R^p + 1 \right)^{2/p}
				\end{align}
				とおけば
				\begin{align}
					|(\refeq{eq:continuity_theorem_1_1})| \leq
					\sum_{k=0}^{N-2} \epsilon C_1' \left( \frac{1}{N-k-1} \right)^{2/p}
					< \epsilon C_1' \zeta \biggl( \frac{2}{p} \biggr)
				\end{align}
				が成立し,$p < 2$より$\zeta(2/p) < \infty$であるから
				$C_1 \coloneqq C_1' \zeta(2/p)$とおいて(\refeq{eq:continuity_theorem_1_3})が従う.
			
			\item[第四段]
				$x_0 = y_0$の仮定により$x_s-y_s = X^1_{0,s} - Y^1_{0,s}$が成り立ち
				\begin{align}
					\left| \tilde{I}_{s,t}(x) - \tilde{I}_{s,t}(y) \right|
					&= \left| f(x_s)X^1_{s,t} - f(y_s)Y^1_{s,t} \right| \\
					&\leq \left| f(x_s)X^1_{s,t} - f(x_s)Y^1_{s,t} \right|
						+ \left| f(x_s)Y^1_{s,t} - f(y_s)Y^1_{s,t} \right| \\
					&\leq M \left| X^1_{s,t} - Y^1_{s,t} \right|
						+ \left| \int_0^1 (\nabla f)(y_s + \theta(x_s - y_s)) \left[ \left( X^1_{0,s} - Y^1_{0,s}\right) \otimes Y^1_{s,t} \right]\ d\theta \right| \\
					&\leq M \left| X^1_{s,t} - Y^1_{s,t} \right|
						+ M \left| X^1_{0,s} - Y^1_{0,s}\right| \left| Y^1_{s,t} \right| \\
					&\leq M \epsilon \omega(s,t)^{1/p} + M \epsilon \omega(0,s)^{1/p} \omega(s,t)^{1/p} \\
					&\leq \epsilon M\left[\left( 2 R^p + 1 \right)^{1/p} + \left( 2 R^p + 1 \right)^{2/p} \right]
				\end{align}
				が従う.ここで$C_2 \coloneqq M\left[\left( 2 R^p + 1 \right)^{1/p} + \left( 2 R^p + 1 \right)^{2/p} \right]$とおく.
				
			\item[第五段]
				第二段と第三段より,任意の$D \in \delta[s,t]$に対し
				\begin{align}
					\left| \tilde{I}_{s,t}(x,D) - \tilde{I}_{s,t}(y,D) \right|
					\leq \epsilon (C_1 + C_2)
				\end{align}
				が成立し.定理\ref{thm:Riemann_Stieltjes_approximation}により
				$|D| \longrightarrow 0$として
				\begin{align}
					\left| I_{s,t}(x) - I_{s,t}(y) \right|
					\leq \epsilon (C_1 + C_2)
				\end{align}
				が出る.
				\QED
		\end{description}
	\end{prf}
	
	\begin{screen}
		\begin{thm}[$2 \leq p < 3$の場合の連続性定理]\label{thm:continuity_theorem_2}
			$2 \leq p < 3$とし,
			$x_0 = y_0$を満たす$x,y \in C^1$と$f \in C^2_b(\R^d,L(\R^d \rightarrow \R^m)),\ 0 < \epsilon, R < \infty$を任意に取る.
			このとき,
			\begin{align}
				&\Norm{X^1}{p},\Norm{Y^1}{p},\Norm{X^2}{p/2},\Norm{Y^2}{p/2} \leq R < \infty,\\
				&\Norm{X^1 - Y^1}{p},\Norm{X^2 - Y^2}{p/2} \leq \epsilon
			\end{align}
			なら,或る定数$C = C(p,R,f)$が存在し,任意の$0 \leq s \leq t \leq T$に対して次が成立する:
			\begin{align}
				\left| I_{s,t}(x) - I_{s,t}(y) \right| \leq \epsilon C.
			\end{align}
		\end{thm}
	\end{screen}
	
	\begin{screen}
		\begin{cor}\label{cor:continuity_theorem_2}
			$1 \leq p < 2$とし,
			$x_0 = y_0$を満たす$x,y \in C^1$と$f \in C^2_b(\R^d,L(\R^d \rightarrow \R^m)),\ 0 < R < \infty,\ [s,t] \subset [0,T]$を任意に取る.
			このとき,
			\begin{align}
				\Norm{X^1}{p,[s,t]},\Norm{Y^1}{p,[s,t]}, \Norm{X^2}{p/2,[s,t]},\Norm{Y^2}{p/2,[s,t]} \leq R
			\end{align}
			なら,或る定数$C = C(p,R,f)$が存在して次を満たす:
			\begin{align}
				\left| I_{s,t}(x) - I_{s,t}(y) \right| \leq C\left( \Norm{X^1 - Y^1}{p,[s,t]}+\Norm{X^2 - Y^2}{p/2,[s,t]} \right).
			\end{align}
		\end{cor}
	\end{screen}
	
	\begin{prf}[系\ref{cor:continuity_theorem_2}]
		定理\ref{thm:continuity_theorem_2}において,
		$\epsilon = \Norm{X^1 - Y^1}{p} + \Norm{X^2 - Y^2}{p/2,[s,t]}\ (x \neq y)$
		として証明が通る.
		$[0,T]$で考察したことを$[s,t]$に置き換えれば系を得る.
		\QED
	\end{prf}
	
	\begin{prf}[定理\ref{cor:continuity_theorem_2}]
		$[s,t] \subset [0,T]$とする.
		\begin{description}
			\item[第一段]
				$\omega:\Delta_T \longrightarrow [0,\infty)$を
				\begin{align}
					\omega(\alpha,\beta) &= \Norm{X^1}{p,[\alpha,\beta]}^p + \Norm{Y^1}{p,[\alpha,\beta]}^p 
						+ \Norm{X^2}{p/2,[\alpha,\beta]}^{p/2} + \Norm{Y^2}{p/2,[\alpha,\beta]}^{p/2} \\
						&\quad + \epsilon^{-p} \Norm{X^1 - Y^1}{p,[\alpha,\beta]}^p +  \epsilon^{-p/2} \Norm{X^2 - Y^2}{p/2,[\alpha,\beta]}^{p/2},
					\quad ((\alpha,\beta) \in \Delta_T)
				\end{align}
				で定めれば,定理\ref{thm:examples_of_control_functions}により$2 \leq p$の下で$\omega$はcontrol functionである.
				
			\item[第二段]
				$D \in \delta[s,t]$に対し,
				定理\ref{thm:continuity_theorem_1}の証明と同様にして
				$t_{(k)},D_{-k}$を構成すれば
				\begin{align}
					&J_{s,t}(x,D) - J_{s,t}(y,D) \\
					&\qquad= \sum_{k=0}^{N-2} \left[ \left\{ J_{s,t}(x,D_{-k}) - J_{s,t}(x,D_{-k-1}) \right\} - 
						\left\{ J_{s,t}(y,D_{-k}) - J_{s,t}(y,D_{-k-1}) \right\} \right] \label{eq:continuity_theorem_2_1}\\
					&\quad\qquad + \left\{ J_{s,t}(x) - J_{s,t}(y) \right\}	\label{eq:continuity_theorem_2_2}
				\end{align}
				と表現できる.
			
			\item[第三段]
				$J_{s,t}(x,D_{-k}) - J_{s,t}(x,D_{-k-1})$を変形する.
				以降$t_k = t_{(k)}$と書き直せば
				\begin{align}
					&J_{s,t}(x,D_{-k}) - J_{s,t}(x,D_{-k-1}) \\
					&\qquad = J_{t_{k-1},t_k}(x) + J_{t_k,t_{k+1}}(x) - J_{t_{k-1},t_{k+1}}(x) \\
					&\qquad = f(x_{t_{k-1}}) X^1_{t_{k-1},t_k} + f(x_{t_k}) X^1_{t_k,t_{k+1}} - f(x_{t_{k-1}}) X^1_{t_{k-1},t_{k+1}} \\
						&\qquad \qquad + (\nabla f)(x_{t_{k-1}})X^2_{t_{k-1},t_k} + (\nabla f)(x_{t_k})X^2_{t_k,t_{k+1}} - (\nabla f)(x_{t_{k-1}})X^2_{t_{k-1},t_{k+1}} \\
					&\qquad = \left\{ f(x_{t_k}) - f(x_{t_{k-1}}) \right\} X^1_{t_k,t_{k+1}} \\
						&\qquad \qquad + (\nabla f)(x_{t_{k-1}})X^2_{t_{k-1},t_k} + (\nabla f)(x_{t_k})X^2_{t_k,t_{k+1}} - (\nabla f)(x_{t_{k-1}})X^2_{t_{k-1},t_{k+1}} \\
					&\qquad = \int_0^1 \left\{ (\nabla f)(x_{t_{k-1}} + \theta(x_{t_k} - x_{t_{k-1}})) - (\nabla f)(x_{t_{k-1}}) \right\} X^1_{t_{k-1},t_k} \otimes X^1_{t_k,t_{k+1}}\ d\theta \\
						&\qquad \qquad + (\nabla f)(x_{t_{k-1}})X^1_{t_{k-1},t_k} \otimes X^1_{t_k,t_{k+1}} \\
						&\qquad \qquad + (\nabla f)(x_{t_{k-1}})X^2_{t_{k-1},t_k} + (\nabla f)(x_{t_k})X^2_{t_k,t_{k+1}} - (\nabla f)(x_{t_{k-1}})X^2_{t_{k-1},t_{k+1}} \\
					&\qquad = \int_0^1 \int_0^\theta (\nabla f)(x_{t_{k-1}} + r(x_{t_k} - x_{t_{k-1}}))  X^1_{t_{k-1},t_k} \otimes X^1_{t_{k-1},t_k} \otimes X^1_{t_k,t_{k+1}}\ dr\ d\theta \\
						&\qquad \qquad + (\nabla f)(x_{t_{k-1}})\left( X^1_{t_{k-1},t_k} \otimes X^1_{t_k,t_{k+1}} + X^2_{t_{k-1},t_k} - X^2_{t_{k-1},t_{k+1}} \right) \\
						&\qquad \qquad + (\nabla f)(x_{t_k})X^2_{t_k,t_{k+1}} \\
					&\qquad = \int_0^1 \int_0^\theta (\nabla f)(x_{t_{k-1}} + r(x_{t_k} - x_{t_{k-1}}))  X^1_{t_{k-1},t_k} \otimes X^1_{t_{k-1},t_k} \otimes X^1_{t_k,t_{k+1}}\ dr\ d\theta \\
						&\qquad \qquad + \left\{ (\nabla f)(x_{t_k}) - (\nabla f)(x_{t_{k-1}}) \right\}X^2_{t_k,t_{k+1}} \\
					&\qquad = \int_0^1 \int_0^\theta (\nabla f)(x_{t_{k-1}} + r(x_{t_k} - x_{t_{k-1}}))  X^1_{t_{k-1},t_k} \otimes X^1_{t_{k-1},t_k} \otimes X^1_{t_k,t_{k+1}}\ dr\ d\theta \\
						&\qquad \qquad + \int_0^1 (\nabla^2 f)(x_{t_{k-1}} + \theta(x_{t_k} - x_{t_{k-1}})) X^1_{t_{k-1},t_k} \otimes X^2_{t_k,t_{k+1}}\ d\theta
				\end{align}
				を得る.
				
			\item[第四段]
				式(\refeq{eq:continuity_theorem_2_1})について,次を満たす定数$C_1$が存在することを示す:
				\begin{align}
					|(\refeq{eq:continuity_theorem_2_1})| \leq \epsilon C_1.
					\label{eq:continuity_theorem_2_3}
				\end{align}
				実際,前段の結果より
				\begin{align}
					&\left\{ J_{s,t}(x,D_{-k}) - J_{s,t}(x,D_{-k-1}) \right\} - 
						\left\{ J_{s,t}(y,D_{-k}) - J_{s,t}(y,D_{-k-1}) \right\} \\
					&=	\int_0^1 \int_0^\theta (\nabla^2 f)(x_{t_{k-1}}+r(x_{t_k}-x_{t_{k-1}})) X^1_{t_{k-1},t_k} \otimes X^1_{t_{k-1},t_k} \otimes X^1_{t_k,t_{k+1}}\ dr\ d\theta \\
						&\quad + \int_0^1 (\nabla^2 f)(x_{t_{k-1}}+\theta(x_{t_k}-x_{t_{k-1}}))X^1_{t_{k-1},t_k} \otimes X^2_{t_k,t_{k+1}}\ d\theta \\
						&\quad - \int_0^1 \int_0^\theta (\nabla^2 f)(y_{t_{k-1}}+r(y_{t_k}-y_{t_{k-1}})) Y^1_{t_{k-1},t_k} \otimes Y^1_{t_{k-1},t_k} \otimes Y^1_{t_k,t_{k+1}}\ dr\ d\theta \\
						&\quad - \int_0^1 (\nabla^2 f)(y_{t_{k-1}}+\theta(y_{t_k}-y_{t_{k-1}}))Y^1_{t_{k-1},t_k} \otimes Y^2_{t_k,t_{k+1}}\ d\theta \\
					&= \int_0^1 \int_0^\theta (\nabla^2 f)(x_{t_{k-1}}+r(x_{t_k}-x_{t_{k-1}})) X^1_{t_{k-1},t_k} \otimes X^1_{t_{k-1},t_k} \otimes \left(X^1_{t_k,t_{k+1}} - Y^1_{t_k,t_{k+1}}\right)\ dr\ d\theta \\
						&\quad + \int_0^1 \int_0^\theta (\nabla^2 f)(x_{t_{k-1}}+r(x_{t_k}-x_{t_{k-1}})) X^1_{t_{k-1},t_k} \otimes \left(X^1_{t_{k-1},t_k} - Y^1_{t_{k-1},t_k} \right) \otimes Y^1_{t_k,t_{k+1}}\ dr\ d\theta \\
						&\quad +  \int_0^1 \int_0^\theta \left\{ (\nabla^2 f)(x_{t_{k-1}}+r(x_{t_k}-x_{t_{k-1}})) - (\nabla^2 f)(y_{t_{k-1}}+r(y_{t_k}-y_{t_{k-1}})) \right\} \\
						&\qquad\qquad\qquad X^1_{t_{k-1},t_k} \otimes Y^1_{t_{k-1},t_k} \otimes Y^1_{t_k,t_{k+1}}\ dr\ d\theta \\
						&\quad + \int_0^1 \int_0^\theta (\nabla^2 f)(y_{t_{k-1}}+r(y_{t_k}-y_{t_{k-1}})) \left( X^1_{t_{k-1},t_k} - Y^1_{t_{k-1},t_k} \right) \otimes Y^1_{t_{k-1},t_k} \otimes Y^1_{t_k,t_{k+1}}\ dr\ d\theta \\
						&\quad + \int_0^1 (\nabla^2 f)(x_{t_{k-1}}+\theta(x_{t_k}-x_{t_{k-1}}))X^1_{t_{k-1},t_k} \otimes \left(X^2_{t_k,t_{k+1}} -  Y^2_{t_k,t_{k+1}}\right)\ d\theta \\
						&\quad + \int_0^1 \left\{ (\nabla^2 f) (x_{t_{k-1}}+\theta(x_{t_k}-x_{t_{k-1}})) - (\nabla^2 f)(y_{t_{k-1}}+\theta(y_{t_k}-y_{t_{k-1}})) \right\} \\
						&\qquad\qquad\qquad X^1_{t_{k-1},t_k} \otimes Y^2_{t_k,t_{k+1}}\ d\theta \\
						&\quad + \int_0^1 (\nabla^2 f)(y_{t_{k-1}}+\theta(y_{t_k}-y_{t_{k-1}})) \left( X^1_{t_{k-1},t_k} - Y^1_{t_{k-1},t_k} \right) \otimes Y^2_{t_k,t_{k+1}}\ d\theta \\
					&= \int_0^1 \int_0^\theta (\nabla^2 f)(x_{t_{k-1}}+r(x_{t_k}-x_{t_{k-1}})) X^1_{t_{k-1},t_k} \otimes X^1_{t_{k-1},t_k} \otimes \left(X^1_{t_k,t_{k+1}} - Y^1_{t_k,t_{k+1}}\right)\ dr\ d\theta \\
						&\quad + \int_0^1 \int_0^\theta (\nabla^2 f)(x_{t_{k-1}}+r(x_{t_k}-x_{t_{k-1}})) X^1_{t_{k-1},t_k} \otimes \left(X^1_{t_{k-1},t_k} - Y^1_{t_{k-1},t_k} \right) \otimes Y^1_{t_k,t_{k+1}}\ dr\ d\theta \\
						&\quad +  \int_0^1 \int_0^\theta \int_0^1 (\nabla^3 f)(y_{t_{k-1}}+r(y_{t_k}-y_{t_{k-1}}) + u(x_{t_{k-1}}+r(x_{t_k}-x_{t_{k-1}}) - y_{t_{k-1}}-r(y_{t_k}-y_{t_{k-1}}))) \\
						&\qquad\qquad\qquad \left\{ \left( X^1_{0,t_{k-1}} - Y^1_{0,t_{k-1}} \right) + r\left( X^1_{t_{k-1},t_k} - Y^1_{t_{k-1},t_k} \right) \right\} \otimes X^1_{t_{k-1},t_k} \otimes Y^1_{t_{k-1},t_k} \otimes Y^1_{t_k,t_{k+1}}\ du\ dr\ d\theta \\
						&\quad + \int_0^1 \int_0^\theta (\nabla^2 f)(y_{t_{k-1}}+r(y_{t_k}-y_{t_{k-1}})) \left( X^1_{t_{k-1},t_k} - Y^1_{t_{k-1},t_k} \right) \otimes Y^1_{t_{k-1},t_k} \otimes Y^1_{t_k,t_{k+1}}\ dr\ d\theta \\
						&\quad + \int_0^1 (\nabla^2 f)(x_{t_{k-1}}+\theta(x_{t_k}-x_{t_{k-1}}))X^1_{t_{k-1},t_k} \otimes \left(X^2_{t_k,t_{k+1}} -  Y^2_{t_k,t_{k+1}}\right)\ d\theta \\
						&\quad + \int_0^1 \int_0^1 (\nabla^3 f) (y_{t_{k-1}}+\theta(y_{t_k}-y_{t_{k-1}}) + r(x_{t_{k-1}}+\theta(x_{t_k}-x_{t_{k-1}})-y_{t_{k-1}}-\theta(y_{t_k}-y_{t_{k-1}}))) \\
						&\qquad\qquad\qquad \left\{ \left( X^1_{0,t_{k-1}} - Y^1_{0,t_{k-1}} \right) + \theta\left( X^1_{t_{k-1},t_k} - Y^1_{t_{k-1},t_k} \right) \right\} \otimes X^1_{t_{k-1},t_k} \otimes Y^2_{t_k,t_{k+1}}\ dr\ d\theta \\
						&\quad + \int_0^1 (\nabla^2 f)(y_{t_{k-1}}+\theta(y_{t_k}-y_{t_{k-1}})) \left( X^1_{t_{k-1},t_k} - Y^1_{t_{k-1},t_k} \right) \otimes Y^2_{t_k,t_{k+1}}\ d\theta
				\end{align}
				が成り立つから,
				\begin{align}
					M &\coloneqq \sum_{i,j} \sup{x \in \R^d}{|f^i_j(x)|} + \sum_{i,j,k} \sup{x \in \R^d}{|\partial_k f^i_j(x)|} \\
						&\qquad + \sum_{i,j,k,v} \sup{x \in \R^d}{|\partial_v \partial_k f^i_j(x)|}
						+ \sum_{i,j,k,v,w} \sup{x \in \R^d}{|\partial_w \partial_v \partial_k f^i_j(x)|}
					\label{eq:continuity_theorem_2_4}
				\end{align}
				とおいて
				\begin{align}
					&\left| \left\{ J_{s,t}(x,D_{-k}) - J_{s,t}(x,D_{-k-1}) \right\} - 
						\left\{ J_{s,t}(y,D_{-k}) - J_{s,t}(y,D_{-k-1}) \right\} \right| \\
					&\leq M \left|X^1_{t_{k-1},t_k}\right| \left|X^1_{t_{k-1},t_k}\right| \left| X^1_{t_k,t_{k+1}} - Y^1_{t_k,t_{k+1}} \right| \\
						&\quad + M \left| X^1_{t_{k-1},t_k} \right| \left| X^1_{t_{k-1},t_k} - Y^1_{t_{k-1},t_k} \right| \left| Y^1_{t_k,t_{k+1}} \right| \\
						&\quad + M \left| X^1_{0,t_{k-1}} - Y^1_{0,t_{k-1}} \right| \left| X^1_{t_{k-1},t_k} \right| \left| Y^1_{t_{k-1},t_k} \right| \left| Y^1_{t_k,t_{k+1}} \right| \\
						&\quad + M \left| X^1_{t_{k-1},t_k} - Y^1_{t_{k-1},t_k} \right| \left| X^1_{t_{k-1},t_k} \right| \left| Y^1_{t_{k-1},t_k} \right| \left| Y^1_{t_k,t_{k+1}} \right| \\
						&\quad + M \left| X^1_{t_{k-1},t_k} - Y^1_{t_{k-1},t_k} \right| \left| Y^1_{t_{k-1},t_k} \right| \left| Y^1_{t_k,t_{k+1}} \right| \\
						&\quad + M \left| X^1_{t_{k-1},t_k} \right| \left| X^2_{t_k,t_{k+1}} - Y^2_{t_k,t_{k+1}} \right| \\
						&\quad + M \left| X^1_{0,t_{k-1}} - Y^1_{0,t_{k-1}} \right| \left| X^1_{t_{k-1},t_k} \right| \left| Y^2_{t_k,t_{k+1}} \right| \\
						&\quad + M \left| X^1_{t_{k-1},t_k} - Y^1_{t_{k-1},t_k} \right| \left| X^1_{t_{k-1},t_k} \right| \left| Y^2_{t_k,t_{k+1}} \right| \\
						&\quad + M \left| X^1_{t_{k-1},t_k} - Y^1_{t_{k-1},t_k} \right| \left| Y^2_{t_k,t_{k+1}} \right| \\
					&\leq \epsilon M \left[5 + 2\omega(0,t_{k-1})^{1/p} + 2\omega(t_{k-1},t_k)^{1/p} \right] \left( \frac{2\omega(s,t)}{N-k-1} \right)^{3/p} \\
					&\leq \epsilon M \left[2 + 4\left( 2 R^p + 2R^{p/2} + 2 \right)^{1/p} \right] 2^{3/p} \left( 2 R^p + 2R^{p/2} + 2 \right)^{3/p} \left( \frac{1}{N-k-1} \right)^{3/p}
				\end{align}
				を得る.ここで
				\begin{align}
					C_1' \coloneqq  M \left[2 + 4\left( 2 R^p + 2R^{p/2} + 2 \right)^{1/p} \right] 2^{3/p} \left( 2 R^p + 2R^{p/2} + 2 \right)^{3/p}
				\end{align}
				と定めれば
				\begin{align}
					|(\refeq{eq:continuity_theorem_2_1})| \leq
					\sum_{k=0}^{N-2} \epsilon C_1' \left( \frac{1}{N-k-1} \right)^{3/p}
					< \epsilon C_1' \zeta \biggl( \frac{3}{p} \biggr)
				\end{align}
				が成立し,$p < 3$より$\zeta(3/p) < \infty$であるから
				$C_1 \coloneqq C_1' \zeta(3/p)$とおいて(\refeq{eq:continuity_theorem_2_3})が出る.
			
			\item[第五段]
				$x_0 = y_0$の仮定により
				\begin{align}
					&\left| J_{s,t}(x) - J_{s,t}(y) \right| \\
					&\leq \left| f(x_s)X^1_{s,t} - f(y_s)Y^1_{s,t} \right| + \left| (\nabla f)(x_s)X^2_{s,t} - (\nabla f)(y_s)Y^2_{s,t} \right| \\
					&\leq \left| f(x_s)X^1_{s,t} - f(x_s)Y^1_{s,t} \right|
						+ \left| f(x_s)Y^1_{s,t} - f(y_s)Y^1_{s,t} \right| \\
						&\qquad + \left| (\nabla f)(x_s)X^2_{s,t} - (\nabla f)(x_s)Y^2_{s,t} \right|
						+ \left| (\nabla f)(x_s)Y^2_{s,t} - (\nabla f)(y_s)Y^2_{s,t} \right| \\
					&\leq M \left| X^1_{s,t} - Y^1_{s,t} \right|
						+ \left| \int_0^1 (\nabla f)(y_s + \theta(x_s - y_s))(x_s - y_s) \otimes Y^1_{s,t}\ d\theta \right| \\
						&\qquad + M \left| X^2_{s,t} - Y^2_{s,t} \right|
						+ \left| \int_0^1 (\nabla^2 f)(y_s + \theta(x_s - y_s))(x_s - y_s) \otimes Y^2_{s,t}\ d\theta \right| \\
					&\leq M \left| X^1_{s,t} - Y^1_{s,t} \right|
						+ M \left| X^1_{0,s} - Y^1_{0,s}\right| \left| Y^1_{s,t} \right| \\
						&\qquad + M \left| X^2_{s,t} - Y^2_{s,t} \right| 
						+ M \left| X^1_{0,s} - Y^1_{0,s}\right| \left| Y^2_{s,t} \right| \\
					&\leq \epsilon M \omega(s,t)^{1/p} + \epsilon M \omega(0,s)^{1/p} \omega(s,t)^{1/p} \\
						&\quad + \epsilon M \omega(s,t)^{2/p} + \epsilon M \omega(0,s)^{1/p} \omega(s,t)^{2/p} \\
					&\leq \epsilon M \left[ \omega(0,T)^{1/p} + 2\omega(0,T)^{2/p} + \omega(0,T)^{3/p} \right] \\
					&\leq \epsilon M \left[ \left( 2 R^p + 2R^{p/2} + 2 \right)^{1/p}
						+ 2\left( 2 R^p + 2R^{p/2} + 2 \right)^{2/p}
						+\left( 2 R^p + 2R^{p/2} + 2 \right)^{3/p} \right]
				\end{align}
				が従う.ここで最下段の$\epsilon$の係数を$C_2$とおく.
				
			\item[第六段]
				以上より,任意の$D \in \delta[s,t]$に対し
				\begin{align}
					\left| J_{s,t}(x,D) - J_{s,t}(y,D) \right|
					\leq \epsilon (C_1 + C_2)
				\end{align}
				が成り立ち,定理\ref{thm:Riemann_Stieltjes_approximation}により
				$|D| \longrightarrow 0$として
				\begin{align}
					\left| I_{s,t}(x) - I_{s,t}(y) \right|
					\leq \epsilon (C_1 + C_2)
				\end{align}
				が出る.
				\QED
		\end{description}
	\end{prf}
	
	\begin{screen}
		\begin{cor}[パスが0出発なら$f$の有界性は要らない]
			定理\ref{thm:continuity_theorem_1}と定理\ref{thm:continuity_theorem_2}について,
			$x,y \in \tilde{C}^1$ならば
			$f \in C^2(\R^d,L(\R^d \rightarrow \R^m))$として主張が成り立つ.
		\end{cor}
	\end{screen}
	
	\begin{prf}
		$x_0 = 0$なら
		\begin{align}
			\Norm{X^1}{p} \leq R \quad \Rightarrow \quad |x_t| \leq R \quad (\forall t \in [0,T])
		\end{align}
		が成り立つから,式(\refeq{eq:continuity_theorem_1_4})と(\refeq{eq:continuity_theorem_2_4})において
		$\sup{x\in\R^d}{}$を$\sup{|x| \leq 9R}$に替えればよい.
		\QED
	\end{prf}
\end{document}