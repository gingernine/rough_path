\section{The notion of rough path}
	$(V,\Norm{\cdot}{})$を$\R$上のBanach空間とする$(V \neq \{0\})$.また
	$\otimes_a$により代数的テンソル積,或はその標準写像を表す.
	$k \geq 2$の場合,$k$重テンソル積$V^{\otimes_a k} = V \otimes_a \cdots \otimes_a V$に
	プロジェクティブノルム$\projectivenorm{\cdot}{k}$を導入し,
	その完備拡大を$(V^{\otimes k},\compcrossnorm{\cdot}{k})$と書く
	\footnote{
		$V$が有限次元なら$V^{\otimes_a k}$も有限次元であるから,
		有限次元ノルム空間の完備性より
		$\left( V^{\otimes_a k},\projectivenorm{\cdot}{k} \right)$を完備化する必要はない.
	}.
	$k=0,1$に対しては$V^{\otimes 0} \coloneqq \R,\ V^{\otimes 1} \coloneqq V$とし,
	$\compcrossnorm{\cdot}{0} = \projectivenorm{\cdot}{0} \coloneqq \mbox{$\R$の絶対値}$,及び
	$\compcrossnorm{\cdot}{1} = \projectivenorm{\cdot}{1} \coloneqq \Norm{\cdot}{}$と定める.
	定理\ref{thm:tensor_product_with_scalar}
	と定理\ref{thm:associativity_of_tensor_products}
	によれば,任意の$0 \leq j \leq k$に対し
	$V^{\otimes_a k}$と$V^{\otimes_a j} \otimes_a V^{\otimes_a k-j}$は線型同型となる.
	この同型写像を
	\begin{align}
		F_{j,k}:V^{\otimes_a k} \longrightarrow V^{\otimes_a j} \otimes_a V^{\otimes_a k-j},
		\quad 0 \leq j \leq k
	\end{align}
	と表せば,$F_{j,k}$は
	\begin{align}
		F_{0,k}(v) &= 1 \otimes_a v, && (\forall v \in V^{\otimes_a k}), \\
		F_{j,k}(v_1 \otimes_a \cdots \otimes_a v_k) 
			&= (v_1 \otimes_a \cdots \otimes_a v_{j}) \otimes_a (v_{j+1} \otimes_a \cdots \otimes_a v_k), 
			&& (\forall v_1 \otimes_a \cdots \otimes_a v_k \in V^{\otimes_a k},\ 1 \leq j \leq k-1), \\
		F_{k,k}(v) &= v \otimes_a 1, && (\forall v \in V^{\otimes_a k})
	\end{align}
	を満たす.また$V^{\otimes_a j} \otimes_a V^{\otimes_a k-j}$上にプロジェクティブノルムを導入し,
	これを$\pi_{j,k}$と書く.
	\begin{screen}
		\begin{thm}
			このとき次式が成立する.特に,$F_{j,k},\ (0 \leq j \leq k)$は等長同型である.
			\begin{align}
				\pi_k \circ F^{-1}_{j,k} = \pi_{j,k}, \quad 0 \leq j \leq k.
			\end{align}
			
		\end{thm}
	\end{screen}
	
	\begin{prf}\mbox{}
		\begin{description}
			\item[第一段]
				$j=0$のとき,任意の$v \in V^{\otimes_a k}$に対し
				\begin{align}
					\projectivenorm{F_{0,k}(v)}{0,k}
					= \projectivenorm{1 \otimes_a v}{0,k}
					= \projectivenorm{1}{0}\projectivenorm{v}{k}
					= \projectivenorm{v}{k}
				\end{align}
				が成り立ち$\pi_k \circ F^{-1}_{0,k} = \pi_{0,k}$を得る.
				同様にして$\pi_k \circ F^{-1}_{k,k} = \pi_{k,k}$も出る.
				
			\item[第二段]
				$\pi_k \circ F^{-1}_{j,k} \leq \pi_{j,k},\ (1 \leq j \leq k-1)$が成り立つことを示す.
				$v \in V^{\otimes_a j} \otimes_a V^{\otimes_a k-j}$の分割
				\begin{align}
					v = \sum_{r} u^r \otimes_a v^r,
					\quad (u^r \in V^{\otimes_a j},\ v^r \in V^{\otimes_a k-j})
				\end{align}
				を任意に取り,一旦固定する.このとき$u^r,v^r$の任意の分割
				\begin{align}
					u^r = \sum_{n(r)} u^{n(r)}_1 \otimes_a \cdots \otimes_a u^{n(r)}_j, 
					\quad v^r = \sum_{m(r)} v^{m(r)}_{j+1} \otimes_a \cdots \otimes_a v^{m(r)}_{k},
					\quad (v^{n(r)}_i,v^{m(r)}_i \in V)
				\end{align}
				に対して
				\begin{align}
					\projectivenorm{F^{-1}_{j,k}(v)}{k}
					&\leq \sum_{r} \sum_{n(r),m(r)} \projectivenorm{u^{n(r)}_1 \otimes_a \cdots \otimes_a u^{n(r)}_j \otimes_a v^{m(r)}_{j+1} \otimes_a \cdots \otimes_a v^{m(r)}_{k}}{k} \\
					&= \sum_{r} \sum_{n(r),m(r)} \Norm{u^{n(r)}_1}{} \cdots \Norm{u^{n(r)}_j}{} \Norm{v^{m(r)}_{j+1}}{} \cdots \Norm{v^{m(r)}_k}{} \\
					&= \sum_{r} \left\{ \sum_{n(r)} \Norm{u^{n(r)}_1}{} \cdots \Norm{u^{n(r)}_j}{} \right\} \left\{ \sum_{m(r)} \Norm{v^{m(r)}_{j+1}}{} \cdots \Norm{v^{m(r)}_k}{} \right\} 
				\end{align}
				が成り立つから,分割の任意性と定理\ref{thm:expression_of_projective_norm}より
				\begin{align}
					\projectivenorm{F^{-1}_{j,k}(v)}{k} 
					\leq \sum_r \projectivenorm{u^r}{j}\projectivenorm{v^r}{k-j}
				\end{align}
				を得る.$v$の分割について下限を取れば,
				再び定理\ref{thm:expression_of_projective_norm}により
				\begin{align}
					\projectivenorm{F^{-1}_{j,k}(v)}{k} \leq \projectivenorm{v}{j,k}
				\end{align}
				が出る.
			
			\item[第三段]
				$\pi_k \circ F^{-1}_{j,k} \geq \pi_{j,k},\ (1 \leq j \leq k-1)$が成り立つことを示す.
				$v \in V^{\otimes_a k}$の任意の分割
				\begin{align}
					v = \sum_n v^n_1 \otimes_a \cdots \otimes_a v^n_k,
					\quad (v^n_i \in V,\ i=1,\cdots,k)
				\end{align}
				を取れば,
				\begin{align}
					\projectivenorm{F_{j,k}(v)}{j,k}
					&\leq \sum_n \projectivenorm{\left( v^n_1 \otimes_a \cdots \otimes_a v^n_j \right) \otimes_a \left( v^n_j \otimes_a \cdots \otimes_a v^n_k \right)}{j,k} \\
					&= \sum_n \projectivenorm{v^n_1 \otimes_a \cdots \otimes_a v^n_j}{j}
						\projectivenorm{v^n_j \otimes_a \cdots \otimes_a v^n_k}{k-j} \\
					&= \sum_n \Norm{v^n_1}{} \cdots \Norm{v^n_k}{}
				\end{align}
				が成立する.従って定理\ref{thm:expression_of_projective_norm}より
				\begin{align}
					\projectivenorm{F_{j,k}(v)}{j,k} \leq \projectivenorm{v}{k}
				\end{align}
				が得られる.
				\QED
		\end{description}
	\end{prf}
	
	
	$V^{\otimes_a i}$の$V^{\otimes i}$への等長埋め込みを$J_i$で表し($i=0,1$の場合$J_i$は恒等写像),
	\begin{align}
		J_j V^{\otimes_a j} \times J_{k-j} V^{\otimes_a k-j} \ni (u,v)
		& \longmapsto ( J_j^{-1}u,J_{k-j}^{-1}v ) &&\in V^{\otimes_a j} \times V^{\otimes_a k-j} \\
		& \longmapsto F_{j,k}^{-1} (J_j^{-1}u \otimes_a J_{k-j}^{-1}v) &&\in V^{\otimes_a k} \\
		& \longmapsto J_k F_{j,k}^{-1} (J_j^{-1}u \otimes_a J_{k-j}^{-1}v) &&\in V^{\otimes k}
	\end{align}
	の対応関係により定まる写像$:J_j V^{\otimes_a j} \times J_{k-j} V^{\otimes_a k-j}
	\longrightarrow V^{\otimes k}$を$\varphi_{j,k}$と書けば,$\varphi_{j,k}$は有界双線型写像である.
	実際,$\otimes_a$の双線型性と埋め込み及び$F_{j,k}^{-1}$の線型性より
	$\varphi_{j,k}$の双線型性が従い,また
	\begin{align}
		\compcrossnorm{\varphi_{j,k}(u,v)}{k}
		&= \projectivenorm{F_{j,k}^{-1} (J_j^{-1}u \otimes_a J_{k-j}^{-1}v)}{k} \\
		&= \projectivenorm{J_j^{-1}u \otimes_a J_{k-j}^{-1}v}{j,k} \\
		&= \projectivenorm{J_j^{-1}u}{j} \projectivenorm{J_{k-j}^{-1}v}{k-j} \\
		&= \compcrossnorm{u}{j} \compcrossnorm{v}{k-j}
	\end{align}
	が任意の$(u,v) \in J_j V^{\otimes_a j} \times J_{k-j} V^{\otimes_a k-j}$に対して
	成り立つから$\Norm{\varphi_{j,k}}{\ContLn{J_j V^{\otimes_a j} \times J_{k-j} V^{\otimes_a k-j}}{V^{\otimes k}}{2}} = 1$を得る.
	従って,定理\ref{thm:expansion_of_multilinear_mapping}より
	$\varphi_{j,k}$は$V^{\otimes j} \times V^{\otimes k-j}$上の或るただ一つの双線型写像
	$\psi_{j,k}$にノルム保存拡張される.
	
	\begin{screen}
		\begin{thm}\label{thm:property_of_the_completion_of_the_projective_norm}
			$0 \leq j \leq k$とする.
			このとき,$\psi_{j,k}:V^{\otimes j} \times V^{\otimes k-j} \longrightarrow V^{\otimes k}$
			は次を満たす:
			\begin{align}
				\compcrossnorm{\psi_{j,k}(u,v)}{k} = \compcrossnorm{u}{j} \compcrossnorm{v}{k-j},
				\quad (\forall (u,v) \in V^{\otimes j} \times V^{\otimes k-j}).
			\end{align}
		\end{thm}
	\end{screen}
	
	\begin{prf}
		$(u,v)$に直積ノルムで収束する点列$(u_n,v_n) \in J_j V^{\otimes_a j} \times J_{k-j} V^{\otimes_a k-j}\ (n=1,2,\cdots)$を取れば
		\begin{align}
			\compcrossnorm{\varphi_{j,k}(u_n,v_n) - \psi_{j,k}(u,v)}{k} \longrightarrow 0,
			\quad (n \longrightarrow \infty)
		\end{align}
		が成り立つ.また
		\begin{align}
			\left| \compcrossnorm{u_n}{j} \compcrossnorm{v_n}{k-j} - 
			\compcrossnorm{u}{j} \compcrossnorm{v}{k-j} \right|
			&\leq \left| \compcrossnorm{u_n}{j} \compcrossnorm{v_n}{k-j} - 
			\compcrossnorm{u_n}{j} \compcrossnorm{v}{k-j} \right| 
			+ \left| \compcrossnorm{u_n}{j} \compcrossnorm{v}{k-j} - 
			\compcrossnorm{u}{j} \compcrossnorm{v}{k-j} \right| \\
			&\leq \compcrossnorm{u_n}{j} \compcrossnorm{v_n - v}{k-j} 
			+ \compcrossnorm{u_n - u}{j} \compcrossnorm{v}{k-j} \\
			&\longrightarrow 0, \quad (n \longrightarrow \infty)
		\end{align}
		も成立するから
		\begin{align}
			\left|\, \compcrossnorm{\psi_{j,k}(u,v)}{k} - \compcrossnorm{u}{j} \compcrossnorm{v}{k-j}\, \right|
			&\leq \compcrossnorm{\varphi_{j,k}(u_n,v_n) - \psi_{j,k}(u,v)}{k}
				+ \left| \compcrossnorm{u_n}{j} \compcrossnorm{v_n}{k-j} - 
			\compcrossnorm{u}{j} \compcrossnorm{v}{k-j} \right| \\
			& \longrightarrow 0, \quad (n \longrightarrow \infty)
		\end{align}
		が従い$\compcrossnorm{\psi_{j,k}(u,v)}{k} = \compcrossnorm{u}{j} \compcrossnorm{v}{k-j}$
		が得られる.
		\QED
	\end{prf}
	
	$T(V) \coloneqq \bigoplus_{k=0}^{\infty} V^{\otimes k}$
	とおく.また上で定めた双線型写像$\psi_{j,k}$を$\otimes_{j,k}$と書き直す:
	\begin{align}
		u \otimes_{j,k} v = \psi_{j,k}(u,v),
		\quad (\forall (u,v) \in V^{\otimes j} \times V^{\otimes k-j},\ 0 \leq j \leq k,\ k \geq 0).
		\label{eq:def_of_otimes_for_completion_V_tensor_k}
	\end{align}
	このとき,$(a^k)_{k=0}^{\infty},(b^k)_{k=0}^{\infty} \in T(V)$に対する
	二項関係$(a^k)_{k=0}^{\infty} \otimes (b^k)_{k=0}^{\infty} \eqqcolon (c^k)_{k=0}^{\infty}$を
	\begin{align}
		c^k = \sum_{j=0}^{k} a^j \otimes_{j,k} b^{k-j},
		\quad (k=0,1,2,\cdots)
	\end{align}
	により定めれば,
	$c^k \in V^{\otimes k}\ (k=0,1,\cdots)$かつ
	有限個の$k$を除いて$c^k = 0$となるから$(c^k)_{k=0}^\infty \in T(V)$が成立し,
	$\otimes$は$T(V)$において結合則?を満たす双線型な積となる.$n \geq 0$に対して
	\begin{align}
		T^{(n)}(V) \coloneqq \bigoplus_{k=0}^{n} V^{\otimes k}
	\end{align}
	とおき,同様に$\otimes$を
	\begin{align}
		(c^k)_{k=0}^n = (a^k)_{k=0}^n \otimes (b^k)_{k=0}^n,
		\quad c^k = \sum_{j=0}^{k} a^j \otimes_{j,k} b^{k-j},
		\quad (k=0,\cdots,n)
	\end{align}
	により定め,次の直積ノルムを導入する:
	\begin{align}
		\compcrossnorm{a}{} \coloneqq \sum_{k=0}^n \compcrossnorm{a^k}{k},
		\quad (a = (a^k)_{k=0}^n \in T^{(n)}(V)).
		\label{eq:def_norm_on_truncated_tensor_algebra}
	\end{align}
	いま,写像$X:\Delta_T \longrightarrow T^{(n)}(V)$に対して
	$X_{s,t} = (X^0_{s,t},\cdots,X^n_{s,t}),\ ((s,t) \in \Delta_T)$と書いて
	\begin{align}
		C_0 \left(\Delta_T,T^{(n)}(V) \right)
		\coloneqq \Set{X:\Delta_T \longrightarrow T^{(n)}(V)}{\mbox{continuous},\ X^0 \equiv 1}
	\end{align}
	とおく.
	
	\begin{screen}
		\begin{dfn}[finite $p$-variation]
			$p \geq 1$とする.$X:\Delta_T \longrightarrow T^{(n)}(V)$に対して
			或るコントロール関数$\omega$が存在して
			\begin{align}
				\compcrossnorm{X^i_{s,t}}{i} \leq \omega(s,t)^{i/p},
				\quad (\forall i=1,\cdots,n,\ \forall (s,t) \in \Delta_T)
				\label{eq:def_finite_p_variation}
			\end{align}
			を満たすとき,$X$は有限$p$-変動(finite $p$-variation)であるという.\footnotemark
		\end{dfn}
	\end{screen}
	\footnotetext{
		$X:\Delta_T \longrightarrow T^{(n)}(V)$が有限$p$-変動であることと
		$\Norm{X}{p} < \infty$が満たされることは同値であるかどうかを考察する.
		実際,$X^0 \equiv 1$を満たし,かつ有限$p$-変動を持つような$X$が存在する場合,
		\begin{align}
			\mbox{$D$の分割小区間の数}
			= \sum_D \compcrossnorm{X^0_{t_{i-1},t_i}}{0}^p
			\leq \sum_D \compcrossnorm{X_{t_{i-1},t_i}}{}^p
			\leq \Norm{X}{p}^p,
			\quad (\forall D \in \delta[0,T])
		\end{align}
		が成り立つから,$\Norm{X}{p} = \infty$となる.
		しかしこのような$X$が存在するかは未だ示していないので脚注メモ.
	}
	
	\begin{screen}
		\begin{dfn}[finite total $p$-variation]
			$p \geq 1$とする.$X \in C_0 \left(\Delta_T,T^{(n)}(V) \right)$
			が有限総$p$-変動(finite total $p$-variation)とは,任意の$1 \leq i \leq n$に対して
			\begin{align}
				\Norm{X^i}{p/i} < \infty
			\end{align}
			が満たされることをいう.また次の線型空間を定める:
			\begin{align}
				C_{0,p} \left(\Delta_T,T^{(n)}(V) \right)
				\coloneqq \Set{X \in C_0 \left(\Delta_T,T^{(n)}(V) \right)}{\mbox{$X$ has finite total $p$-variation}}	.
			\end{align}
		\end{dfn}
	\end{screen}
	
	\begin{screen}
		\begin{dfn}[乗法的汎関数]
			次の関係式(Chen's identity)を満たす$X \in C_0 \left(\Delta_T,T^{(n)}(V) \right)$
			を$n$次の乗法的汎関数(multiplicative functional of degree $n$)と呼ぶ:
			\begin{align}
				X_{s,u} \otimes X_{u,t} = X_{s,t},
				\quad (\forall 0 \leq s \leq u \leq t \leq T).
			\end{align}
		\end{dfn}
	\end{screen}
	
	$C_0 \left(\Delta_T,T^{(n)}(V) \right)$の定義には
	$X^0 \equiv 1$という条件が含まれている.
	実際,$X:\Delta_T \longrightarrow T^{(n)}(V)$が Chen's identity を満たすには
	$X^0 \equiv 1$或は0である必要がある.理由は,
	$X^0_{s,t} = X^0_{s,t} \otimes_{0,0} X^0_{s,t} = X^0_{s,t}X^0_{s,t}$を得るためである.
	特に,次の定理が成立するためには$X^0 \equiv 1$が満たされていなくてはならない.
	\begin{screen}
		\begin{lem}\label{lem:multiplicative_functional_vanishes_on_diagonal}
			$X:\Delta_T \longrightarrow T^{(n)}(V)$が$X^0 \equiv 0$かつ
			Chen's identity を満たせば$X^k_{t,t} = 0\ (1 \leq k \leq n)$が成り立つ.
		\end{lem}
	\end{screen}
	
	\begin{prf}
		任意に$t \in [0,T]$を取る.
		$X^k_{t,t} = \sum_{j=0}^{k} X^j_{t,t} \otimes_{j,k} X^{k-j}_{t,t}$より,先ず
		\begin{align}
			X^1_{t,t} = X^0_{t,t} \otimes_{0,1} X^1_{t,t} + X^1_{t,t} \otimes_{0,1} X^0_{t,t}
			= X^1_{t,t} + X^1_{t,t}
		\end{align}
		が成り立ち$X^1_{t,t} = 0$が従う.同様に
		\begin{align}
			X^2_{t,t} = X^0_{t,t} \otimes_{0,1} X^2_{t,t} + X^1_{t,t} \otimes_{0,1} X^1_{t,t}
				+ X^2_{t,t} \otimes_{0,1} X^0_{t,t}
			= X^2_{t,t} + X^2_{t,t}
		\end{align}
		より$X^2_{t,t} = 0$が成立し,帰納的に$X^k_{t,t} = 0\ (1 \leq k \leq n)$を得る.
		\QED
	\end{prf}
	
	\begin{screen}
		\begin{thm}
			$n$次乗法的汎関数$X \in C_0 \left(\Delta_T,T^{(n)}(V) \right)$に対し,
			$X$が有限$p$-変動であることと有限総$p$-変動であることは同値である.
		\end{thm}
	\end{screen}
	
	\begin{prf}
		$n$次乗法的汎関数$X \in C_0 \left(\Delta_T,T^{(n)}(V) \right)$が有限総$p$-変動のとき,
		\begin{align}
			\omega(s,t) \coloneqq \sum_{i=1}^n \Norm{X^i}{p/i,[s,t]}^{p/i},
			\quad ((s,t) \in \Delta_T)
		\end{align}
		で$\omega$を定めれば,
		補題\ref{lem:multiplicative_functional_vanishes_on_diagonal}と
		定理\ref{thm:control_function_defined_by_p_variation}により
		$\omega$はコントロール関数となる.このとき
		\begin{align}
			\compcrossnorm{X^i_{s,t}}{i}
			\leq \Norm{X^i}{p/i,[s,t]}
			\leq \omega(s,t)^{i/p},
			\quad (\forall i = 1,\cdots,n,\ \forall (s,t) \in \Delta_T)
		\end{align}
		が成り立つから$X$は有限$p$-変動である.逆に$X$が有限$p$-変動なら,
		(\refeq{eq:def_finite_p_variation})を満たす$\omega$を取れば
		\begin{align}
			\sum_D \compcrossnorm{X^i_{t_{i-1},t_i}}{i}^{p/i}
			\leq \sum_D \omega(t_{i-1},t_i)
			\leq \omega(0,T),
			\quad (\forall D \in \delta[0,T],\ i=1,\cdots,n)
		\end{align}
		が成立し$\Norm{X^i}{p/i} < \infty\ (i=1,\cdots,n)$が従うので$X$は有限総$p$-変動である.
		\QED
	\end{prf}
	
	実際に乗法的汎関数を構成する.有界変動な連続写像$x:[0,T] \longrightarrow V = V^{\otimes 1}$に対して
	\begin{align}
		X^1_{s,t} \coloneqq x_t - x_s,
		\quad (\forall (s,t) \in \Delta_T)
	\end{align}
	とおけば,$X^1:\Delta_T \longrightarrow V$は連続かつ$\Norm{X^1}{1} < \infty$を満たす.
	このとき次の積分
	\begin{align}
		\int_s^t X^1_{s,u} \otimes d x_u
		\coloneqq \lim_{|D| \to 0} \sum_{D} X^1_{s,t_{i-1}} \otimes_{1,2} X^1_{t_{i-1},t_i}
	\end{align}
	を定めたい.右辺が$V^{\otimes 2}$で収束することを示せばよい.
	いま,細分$D=\{s=u_0 < \cdots <u_n= t\},D'=\{s=v_0 < \cdots <v_m= t\} \in \delta[s,t]$
	を任意に取り,これらの共通細分を$D''=\{s=w_0 < \cdots < w_r = t\}$と表して
	\begin{align}
		\begin{cases}
			\tilde{X}^1_{s,w_k} \coloneqq X^1_{s,u_i}, & (u_i \leq w_k \leq u_{i+1}), \\
			\hat{X}^1_{s,w_k} \coloneqq X^1_{s,v_j}, & (v_j \leq w_k \leq v_{j+1}),
		\end{cases}
		\quad (k=0,1,\cdots,r)
	\end{align}
	で$\tilde{X}^1,\hat{X}^1$を定めれば,
	定理\ref{thm:property_of_the_completion_of_the_projective_norm}より
	\begin{align}
		&\compcrossnorm{ \sum_D X^1_{s,u_{i-1}} \otimes_{1,2} X^1_{u_{i-1},u_i} - 
			\sum_{D'} X^1_{s,v_{j-1}} \otimes_{1,2} X^1_{v_{j-1},v_j}}{2} \\
		&\qquad\leq \compcrossnorm{ \sum_D X^1_{s,u_{i-1}} \otimes_{1,2} X^1_{u_{i-1},u_i} - 
			\sum_{D''} X^1_{s,w_{k-1}} \otimes_{1,2} X^1_{w_{k-1},w_k}}{2}
			+ \compcrossnorm{ \sum_{D'} X^1_{s,v_{j-1}} \otimes_{1,2} X^1_{v_{j-1},v_j} - 
			\sum_{D''} X^1_{s,w_{k-1}} \otimes_{1,2} X^1_{w_{k-1},w_k}}{2} \\
		&\qquad= \compcrossnorm{ \sum_{D''} \tilde{X}^1_{s,w_{k-1}} \otimes_{1,2} X^1_{w_{k-1},w_k} - 
			\sum_{D''} X^1_{s,w_{k-1}} \otimes_{1,2} X^1_{w_{k-1},w_k}}{2} 
			+ \compcrossnorm{ \sum_{D''} \hat{X}^1_{s,w_{k-1}} \otimes_{1,2} X^1_{w_{k-1},w_k} - 
			\sum_{D''} X^1_{s,w_{k-1}} \otimes_{1,2} X^1_{w_{k-1},w_k}}{2} \\
		&\qquad= \compcrossnorm{ \sum_{D''} \left(\tilde{X}^1_{s,w_{k-1}} -  X^1_{s,w_{k-1}} \right)
		 	\otimes_{1,2} X^1_{w_{k-1},w_k}}{2}
			+ \compcrossnorm{ \sum_{D''} \left( \hat{X}^1_{s,w_{k-1}} - X^1_{s,w_{k-1}} \right)
			 \otimes_{1,2} X^1_{w_{k-1},w_k}}{2} \\
		&\qquad\leq \sum_{D''} \compcrossnorm{\tilde{X}^1_{s,w_{k-1}} -  X^1_{s,w_{k-1}}}{1} \compcrossnorm{X^1_{w_{k-1},w_k}}{1}
			+ \sum_{D''} \compcrossnorm{\hat{X}^1_{s,w_{k-1}} -  X^1_{s,w_{k-1}}}{1} \compcrossnorm{X^1_{w_{k-1},w_k}}{1} \\
		&\qquad\leq \max{k}{\compcrossnorm{\tilde{X}^1_{s,w_{k-1}} -  X^1_{s,w_{k-1}}}{1}} 
			\Norm{X^1}{1,[s,t]} + \max{k}{\compcrossnorm{\hat{X}^1_{s,w_{k-1}} -  X^1_{s,w_{k-1}}}{1}} 
			\Norm{X^1}{1,[s,t]}
	\end{align}
	が成立する.$[s,t] \ni u \longmapsto X^1_{s,u}$は一様連続であるから,
	$|D|,|D'|\longrightarrow 0$として右辺は0に収束する.
	従って,$|D_n| \longrightarrow 0\ (n \longrightarrow \infty)$を満たす細分列$D_n \in \delta[s,t]$を
	取れば$\left(\sum_{D_n} X^1_{s,u_{i-1}} \otimes_{1,2} X^1_{u_{i-1},u_i} \right)_{n=1}^{\infty}$
	は$V^{\otimes 2}$のCauchy列となり$V^{\otimes 2}$で収束する.別の細分列
	$(\tilde{D}_m)_{m=1}^{\infty} \subset \delta[s,t]\ (|\tilde{D}_m| \longrightarrow 0)$を取っても
	\begin{align}
		\compcrossnorm{ \sum_{D_n} X^1_{s,u_{i-1}} \otimes_{1,2} X^1_{u_{i-1},u_i} - 
			\sum_{\tilde{D}_m} X^1_{s,v_{j-1}} \otimes_{1,2} X^1_{v_{j-1},v_j}}{2}
		\longrightarrow 0,
		\quad (n,m \longrightarrow \infty)
	\end{align}
	が成り立つから極限は細分列に依らず確定し,これにより
	$\lim_{|D| \to 0} \sum_{D} X^1_{s,u_{i-1}} \otimes_{1,2} X^1_{u_{i-1},u_i}$が存在する.この極限を
	\begin{align}
		X^2_{s,t} = \int_s^t X^1_{s,u} \otimes d x_u \coloneqq 
		\lim_{|D| \to 0} \sum_{D} X^1_{s,u_{i-1}} \otimes_{1,2} X^1_{u_{i-1},u_i}
		\label{eq:signature_of_path_1}
	\end{align}
	と表記すれば次が成立する:
	
	\begin{screen}
		\begin{thm}
			(\refeq{eq:signature_of_path_1})で定める
			$\Delta_T \ni (s,t) \longmapsto X^2_{s,t} \in V^{\otimes 2}$は連続かつ有界変動であり,
			更に次を満たす:
			\begin{align}
				X^2_{s,t} = X^2_{s,u} + X^1_{s,u} \otimes_{1,2} X^1_{u,t} + X^2_{u,t},
				\quad (\forall 0 \leq s \leq t \leq T).
				\label{eq:signature_of_path_2}
			\end{align}
		\end{thm}
	\end{screen}
	
	\begin{prf}\mbox{}
		\begin{description}
			\item[第一段]
				$X^2$が有界変動であることを示す.任意に$(s,t) \in \Delta_T\ (s < t)$
				\footnote{
					$s=t$なら,$X^1_{s,t} = 0$より$X^2_{s,t} = 0$が成り立つ.
				}を取る.
				\begin{align}
					M \coloneqq \sup{(x,y) \in \Delta_T}{\compcrossnorm{X^1_{x,y}}{1}}
				\end{align}
				とおけば$X^1$の連続性より$M < \infty$であり,
				任意の$\epsilon > 0$に対し或る$D \in \delta[s,t]$が存在して
				\begin{align}
					\compcrossnorm{X^2_{s,t}}{2}
					\leq \epsilon + \compcrossnorm{\sum_{D} X^1_{s,u_{i-1}} \otimes_{1,2} X^1_{u_{i-1},u_i}}{2}
					\leq \epsilon + \sum_D \compcrossnorm{X^1_{s,u_{i-1}}}{1}
					\compcrossnorm{X^1_{t_{i-1},t_i}}{1} 
					\leq \epsilon + M \Norm{X^1}{1,[s,t]}
				\end{align}
				が成立するから,$\epsilon > 0$と$(s,t)$の任意性より
				\begin{align}
					\compcrossnorm{X^2_{s,t}}{2} \leq M \Norm{X^1}{1,[s,t]},
					\quad (\forall (s,t) \in \Delta_T)
				\end{align}
				が従い$\Norm{X^2}{1} \leq M \Norm{X^1}{1}$を得る.
				
			\item[第二段]
				点$(s,s)\ (\forall s \in [0,T])$において$X^2$が連続であること示す.
				実際,定理\ref{thm:control_function_defined_by_p_variation}より
				\begin{align}
					\Delta_T \ni (s,t) \longmapsto \Norm{X^1}{1,[s,t]}
				\end{align}
				はコントロール関数であるから,
				\begin{align}
					&\compcrossnorm{X^2_{t,u} - X^2_{s,s}}{2}
					= \compcrossnorm{X^2_{t,u}}{2}
					\leq M \Norm{X^1}{1,[t,u]}
					\longrightarrow 0 \quad ((t,u) \longrightarrow (s,s))
				\end{align}
				が成立し$X^2$の$(s,s)$における連続性を得る.
			
			\item[第三段]
				$s < t$を満たす点$(s,t) \in \Delta_T$において$X^2$が連続であること示す.
				いま,任意に$\epsilon > 0$を取る.
				このとき,$X^1$の一様連続性により,或る$0 < c < t-s$が存在して,
				$(a,b) \in \Delta_T$が$|a-s| < c$を満たす限り
				\begin{align}
					&\compcrossnorm{X^1_{s,u} - X^1_{a,u}}{1} < \epsilon,
					\quad (s \vee a \leq \forall u \leq T)
				\end{align}
				が成立する.一方(\refeq{eq:signature_of_path_1})より,
				或る$\eta > 0$が存在して
				$D_1 \in \delta[s,t],\ D_2 \in \delta[a,b]$
				が$|D_1|,|D_2| < \eta$を満たす限り
				\begin{align}
					\compcrossnorm{X^2_{s,t} - \sum_{D_1} X^1_{s,u_{i-1}} \otimes_{1,2} X^1_{u_{i-1},u_i}}{2} < \epsilon,
					\quad \compcrossnorm{X^2_{a,b} - \sum_{D_2} X^1_{a,v_{j-1}} \otimes_{1,2} X^1_{v_{j-1},v_j}}{2} < \epsilon
				\end{align}
				が成立する.また或る$\eta' > 0$が存在して$0 < v-u < \eta'$なら
				\begin{align}
					\compcrossnorm{X^1_{u,v}}{1} < \epsilon
				\end{align}
				となる.ここで$|s-a| < c,|D_1|,|D_2| < \eta \wedge \eta'$を満たす
				$(a,b),D_1,D_2$を取り
				\begin{align}
					D_3 \coloneqq (D_1 \cup D_2) \cap [s,t] \cap [a,b],
					\quad D'_1 \coloneqq D_1 \cup D_3,
					\quad D'_2 \coloneqq D_2 \cup D_3
				\end{align}
				とおけば,$[s,t] \cap [a,b]$上で$D'_1$と$D'_2$の分割点は一致するので
				\begin{align}
					&\compcrossnorm{\sum_{D'_1} X^1_{s,u_{i-1}} \otimes_{1,2} X^1_{u_{i-1},u_i}
						- \sum_{D'_2} X^1_{a,v_{j-1}} \otimes_{1,2} X^1_{v_{j-1},v_j}}{2} \\
					&\qquad\leq \compcrossnorm{\sum_{D_3} X^1_{s,u_{i-1}} \otimes_{1,2} X^1_{u_{i-1},u_i}
						- \sum_{D_3} X^1_{a,u_{i-1}} \otimes_{1,2} X^1_{u_{i-1},u_i}}{2} \\
					&\quad\qquad + \compcrossnorm{\sum_{D'_1 \backslash D_3} X^1_{s,u_{i-1}} \otimes_{1,2} X^1_{u_{i-1},u_i}}{2}
					+ \compcrossnorm{\sum_{D'_2 \backslash D_3} X^1_{a,v_{j-1}} \otimes_{1,2} X^1_{v_{j-1},v_j}}{2} \\
					&\qquad\leq \sum_{D_3} \compcrossnorm{X^1_{s,u_{i-1}} - X^1_{a,u_{i-1}}}{1}
					\compcrossnorm{X^1_{u_{i-1},u_i}}{1}
					+ \sum_{D'_1 \backslash D_3} \compcrossnorm{X^1_{s,u_{i-1}}}{1}
						\compcrossnorm{X^1_{u_{i-1},u_i}}{1}
						+ \sum_{D'_2 \backslash D_3} \compcrossnorm{X^1_{a,v_{j-1}}}{1}
						\compcrossnorm{X^1_{v_{j-1},v_i}}{1} \\
					&\qquad< 3 \Norm{X^1}{1} \epsilon 
				\end{align}
				が成り立ち
				\begin{align}
					\compcrossnorm{X^2_{s,t} - X^2_{a,b}}{2} < \left[ 3 \Norm{X^1}{1} + 2\right]\epsilon
				\end{align}
				が従う.これにより$X^2$の$(s,t)$における連続性を得る.
			
			\item[第四段]
				(\refeq{eq:signature_of_path_2})を示す.
		\end{description}
		
	\end{prf}
	
	\begin{screen}
		\begin{dfn}[$p$-ラフパス]
			$p \geq 1$とし,$p$を超えない最大の整数を$[p]$で表す.
			有限$p$-変動を持つ$[p]$次乗法的汎関数を
			$p$-ラフパス($p$-rough path)と呼び,その全体を$\Omega_p(V)$と書く:
			\begin{align}
				\Omega_p(V) 
				= \Set{X \in C_0\left( \Delta_T,T^{([p])}(V) \right)}{
					\mbox{$[p]$次乗法的,有限$p$-変動.}}.
			\end{align}
		\end{dfn}
	\end{screen}
	
	\begin{screen}
		\begin{thm}\label{thm:p_rough_path_complete_dist}
			$\Omega_p(V)$は次で定める距離により完備距離空間となる:
			\begin{align}
				d_p(X,Y) \coloneqq \max{1 \leq i \leq [p]}{\Norm{X^i - Y^i}{p/i}}.
			\end{align}
		\end{thm}
	\end{screen}
	
	$X \in \Omega_p(V)$は$X^0 \equiv 1$を満たすから,
	$\max{1 \leq i \leq [p]}{\Norm{\cdot}{p/i}}$は$\Omega_p(V)$においてノルムとはならない.
	
	\begin{prf}完備性を示す.
		\begin{description}
			\item[第一段] 極限を構成する.いま,
			任意の$X = (X^0,X^1,\cdots,X^{[p]}) \in \Omega_p(V)$に対して
			\begin{align}
				X^i \in \tilde{B}_{p/i,T}\left( V^{\otimes i} \right),
				\quad (\forall i=1,\cdots,[p]) 
			\end{align}
			が満たされるから,定理\ref{thm:B_p_T_Banach_space_2}より任意のCauchy列
			$\left( X^k = (X^{k,0},\cdots,X^{k,[p]}) \right)_{k=1}^{\infty} \subset \Omega_p(V)$
			に対して
			\begin{align}
				\Norm{X^{k,i} - X^i}{p/i} \longrightarrow 0
				\quad (k \longrightarrow \infty,\ \forall i=1,\cdots,[p])
				\label{eq:thm_p_rough_path_complete_dist_2}
			\end{align}
			を満たす$X^i \in B_{p/i,T}\left( V^{\otimes i} \right)$が存在する.
			ここで$X:\Delta_T \longrightarrow T^{([p])}(V)$を
			\begin{align}
				X_{s,t} \coloneqq (1,X^1_{s,t},\cdots,X^n_{s,t}),
				\quad (\forall (s,t) \in \Delta_T)
			\end{align}
			により定めれば,$X^i$の連続性と$T^{([p])}(V)$におけるノルムの定義
			(\refeq{eq:def_norm_on_truncated_tensor_algebra})より
			$X$は連続写像である.
		
		\item[第二段] $X$が Chen's identity を満たすことを示す.
			各$1 \leq i \leq [p]$について,
			\begin{align}
				X^i_{s,t} = \sum_{j=0}^i X^j_{s,u} \otimes_{j,i} X^{i-j}_{u,t},
				\quad (\forall 0 \leq s \leq u \leq t \leq T)
				\label{eq:thm_p_rough_path_complete_dist_1}
			\end{align}
			が成立すればよい.実際,
			\begin{align}
				\compcrossnorm{X^{k,i}_{s,t} - X^i_{s,t}}{i}
				\leq \Norm{X^{k,i} - X^i}{p/i} \longrightarrow 0,
				\quad (k \longrightarrow \infty,\ \forall 0 \leq s \leq t \leq T)
			\end{align}
			かつ,定理\ref{thm:property_of_the_completion_of_the_projective_norm}及び
			双線型写像$\otimes_{j,i}:V^{\otimes j} \times V^{\otimes i-j} 
			\longrightarrow V^{\otimes i}$
			の定義(\refeq{eq:def_of_otimes_for_completion_V_tensor_k})より
			\begin{align}
				\compcrossnorm{X^j_{s,u} \otimes_{j,i} X^{i-j}_{u,t}}{i}
				= \compcrossnorm{X^j_{s,u}}{j} \compcrossnorm{X^{i-j}_{u,t}}{i-j},
				\quad (\forall 0 \leq j \leq i)
			\end{align}
			が満たされているから,任意の$0 \leq s \leq t \leq T$に対して
			\begin{align}
				\compcrossnorm{X^i_{s,t} - \sum_{j=0}^i X^j_{s,u} \otimes_{j,i} X^{i-j}_{u,t}}{i}
				&\leq \compcrossnorm{X^i_{s,t} - X^{k,i}_{s,t}}{i}
					+ \compcrossnorm{\sum_{j=0}^i X^j_{s,u} \otimes_{j,i} X^{i-j}_{u,t} 
					- \sum_{j=0}^i X^{k,j}_{s,u} \otimes_{j,i} X^{k,i-j}_{u,t}}{i} \\
				&\leq \compcrossnorm{X^i_{s,t} - X^{k,i}_{s,t}}{i}
					+ \sum_{j=0}^i \compcrossnorm{X^j_{s,u} - X^{k,j}_{s,u}}{j}
						\compcrossnorm{X^{i-j}_{u,t}}{i-j} \\
					&\qquad + \sum_{j=0}^i \compcrossnorm{X^{k,j}_{s,u}}{j}
						\compcrossnorm{X^{i-j}_{u,t} - X^{k,i-j}_{u,t}}{i-j} \\
				&\longrightarrow 0,
				\quad (k \longrightarrow \infty)
			\end{align}
			が従い(\refeq{eq:thm_p_rough_path_complete_dist_1})を得る.
			則ち$X$は$p$-ラフパスであり,(\refeq{eq:thm_p_rough_path_complete_dist_2})より
			$d_p(X^k,X) \longrightarrow 0\ (k \longrightarrow \infty)$が成り立つ.
			\QED
	\end{description}
\end{prf}