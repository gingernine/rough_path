\section{The notion of rough path}
	$(V,\Norm{\cdot}{})$を$\R$上のBanach空間とする$(V \neq \{0\})$.また
	$\otimes_a$により代数的テンソル積,或はその標準写像を表す.
	$k \geq 2$の場合,$k$重テンソル積$V^{\otimes_a k} = V \otimes_a \cdots \otimes_a V$に
	プロジェクティブノルム$\projectivenorm{\cdot}{k}$を導入し,
	その完備拡大を$(V^{\otimes k},\compcrossnorm{\cdot}{k})$と書く
	\footnote{
		$V$が有限次元なら$V^{\otimes_a k}$も有限次元であるから
		$V^{\otimes k} = V^{\otimes_a k},\ \compcrossnorm{\cdot}{k} = \projectivenorm{\cdot}{k}$
		でよい.しかし一般に$\left( V^{\otimes_a k},\projectivenorm{\cdot}{k} \right)$
		は完備ではない(\cite{key10} (p. 17), \cite{key9}).
	}.
	$k=0,1$に対しては$V^{\otimes 0} \coloneqq \R,\ V^{\otimes 1} \coloneqq V$とし,
	$\compcrossnorm{\cdot}{0} = \projectivenorm{\cdot}{0} \coloneqq \mbox{$\R$の絶対値}$,及び
	$\compcrossnorm{\cdot}{1} = \projectivenorm{\cdot}{1} \coloneqq \Norm{\cdot}{}$と定める.
	定理\ref{thm:tensor_product_with_scalar}
	と定理\ref{thm:associativity_of_tensor_products}
	により,任意の$0 \leq j \leq k$に対し
	$V^{\otimes_a k}$と$V^{\otimes_a j} \otimes_a V^{\otimes_a k-j}$は線型同型となる.
	この同型写像を
	\begin{align}
		F_{j,k}:V^{\otimes_a k} \longrightarrow V^{\otimes_a j} \otimes_a V^{\otimes_a k-j},
		\quad 0 \leq j \leq k
	\end{align}
	書けば,$F_{j,k}$は
	\begin{align}
		F_{0,k}(v) &= 1 \otimes_a v, && (\forall v \in V^{\otimes_a k}), \\
		F_{j,k}(v_1 \otimes_a \cdots \otimes_a v_k) 
			&= (v_1 \otimes_a \cdots \otimes_a v_{j}) \otimes_a (v_{j+1} \otimes_a \cdots \otimes_a v_k), 
			&& (\forall v_1 \otimes_a \cdots \otimes_a v_k \in V^{\otimes_a k},\ 1 \leq j \leq k-1), \\
		F_{k,k}(v) &= v \otimes_a 1, && (\forall v \in V^{\otimes_a k})
	\end{align}
	を満たす.また$V^{\otimes_a j} \otimes_a V^{\otimes_a k-j}$上にもプロジェクティブノルムを導入し,
	これを$\pi_{j,k}$と書く.
	\begin{screen}
		\begin{thm}
			このとき次式が成立する.特に,$F_{j,k},\ (0 \leq j \leq k)$は等長同型である.
			\begin{align}
				\pi_k \circ F^{-1}_{j,k} = \pi_{j,k}, \quad 0 \leq j \leq k.
			\end{align}
			
		\end{thm}
	\end{screen}
	
	\begin{prf}\mbox{}
		\begin{description}
			\item[第一段]
				$j=0$のとき,任意の$v \in V^{\otimes_a k}$に対し
				\begin{align}
					\projectivenorm{F_{0,k}(v)}{0,k}
					= \projectivenorm{1 \otimes_a v}{0,k}
					= \projectivenorm{1}{0}\projectivenorm{v}{k}
					= \projectivenorm{v}{k}
				\end{align}
				が成り立ち$\pi_k \circ F^{-1}_{0,k} = \pi_{0,k}$を得る.
				同様にして$\pi_k \circ F^{-1}_{k,k} = \pi_{k,k}$も出る.
				
			\item[第二段]
				$\pi_k \circ F^{-1}_{j,k} \leq \pi_{j,k},\ (1 \leq j \leq k-1)$が成り立つことを示す.
				$v \in V^{\otimes_a j} \otimes_a V^{\otimes_a k-j}$の分割
				\begin{align}
					v = \sum_{r} u^r \otimes_a v^r,
					\quad (u^r \in V^{\otimes_a j},\ v^r \in V^{\otimes_a k-j})
				\end{align}
				を任意に取り,一旦固定する.このとき$u^r,v^r$の任意の分割
				\begin{align}
					u^r = \sum_{n(r)} u^{n(r)}_1 \otimes_a \cdots \otimes_a u^{n(r)}_j, 
					\quad v^r = \sum_{m(r)} v^{m(r)}_{j+1} \otimes_a \cdots \otimes_a v^{m(r)}_{k},
					\quad (v^{n(r)}_i,v^{m(r)}_i \in V)
				\end{align}
				に対して
				\begin{align}
					\projectivenorm{F^{-1}_{j,k}(v)}{k}
					&\leq \sum_{r} \sum_{n(r),m(r)} \projectivenorm{u^{n(r)}_1 \otimes_a \cdots \otimes_a u^{n(r)}_j \otimes_a v^{m(r)}_{j+1} \otimes_a \cdots \otimes_a v^{m(r)}_{k}}{k} \\
					&= \sum_{r} \sum_{n(r),m(r)} \Norm{u^{n(r)}_1}{} \cdots \Norm{u^{n(r)}_j}{} \Norm{v^{m(r)}_{j+1}}{} \cdots \Norm{v^{m(r)}_k}{} \\
					&= \sum_{r} \left\{ \sum_{n(r)} \Norm{u^{n(r)}_1}{} \cdots \Norm{u^{n(r)}_j}{} \right\} \left\{ \sum_{m(r)} \Norm{v^{m(r)}_{j+1}}{} \cdots \Norm{v^{m(r)}_k}{} \right\} 
				\end{align}
				が成り立つから,分割の任意性と定理\ref{thm:expression_of_projective_norm}より
				\begin{align}
					\projectivenorm{F^{-1}_{j,k}(v)}{k} 
					\leq \sum_r \projectivenorm{u^r}{j}\projectivenorm{v^r}{k-j}
				\end{align}
				を得る.$v$の分割について下限を取れば,
				再び定理\ref{thm:expression_of_projective_norm}により
				\begin{align}
					\projectivenorm{F^{-1}_{j,k}(v)}{k} \leq \projectivenorm{v}{j,k}
				\end{align}
				が出る.
			
			\item[第三段]
				$\pi_k \circ F^{-1}_{j,k} \geq \pi_{j,k},\ (1 \leq j \leq k-1)$が成り立つことを示す.
				$v \in V^{\otimes_a k}$の任意の分割
				\begin{align}
					v = \sum_n v^n_1 \otimes_a \cdots \otimes_a v^n_k,
					\quad (v^n_i \in V,\ i=1,\cdots,k)
				\end{align}
				を取れば,
				\begin{align}
					\projectivenorm{F_{j,k}(v)}{j,k}
					&\leq \sum_n \projectivenorm{\left( v^n_1 \otimes_a \cdots \otimes_a v^n_j \right) \otimes_a \left( v^n_j \otimes_a \cdots \otimes_a v^n_k \right)}{j,k} \\
					&= \sum_n \projectivenorm{v^n_1 \otimes_a \cdots \otimes_a v^n_j}{j}
						\projectivenorm{v^n_j \otimes_a \cdots \otimes_a v^n_k}{k-j} \\
					&= \sum_n \Norm{v^n_1}{} \cdots \Norm{v^n_k}{}
				\end{align}
				が成立する.従って定理\ref{thm:expression_of_projective_norm}より
				\begin{align}
					\projectivenorm{F_{j,k}(v)}{j,k} \leq \projectivenorm{v}{k}
				\end{align}
				が得られる.
				\QED
		\end{description}
	\end{prf}
	
	
	$V^{\otimes_a i}$の$V^{\otimes i}$への等長埋め込みを$J_i$で表し
	($J_i$の終集合を$J_i V^{\otimes_a i}$と考える
	\footnote{
		$J_i$の終集合を$J_i V^{\otimes_a i}$と考えれば全単射であるから
		逆写像$J_i^{-1}$が存在する.
	}.$i=0,1$の場合$J_i$は恒等写像),
	\begin{align}
		J_j V^{\otimes_a j} \times J_{k-j} V^{\otimes_a k-j} \ni (u,v)
		& \longmapsto ( J_j^{-1}u,J_{k-j}^{-1}v ) &&\in V^{\otimes_a j} \times V^{\otimes_a k-j} \\
		& \longmapsto F_{j,k}^{-1} (J_j^{-1}u \otimes_a J_{k-j}^{-1}v) &&\in V^{\otimes_a k} \\
		& \longmapsto J_k F_{j,k}^{-1} (J_j^{-1}u \otimes_a J_{k-j}^{-1}v) &&\in V^{\otimes k}
		\label{bilinear_map_on_algebraic_Banach_tensor_products}
	\end{align}
	の対応関係により定まる写像$:J_j V^{\otimes_a j} \times J_{k-j} V^{\otimes_a k-j}
	\longrightarrow V^{\otimes k}$を$\varphi_{j,k}$と書けば,$\varphi_{j,k}$は有界双線型写像である.
	実際,$\otimes_a$の双線型性と埋め込み及び$F_{j,k}^{-1}$の線型性より
	$\varphi_{j,k}$の双線型性が従い,また
	\begin{align}
		\compcrossnorm{\varphi_{j,k}(u,v)}{k}
		&= \projectivenorm{F_{j,k}^{-1} (J_j^{-1}u \otimes_a J_{k-j}^{-1}v)}{k} \\
		&= \projectivenorm{J_j^{-1}u \otimes_a J_{k-j}^{-1}v}{j,k} \\
		&= \projectivenorm{J_j^{-1}u}{j} \projectivenorm{J_{k-j}^{-1}v}{k-j} \\
		&= \compcrossnorm{u}{j} \compcrossnorm{v}{k-j}
	\end{align}
	が任意の$(u,v) \in J_j V^{\otimes_a j} \times J_{k-j} V^{\otimes_a k-j}$に対して
	成り立つから$\Norm{\varphi_{j,k}}{\ContLn{J_j V^{\otimes_a j} \times J_{k-j} V^{\otimes_a k-j}}{V^{\otimes k}}{2}} = 1$を得る.
	従って,定理\ref{thm:expansion_of_multilinear_mapping}より
	$\varphi_{j,k}$は$V^{\otimes j} \times V^{\otimes k-j}$上の或るただ一つの双線型写像
	$\psi_{j,k}$にノルム保存拡張される.
	
	\begin{screen}
		\begin{thm}\label{thm:property_of_the_completion_of_the_projective_norm}
			$0 \leq j \leq k$とする.
			このとき,$\psi_{j,k}:V^{\otimes j} \times V^{\otimes k-j} \longrightarrow V^{\otimes k}$
			は次を満たす:
			\begin{align}
				\compcrossnorm{\psi_{j,k}(u,v)}{k} = \compcrossnorm{u}{j} \compcrossnorm{v}{k-j},
				\quad (\forall (u,v) \in V^{\otimes j} \times V^{\otimes k-j}).
			\end{align}
		\end{thm}
	\end{screen}
	
	\begin{prf}
		$(u,v)$に直積ノルムで収束する点列$(u_n,v_n) \in J_j V^{\otimes_a j} \times J_{k-j} V^{\otimes_a k-j}\ (n=1,2,\cdots)$を取れば
		\begin{align}
			\compcrossnorm{\varphi_{j,k}(u_n,v_n) - \psi_{j,k}(u,v)}{k} \longrightarrow 0,
			\quad (n \longrightarrow \infty)
		\end{align}
		が成り立つ.また
		\begin{align}
			\left| \compcrossnorm{u_n}{j} \compcrossnorm{v_n}{k-j} - 
			\compcrossnorm{u}{j} \compcrossnorm{v}{k-j} \right|
			&\leq \left| \compcrossnorm{u_n}{j} \compcrossnorm{v_n}{k-j} - 
			\compcrossnorm{u_n}{j} \compcrossnorm{v}{k-j} \right| 
			+ \left| \compcrossnorm{u_n}{j} \compcrossnorm{v}{k-j} - 
			\compcrossnorm{u}{j} \compcrossnorm{v}{k-j} \right| \\
			&\leq \compcrossnorm{u_n}{j} \compcrossnorm{v_n - v}{k-j} 
			+ \compcrossnorm{u_n - u}{j} \compcrossnorm{v}{k-j} \\
			&\longrightarrow 0, \quad (n \longrightarrow \infty)
		\end{align}
		も成立するから
		\begin{align}
			\left|\, \compcrossnorm{\psi_{j,k}(u,v)}{k} - \compcrossnorm{u}{j} \compcrossnorm{v}{k-j}\, \right|
			&\leq \compcrossnorm{\varphi_{j,k}(u_n,v_n) - \psi_{j,k}(u,v)}{k}
				+ \left| \compcrossnorm{u_n}{j} \compcrossnorm{v_n}{k-j} - 
			\compcrossnorm{u}{j} \compcrossnorm{v}{k-j} \right| \\
			& \longrightarrow 0, \quad (n \longrightarrow \infty)
		\end{align}
		が従い$\compcrossnorm{\psi_{j,k}(u,v)}{k} = \compcrossnorm{u}{j} \compcrossnorm{v}{k-j}$
		が得られる.
		\QED
	\end{prf}
	
	$T(V) \coloneqq \bigoplus_{k=0}^{\infty} V^{\otimes k}$
	とおく.また上で定めた双線型写像$\psi_{j,k}$を$\otimes_{j,k}$と書き直す:
	\begin{align}
		u \otimes_{j,k} v = \psi_{j,k}(u,v),
		\quad (\forall (u,v) \in V^{\otimes j} \times V^{\otimes k-j},\ 0 \leq j \leq k,\ k \geq 0).
		\label{eq:def_of_otimes_for_completion_V_tensor_k}
	\end{align}
	このとき,任意の$a=(a_k)_{k=0}^{\infty},\ b=(b_k)_{k=0}^{\infty} \in T(V)$に対し
	\begin{align}
		c_k \coloneqq \sum_{j=0}^{k} a_j \otimes_{j,k} b_{k-j},
		\quad (k=0,1,2,\cdots)
	\end{align}
	で$c_k \in V^{\otimes k}$を定めれば,
	有限個の$k$を除いて$c_k = 0$となり$c = (c_k)_{k=0}^{\infty} \in T(V)$が満たされる.
	これで定まる二項算法を
	\begin{align}
		c = a \otimes b
	\end{align}
	と書けば,以下の主張により$T(V)$は$\otimes$を乗法として代数となる.
	
	\begin{screen}
		\begin{lem}\label{lem:associative_law}
			任意の$a,b,c \in T(V)$及び$0 \leq r \leq j \leq k$に対して
			\begin{align}
				\left( a_r \otimes_{r,j} b_{j-r} \right) \otimes_{j,k} c_{k-j}
				= a_r \otimes_{r,k} \left( b_{j-r} \otimes_{j-r,k-r} c_{k-j} \right)
				\label{eq:thm_otimes_is_a_multiplication_1}
			\end{align}
			が成立する.
		\end{lem}
	\end{screen}
	
	\begin{prf}\mbox{}
		\begin{description}
			\item[第一段]
				$a_r \in J_r V^{\otimes_a r},\ b_{j-r} \in J_{j-r} V^{\otimes_a j-r},
				\ c_{k-j} \in J_{k-j} V^{\otimes_a k-j}$として
				(\refeq{eq:thm_otimes_is_a_multiplication_1})を示す.
				この場合,(\refeq{bilinear_map_on_algebraic_Banach_tensor_products})の対応関係により
				\begin{align}
					\left( a_r \otimes_{r,j} b_{j-r} \right) \otimes_{j,k} c_{k-j}
					&= J_k F_{j,k}^{-1}\left( F_{r,j}^{-1} \left( J_r^{-1}a_r \otimes_a J_{j-r}^{-1} b_{j-r} \right) \otimes_a J_{k-j}^{-1} c_{k-j} \right), \\
					a_r \otimes_{r,k} \left( b_{j-r} \otimes_{j-r,k-r} c_{k-j} \right)
					&= J_k F_{r,k}^{-1}\left( J_r^{-1}a_r \otimes_a F_{j-r,k-j}^{-1} \left( J_{j-r}^{-1} b_{j-r} \otimes_a J_{k-j}^{-1} c_{k-j} \right) \right)
				\end{align}
				となる.ここで
				\begin{align}
					J_r^{-1} a_r = \sum_{i_1 = 1}^{I_1} v_1^{i_1} \otimes_a \cdots \otimes_a v_r^{i_1},
					\quad J_{j-r}^{-1} b_{j-r} = \sum_{i_2 = 1}^{I_2} v_{r+1}^{i_2} \otimes_a \cdots \otimes_a v_j^{i_2},
					\quad J_{k-j}^{-1} c_{k-j} = \sum_{i_3 = 1}^{I_3} v_{j+1}^{i_3} \otimes_a \cdots \otimes_a v_k^{i_3}
				\end{align}
				と表現できるから
				\begin{align}
					&F_{j,k}^{-1}\left( F_{r,j}^{-1} \left( J_r^{-1}a_r \otimes_a J_{j-r}^{-1} b_{j-r} \right) \otimes_a J_{k-j}^{-1} c_{k-j} \right) \\
					&\qquad = \sum_{i_1=1}^{I_1}\sum_{i_2=1}^{I_2}\sum_{i_3=1}^{I_3} 
						v_1^{i_1} \otimes_a \cdots \otimes_a v_r^{i_1}
						\otimes_a v_{r+1}^{i_2} \otimes_a \cdots \otimes_a v_j^{i_2}
						\otimes_a v_{j+1}^{i_3} \otimes_a \cdots \otimes_a v_k^{i_3} \\
					&\qquad = 	F_{r,k}^{-1}\left( J_r^{-1}a_r \otimes_a F_{j-r,k-j}^{-1} \left( J_{j-r}^{-1} b_{j-r} \otimes_a J_{k-j}^{-1} c_{k-j} \right) \right)
				\end{align}
				が成立し,$J_k$の単射性から(\refeq{eq:thm_otimes_is_a_multiplication_1})が得られる.
			
			\item[第二段]
				一般に$a_r \in V^{\otimes r},\ b_{j-r} \in V^{\otimes j-r},
				\ c_{k-j} \in V^{\otimes k-j}$の場合,
				\begin{align}
					\compcrossnorm{a_r^n - a_r}{r} \longrightarrow 0,
					\quad \compcrossnorm{b_{j-r}^n - b_{j-r}}{j-r} \longrightarrow 0,
					\quad \compcrossnorm{c_{k-j}^n - c_{k-j}}{k-j} \longrightarrow 0,
					\quad (n \longrightarrow \infty)
				\end{align}
				を満たす点列$\left\{ a_r^n \right\}_{n=1}^\infty \subset J_r V^{\otimes_a r},\ 
				\left\{ b_{j-r}^n \right\}_{n=1}^\infty \subset J_{j-r} V^{\otimes_a j-r},\ 
				\left\{ c_{k-j}^n \right\}_{n=1}^\infty \subset J_{k-j} V^{\otimes_a k-j}$
				を取れば,前段の結果より
				\begin{align}
					&\compcrossnorm{\left( a_r \otimes_{r,j} b_{j-r} \right) \otimes_{j,k} c_{k-j}
					- a_r \otimes_{r,k} \left( b_{j-r} \otimes_{j-r,k-r} c_{k-j} \right)}{k} \\
					&\qquad \leq \compcrossnorm{\left( a_r \otimes_{r,j} b_{j-r} \right) \otimes_{j,k} c_{k-j} - \left( a^n_r \otimes_{r,j} b^n_{j-r} \right) \otimes_{j,k} c^n_{k-j}}{k} \\
					&\qquad\qquad + \compcrossnorm{a_r \otimes_{r,k} \left( b_{j-r} \otimes_{j-r,k-r} c_{k-j} \right) - a^n_r \otimes_{r,k} \left( b^n_{j-r} \otimes_{j-r,k-r} c^n_{k-j} \right)}{k} \\
					&\qquad \leq \compcrossnorm{a_r \otimes_{r,j} b_{j-r}}{j} \compcrossnorm{c_{k-j} - c^n_{k-j}}{k-j} 
						+ \compcrossnorm{a_r \otimes_{r,j} b_{j-r} - a^n_r \otimes_{r,j} b^n_{j-r}}{j} \compcrossnorm{c^n_{k-j}}{k-j} \\
					&\qquad\qquad + \compcrossnorm{a_r}{r} \compcrossnorm{b_{j-r} \otimes_{j-r,k-r} c_{k-j} - b^n_{j-r} \otimes_{j-r,k-r} c^n_{k-j}}{k-r}
						+ \compcrossnorm{a_r - a_r^n}{r} \compcrossnorm{b^n_{j-r} \otimes_{j-r,k-r} c^n_{k-j}}{k-r} \\
					&\qquad \leq \compcrossnorm{a_r}{r} \compcrossnorm{b_{j-r}}{j-r} \compcrossnorm{c_{k-j} - c^n_{k-j}}{k-j} 
						+ \left\{ \compcrossnorm{a_r-a^n_r}{r} \compcrossnorm{b_{j-r}}{j-r} 
						+ \compcrossnorm{a^n_r}{r} \compcrossnorm{b_{j-r} - b^n_{j-r}}{j-r} \right\} \compcrossnorm{c^n_{k-j}}{k-j} \\
					&\qquad\qquad + \compcrossnorm{a_r}{r} 
						\left\{ \compcrossnorm{b_{j-r} - b^n_{j-r}}{j-r} \compcrossnorm{c_{k-j}}{k-j} 
						+ \compcrossnorm{b^n_{j-r}}{j-r} \compcrossnorm{c_{k-j}-c^n_{k-j}}{k-j} \right\}
						+ \compcrossnorm{a_r - a_r^n}{r} \compcrossnorm{b^n_{j-r}}{j-r} \compcrossnorm{c^n_{k-j}}{k-j} \\
					&\qquad \longrightarrow 0
					\quad (n \longrightarrow \infty) 
				\end{align}
				が従い(\refeq{eq:thm_otimes_is_a_multiplication_1})が出る.
				\QED
		\end{description}	
	\end{prf}
	
	\begin{screen}
		\begin{thm}[$\otimes$は$T(V)$の乗法となる]\label{thm:otimes_is_a_multiplication}
			$\otimes$は$T(V)$において結合則を満たす双線型写像である.
		\end{thm}
	\end{screen}
	
	\begin{prf}
		$\otimes$の双線型性は各$\otimes_{j,k}$の双線型性より従う.
		また任意の$a,b,c \in T(V)$に対し,補題\ref{lem:associative_law}より
		\begin{align}
			&((a \otimes b) \otimes c)_k
			= \sum_{j=0}^k (a \otimes b)_j \otimes_{j,k} c_{k-j} \\
			&= \sum_{j=0}^k \sum_{r=0}^j \left( a_r \otimes_{r,j} b_{j-r} \right) \otimes_{j,k} c_{k-j}
			= \sum_{j=0}^k \sum_{r=0}^j a_r \otimes_{r,k} \left( b_{j-r} \otimes_{j-r,k-r} c_{k-j} \right)
			= \sum_{r=0}^k \sum_{j=r}^k a_r \otimes_{r,k} \left( b_{j-r} \otimes_{j-r,k-r} c_{k-j} \right) \\
			&= \sum_{r=0}^k a_r \otimes_{r,k} (b \otimes c)_{k-r} \\
			&= (a \otimes (b \otimes c))_k,
			\quad (\forall k = 0,1,2,\cdots).
		\end{align}
		が成立する.
		\QED
	\end{prf}
	
	\begin{screen}
		\begin{thm}[$T(V)$の単位元]
			任意の$k \geq 0$,および任意の$\alpha \in \R$と$v \in V^{\otimes k}$に対し
			\begin{align}
				\alpha \otimes_{0,k} v = v \otimes_{k,k} \alpha = \alpha v
				\label{eq:thm_identity_element_of_T_V}
			\end{align}
			が成立する.特に,$e_0 \coloneqq 1,\ e_k \coloneqq 0\ (k \geq 0)$
			として定める$e=(e_k)_{k=1}^\infty \in T(V)$は$\otimes$に関する単位元である:
			\begin{align}
				\mbox{i.e.}\quad
				e \otimes a = a \otimes e = a,
				\quad (\forall a \in T(V))
			\end{align}
		\end{thm}
	\end{screen}
	
	\begin{prf}
		$v \in J_k V^{\otimes_a k}$の場合は
		(\refeq{bilinear_map_on_algebraic_Banach_tensor_products})
		の対応関係により
		\begin{align}
			\alpha \otimes_{0,k} v
			= J_k F_{0,k}^{-1}\left( \alpha \otimes_a J_k^{-1}v \right)
			= J_k \left( \alpha J_k^{-1}v \right)
			= J_k J_k^{-1} \alpha v
			= \alpha v
		\end{align}
		が成立する.一般の$v \in V^{\otimes k}$に対しては,
		$\lim_{n \to \infty} \compcrossnorm{v-v_n}{k} = 0$を満たす
		$\{v_n\}_{n=1}^\infty \subset J_k V^{\otimes_a k}$を取れば
		\begin{align}
			\compcrossnorm{\alpha v - \alpha \otimes_{0,k} v}{k}
			\leq \compcrossnorm{\alpha v - \alpha v_n}{k}
				+ \compcrossnorm{\alpha \otimes_{0,k} v_n - \alpha \otimes_{0,k} v}{k}
			= 2|\alpha| \compcrossnorm{v-v_n}{k}
			\longrightarrow 0
			\quad (n \longrightarrow \infty)
		\end{align}
		が成立し$\alpha v = \alpha \otimes_{0,k} v$が得られる.
		同様にして$\alpha v = v \otimes_{k,k} \alpha$も成り立つ.
		従って,任意の$a \in T(V)$に対し
		\begin{align}
			(e \otimes a)_k
			&= \sum_{j=0}^k e_j \otimes_{j,k} a_{k-j}
			= 1 \otimes_{0,k} a_k \\
			&= a_k
			= a_k \otimes_{k,k} 1
			= \sum_{j=0}^k a_j \otimes_{j,k} e_{k-j}
			= (a \otimes e)_k,
			\quad (k=0,1,2,\cdots)
		\end{align}
		が満たされ$e \otimes a = a \otimes e = a$が出る.
		\QED
	\end{prf}
	
	$n \geq 0$に対して
	\begin{align}
		T^{(n)}(V) \coloneqq \bigoplus_{k=0}^{n} V^{\otimes k}
	\end{align}
	とおき,$T(V)$の場合と同様に乗法$\otimes$を
	\begin{align}
		a \otimes b,\quad
		\Biggl((a \otimes b)_k \coloneqq \sum_{j=0}^{k} a_j \otimes_{j,k} b_{k-j},\ k=1,\cdots,n \Biggr),
		\quad a,b \in T^{(n)}(V)
	\end{align}
	により定め,次の直積ノルムでノルム位相を導入する:
	\begin{align}
		\compcrossnorm{a}{} \coloneqq \sum_{k=0}^n \compcrossnorm{a_k}{k},
		\quad (a = (a_k)_{k=0}^n \in T^{(n)}(V)).
		\label{eq:def_norm_on_truncated_tensor_algebra}
	\end{align}
	また,写像$X:\Delta_T \longrightarrow T^{(n)}(V)$に対して
	$X_{s,t} = (X^0_{s,t},\cdots,X^n_{s,t}),\ ((s,t) \in \Delta_T)$と表記し
	\begin{align}
		C_0 \left(\Delta_T,T^{(n)}(V) \right)
		\coloneqq \Set{X:\Delta_T \longrightarrow T^{(n)}(V)}{\mbox{continuous},\ X^0 \equiv 1}
	\end{align}
	を定める.
	
	\begin{screen}
		\begin{dfn}[有限$p$-変動]
			$p \geq 1$とする.$X:\Delta_T \longrightarrow T^{(n)}(V)$に対して
			或るコントロール関数$\omega$が存在して
			\begin{align}
				\compcrossnorm{X^i_{s,t}}{i} \leq \omega(s,t)^{i/p},
				\quad (\forall i=1,\cdots,n,\ \forall (s,t) \in \Delta_T)
				\label{eq:def_finite_p_variation}
			\end{align}
			を満たすとき,$X$は有限$p$-変動(finite $p$-variation)であるという.\footnotemark
		\end{dfn}
	\end{screen}
	\footnotetext{
		$X:\Delta_T \longrightarrow T^{(n)}(V)$が有限$p$-変動であることと
		$\Norm{X}{p}$が有限であることは一致しない.
		実際,後述のシグネチャー$X=(X^0,\cdots,X^n)$
		は有限$p$-変動であるが,その定義より$X^0 \equiv 1$が満たされているから
		\begin{align}
			\mbox{$D$の分割小区間の数}
			= \sum_D \compcrossnorm{X^0_{t_{i-1},t_i}}{0}^p
			\leq \sum_D \compcrossnorm{X_{t_{i-1},t_i}}{}^p
			\leq \Norm{X}{p}^p,
			\quad (\forall D \in \delta[0,T])
		\end{align}
		が成り立ち$\Norm{X}{p} = \infty$となる.
	}
	
	\begin{screen}
		\begin{dfn}[有限総$p$-変動]
			$p \geq 1$とする.$X \in C_0 \left(\Delta_T,T^{(n)}(V) \right)$
			が有限総$p$-変動(finite total $p$-variation)とは
			\begin{align}
				\Norm{X^i}{p/i} < \infty,
				\quad \forall i=1,\cdots,n
			\end{align}
			が満たされることをいう.また次の線型空間を定める:
			\begin{align}
				C_{0,p} \left(\Delta_T,T^{(n)}(V) \right)
				\coloneqq \Set{X \in C_0 \left(\Delta_T,T^{(n)}(V) \right)}{\mbox{$X$ has finite total $p$-variation}}	.
			\end{align}
		\end{dfn}
	\end{screen}
	
	\begin{screen}
		\begin{dfn}[乗法的汎関数]
			次の関係式(Chen's identity)を満たす$X \in C_0 \left(\Delta_T,T^{(n)}(V) \right)$
			を$n$次の乗法的汎関数(multiplicative functional of degree $n$)と呼ぶ:
			\begin{align}
				X_{s,u} \otimes X_{u,t} = X_{s,t},
				\quad (\forall 0 \leq s \leq u \leq t \leq T).
			\end{align}
		\end{dfn}
	\end{screen}
	
	\begin{screen}
		\begin{lem}\label{lem:multiplicative_functional_vanishes_on_diagonal}
			$X:\Delta_T \longrightarrow T^{(n)}(V)$が$X^0 \equiv 1$かつ
			Chen's identity を満たせば$X^k_{t,t} = 0,
			\ (0 \leq \forall t \leq T,\ 1 \leq \forall k \leq n)$.
		\end{lem}
	\end{screen}
	
	\begin{prf}
		任意に$t \in [0,T]$を取る.
		$X^k_{t,t} = \sum_{j=0}^{k} X^j_{t,t} \otimes_{j,k} X^{k-j}_{t,t}$
		と式(\refeq{eq:thm_identity_element_of_T_V})より,先ず
		\begin{align}
			X^1_{t,t} = X^0_{t,t} \otimes_{0,1} X^1_{t,t} + X^1_{t,t} \otimes_{1,1} X^0_{t,t}
			= X^1_{t,t} + X^1_{t,t}
		\end{align}
		が成り立ち$X^1_{t,t} = 0$が従う.同様に
		\begin{align}
			X^2_{t,t} = X^0_{t,t} \otimes_{0,2} X^2_{t,t} + X^1_{t,t} \otimes_{1,2} X^1_{t,t}
				+ X^2_{t,t} \otimes_{2,2} X^0_{t,t}
			= X^2_{t,t} + X^2_{t,t}
		\end{align}
		より$X^2_{t,t} = 0$が成立し,帰納的に$X^k_{t,t} = 0\ (1 \leq k \leq n)$が出る.
		\QED
	\end{prf}
	
	\begin{screen}
		\begin{thm}\label{thm:fin_p_var_and_fin_ttl_p_var_is_equiv_for_multiplicative}
			$n$次乗法的汎関数については
			有限$p$-変動であることと有限総$p$-変動であることは同値である($p \geq 1$).
		\end{thm}
	\end{screen}
	
	\begin{prf}
		$X \in C_0 \left(\Delta_T,T^{(n)}(V) \right)$を$n$次乗法的とする.
		$X$が有限総$p$-変動ならば,補題\ref{lem:multiplicative_functional_vanishes_on_diagonal}と
		定理\ref{thm:control_function_defined_by_p_variation}により
		\begin{align}
			\omega(s,t) \coloneqq \sum_{i=1}^n \Norm{X^i}{p/i,[s,t]}^{p/i},
			\quad ((s,t) \in \Delta_T)
		\end{align}
		で定める$\omega$はコントロール関数となる.このとき
		\begin{align}
			\compcrossnorm{X^i_{s,t}}{i}
			\leq \Norm{X^i}{p/i,[s,t]}
			\leq \omega(s,t)^{i/p},
			\quad (\forall i = 1,\cdots,n,\ \forall (s,t) \in \Delta_T)
		\end{align}
		が成り立つから$X$は有限$p$-変動である.逆に$X$が有限$p$-変動なら,
		(\refeq{eq:def_finite_p_variation})を満たす$\omega$に対し
		\begin{align}
			\sum_D \compcrossnorm{X^i_{t_{i-1},t_i}}{i}^{p/i}
			\leq \sum_D \omega(t_{i-1},t_i)
			\leq \omega(0,T),
			\quad (\forall D \in \delta[0,T],\ \forall i=1,\cdots,n)
		\end{align}
		が満たされ$\Norm{X^i}{p/i} < \infty,\ i=1,\cdots,n$が従うので$X$は有限総$p$-変動である.
		\QED
	\end{prf}
	
	実際に乗法的汎関数を構成する.有界変動な連続写像$x:[0,T] \longrightarrow V$に対して
	\begin{align}
		X^1_{s,t} \coloneqq x_t - x_s,
		\quad (\forall (s,t) \in \Delta_T)
	\end{align}
	とおけば,$X^1:\Delta_T \longrightarrow V$は連続かつ$\Norm{X^1}{1} < \infty$を満たす.
	このとき,
	\begin{screen}
		\begin{lem}\label{lem:def_integration_of_continuous_mapping_by_X_1}
			任意の$(s,t) \in \Delta_T$と
			連続写像$Y:\Delta_T \longrightarrow V^{\otimes k},\ (k \geq 1)$
			に対して次の積分が$V^{\otimes k+1}$で確定する:
			\begin{align}
				\int_s^t Y_{s,u} \otimes d x_u
				\coloneqq \lim_{|D| \to 0} \sum_{D} Y_{s,u_{i-1}} \otimes_{k,k+1} X^1_{u_{i-1},u_i},
				\quad (D \in \delta[s,t]).
				\label{eq:def_integration_of_continuous_mapping_by_X_1}
			\end{align}
		\end{lem}
	\end{screen}
	
	\begin{prf}
		$D=\{s=u_0 < \cdots <u_n= t\},\ D'=\{s=v_0 < \cdots <v_m= t\} \in \delta[s,t]$
		を任意に取り,共通細分を$D''=\{s=w_0 < \cdots < w_r = t\}$と表して
		\begin{align}
			\begin{cases}
				\tilde{Y}_{s,w_\ell} \coloneqq Y_{s,u_i}, & (u_i \leq w_\ell < u_{i+1}), \\
				\hat{Y}_{s,w_\ell} \coloneqq Y_{s,v_j}, & (v_j \leq w_\ell < v_{j+1}),
			\end{cases}
			\quad (\ell=0,1,\cdots,r)
		\end{align}
		で$\tilde{Y},\hat{Y}$を定めれば,
		定理\ref{thm:property_of_the_completion_of_the_projective_norm}より
		\begin{align}
			&\compcrossnorm{ \sum_D Y_{s,u_{i-1}} \otimes_{k,k+1} X^1_{u_{i-1},u_i} - 
				\sum_{D'} Y_{s,v_{j-1}} \otimes_{k,k+1} X^1_{v_{j-1},v_j}}{k+1} \\
			&\quad\leq \compcrossnorm{ \sum_D Y_{s,u_{i-1}} \otimes_{k,k+1} X^1_{u_{i-1},u_i} - 
				\sum_{D''} Y_{s,w_{\ell-1}} \otimes_{k,k+1} X^1_{w_{\ell-1},w_\ell}}{k+1}
				+ \compcrossnorm{ \sum_{D'} Y_{s,v_{j-1}} \otimes_{k,k+1} X^1_{v_{j-1},v_j} - 
				\sum_{D''} Y_{s,w_{\ell-1}} \otimes_{k,k+1} X^1_{w_{\ell-1},w_\ell}}{k+1} \\
			&\quad= \compcrossnorm{ \sum_{D''} \tilde{Y}_{s,w_{\ell-1}} \otimes_{k,k+1} X^1_{w_{\ell-1},w_\ell} - 
				\sum_{D''} Y_{s,w_{\ell-1}} \otimes_{k,k+1} X^1_{w_{\ell-1},w_\ell}}{k+1} 
				+ \compcrossnorm{ \sum_{D''} \hat{Y}_{s,w_{\ell-1}} \otimes_{k,k+1} X^1_{w_{\ell-1},w_\ell} - 
				\sum_{D''} Y_{s,w_{\ell-1}} \otimes_{k,k+1} X^1_{w_{\ell-1},w_\ell}}{k+1} \\
			&\quad= \compcrossnorm{ \sum_{D''} \left(\tilde{Y}_{s,w_{\ell-1}} - Y_{s,w_{\ell-1}} \right)
		 		\otimes_{k,k+1} X^1_{w_{\ell-1},w_\ell}}{k+1}
				+ \compcrossnorm{ \sum_{D''} \left( \hat{Y}_{s,w_{\ell-1}} - Y_{s,w_{\ell-1}} \right)
				\otimes_{k,k+1} X^1_{w_{\ell-1},w_\ell}}{k+1} \\
			&\quad\leq \sum_{D''} \compcrossnorm{\tilde{Y}_{s,w_{\ell-1}} - Y_{s,w_{\ell-1}}}{k} \compcrossnorm{X^1_{w_{\ell-1},w_\ell}}{1}
				+ \sum_{D''} \compcrossnorm{\hat{Y}_{s,w_{\ell-1}} - Y_{s,w_{\ell-1}}}{k} \compcrossnorm{X^1_{w_{\ell-1},w_\ell}}{1} \\
			&\quad\leq \max{k}{\compcrossnorm{\tilde{Y}_{s,w_{\ell-1}} - Y_{s,w_{\ell-1}}}{k}} 
				\Norm{X^1}{1,[s,t]} + \max{k}{\compcrossnorm{\hat{Y}_{s,w_{\ell-1}} - Y_{s,w_{\ell-1}}}{k}} \Norm{X^1}{1,[s,t]}
		\end{align}
		が成立する.いま,$[s,t] \ni u \longmapsto Y_{s,u}$は一様連続であるから,
		$|D|,|D'|\longrightarrow 0$として右辺は0に収束する.
		従って$|D_n| \longrightarrow 0\ (n \longrightarrow \infty)$を満たす細分列$D_n \in \delta[s,t]$を
		取れば,$\left(\sum_{D_n} Y_{s,u_{i-1}} \otimes_{k,k+1} X^1_{u_{i-1},u_i} \right)_{n=1}^{\infty}$
		は$V^{\otimes k+1}$のCauchy列となり$V^{\otimes k+1}$で収束する.別の細分列
		$\tilde{D}_m \in \delta[s,t],\ (|\tilde{D}_m| \longrightarrow 0)$を取っても
		\begin{align}
			\compcrossnorm{ \sum_{D_n} Y_{s,u_{i-1}} \otimes_{k,k+1} X^1_{u_{i-1},u_i} - 
				\sum_{\tilde{D}_m} Y_{s,v_{j-1}} \otimes_{k,k+1} X^1_{v_{j-1},v_j}}{k+1}
			\longrightarrow 0,
			\quad (n,m \longrightarrow \infty)
		\end{align}
		が成り立つから,極限は細分列に依らず確定する.従って補題の主張が得られる.
		\QED
	\end{prf}
	
	\begin{screen}
		\begin{lem}\label{lem:signature_of_path}
			(\refeq{eq:def_integration_of_continuous_mapping_by_X_1})の積分を
			\begin{align}
				Z_{s,t} \coloneqq \int_s^t Y_{s,u} \otimes d x_u,
				\quad (\forall (s,t) \in \Delta_T)
				\label{eq:signature_of_path_1}
			\end{align}
			とおけば,$Z:\Delta_T \longrightarrow V^{\otimes k+1}$は連続かつ有界変動である.
		\end{lem}
	\end{screen}
	
	\begin{prf}\mbox{}
		\begin{description}
			\item[第一段]
				$Z$が有界変動であることを示す.いま,任意に$(s,t) \in \Delta_T\ (s < t)$
				\footnote{
					$s=t$なら,$X^1_{s,t} = 0$より$Z_{s,t} = 0$が成り立つ.
				}を取る.
				\begin{align}
					M \coloneqq \sup{(x,y) \in \Delta_T}{\compcrossnorm{Y_{x,y}}{k}}
				\end{align}
				とおけば$Y$の連続性より$M < \infty$となり,
				任意の$\epsilon > 0$に対し或る$D \in \delta[s,t]$が存在して
				\begin{align}
					\compcrossnorm{Z_{s,t}}{k+1}
					\leq \epsilon + \compcrossnorm{\sum_{D} Y_{s,u_{i-1}} \otimes_{k,k+1} X^1_{u_{i-1},u_i}}{k+1}
					\leq \epsilon + \sum_D \compcrossnorm{Y_{s,u_{i-1}}}{k}
					\compcrossnorm{X^1_{t_{i-1},t_i}}{1} 
					\leq \epsilon + M \Norm{X^1}{1,[s,t]}
				\end{align}
				が成立する.$\epsilon > 0$と$(s,t)$の任意性より
				\begin{align}
					\compcrossnorm{Z_{s,t}}{k+1} \leq M \Norm{X^1}{1,[s,t]},
					\quad (\forall (s,t) \in \Delta_T)
				\end{align}
				が従い$\Norm{Z}{1} \leq M \Norm{X^1}{1}$\ (1-変動ノルム)を得る.
				
			\item[第二段]
				点$(s,s)\ (\forall s \in [0,T])$において$Z$が連続であること示す.
				実際,定理\ref{thm:control_function_defined_by_p_variation}より
				\begin{align}
					\Delta_T \ni (s,t) \longmapsto \Norm{X^1}{1,[s,t]}
					\label{map:lem_signature_of_path_1}
				\end{align}
				はコントロール関数であるから,
				\begin{align}
					&\compcrossnorm{Z_{t,u} - Z_{s,s}}{k+1}
					= \compcrossnorm{Z_{t,u}}{k+1}
					\leq M \Norm{X^1}{1,[t,u]}
					\longrightarrow 0 \quad ((t,u) \longrightarrow (s,s))
				\end{align}
				が成立し$Z$の$(s,s)$における連続性を得る.
			
			\item[第三段]
				$s < t$を満たす点$(s,t) \in \Delta_T$において$Z$が連続であること示す.
				いま,任意に$\epsilon > 0$を取れば,
				\begin{description}
					\item[(i)] (\refeq{map:lem_signature_of_path_1})がコントロール関数であるから,
						或る$\eta_1 > 0$が存在して$|s-a|,|t-b| < \eta_1$ならば
						\begin{align}
							\Norm{X^1}{1,[s \wedge a,s \vee a]} < \epsilon, 
							\quad \Norm{X^1}{1,[t \wedge b,t \vee b]} < \epsilon
						\end{align}
						が満たされる.
						
					\item[(ii)] 或る$\eta_2 > 0$が存在して$|s-a|,|t-b| < \eta_2$ならば
						$[s,t] \cap [a,b] \neq \emptyset$が満たされる.
						
					\item[(iii)] $Y$は$\Delta_T$上一様連続であるから,
						或る$\eta_3 > 0$が存在して$|s-a| < \eta_3$なら
						\begin{align}
							\sup{}{\Set{\compcrossnorm{Y_{s,u} - Y_{a,u}}{k}}{(s \vee a) \leq u \leq T}} < \epsilon
						\end{align}
						が満たされる.
						
					\item[(iv)] 補題\ref{lem:def_integration_of_continuous_mapping_by_X_1}
						より或る$\eta_4 > 0$が存在して$|D_1|,|D_2| < \eta_4,\ 
						(D_1 \in \delta[s,t],\ D_2 \in \delta[a,b])$なら
						\begin{align}
							\compcrossnorm{Z_{s,t} - \sum_{D_1} Y_{s,u_{i-1}} \otimes_{k,k+1} X^1_{u_{i-1},u_i}}{k+1} < \epsilon,
					\quad \compcrossnorm{Z_{a,b} - \sum_{D_2} Y_{a,v_{j-1}} \otimes_{k,k+1} X^1_{v_{j-1},v_j}}{k+1} < \epsilon
						\end{align}
						が満たされる.
				\end{description}
				ここで$\eta \coloneqq \eta_1 \wedge \eta_2 \wedge \eta_3$として,
				$|s-a|,|t-b| < \eta,\ |D_1|,|D_2| < \eta_4$を満たす$(a,b),D_1,D_2$を取り
				\begin{align}
					\Omega_3 &\coloneqq (D_1 \cup D_2) \cap [s,t] \cap [a,b], \\
					\Omega_1 &\coloneqq D_1 \cup \Omega_3,
						\quad \Omega_1^< \coloneqq \Omega_1 \cap [0,a],
						\quad \Omega_1^> \coloneqq \Omega_1 \cap [b,T], \\
					\Omega_2 &\coloneqq D_2 \cup \Omega_3,
						\quad \Omega_2^< \coloneqq \Omega_2 \cap [0,s],
						\quad \Omega_2^> \coloneqq \Omega_2 \cap [t,T]
				\end{align}
				とおけば,$\Omega_1 = \Omega_1^< \cup \Omega_3 \cup \Omega_1^>,\ 
				\Omega_2 = \Omega_2^< \cup \Omega_3 \cup \Omega_2^>$と分割できる.
				このとき(i)(ii)(iii)が満たされるから
				\begin{align}
					&\compcrossnorm{\sum_{\Omega_1} Y_{s,u_{i-1}} \otimes_{k,k+1} X^1_{u_{i-1},u_i}
						- \sum_{\Omega_2} Y_{a,v_{j-1}} \otimes_{k,k+1} X^1_{v_{j-1},v_j}}{k+1} \\
					&\quad\leq \compcrossnorm{\sum_{\Omega_3} Y_{s,u_{i-1}} \otimes_{k,k+1} X^1_{u_{i-1},u_i}
						- \sum_{\Omega_3} Y_{a,u_{i-1}} \otimes_{k,k+1} X^1_{u_{i-1},u_i}}{k+1} \\
					&\quad\qquad + \compcrossnorm{\sum_{\Omega_1^<} Y_{s,u_{i-1}} \otimes_{k,k+1} X^1_{u_{i-1},u_i}}{k+1} 
					+ \compcrossnorm{\sum_{\Omega_1^>} Y_{s,u_{i-1}} \otimes_{k,k+1} X^1_{u_{i-1},u_i}}{k+1} \\
					&\quad\qquad + \compcrossnorm{\sum_{\Omega_2^<} Y_{a,u_{i-1}} \otimes_{k,k+1} X^1_{u_{i-1},u_i}}{k+1} 
					+ \compcrossnorm{\sum_{\Omega_2^>} Y_{a,v_{j-1}} \otimes_{k,k+1} X^1_{v_{j-1},v_j}}{k+1} \\
					&\quad\leq \sum_{\Omega_3} \compcrossnorm{Y_{s,u_{i-1}} - Y_{a,u_{i-1}}}{k}
					\compcrossnorm{X^1_{u_{i-1},u_i}}{1}
					+ \sum_{\Omega_1^<} \compcrossnorm{Y_{s,u_{i-1}}}{k}
						\compcrossnorm{X^1_{u_{i-1},u_i}}{1}
					+ \sum_{\Omega_1^>} \compcrossnorm{Y_{s,u_{i-1}}}{k}
						\compcrossnorm{X^1_{u_{i-1},u_i}}{1} \\
					&\quad\qquad + \sum_{\Omega_2^<} \compcrossnorm{Y_{a,v_{j-1}}}{k}
						\compcrossnorm{X^1_{v_{j-1},v_i}}{1} 
					+ \sum_{\Omega_2^>} \compcrossnorm{Y_{a,v_{j-1}}}{k}
						\compcrossnorm{X^1_{v_{j-1},v_i}}{1} \\
					&\quad\leq \sup{u \in [s \vee a, T]}{\compcrossnorm{Y_{s,u} - Y_{a,u}}{k}}
						\Norm{X^1}{1}
						+ M \left( \Norm{X^1}{1,[s,s \vee a]}
						+ \Norm{X^1}{1,[a,s \vee a]} 
						+ \Norm{X^1}{1,[t,t \vee b]} 
						+ \Norm{X^1}{1,[b,t \vee b]} \right) \\
					&\quad< \left( \Norm{X^1}{1} + 4 M \right) \epsilon 
				\end{align}
				が成立し,(iv)と併せれば
				\begin{align}
					\compcrossnorm{Z_{s,t} - Z_{a,b}}{k+1} 
					< \left( \Norm{X^1}{1} + 4 M + 2\right)\epsilon,
					\quad (|s-a|,|t-b| < \eta)
				\end{align}
				が従い$Z$の$(s,t)$における連続性が得られる.
				\QED
		\end{description}	
	\end{prf}
	
	\begin{screen}
		\begin{dfn}[パスのシグネチャー]
			有界変動な連続写像$x:[0,T] \longrightarrow V$に対して
			$X^1_{s,t} \coloneqq x_t-x_s,\ (\forall (s,t) \in \Delta_T)$とおけば,
			補題\ref{lem:def_integration_of_continuous_mapping_by_X_1}と
			補題\ref{lem:signature_of_path}により逐次的に
			次を構成することができる:
			\begin{align}
				V^{\otimes i+1} \ni X^{i+1}_{s,t} \coloneqq \int_s^t X^i_{s,u} \otimes dx_u,
				\quad (\forall (s,t) \in \Delta_T,\ i=1,2,\cdots)
			\end{align}
			ここで
			$S(x)_{[s,t]} \coloneqq (1,X^1_{s,t},X^2_{s,t},\cdots)$
			とおき,特に$S(x)_{[0,T]}$をパス$x$のシグネチャー(signature of path $x$)と呼ぶ.
		\end{dfn}
	\end{screen}
	
	\begin{screen}
		\begin{thm}[逐次積分により定まる乗法的汎関数]
		\label{thm:signature_of_path_multiplicative}
			任意の$n \geq 1$に対し,$S(x)_{[s,t]}$の最初の
			$n+1$個\footnotemark
			の元の族を$S_n(x)_{[s,t]} = (X^0_{s,t},X^1_{s,t},\cdots,X^n_{s,t}),\ 
			(\forall (s,t) \in \Delta_T)$
			と書けば,$S_n(x)$は$n$次乗法的である.
		\end{thm}
	\end{screen}
	
	\begin{prf}
		補題\refeq{lem:signature_of_path}より
		$S_n(x) \in C_0 \left(\Delta_T,T^{(n)}(V) \right)$であるから,以下では
		任意の$k \geq 0$と$0 \leq s \leq u \leq t \leq T$に対して
		\begin{align}
			X^k_{s,t} = \sum_{j=0}^k X^j_{s,u} \otimes_{j,k} X^{k-j}_{u,t}.
			\label{eq:thm_signature_of_path_multiplicative_1}
		\end{align}
		が成り立つことを数学的帰納法により示す.
		\begin{description}
			\item[第一段]$k=1$の場合,$X^1_{s,t} = x_t - x_s,\ (0 \leq \forall s \leq \forall t \leq T)$より
				\begin{align}
					\sum_{j=0}^1 X^j_{s,u} \otimes_{j,k} X^{k-j}_{u,t}
					= X^0_{s,u} \otimes_{0,1} X^1_{u,t}
					+ X^1_{s,u} \otimes_{1,1} X^0_{u,t}
					= X^1_{u,t} + X^1_{s,u}
					= X^1_{s,t}
				\end{align}
				となる.
				
			\item[第二段]
				$k=m-1\ (m \geq 2)$まで式(\refeq{eq:thm_signature_of_path_multiplicative_1})
				が満たされると仮定するとき,任意の$D \in \delta[u,t]$に対して
				\begin{align}
					\sum_D \left( X^{m-1}_{s,u_{i-1}} - X^{m-1}_{u,u_{i-1}} \right) \otimes_{m-1,m} X^1_{u_{i-1},u_i}
					= \sum_{j=1}^{m-1} X^j_{s,u} \otimes_{j,m} \left\{ \sum_D X^{m-1-j}_{u,u_{i-1}} \otimes_{m-1-j,m-j} X^1_{u_{i-1},u_i} \right\}
					\label{eq:thm_signature_of_path_multiplicative_2}
				\end{align}
				が成り立つことを次段で示す.実際これが示されれば,$D' \in \delta[s,u]$として
				\begin{align}
					\compcrossnorm{X^m_{s,t} - \sum_{j=0}^m X^j_{s,u} \otimes_{j,m} X^{m-j}_{u,t}}{m}
					&\leq \compcrossnorm{X^m_{s,t} - \sum_{D'} X^{m-1}_{s,u_{i-1}} \otimes_{m-1,m} X^1_{u_{i-1},u_i} 
						- \sum_{D} X^{m-1}_{s,u_{i-1}} \otimes_{m-1,m} X^1_{u_{i-1},u_i}}{m} \\
						&\quad+ \compcrossnorm{\sum_{D'} X^{m-1}_{s,u_{i-1}} \otimes_{m-1,m} X^1_{u_{i-1},u_i} - X^m_{s,u}}{m}
						+ \compcrossnorm{\sum_{D} X^{m-1}_{u,u_{i-1}} \otimes_{m-1,m} X^1_{u_{i-1},u_i} - X^m_{u,t}}{m} \\
						&\quad+ \compcrossnorm{\sum_D \left( X^{m-1}_{s,u_{i-1}} - X^{m-1}_{u,u_{i-1}} \right) \otimes_{m-1,m} X^1_{u_{i-1},u_i} - 
						\sum_{j=1}^{m-1} X^j_{s,u} \otimes_{j,m} X^{m-j}_{u,t}}{m} \\
					&= \compcrossnorm{X^m_{s,t} - \sum_{D'} X^{m-1}_{s,u_{i-1}} \otimes_{m-1,m} X^1_{u_{i-1},u_i} 
						- \sum_{D} X^{m-1}_{s,u_{i-1}} \otimes_{m-1,m} X^1_{u_{i-1},u_i}}{m} \\
						&\quad+ \compcrossnorm{\sum_{D'} X^{m-1}_{s,u_{i-1}} \otimes_{m-1,m} X^1_{u_{i-1},u_i} - X^m_{s,u}}{m}
						+ \compcrossnorm{\sum_{D} X^{m-1}_{u,u_{i-1}} \otimes_{m-1,m} X^1_{u_{i-1},u_i} - X^m_{u,t}}{m} \\
						&\quad+ \compcrossnorm{\sum_{j=1}^{m-1} X^j_{s,u} \otimes_{j,m} \left\{ \sum_D X^{m-1-j}_{u,u_{i-1}} \otimes_{m-1-j,m-j} X^1_{u_{i-1},u_i} \right\} 
						- \sum_{j=1}^{m-1} X^j_{s,u} \otimes_{j,m} X^{m-j}_{u,t}}{m} \\
					&\leq \compcrossnorm{X^m_{s,t} - \sum_{D'} X^{m-1}_{s,u_{i-1}} \otimes_{m-1,m} X^1_{u_{i-1},u_i} 
						- \sum_{D} X^{m-1}_{s,u_{i-1}} \otimes_{m-1,m} X^1_{u_{i-1},u_i}}{m} \\
						&\quad+ \compcrossnorm{\sum_{D'} X^{m-1}_{s,u_{i-1}} \otimes_{m-1,m} X^1_{u_{i-1},u_i} - X^m_{s,u}}{m}
						+ \compcrossnorm{\sum_{D} X^{m-1}_{u,u_{i-1}} \otimes_{m-1,m} X^1_{u_{i-1},u_i} - X^m_{u,t}}{m} \\
						&\quad+  \sum_{j=1}^{m-1} \compcrossnorm{X^j_{s,u}}{j} 
						\compcrossnorm{\left\{ \sum_D X^{m-1-j}_{u,u_{i-1}} \otimes_{m-1-j,m-j} X^1_{u_{i-1},u_i} \right\} - X^{m-j}_{u,t}}{m-j} \\
					&\longrightarrow 0 \quad (|D'|,|D| \longrightarrow 0)
				\end{align}
				が従い,$k=m$について(\refeq{eq:thm_signature_of_path_multiplicative_1})が得られる.
				
			\item[第三段]
				式(\refeq{eq:thm_signature_of_path_multiplicative_2})を示す.仮定より
				$k=m-1$に対して(\refeq{eq:thm_signature_of_path_multiplicative_2})は満たされているから,
				補題\ref{lem:associative_law}と併せれば
				\begin{align}
					\sum_D \left( X^{m-1}_{s,u_{i-1}} - X^{m-1}_{u,u_{i-1}} \right) \otimes_{m-1,m} X^1_{u_{i-1},u_i}
					&= \sum_D \Biggl( \sum_{j=0}^{m-1} X^j_{s,u} \otimes_{j,m-1} X^{m-1-j}_{u,u_{i-1}} - X^{m-1}_{u,u_{i-1}} \Biggr) \otimes_{m-1,m} X^1_{u_{i-1},u_i} \\
					&= \sum_D \Biggl( \sum_{j=1}^{m-1} X^j_{s,u} \otimes_{j,m-1} X^{m-1-j}_{u,u_{i-1}} \Biggr) \otimes_{m-1,m} X^1_{u_{i-1},u_i} \\
					&= \sum_D \sum_{j=1}^{m-1} \left( X^j_{s,u} \otimes_{j,m-1} X^{m-1-j}_{u,u_{i-1}} \right) \otimes_{m-1,m} X^1_{u_{i-1},u_i} \\
					&= \sum_{j=1}^{m-1} \sum_D X^j_{s,u} \otimes_{j,m-j} \left( X^{m-1-j}_{u,u_{i-1}} \otimes_{m-1-j,m-j} X^1_{u_{i-1},u_i} \right) \\
					&= \sum_{j=1}^{m-1} X^j_{s,u} \otimes_{j,m} \left\{ \sum_D X^{m-1-j}_{u,u_{i-1}} \otimes_{m-1-j,m-j} X^1_{u_{i-1},u_i} \right\}
				\end{align}
				が成立する.
				\QED
		\end{description}
	\end{prf}
	
	\begin{screen}
		\begin{dfn}[$p$-ラフパス]
			$p \geq 1$とし,$p$を超えない最大の整数を$[p]$で表す.
			有限$p$-変動を持つ$[p]$次乗法的汎関数を
			$p$-ラフパス($p$-rough path)と呼び,その全体を$\Omega_p(V)$と書く:
			\begin{align}
				\Omega_p(V) 
				= \Set{X \in C_0\left( \Delta_T,T^{([p])}(V) \right)}{
					\mbox{$[p]$次乗法的,有限$p$-変動.}}.
			\end{align}
		\end{dfn}
	\end{screen}
	
	\begin{screen}
		\begin{thm}\label{thm:p_rough_path_complete_dist}
			$\Omega_p(V)$は次で定める距離により完備距離空間となる:
			\begin{align}
				d_p(X,Y) \coloneqq \max{1 \leq i \leq [p]}{\Norm{X^i - Y^i}{p/i}}.
			\end{align}
		\end{thm}
	\end{screen}
	
	$X \in \Omega_p(V)$は$X^0 \equiv 1$を満たすから,
	$\max{1 \leq i \leq [p]}{\Norm{\cdot}{p/i}}$は$\Omega_p(V)$においてノルムとはならない.
	
	\begin{prf}完備性を示す.
		\begin{description}
			\item[第一段] 極限を構成する.いま,
			任意の$X = (X^0,X^1,\cdots,X^{[p]}) \in \Omega_p(V)$に対して
			\begin{align}
				X^i \in B_{p/i,T}\left( V^{\otimes i} \right),\footnotemark
				\quad (\forall i=1,\cdots,[p]) 
			\end{align}
			\footnotetext{
				$B_{p/i,T}\left( V^{\otimes i} \right)$の定義は
				(p. \pageref{def:Banach_space_of_continuous_mapping_on_V})
				の(\refeq{def:Banach_space_of_continuous_mapping_on_V}).
			}
			が満たされるから,定理\ref{thm:B_p_T_Banach_space_2}より$\Omega_p(V)$の任意のCauchy列
			$\left( X^{(k)} = (X^{(k),0},\cdots,X^{(k),[p]}) \right)_{k=1}^{\infty}$
			に対して
			\begin{align}
				\Norm{X^{(k),i} - X^i}{p/i} \longrightarrow 0
				\quad (k \longrightarrow \infty,\ \forall i=1,\cdots,[p])
				\label{eq:thm_p_rough_path_complete_dist_2}
			\end{align}
			を満たす$X^i \in B_{p/i,T}\left( V^{\otimes i} \right)$が存在する.
			ここで$X:\Delta_T \longrightarrow T^{([p])}(V)$を
			\begin{align}
				X_{s,t} \coloneqq (1,X^1_{s,t},\cdots,X^n_{s,t}),
				\quad (\forall (s,t) \in \Delta_T)
			\end{align}
			により定めれば,(\refeq{eq:thm_p_rough_path_complete_dist_2})及び
			$X^i,\ i=1,\cdots,n$の連続性より
			$X \in C_{0,p} \left(\Delta_T,T^{([p])}(V) \right)$となる.
		
			\item[第二段] $X$が Chen's identity を満たすことを示す.
			これが示されれば,前段の結果と
			定理\ref{thm:fin_p_var_and_fin_ttl_p_var_is_equiv_for_multiplicative}
			より$X$は有限$p$-変動となり$X \in \Omega_p(V)$が従う.各$1 \leq i \leq [p]$に対して
			次が成立すればよい:
			\begin{align}
				X^i_{s,t} = \sum_{j=0}^i X^j_{s,u} \otimes_{j,i} X^{i-j}_{u,t},
				\quad (\forall 0 \leq s \leq u \leq t \leq T).
				\label{eq:thm_p_rough_path_complete_dist_1}
			\end{align}
			実際,(\refeq{eq:thm_p_rough_path_complete_dist_2})より
			\begin{align}
				\compcrossnorm{X^{(k),i}_{s,t} - X^i_{s,t}}{i}
				\leq \Norm{X^{(k),i} - X^i}{p/i} \longrightarrow 0,
				\quad (k \longrightarrow \infty,\ \forall 0 \leq s \leq t \leq T)
			\end{align}
			が成り立ち,かつ定理\ref{thm:property_of_the_completion_of_the_projective_norm}より
			\begin{align}
				\compcrossnorm{X^j_{s,u} \otimes_{j,i} X^{i-j}_{u,t}}{i}
				= \compcrossnorm{X^j_{s,u}}{j} \compcrossnorm{X^{i-j}_{u,t}}{i-j},
				\quad (\forall 0 \leq j \leq i)
			\end{align}
			が満たされるから,任意の$0 \leq s \leq t \leq T$に対して
			\begin{align}
				\compcrossnorm{X^i_{s,t} - \sum_{j=0}^i X^j_{s,u} \otimes_{j,i} X^{i-j}_{u,t}}{i}
				&\leq \compcrossnorm{X^i_{s,t} - X^{(k),i}_{s,t}}{i}
					+ \compcrossnorm{\sum_{j=0}^i X^j_{s,u} \otimes_{j,i} X^{i-j}_{u,t} 
					- \sum_{j=0}^i X^{(k),j}_{s,u} \otimes_{j,i} X^{(k),i-j}_{u,t}}{i} \\
				&\leq \compcrossnorm{X^i_{s,t} - X^{(k),i}_{s,t}}{i}
					+ \sum_{j=0}^i \compcrossnorm{X^j_{s,u} - X^{(k),j}_{s,u}}{j}
						\compcrossnorm{X^{i-j}_{u,t}}{i-j} \\
					&\qquad + \sum_{j=0}^i \compcrossnorm{X^{(k),j}_{s,u}}{j}
						\compcrossnorm{X^{i-j}_{u,t} - X^{(k),i-j}_{u,t}}{i-j} \\
				&\longrightarrow 0,
				\quad (k \longrightarrow \infty)
			\end{align}
			が従い(\refeq{eq:thm_p_rough_path_complete_dist_1})を得る.
			(\refeq{eq:thm_p_rough_path_complete_dist_2})より
			$d_p(X^{(k)},X) \longrightarrow 0\ (k \longrightarrow \infty)$が成り立ち定理の主張が得られる.
			\QED
		\end{description}
	\end{prf}
	
	\begin{screen}
		\begin{dfn}[スムースラフパス]
			
		\end{dfn}
	\end{screen}