\section{The notion of rough path}
	$(V,\Norm{\cdot}{})$を$\R$-Banach空間とする.また
	$\otimes_a$により代数的テンソル積,或はその標準写像を表す.
	$k \geq 2$の場合,$k$重テンソル積$V^{\otimes_a k} = V \otimes_a \cdots \otimes_a V$に
	プロジェクティブノルム$\projectivenorm{\cdot}{k}$を導入し,
	その完備拡大を$(V^{\otimes k},\compcrossnorm{\cdot}{k})$と書く.
	定理\ref{thm:tensor_product_with_scalar}
	と定理\ref{thm:associativity_of_tensor_products}
	によれば,任意の$0 \leq j \leq k$に対し
	$V^{\otimes_a k}$と$V^{\otimes_a j} \otimes_a V^{\otimes_a k-j}$は同型となる.
	特に$k \geq 3$及び$1 \leq j \leq k-1$に対し,
	定理\ref{thm:associativity_of_tensor_products}における線型同型を
	\begin{align}
		F_{j,k}:V^{\otimes_a k} \longrightarrow V^{\otimes_a j} \otimes_a V^{\otimes_a k-j}
	\end{align}
	と表し,$V^{\otimes_a j} \otimes_a V^{\otimes_a k-j}$上にプロジェクティブノルムを導入し,
	これを$\pi_{j,k}$と書く.
	\begin{screen}
		\begin{thm}
			このとき次が成立する.特に,$F_{j,k}$は等長同型である.
			\begin{align}
				\pi_k \circ F^{-1}_{j,k} = \pi_{j,k}.
			\end{align}
			
		\end{thm}
	\end{screen}
	
	\begin{prf}\mbox{}
		\begin{description}
			\item[第一段]
				$\pi_k \circ F^{-1}_{j,k} \leq \pi_{j,k}$が成り立つことを示す.
				$v \in V^{\otimes_a j} \otimes_a V^{\otimes_a k-j}$の分割
				\begin{align}
					v = \sum_{r} u^r \otimes_a v^r,
					\quad (u^r \in V^{\otimes_a j},\ v^r \in V^{\otimes_a k-j})
				\end{align}
				を任意に取り,一旦固定する.このとき$u^r,v^r$の任意の分割
				\begin{align}
					u^r = \sum_{n(r)} u^{n(r)}_1 \otimes_a \cdots \otimes_a u^{n(r)}_j, 
					\quad v^r = \sum_{m(r)} v^{m(r)}_{j+1} \otimes_a \cdots \otimes_a v^{m(r)}_{k},
					\quad (v^{n(r)}_i,v^{m(r)}_i \in V)
				\end{align}
				に対して
				\begin{align}
					\projectivenorm{F^{-1}_{j,k}(v)}{k}
					&\leq \sum_{r} \sum_{n(r),m(r)} \projectivenorm{u^{n(r)}_1 \otimes_a \cdots \otimes_a u^{n(r)}_j \otimes_a v^{m(r)}_{j+1} \otimes_a \cdots \otimes_a v^{m(r)}_{k}}{k} \\
					&= \sum_{r} \sum_{n(r),m(r)} \Norm{u^{n(r)}_1}{} \cdots \Norm{u^{n(r)}_j}{} \Norm{v^{m(r)}_{j+1}}{} \cdots \Norm{v^{m(r)}_k}{} \\
					&= \sum_{r} \left\{ \sum_{n(r)} \Norm{u^{n(r)}_1}{} \cdots \Norm{u^{n(r)}_j}{} \right\} \left\{ \sum_{m(r)} \Norm{v^{m(r)}_{j+1}}{} \cdots \Norm{v^{m(r)}_k}{} \right\} 
				\end{align}
				が成り立つから,分割の任意性と定理\ref{thm:expression_of_projective_norm}より
				\begin{align}
					\projectivenorm{F^{-1}_{j,k}(v)}{k} 
					\leq \sum_r \projectivenorm{u^r}{j}\projectivenorm{v^r}{k-j}
				\end{align}
				を得る.$v$の分割について下限を取れば,
				再び定理\ref{thm:expression_of_projective_norm}により
				\begin{align}
					\projectivenorm{F^{-1}_{j,k}(v)}{k} \leq \projectivenorm{v}{j,k}
				\end{align}
				が出る.
			
			\item[第二段]
				$\pi_k \circ F^{-1}_{j,k} \geq \pi_{j,k}$が成り立つことを示す.
				$v \in V^{\otimes_a k}$の任意の分割
				\begin{align}
					v = \sum_n v^n_1 \otimes_a \cdots \otimes_a v^n_k,
					\quad (v^n_i \in V,\ i=1,\cdots,k)
				\end{align}
				を取れば,
				\begin{align}
					\projectivenorm{F_{j,k}(v)}{j,k}
					&\leq \sum_n \projectivenorm{\left( v^n_1 \otimes_a \cdots \otimes_a v^n_j \right) \otimes_a \left( v^n_j \otimes_a \cdots \otimes_a v^n_k \right)}{j,k} \\
					&= \sum_n \projectivenorm{v^n_1 \otimes_a \cdots \otimes_a v^n_j}{j}
						\projectivenorm{v^n_j \otimes_a \cdots \otimes_a v^n_k}{k-j} \\
					&= \sum_n \Norm{v^n_1}{} \cdots \Norm{v^n_k}{}
				\end{align}
				が成立する.従って定理\ref{thm:expression_of_projective_norm}より
				\begin{align}
					\projectivenorm{F_{j,k}(v)}{j,k} \leq \projectivenorm{v}{k}
				\end{align}
				が得られる.
				\QED
		\end{description}
	\end{prf}
	
	
	$V^{\otimes_a i}$の$V^{\otimes i}$への等長埋め込みを$J_i$で表し
	\begin{align}
		\varphi_{j,k}:J_j V^{\otimes_a j} \times J_{k-j} V^{\otimes_a k-j} \ni (u,v)
		& \longmapsto ( J_j^{-1}u,J_{k-j}^{-1}v ) &&\in V^{\otimes_a j} \times V^{\otimes_a k-j} \\
		& \longmapsto F_{j,k}^{-1} (J_j^{-1}u \otimes_a J_{k-j}^{-1}v) &&\in V^{\otimes_a k} \\
		& \longmapsto J_k F_{j,k}^{-1} (J_j^{-1}u \otimes_a J_{k-j}^{-1}v) &&\in V^{\otimes k}
	\end{align}
	により$\varphi_{j,k}:J_j V^{\otimes_a j} \times J_{k-j} V^{\otimes_a k-j}
	\longrightarrow V^{\otimes k}$を定めれば,$\varphi_{j,k}$は有界双線型写像である.
	実際,$\otimes_a$の双線型性と埋め込み及び$F_{j,k}^{-1}$の線型性より
	$\varphi_{j,k}$の双線型性が従い,また
	\begin{align}
		\compcrossnorm{\varphi_{j,k}(u,v)}{k}
		&= \projectivenorm{F_{j,k}^{-1} (J_j^{-1}u \otimes_a J_{k-j}^{-1}v)}{k} \\
		&= \projectivenorm{J_j^{-1}u \otimes_a J_{k-j}^{-1}v}{j,k} \\
		&= \projectivenorm{J_j^{-1}u}{j} \projectivenorm{J_{k-j}^{-1}v}{k-j} \\
		&= \compcrossnorm{u}{j} \compcrossnorm{v}{k-j}
	\end{align}
	が任意の$(u,v) \in J_j V^{\otimes_a j} \times J_{k-j} V^{\otimes_a k-j}$に対して
	成り立つから$\Norm{\varphi_{j,k}}{\ContLn{J_j V^{\otimes_a j} \times J_{k-j} V^{\otimes_a k-j}}{V^{\otimes k}}{2}} = 1$を得る.
	従って,定理\ref{thm:expansion_of_multilinear_mapping}より
	$\varphi_{j,k}$は$V^{\otimes j} \times V^{\otimes k-j}$上の或る双線型
	$\psi_{j,k}$にノルム保存拡張される.
	
	\begin{screen}
		\begin{thm}
			$\psi_{j,k}$は次を満たす:
			\begin{align}
				\compcrossnorm{\psi_{j,k}(u,v)}{k} = \compcrossnorm{u}{j} \compcrossnorm{v}{k-j},
				\quad (\forall (u,v) \in V^{\otimes j} \times V^{\otimes k-j}).
			\end{align}
		\end{thm}
	\end{screen}
	
	\begin{prf}
		$(u,v)$に直積ノルムで収束する点列$(u_n,v_n) \in J_j V^{\otimes_a j} \times J_{k-j} V^{\otimes_a k-j}\ (n=1,2,\cdots)$を取れば
		\begin{align}
			\compcrossnorm{\varphi_{j,k}(u_n,v_n) - \psi_{j,k}(u,v)}{k} \longrightarrow 0,
			\quad (k \longrightarrow \infty)
		\end{align}
		が成り立つ.また
		\begin{align}
			\left| \compcrossnorm{u_n}{j} \compcrossnorm{v_n}{k-j} - 
			\compcrossnorm{u}{j} \compcrossnorm{v}{k-j} \right|
			&\leq \left| \compcrossnorm{u_n}{j} \compcrossnorm{v_n}{k-j} - 
			\compcrossnorm{u_n}{j} \compcrossnorm{v}{k-j} \right| 
			+ \left| \compcrossnorm{u_n}{j} \compcrossnorm{v}{k-j} - 
			\compcrossnorm{u}{j} \compcrossnorm{v}{k-j} \right| \\
			&\leq \compcrossnorm{u_n}{j} \compcrossnorm{v_n - v}{k-j} 
			+ \compcrossnorm{u_n - u}{j} \compcrossnorm{v}{k-j} \\
			&\longrightarrow 0, \quad (k \longrightarrow \infty)
		\end{align}
		も成立するから
		\begin{align}
			\left|\, \compcrossnorm{\psi_{j,k}(u,v)}{k} - \compcrossnorm{u}{j} \compcrossnorm{v}{k-j}\, \right|
			&\leq \compcrossnorm{\varphi_{j,k}(u_n,v_n) - \psi_{j,k}(u,v)}{k}
				+ \left| \compcrossnorm{u_n}{j} \compcrossnorm{v_n}{k-j} - 
			\compcrossnorm{u}{j} \compcrossnorm{v}{k-j} \right| \\
			& \longrightarrow 0, \quad (k \longrightarrow \infty)
		\end{align}
		が従い$\compcrossnorm{\psi_{j,k}(u,v)}{k} = \compcrossnorm{u}{j} \compcrossnorm{v}{k-j}$
		が得られる.
		\QED
	\end{prf}
	
	
	$V^{\otimes 0} = \R$として
	$T(V) \coloneqq \bigoplus_{k=0}^{\infty} V^{\otimes k}$
	とおく.また上で定めた双線型写像$\psi_{j,k}$を$\otimes_{j,k}$と書き直す:
	\begin{align}
		u \otimes_{j,k} v = \psi_{j,k}(u,v),
		\quad (\forall (u,v) \in V^{\otimes j} \times V^{\otimes k-j}).
	\end{align}
	このとき,$(a^k)_{k=0}^{\infty},(b^k)_{k=0}^{\infty} \in T(V)$に対する
	二項関係$(a^k)_{k=0}^{\infty} \otimes (b^k)_{k=0}^{\infty} \eqqcolon (c^k)_{k=0}^{\infty}$を
	\begin{align}
		c^k = \sum_{j=0}^{k} a^j \otimes_{j,k} b^{k-j},
		\quad (k=0,1,2,\cdots)
	\end{align}
	により定めれば,
	$c^k \in V^{\otimes k}\ (k=0,1,\cdots)$かつ
	有限個の$k$を除いて$c^k = 0$となるから$(c^k)_{k=0}^\infty \in T(V)$が成立し,
	$\otimes$は$T(V)$において結合則?を満たす双線型な積となる.$n \geq 0$に対して
	\begin{align}
		T^{(n)}(V) \coloneqq \bigoplus_{k=0}^{n} V^{\otimes k}
	\end{align}
	とおき,同様に$\otimes$を
	\begin{align}
		(c^k)_{k=0}^n = (a^k)_{k=0}^n \otimes (b^k)_{k=0}^n,
		\quad c^k = \sum_{j=0}^{k} a^j \otimes_{j,k} b^{k-j},
		\quad (k=0,\cdots,n)
	\end{align}
	により定め,次の直積ノルムを導入する:
	\begin{align}
		|a| \coloneqq \sum_{k=0}^n \compcrossnorm{a^k}{k},
		\quad (a = (a^k)_{k=0}^n \in T^{(n)}(V)).
	\end{align}
	いま,写像$X:\Delta_T \longrightarrow T^{(n)}(V)$に対して
	$X_{s,t} = (X^0_{s,t},\cdots,X^n_{s,t}),\ ((s,t) \in \Delta_T)$と書いて
	\begin{align}
		C_0 \left(\Delta_T,T^{(n)}(V) \right)
		\coloneqq \Set{X:\Delta_T \longrightarrow T^{(n)}(V)}{\mbox{continuous},\ X^0 \equiv 1}
	\end{align}
	とおく.
	
	\begin{screen}
		\begin{dfn}[finite total $p$-variation]
			$X \in C_0 \left(\Delta_T,T^{(n)}(V) \right)$
			について,任意の$1 \leq i \leq n$に対して
			\begin{align}
				\Norm{X^i}{p/i} < \infty
			\end{align}
			が満たされるとき,$X$は finite total $p$-variation を持つという.また次を定める:
			\begin{align}
				C_{0,p} \left(\Delta_T,T^{(n)}(V) \right)
				\coloneqq \Set{X \in C_0 \left(\Delta_T,T^{(n)}(V) \right)}{\mbox{$X$ has finite total $p$-variation}}	.
			\end{align}
		\end{dfn}
	\end{screen}
	
	\begin{screen}
		\begin{dfn}[乗法的汎関数]
			次の式(Chen's identity)を満たす$X \in C_0 \left(\Delta_T,T^{(n)}(V) \right)$
			を$n$次の乗法的汎関数(multiplicative functional)と呼ぶ:
			\begin{align}
				X_{s,u} \otimes X_{u,t} = X_{s,t},
				\quad (\forall 0 \leq s \leq u \leq t \leq T).
			\end{align}
		\end{dfn}
	\end{screen}
	
	$C_0 \left(\Delta_T,T^{(n)}(V) \right)$の定義で
	$X^0 \equiv 1$としたが,
	$X:\Delta_T \longrightarrow T^{(n)}(V)$が Chen's identity を満たすには
	$X^0 \equiv 1$或は0である必要がある.実際,
	$X^0_{s,t} = X^0_{s,t} \otimes_{0,0} X^0_{s,t} = X^0_{s,t}X^0_{s,t}$が成り立たなければならない.
	
	\begin{screen}
		\begin{lem}
			$X:\Delta_T \longrightarrow T^{(n)}(V)$が
			Chen's identity を満たせば$X^k_{t,t} = 0\ (1 \leq k \leq n)$が成り立つ.
		\end{lem}
	\end{screen}
	
	\begin{prf}
		任意に$t \in [0,T]$を取る.
		$X^k_{t,t} = \sum_{j=0}^{k} X^j_{t,t} \otimes_{j,k} X^{k-j}_{t,t}$より,先ず
		\begin{align}
			X^1_{t,t} = X^0_{t,t} \otimes_{0,1} X^1_{t,t} + X^1_{t,t} \otimes_{0,1} X^0_{t,t}
			= X^1_{t,t} + X^1_{t,t}
		\end{align}
		が成り立ち$X^1_{t,t} = 0$が従う.同様に
		\begin{align}
			X^2_{t,t} = X^0_{t,t} \otimes_{0,1} X^2_{t,t} + X^1_{t,t} \otimes_{0,1} X^1_{t,t}
				+ X^2_{t,t} \otimes_{0,1} X^0_{t,t}
			= X^2_{t,t} + X^2_{t,t}
		\end{align}
		より$X^2_{t,t} = 0$が成立し,帰納的に$X^k_{t,t} = 0\ (1 \leq k \leq n)$が出る.
		\QED
	\end{prf}
	
	\begin{screen}
		\begin{thm}
		\end{thm}
	\end{screen}
	
	連続かつ有界変動なパス$x:[0,T] \longrightarrow V$に対して次の積分
	\begin{align}
		\int_s^t X^1_{s,u} \otimes d x_u
	\end{align}
	を定めたい.ただし$X^1_{s,t} = x_t - x_s\ ((s,t) \in \Delta_T)$である.
	いま,細分$D=\{s=u_0 < \cdots <u_n= t\},D'=\{s=v_0 < \cdots <v_m= t\} \in \delta[s,t]$
	に対して,共通細分を$D''=\{s=w_0 < w_1 \cdots \leq t\}$と表し
	\begin{align}
		\tilde{X}^1_{s,w_k} &\coloneqq X^1_{s,u_i},
		\quad (u_i \leq w_k \leq u_{i+1}), \\
		\hat{X}^1_{s,w_k} &\coloneqq X^1_{s,v_j},
		\quad (v_j \leq w_k \leq v_{j+1})
	\end{align}
	により$\tilde{X}^1,\hat{X}^1$を定める.このとき
	\begin{align}
		&\Norm{ \sum_D X^1_{s,u_{i-1}} \otimes X^1_{u_{i-1},u_i} - 
			\sum_{D'} X^1_{s,v_{j-1}} \otimes X^1_{v_{j-1},v_j}}{V^{\otimes 2}} \\
		&\qquad= \Norm{ \sum_D X^1_{s,u_{i-1}} \otimes X^1_{u_{i-1},u_i} - 
			\sum_{D''} X^1_{s,w_{k-1}} \otimes X^1_{w_{k-1},w_k}}{V^{\otimes 2}} \\
			&\quad\qquad + \Norm{ \sum_{D'} X^1_{s,v_{j-1}} \otimes X^1_{v_{j-1},v_j} - 
			\sum_{D''} X^1_{s,w_{k-1}} \otimes X^1_{w_{k-1},w_k}}{V^{\otimes 2}} \\
		&\qquad= \Norm{ \sum_{D''} \tilde{X}^1_{s,w_{k-1}} \otimes X^1_{w_{k-1},w_k} - 
			\sum_{D''} X^1_{s,w_{k-1}} \otimes X^1_{w_{k-1},w_k}}{V^{\otimes 2}} \\
			&\quad\qquad + \Norm{ \sum_{D''} \hat{X}^1_{s,w_{k-1}} \otimes X^1_{w_{k-1},w_k} - 
			\sum_{D''} X^1_{s,w_{k-1}} \otimes X^1_{w_{k-1},w_k}}{V^{\otimes 2}} \\
		&\qquad= \Norm{ \sum_{D''} \left(\tilde{X}^1_{s,w_{k-1}} -  X^1_{s,w_{k-1}} \right)
		 	\otimes X^1_{w_{k-1},w_k}}{V^{\otimes 2}} \\
			&\quad\qquad + \Norm{ \sum_{D''} \left( \hat{X}^1_{s,w_{k-1}} - X^1_{s,w_{k-1}} \right)
			 \otimes X^1_{w_{k-1},w_k}}{V^{\otimes 2}} \\
		&\qquad \leq \sum_{D''} \Norm{\tilde{X}^1_{s,w_{k-1}} -  X^1_{s,w_{k-1}}}{} \Norm{X^1_{w_{k-1},w_k}}{}
			+ \sum_{D''} \Norm{\hat{X}^1_{s,w_{k-1}} -  X^1_{s,w_{k-1}}}{} \Norm{X^1_{w_{k-1},w_k}}{} \\
		&\qquad \leq \max{k}{\Norm{\tilde{X}^1_{s,w_{k-1}} -  X^1_{s,w_{k-1}}}{}} 
			\Norm{X^1}{1,[s,t]} + \max{k}{\Norm{\hat{X}^1_{s,w_{k-1}} -  X^1_{s,w_{k-1}}}{}} 
			\Norm{X^1}{1,[s,t]}
	\end{align}
	が成立する.いま,$[s,t] \ni u \longmapsto X^1_{s,u}$は一様連続であるから
	$|D|,|D'|\longrightarrow 0$として右辺は0に収束する.
	従って$D_n \in \delta[s,t]$を$|D_n| \longrightarrow 0\ (n \longrightarrow \infty)$を満たす
	細分列とすれば$\left(\sum_{D_n} X^1_{s,u_{i-1}} \otimes X^1_{u_{i-1},u_i} \right)_{n=1}^{\infty}$
	は$V^{\otimes 2}$のCauchy列となり$\in V^{\otimes 2}$で収束する.別の細分列
	$(D_m)_{m=1}^{\infty}\ (|D_m| \longrightarrow 0)$を取っても
	\begin{align}
		\Norm{ \sum_{D_n} X^1_{s,u_{i-1}} \otimes X^1_{u_{i-1},u_i} - 
			\sum_{D_m} X^1_{s,v_{j-1}} \otimes X^1_{v_{j-1},v_j}}{V^{\otimes 2}}
		\longrightarrow 0,
		\quad (n,m \longrightarrow \infty)
	\end{align}
	が成り立つから極限は細分列に依らず定まり,従って
	$\lim_{|D| \to 0} \sum_{D} X^1_{s,u_{i-1}} \otimes X^1_{u_{i-1},u_i}$が確定する.ここで
	\begin{align}
		X^2_{s,t} = \int_s^t X^1_{s,u} \otimes d x_u \coloneqq 
		\lim_{|D| \to 0} \sum_{D} X^1_{s,u_{i-1}} \otimes X^1_{u_{i-1},u_i}
	\end{align}
	と定める.このとき$\Delta_T \ni (s,t) \longmapsto X^2_{s,t} \in V^{\otimes 2}$は連続である.
	
\begin{screen}
	\begin{thm}
		$\Omega_p(V)$は次で定める距離
		\begin{align}
			d_p(X,Y) \coloneqq \max{1 \leq i \leq [p]}{\Norm{X^i - Y^i}{p/i}}
		\end{align}
		により完備距離空間となる.
	\end{thm}
\end{screen}

\begin{prf}
	$(X^k)_{k=1}^{\infty}$をCauchy列とすれば,
	任意の$\epsilon > 0$に対し或る$K \in \N$が存在して
	\begin{align}
		\Norm{X^{k,i} - X^{\ell,i}}{p/i} < \epsilon,
		\quad (\forall k,\ell > K,\ 1 \leq i \leq [p])
	\end{align}
	が成立する.よって定理(\ref{thm:B_p_T_Banach_space})より
	各$i$に対し或る$X^i \in B_{p/i,T}(V)$が存在して
	\begin{align}
		\Norm{X^{k,i} - X^i}{p/i}
		\longrightarrow 0
		\quad (k \longrightarrow \infty,\ 1 \leq i \leq [p])
	\end{align}
	を満たす.$X:\Delta_T \longrightarrow T^{(n)}(V)$を
	\begin{align}
		X_{s,t} \coloneqq (1,X^1_{s,t},\cdots,X^n_{s,t}),
		\quad ((s,t) \in \Delta_T)
	\end{align}
	により定めれば,$X^i$の連続性より$X$も連続である.さらに任意の$D \in \delta[0,T]$に対して
	\begin{align}
		\sum_D \Norm{X^k_{t_{i-1},t_i} - X_{t_{i-1},t_i}}{}^p
		&= \sum_D \left( \Norm{X^{k,1}_{t_{i-1},t_i} - X^1_{t_{i-1},t_i}}{} + \cdots + \Norm{X^{k,n}_{t_{i-1},t_i} - X^n_{t_{i-1},t_i}}{} \right)^p \\
		&\leq (n+1)^p \sum_D \left( \Norm{X^{k,1}_{t_{i-1},t_i} - X^1_{t_{i-1},t_i}}{}^p + \cdots + \Norm{X^{k,n}_{t_{i-1},t_i} - X^n_{t_{i-1},t_i}}{}^p \right) \\
		&\leq (n+1)^p \left( \Norm{X^{k,1}_{t_{i-1},t_i} - X^1_{t_{i-1},t_i}}{p}^p + \cdots + \Norm{X^{k,n}_{t_{i-1},t_i} - X^n_{t_{i-1},t_i}}{p}^p \right)
	\end{align}
	が成立するから,定理より
	\begin{align}
		\sup{D}{\sum_D \Norm{X^k_{t_{i-1},t_i} - X_{t_{i-1},t_i}}{}^p}
		\longrightarrow 0 \quad (k \longrightarrow \infty)
	\end{align}
	が従い$X$の$p$-variationの有界性が出る.
	\QED
\end{prf}