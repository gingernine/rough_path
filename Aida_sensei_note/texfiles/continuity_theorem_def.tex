\section{連続性定理の主張}
	\begin{screen}
		\begin{dfn}[記号の定義]
			$x \in C^1,\ f \in C^2(\R^d,L(\R^d \rightarrow \R^m))$に対し次を定める.
			\begin{align}
				\Delta_T &\coloneqq \Set{(s,t)}{0 \leq s \leq t \leq T}, \\
				X^1 &:\Delta_T \longrightarrow \R^d\ \left( (s,t) \longmapsto X^1_{s,t} = x_t - x_s \right), \\
				X^2 &:\Delta_T \longrightarrow \R^m \otimes \R^d\ \left( (s,t) \longmapsto X^2_{s,t} = \int_s^t (x_u - x_s) \otimes dx_u \right), \\
				\tilde{I}^f_{s,t}(x) &\coloneqq f(x_s)X^1_{s,t} = f(x_s)(x_t - x_s), \\
				J^f_{s,t}(x) &\coloneqq f(x_s)X^1_{s,t} + (\nabla f)(x_s)X^2_{s,t}, \\
					&\quad \mbox{where\ } \left[ (\nabla f)(x_s)X^2_{s,t} \right]^i = \sum_{j,k=1}^d \partial_k f^i_j(x_s) \int_s^t \left(x^k_u - x^k_s \right)\ dx^j_u.
			\end{align}
		\end{dfn}
	\end{screen}
	
	\begin{screen}
		\begin{thm}
			$[s,t] \subset [0,T],\ x \in C^1,\ f \in C^2(\R^d,L(\R^d \rightarrow \R^m))$とする.$D \in \delta[s,t]$に対し
			\begin{align}
				\tilde{I}^f_{s,t}(x,D) \coloneqq \sum_D \tilde{I}^f_{t_{i-1},t_i}(x),
				\quad J^f_{s,t}(x,D) \coloneqq \sum_D J^f_{t_{i-1},t_i}(x)
			\end{align}
			を定めるとき,次が成立する:
			\begin{align}
				I^f_{s,t}(x) = \lim_{|D| \to 0} \tilde{I}^f_{s,t}(x,D)
				= \lim_{|D| \to 0} J^f_{s,t}(x,D).
			\end{align}
		\end{thm}
	\end{screen}
	
	\begin{prf}
		第一の等号は$I^f_{s,t}(x)$の定義によるから,第二の等号を証明する.各$1 \leq i \leq m$について
		\begin{align}
			\left[ I_{s,t}(x) \right]^i
			&= \sum_{j=1}^d \int_s^t f_j^i(x_u)\ dx^j_u \\
			&= \sum_{j=1}^d \int_s^t f_j^i(x_s) + f_j^i(x_u) - f_j^i(x_s)\ dx^j_u \\
			&= \sum_{j=1}^d \int_s^t f_j^i(x_s) 
				+ \sum_{k=1}^d  \left\{ \int_0^1 \partial_k f_j^i(x_s + \theta(x_u - x_s))\ d\theta \right\} (x^k_u - x^k_s)\ dx^j_u \\
			&= \sum_{j=1}^d \int_s^t f_j^i(x_s)\ dx^j_u + \sum_{j,k=1}^d \partial_k f_j^i(x_s) \int_s^t (x^k_u - x^k_s)\ dx^j_u \\
				&\quad + \sum_{j,k=1}^d \int_s^t 
				\left\{ \int_0^1 \partial_k f_j^i(x_s + \theta(x_u - x_s)) - \partial_k f_j^i(x_s)\ d\theta \right\} (x^k_u - x^k_s)\ dx^j_u, \\
			&= \left[ J^f_{s,t}(x) \right]^i
				+ \sum_{j,k,\ell=1}^d \int_s^t 
				\left\{ \int_0^1 \left( \int_0^\theta \partial_\ell \partial_k f_j^i(x_s + r(x_u - x_s))\ dr\right) d\theta \right\} (x^\ell_u - x^\ell_s)(x^k_u - x^k_s)\ dx^j_u
		\end{align}
		が成り立つ.$x([0,T])$はコンパクトであるから,原点中心の半径${}^\exists w$の閉球に含まれる.
		従って$x_s + r(x_u - x_s)\ (0 \leq r \leq 1)$は半径$3w$の閉球に属し,
		$\partial_\ell \partial_k f_j^i$のその球上での最大値を$M$とおけば
		\begin{align}
			&\left| \int_s^t 
				\left\{ \int_0^1 \left( \int_0^\theta \partial_\ell \partial_k f_j^i(x_s + r(x_u - x_s))\ dr\right) d\theta \right\} (x^\ell_u - x^\ell_s)(x^k_u - x^k_s)\ dx^j_u \right| \\
			&\qquad \leq M \int_s^t |x^\ell_u - x^\ell_s||x^k_u - x^k_s| |\dot{x}^j_u|\ du \\
			&\qquad \leq M \Norm{x}{C^1}^3 \int_s^t (u - s)^2\ du
		\end{align}
		となるから,特に$D \in \delta[s,t]$に対して
		\begin{align}
			&\sum_D \int_{t_{i-1}}^{t_i} (u - t_{i-1})^2\ du
			\leq \sum_D |D| \int_{t_{i-1}}^{t_i} (u - t_{i-1})\ du \\
			&\qquad \leq \sum_D |D| \int_{t_{i-1}}^{t_i} (u - s)\ du
			\leq \frac{1}{2}(t-s)^2 |D|
			\longrightarrow 0 \quad (|D| \longrightarrow 0)
		\end{align}
		が成立する.これにより
		\begin{align}
			\left| \left[ I_{s,t}(x,D) \right]^i - \left[ J_{s,t}(x,D) \right]^i \right| \longrightarrow 0 \quad (|D| \longrightarrow 0)
		\end{align}
		が従い定理の主張を得る.
		\QED
	\end{prf}
	
	
	\begin{screen}
		\begin{thm}[$1 \leq p < 2$の場合の連続性定理]
		\end{thm}
	\end{screen}
	
	\begin{screen}
		\begin{thm}[$2 \leq p < 3$の場合の連続性定理]
		\end{thm}
	\end{screen}