\section{導入}
	以下,$x \in \R^d$について成分を込めて表現する場合は
	$x = (x^1,\cdots,x^d)$と書き,
	実$m \times d$行列$a$については$a=(a^i_j)_{1 \leq i \leq m,1 \leq j \leq d}$と表す.
	また$T > 0$を固定し$C^1 = C^1([0,T] \rightarrow \R^d)$とおく.
	ただし端点においては片側微分を考える.
	区間$[s,t] \subset [0,T]$の分割を
	$D = \{s = t_0 < t_1 < \cdots < t_N = t\}$で表現し
	$|D| \coloneqq \max{1 \leq i \leq N}{\left| t_i - t_{i-1} \right|}$とおく.
	また$[s,t]$の分割の全体を$\delta[s,t]$と書く.

	\begin{screen}
		\begin{thm}[Riemann-Stieltjes積分]
			$[s,t] \subset [0,T]$とし,$D \in \delta[s,t]$についてのみ考えるとき,
			任意の$x \in C^1,\ f \in C(\R^d,L(\R^d \rightarrow \R^m))$\footnotemark
			に対して
			次の極限が確定する:
			\begin{align}
				\lim_{|D| \to 0}
				\sum_{D} f(x_{s_{i-1}})(x_{t_i} - x_{t_{i-1}})
				\in \R^m.
			\end{align}
			ここで$s_{i-1}$は区間$[t_{i-1},t_i]$に属する任意の点である.
			極限は$s_{i-1}$の取り方にも依存しない.
		\end{thm}
	\end{screen}
	\footnotetext{
		極限の存在を保証する条件としては,$f$の有界性と微分可能性は必要ない.
	}
	\begin{prf}
		各$x^j$は$C^1$-級であるから,平均値の定理より
		$\sum_{D} f(x_{s_{i-1}})(x_{t_i} - x_{t_{i-1}})$
		の第$k$成分を
		\begin{align}
			\sum_{j=1}^{d} \sum_{D} f^k_j (x_{s_{i-1}})(x^j_{t_i} - x^j_{t_{i-1}})
			= \sum_{j=1}^{d} \sum_{D} f^k_j (x_{s_{i-1}}) \frac{d}{dt}x^j(\xi_{i-1,j})(t_i - t_{i-1}),
			\quad ({}^\exists \xi_{i-1,j} \in [t_{i-1},t_i])
		\end{align}
		と表現できる.各$j,k$について
		\begin{align}
			\lim_{|D| \to 0} \sum_{D} f^k_j (x_{s_{i-1}}) \frac{d}{dt}x^j(\xi_{i-1,j})(t_i - t_{i-1})
		\end{align}
		が確定すれば,第$k$成分の極限が確定し定理の主張を得る.
		いま,$t \longrightarrow f^k_j(x_t)$及び$t \longmapsto (d/dt)x^j_t$は([s,t]上一様)連続であるから,
		分割$D$による各区間$[t_{i-1},t_i]$において次の最大最小値が定まる:
		\begin{align}
			M_{i-1} \coloneqq \sup{t_{i-1} \leq t \leq t_i} f^k_j(x_t)\frac{d}{dt}x^j_t,
			\quad m_{i-1} \coloneqq \inf{t_{i-1} \leq t \leq t_i} f^k_j(x_t)\frac{d}{dt}x^j_t.
		\end{align}
		ここで
		\begin{align}
			S_D \coloneqq \sum_{D} M_{i-1}(t_i - t_{i-1}),
			\quad s_D \coloneqq \sum_{D} m_{i-1}(t_i - t_{i-1}),
			\quad \Sigma_D \coloneqq \sum_{D} f^k_j (x_{s_{i-1}}) \frac{d}{dt}x^j(\xi_{i-1})(t_i - t_{i-1})
		\end{align}
		とおいて
		\begin{align}
			S \coloneqq \inf{D \in \delta[s,t]}{S_D},
			\quad s \coloneqq \sup{D \in \delta[s,t]}{s_D}
		\end{align}
		を定めれば
		\begin{align}
			s_D \leq s \leq S \leq S_D,
			\quad s_D \leq \Sigma_D \leq S_D
		\end{align}
		が満たされる.実際,任意の$D_1,D_2 \in \delta[s,t]$に対して,
		分割の合併を$D_3$とすれば
		\begin{align}
			s_{D_1} \leq s_{D_3} \leq S_{D_3} \leq S_{D_2}
		\end{align}
		が成立し$s \leq S_D\ (\forall D \in \delta[s,t])$すなわち$s \leq S$が出る.
		一方で一様連続性から
		\begin{align}
			0 \leq S - s \leq S_D - s_D = \sum_D (M_{i-1} - m_{i-1})(t_i - t_{i-1})
			\longrightarrow 0
			\quad (|D| \longrightarrow 0)
		\end{align}
		が従い$s = S$を得る.以上より
		\begin{align}
			|S - \Sigma_D| \leq |S - S_D| + |S_D - \Sigma_D|
			\leq |S - S_D| + |S_D - s_D|
			\longrightarrow 0
			\quad (|D| \longrightarrow 0)
		\end{align}
		が成り立つ.
		\QED
\end{prf}

\begin{screen}
	\begin{dfn}
		$I_{0,T}(x)$の定義.
	\end{dfn}
\end{screen}

$C^1$において次でノルム$\Norm{\cdot}{C^1}$を定める:
\begin{align}
	\Norm{x}{\infty} &\coloneqq \sup{t \in [0,T]}{|x(t)|},
	\quad \Norm{x'}{\infty} \coloneqq \sup{t \in [0,T]}{|x'(t)|}, \\
	\Norm{x}{C^1} &\coloneqq
	\Norm{x}{\infty} + \Norm{x}{\infty}.
\end{align}

\begin{screen}
	\begin{thm}
		$x \longmapsto I_{0,T}(x)$は連続である.
	\end{thm}
\end{screen}

\begin{prf}
	$C^1$は次のノルムでBanach空間になる.ゆえに
	$x \longmapsto I_{0,T}(x)$は距離空間から距離空間への対応であり
	点列連続性と連続性が一致するから,
	$x_n \longrightarrow x$なら$I_{0,T}(x_n) \longrightarrow I_{0,T}(x)$
	が従うことを示せばよい.実際,$f,x$の連続性より
	\begin{align}
		\sum_{i=1}^{N} f(x^{(n)}_{s_{i-1}})(x^{(n)}_{t_i} - x^{(n)}_{t_{i-1}})
		\longrightarrow \sum_{i=1}^{N} f(x_{s_{i-1}})(x_{t_i} - x_{t_{i-1}})
	\end{align}
	が成り立つ.
\end{prf}

\begin{screen}
	\begin{dfn}[$p$-variation]
		$[0,T]$上の$\R^d$値関数$x$に対し$p$-variationノルムを次で定める:
		\begin{align}
			\Norm{x}{p}
			\coloneqq \left\{ \sup{D}{\sum_{i=1}^{N} 
				\left| x_{t_i} - x_{t_{i-1}} \right|^p }\right\}^{1/p}.
		\end{align}
		また線形空間$B_{p,T}(\R^d)$を
		\begin{align}
			B_{p,T}(\R^d)
			\coloneqq \Set{x:[0,T] \longrightarrow \R^d}{x_0=0,\ x:\mbox{continuous},\ \Norm{x}{p} < \infty}
		\end{align}
		により定める.
	\end{dfn}
\end{screen}

\begin{screen}
	\begin{thm}
		$\tilde{C}^1 \coloneqq \Set{x \in C^1}{x_0 = 0}$とおくと,
		$\tilde{C}^1 \subset B_{p,T}(\R^d)$が成り立つ.
	\end{thm}
\end{screen}

\begin{prf}
	$x \in \tilde{C}^1$に対して
	\begin{align}
		M \coloneqq \sum_{j=1}^{d} \sup{x \in [0,T]}{|x^j(t)|}
	\end{align}
	とおけば,$x'$の連続性より$M < \infty$が定まる.
	平均値の定理より,$|D| < 1$を満たす分割$D$に対して
	\begin{align}
		\left\{ \sum_{i=1}^{N} |x_{t_i} - x_{t_{i-1}}|^p \right\}^{1/p}
		\leq \left\{ \sum_{i=1}^{N} \Norm{x}{C^1}^p (t_i - t_{i-1})^p \right\}^{1/p}
		\leq M T < \infty
	\end{align}
	が成立し$\Norm{x}{p} \leq MT < \infty$が従う
	\footnote{
		$S_D \geq 0$ならば$(\sup{D}S_D)^{1/p} = \sup{D}S_D^{1/p}$が成り立つ.
	}
	.
\end{prf}

\begin{screen}
	\begin{thm}
		$B_{p,T}(\R^d)$はBanach空間である.
	\end{thm}
\end{screen}

\begin{prf}
	$(x^n)_{n=1}^{\infty} \subset B_{p,T}(\R^d)$をCauchy列とする.
	つまり任意の$\epsilon > 0$に対して或る$n_\epsilon \in \N$が存在し
	\begin{align}
		\Norm{x^n - x^m}{p}
		= \left\{ \sup{D}{\sum_{i=1}^{N} 
		\left| \left( x^n_{t_i} - x^m_{t_i} \right) 
		- \left(x^n_{t_{i-1}} - x^m_{t_{i-1}} \right) \right|^p }\right\}^{1/p} < \epsilon,
		\quad (n,m > n_\epsilon)
	\end{align}
	を満たす.いま,任意の$t \in [0,T]$に対して$[0,T]$の分割$\{0=t_0 \leq t \leq T\}$
	を考えれば
	\begin{align}
		|x^n_t - x^m_t| < \epsilon,
		\quad (n,m > n_\epsilon)
	\end{align}
	が得られ,実数の完備性より或る$x_t \in \R^d$が存在して
	\begin{align}
		|x^n_t - x_t| < \epsilon
		\quad (n > n_\epsilon)
	\end{align}
	を満たす.実際,もし或る$n > n_\epsilon$で$|x^n_t - x_t| \eqqcolon \alpha \geq \epsilon$
	が成り立つと,任意の$m > n_\epsilon$に対して
	\begin{align}
		|x^m_t - x_t| \geq |x^n_t - x_t| - |x^n_t - x^m_t| > \alpha - \epsilon
	\end{align}
	が従い$x^m_t \longrightarrow x_t$に反する.
	ゆえに収束は$t$に関して一様であり,$t \longmapsto x_t$は0出発かつ連続である.
	あとは$\Norm{x^n - x}{p} \longrightarrow 0\ (n \longrightarrow \infty)$であればよい.
\end{prf}

\begin{screen}
	\begin{thm}
		$C^1$空間において,$\Norm{}{C^1}$で定まる位相は
		$\Norm{}{p}$で定まる位相より強い.
		特に,$B_{p,T}(\R^d)$上で考える写像$x \longmapsto I_{p,T}(x)$は
		$\Norm{\cdot}{p}$の定める位相により連続である.
	\end{thm}
\end{screen}

\begin{prf}
	任意の$x \in \tilde{C}^1$に対して
	\begin{align}
		\Norm{x}{p} \leq T \Norm{x}{C^1}
	\end{align}
	を満たすことを証明する.実際,任意の分割$D$に対して
	\begin{align}
		\sum_{i=1}^{N} |x_{t_i} - x_{t_{i-1}}|
		\leq \sum_{i=1}^{N} \Norm{x}{C^1} (t_i - t_{i-1})
		= T \Norm{x}{C^1}
	\end{align}
	が成り立つ.
	\QED
\end{prf}