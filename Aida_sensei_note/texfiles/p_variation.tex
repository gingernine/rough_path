\section{導入}
	以下,$d$次元ベクトル$x \in \R^d$と$(m, d)$行列$a \in \R^m \otimes \R^d$について,
	成分を込めて表現する場合は
	$x = (x^1,\cdots,x^d),\ a=(a^i_j)_{1 \leq i \leq m,1 \leq j \leq d}$と書く.
	また$T > 0$を固定し$C^1 = C^1([0,T] \rightarrow \R^d)$とおく.
	(端点においては片側微分を考える.)
	区間$[s,t] \subset [0,T]$の分割を
	$D = \{s = t_0 < t_1 < \cdots < t_N = t\}$で表現し,
	分割の全体を$\delta[s,t]$とおく.
	$|D|$により$\max{1 \leq i \leq N}{\left| t_i - t_{i-1} \right|}$を表し,
	\begin{align}
		\sum_D = \sum_{i=1}^{N}
	\end{align}
	と略記する.

	\begin{screen}
		\begin{thm}[Riemann-Stieltjes積分]\label{thm:existence_of_Riemann_Stieltjes_integral}
			$[s,t] \subset [0,T]$とし,$D \in \delta[s,t]$についてのみ考えるとき,
			任意の$x \in C^1,\ f \in C(\R^d,L(\R^d \rightarrow \R^m))$
			に対して次の極限が存在する:\footnotemark
			\begin{align}
				\lim_{|D| \to 0}
				\sum_{D} f(x_{s_{i-1}})(x_{t_i} - x_{t_{i-1}})
				\in \R^m.
			\end{align}
			$s_{i-1}$は区間$[t_{i-1},t_i]$に属する任意の点であり,
			極限は$s_{i-1}$の取り方に依らない.
		\end{thm}
	\end{screen}
	\footnotetext{
		極限の存在を保証する条件としては,$f$の有界性と微分可能性は必要ない.
	}
	\begin{prf}
		各$x^j$は$C^1$-級であるから,平均値の定理より
		$\sum_{D} f(x_{s_{i-1}})(x_{t_i} - x_{t_{i-1}})$
		の第$k$成分を
		\begin{align}
			&\sum_{j=1}^{d} \sum_{D} f^k_j (x_{s_{i-1}})(x^j_{t_i} - x^j_{t_{i-1}}) 
				\label{eq:thm_existence_of_Riemann_Stieltjes_integral}\\
			&\qquad 
			= \sum_{j=1}^{d} \sum_{D} f^k_j (x_{s_{i-1}}) \dot{x}^j_{\xi_i}(t_i - t_{i-1}),
			\quad ({}^\exists \xi_i \in [t_{i-1},t_i])
		\end{align}
		と表現できる.各$j,k$について
		\begin{align}
			\lim_{|D| \to 0} \sum_{D} f^k_j (x_{s_{i-1}}) \dot{x}^j_{\xi_i}(t_i - t_{i-1})
		\end{align}
		は通常の連続関数のRiemann積分
		\begin{align}
			\int_s^t f^k_j (x_u) \dot{x}^j_u\ du
		\end{align}
		に収束する. 
		\begin{comment}
		が確定すれば,第$k$成分の極限が確定し定理の主張を得る.
		いま,$t \longrightarrow f^k_j(x_t)$及び$t \longmapsto (d/dt)x^j_t$は($[s,t]$上一様)連続であるから,
		分割$D$による各区間$[t_{i-1},t_i]$において次の最大最小値が定まる:
		\begin{align}
			M_i \coloneqq \sup{t_{i-1} \leq t \leq t_i} f^k_j(x_t)\dot{x}^j_t,
			\quad m_i \coloneqq \inf{t_{i-1} \leq t \leq t_i} f^k_j(x_t)\dot{x}^j_t.
		\end{align}
		ここで
		\begin{align}
			S_D \coloneqq \sum_{D} M_i(t_i - t_{i-1}),
			\quad s_D \coloneqq \sum_{D} m_i(t_i - t_{i-1}),
			\quad \Sigma_D \coloneqq \sum_{D} f^k_j (x_{s_{i-1}}) \dot{x}^j_{\xi_i}(t_i - t_{i-1})
		\end{align}
		とおいて
		\begin{align}
			S \coloneqq \inf{D \in \delta[s,t]}{S_D},
			\quad s \coloneqq \sup{D \in \delta[s,t]}{s_D}
		\end{align}
		を定めれば
		\begin{align}
			s_D \leq s \leq S \leq S_D,
			\quad s_D \leq \Sigma_D \leq S_D
		\end{align}
		が満たされる.実際,任意の$D_1,D_2 \in \delta[s,t]$に対して,
		分割の合併を$D_3$とすれば
		\begin{align}
			s_{D_1} \leq s_{D_3} \leq S_{D_3} \leq S_{D_2}
		\end{align}
		が成立し$s \leq S_D\ (\forall D \in \delta[s,t])$すなわち$s \leq S$が出る.
		一方で一様連続性から
		\begin{align}
			0 \leq S - s \leq S_D - s_D = \sum_D (M_i - m_i)(t_i - t_{i-1})
			\longrightarrow 0
			\quad (|D| \longrightarrow 0)
		\end{align}
		が従い$s = S$を得る.以上より
		\begin{align}
			|S - \Sigma_D| \leq |S - S_D| + |S_D - \Sigma_D|
			\leq |S - S_D| + |S_D - s_D|
			\longrightarrow 0
			\quad (|D| \longrightarrow 0)
		\end{align}
		が成り立つ.
		\end{comment}
		\QED
\end{prf}

\begin{screen}
	\begin{dfn}[$C^1$-級のパスに対する汎関数]
		$x \in C^1$と$f \in C(\R^d,L(\R^d \rightarrow \R^m))$に対して,
		$[s,t] \subset [0,T]$におけるRiemann-Stieltjes積分を$I$で表現する:
		\begin{align}
			I_{s,t}(x) = \int_s^t f(x_u)\ dx_u 
			&\coloneqq \lim_{|D| \to 0}
				\sum_{D} f(x_{t_{i-1}})(x_{t_i} - x_{t_{i-1}}), \\
			\left[ \int_s^t f(x_u)\ dx_u \right]^k
			&= \sum_{j=1}^d \int_s^t f^k_j(x_u)\ dx^j_u,
			\quad (k = 1,\cdots,m).
		\end{align}
		ただし$D \in \delta[s,t]$のみを考える.
	\end{dfn}
\end{screen}

\begin{comment}
\begin{screen}
	\begin{thm}[$I$の加法性・線型性・絶対値]
	\label{thm:linearity_of_Riemann_Stieltjes_integral}
		任意の$x \in C^1,\ f,g \in C(\R^d,L(\R^d \rightarrow \R^m)),\ \alpha,\beta \in \R$に対して次が成立する:
		\begin{description}
			\item[(1)] $I^f_{s,u}(x) + I^f_{u,t}(x) = I^f_{s,t}(x)$,
			
			\item[(2)] $I^{(\alpha f + \beta g)}_{s,t}(x) = \alpha I^f_{s,t}(x) + \beta I^g_{s,t}(x)$.
			
			\item[(3)] $\left| I^f_{s,t}(x) \right| \leq \int_s^t |f(x_u)| |\dot{x}_u|\ du$.
		\end{description}
	\end{thm}
\end{screen}

\begin{prf}\mbox{}
	\begin{description}
		\item[(1)] 
			各$k,j$に対して
			\begin{align}
				\int_s^u f^k_j(x_r)\ dx^j_r
				+ \int_u^t f^k_j(x_r)\ dx^j_r
				= \int_s^t f^k_j(x_r)\ dx^j_r
				\label{eq:thm_linearity_of_Riemann_Stieltjes_integral_1}
			\end{align}
			が成り立つことを示せばよい.
			以下,分割$D$に対するRiemann和$\sum_D f^k_j(x_{t_{i-1}})(x^j_{t_i} - x^j_{t_{i-1}})$を$\Sigma_D$と略記する.
			定理\ref{thm:existence_of_Riemann_Stieltjes_integral}より,
			任意の$\epsilon > 0$に対して或る$\delta > 0$が存在し,
			\begin{align}
				|D_1|,|D_2|,|D_3| < \delta,
				\quad \left(D_1 \in \delta[s,u],\ D_2 \in \delta[u,t],\ D_3 \in \delta[s,t] \right)
			\end{align}
			である限り
			\begin{align}
				\left| \int_s^u f^k_j(x_r)\ dx^j_r - \Sigma_{D_1} \right| < \epsilon,
				\quad \left| \int_u^t f^k_j(x_r)\ dx^j_r - \Sigma_{D_2} \right| < \epsilon,
				\quad \left| \int_s^t f^k_j(x_r)\ dx^j_r - \Sigma_{D_3} \right| < \epsilon
			\end{align}
			が成立する.$|D_1|,|D_2| < \delta/2$を満たす
			$D_1,D_2$を取り$D_3$をその合併とすれば,$|D_3| < \delta$かつ
			\begin{align}
				\Sigma_{D_1} + \Sigma_{D_2} = \Sigma_{D_3}
			\end{align}
			が成り立ち,
			\begin{align}
				&\left| \int_s^u f^k_j(x_r)\ dx^j_r + \int_u^t f^k_j(x_r)\ dx^j_r
					- \int_s^t f^k_j(x_r)\ dx^j_r \right| \\
				&\qquad \leq \left| \int_s^u f^k_j(x_r)\ dx^j_r - \Sigma_{D_1} \right|
				+ \left| \int_u^t f^k_j(x_r)\ dx^j_r - \Sigma_{D_2} \right|
				+ \left| \int_s^t f^k_j(x_r)\ dx^j_r - \Sigma_{D_3} \right| \\
				&\qquad < 3\epsilon
			\end{align}
			が従い(\refeq{eq:thm_linearity_of_Riemann_Stieltjes_integral_1})を得る.
		
		\item[(2)] 略.
		\QED
	\end{description}
\end{prf}

\end{comment}
$C^1$は次で定めるノルム$\Norm{\cdot}{C^1}$によりBanach空間となる:
\begin{align}
	\Norm{x}{C^1} \coloneqq
	\sup{t \in [0,T]}{|x(t)|} + \sup{t \in [0,T]}{|\dot{x}(t)|}.
\end{align}

\begin{screen}
	\begin{thm}[$\Norm{\cdot}{C^1}$に関する連続性]\label{thm:T_s_t_continuous_w_r_t_C1_norm}
		$[s,t] \subset [0,T]$とし,$C^1$には$\Norm{\cdot}{C^1}$でノルム位相を入れる.
		このとき,$C^1 \ni x \longmapsto I_{s,t}(x) \in \R^m$は連続である.
	\end{thm}
\end{screen}

\begin{prf}
	$C^1$の第一可算性により点列連続性と連続性は一致するから,
	$x^n \longrightarrow x$のとき$I_{s,t}(x^n) \longrightarrow I_{s,t}(x)$
	が従うことを示せばよい.各$j,k$について
	\begin{align}
		\int_s^t f^k_j(x^n_u)\ dx^{n,j}_u
		\longrightarrow \int_s^t f^k_j(x_u)\ dx^j_u,
		\quad (n \longrightarrow \infty)
		\label{eq:thm_T_s_t_continuous_w_r_t_C1_norm}
	\end{align}
	が成り立つことを示せば十分である.
	連続性より$M \coloneqq \sup{u \in [s,t]}{|f(x_u)|} < \infty$が定まり
	\begin{align}
		&\left| \int_s^t f^k_j(x^n_u)\ dx^{n,j}_u
			- \int_s^t f^k_j(x_u)\ dx^j_u \right|
		= \left| \int_s^t f^k_j(x^n_u)\dot{x}^{n,j}_u\ du
			- \int_s^t f^k_j(x_u)\dot{x}^j_u\ du \right| \\
		&\qquad \leq \int_s^t \left| f^k_j(x^n_u)\dot{x}^{n,j}_u - f^k_j(x^n_u)\dot{x}^j_u \right|\ du
			+ \int_s^t \left| f^k_j(x^n_u)\dot{x}^j_u - f^k_j(x_u)\dot{x}^j_u \right|\ du \\
		&\qquad \leq M \Norm{x^n - x}{C^1} (t-s)
			+ \sup{u \in [s,t]}{\left| f^k_j(x^n_u) - f^k_j(x_u) \right|} \Norm{x}{C^1}(t-s)
			\label{eq:thm_T_s_t_continuous_w_r_t_C1_norm_2}
	\end{align}
	が成り立つ.いま,任意に$\epsilon > 0$を取れば,或る$\epsilon > \delta > 0$が存在して
	$v,w \in x([s,t]),\ |v - w| < \delta$なら$|f^k_j(v) - f^k_j(w)| < \epsilon$
	を満たす(一様連続).すなわち$\Norm{x^{(n)} - x}{C^1} < \delta$なら
	\begin{align}
		\sup{t \in [s,t]}{\left|f^k_j(x^n_t) - f^k_j(x_t) \right|} < \epsilon
	\end{align}
	が成立する.$\Norm{x^n - x}{C^1} \longrightarrow 0$の仮定より,
	或る自然数$N$が存在して$\Norm{x^n - x}{C^1} < \delta\ (n > N)$が満たされるから,
	$(\refeq{eq:thm_T_s_t_continuous_w_r_t_C1_norm_2}) < \epsilon[M(t-s) + \Norm{x}{C^1}(t-s)]\ (n > N)$
	が成り立ち(\refeq{eq:thm_T_s_t_continuous_w_r_t_C1_norm})が従う.
	\QED
\end{prf}

\begin{screen}
	\begin{dfn}[$p$-variation]
		$[0,T]$上の$\R^d$値関数$x$に対し,$p$-variationを次で定める:
		\begin{align}
			\Norm{x}{p,[s,t]}
			\coloneqq \left\{ \sup{D \in \delta[s,t]}{\sum_{D} 
				\left| x_{t_i} - x_{t_{i-1}} \right|^p }\right\}^{1/p}.
		\end{align}
		特に,$\Norm{\cdot}{p,[0,T]}$を$\Norm{\cdot}{p}$と表記する.
		また$p \geq 1$として,線形空間$B_{p,T}(\R^d)$を
		\begin{align}
			B_{p,T}(\R^d)
			\coloneqq \Set{x:[0,T] \longrightarrow \R^d}{x_0=0,\ x:\mbox{continuous},\ \Norm{x}{p} < \infty}
		\end{align}
		により定める.
	\end{dfn}
\end{screen}

次の結果によれば,$0 < p < 1$に対し$B_{p,T}(\R^d)$を定めても
$0$の定数関数のみの空間でしかない.
\begin{screen}
	\begin{thm}[$0 < p < 1$に対して有界$p$-variationなら定数]
		$x:[0,T] \longrightarrow \R^d$を連続関数とする.
		このとき,$p \in (0, 1)$に対し$\Norm{x}{p} < \infty$が成り立つなら$x$は定数関数である.
	\end{thm}
\end{screen}

\begin{prf}
	$t \in [0,T]$を任意に取り固定する.このとき全ての$D \in \delta[0,t]$に対して,
	\begin{align}
		|x_t - x_0| \leq \sum_D \left| x_{t_i} - x_{t_{i-1}} \right|
		&\leq \max{D}{\left| x_{t_i} - x_{t_{i-1}} \right|^{1-p}} 
			\sum_D \left| x_{t_i} - x_{t_{i-1}} \right|^p \\
		&\leq \max{D}{\left| x_{t_i} - x_{t_{i-1}} \right|^{1-p}} \Norm{x}{p}
	\end{align}
	が成り立ち,$x$の一様連続性から右辺は$|D| \longrightarrow 0$で$0$に収束し,
	$x_t = x_0$が従う.
	\QED
\end{prf}

$p \geq 1$の場合,Minkowskiの不等式によれば,任意の$D \in \delta[s,t]$に対し
\begin{align}
	\left\{ \sum_D \left| (x_{t_i} + y_{t_i}) - (x_{t_{i-1}} + y_{t_{i-1}}) \right|^p \right\}^{1/p}
	&\leq \left\{ \sum_D \left| x_{t_i} - x_{t_{i-1}} \right|^p \right\}^{1/p}
		+ \left\{ \sum_D \left| y_{t_i} - y_{t_{i-1}} \right|^p \right\}^{1/p} \\
	&\leq \Norm{x}{p,[s,t]} + \Norm{y}{p,[s,t]}
\end{align}
が成り立ち$\Norm{x + y}{p,[s,t]} \leq \Norm{x}{p,[s,t]} + \Norm{y}{p,[s,t]}$を得る.

\begin{screen}
	\begin{thm}
		$B_{p,T}(\R^d)$は$\Norm{\cdot}{p}$をノルムとするBanach空間である.
	\end{thm}
\end{screen}

\begin{prf}完備性を示す.
	\begin{description}
		\item[第一段] $(x^n)_{n=1}^{\infty} \subset B_{p,T}(\R^d)$をCauchy列とすれば,
			任意の$\epsilon > 0$に対して或る$n_\epsilon \in \N$が存在し
			\begin{align}
				\Norm{x^n - x^m}{p}
				= \left\{ \sup{D \in \delta[0,T]}{\sum_{D} 
				\left| \left( x^n_{t_i} - x^m_{t_i} \right) 
				- \left(x^n_{t_{i-1}} - x^m_{t_{i-1}} \right) \right|^p }\right\}^{1/p} < \epsilon,
				\quad (n,m > n_\epsilon)
			\end{align}
			を満たす.いま,任意の$t \in [0,T]$に対して$[0,T]$の分割$D = \{0 \leq t \leq T\}$
			を考えれば
			\begin{align}
				|x^n_t - x^m_t| < \epsilon,
				\quad (n,m > n_\epsilon)
			\end{align}
			が得られ,実数の完備性より或る$x_t \in \R^d$が存在して
			\begin{align}
				|x^n_t - x_t| < \epsilon
				\quad (n > n_\epsilon)
			\end{align}
			を満たす.この収束は$t$に関して一様であるから,$t \longmapsto x_t$は0出発かつ連続である.
			
		\item[第二段] $\Norm{x^n - x}{p} \longrightarrow 0\ (n \longrightarrow \infty)$を示す.
			前段によれば,任意の$D \in \delta[0,T]$に対し
			\begin{align}
				\sum_D \left| (x^m_{t_i} - x^n_{t_i}) - (x^m_{t_{i-1}} - x^n_{t_{i-1}}) \right|^p
				< \epsilon^p,
				\quad (n,m > n_\epsilon)
			\end{align}
			が成り立っている.$D$はせいぜい有限個の分割であるから,$m \longrightarrow \infty$として
			\begin{align}
				\sum_D \left| (x_{t_i} - x^n_{t_i}) - (x_{t_{i-1}} - x^n_{t_{i-1}}) \right|^p
				< \epsilon^p,
				\quad (n > n_\epsilon)
			\end{align}
			が従い,$D$の任意性より$\Norm{x^n - x}{p} < \epsilon\ (n > n_\epsilon)$を得る.
			\QED
	\end{description}
\end{prf}

\begin{screen}
	\begin{thm}\label{thm:C1_p_variation_topology}
		$p \geq 1$とする.また$x_0 = 0$を満たす$x \in C^1$の全体が作る線形空間を$\tilde{C}^1$とおく.
		\begin{description}
			\item[(1)] $x \in C^1$ならば$\Norm{x}{p} < \infty$が成り立つ.
				ただちに,$\Norm{\cdot}{p}$は$\tilde{C}^1$においてノルムとなる.
			\item[(2)] $\tilde{C}^1$において,$\Norm{\cdot}{C^1}$で導入する位相は
				$\Norm{\cdot}{p}$で導入する位相より強い.
		\end{description}
	\end{thm}
\end{screen}

\begin{prf}\mbox{}
	\begin{description}
		\item[$p = 1$の場合]
			平均値の定理より,任意の$D \in \delta[0,T]$に対し
			\begin{align}
				\sum_D \left| x_{t_i} - x_{t_{i-1}} \right|
				\leq \sum_D \Norm{x}{C^1}(t_i - t_{i-1})
				= \Norm{x}{C^1}T < \infty
			\end{align}
			が成り立ち$\Norm{x}{1} < \infty$が従う.
		
		\item[$p > 1$の場合] $q$を$p$の共役指数とする.
			任意の$D \in \delta[0,T]$に対し,H\Ddot{o}lderの不等式より
			\begin{align}
				\sum_D \left| x_{t_i} - x_{t_{i-1}} \right|^p
				&= \sum_D \left| \int_{t_{i-1}}^{t_i} \dot{x}_{u}\ du \right|^p 
				\leq \sum_D (t_i - t_{i-1})
					\biggl( \int_{t_{i-1}}^{t_i} | \dot{x}_{u} |^q\ du \biggr)^{p/q} \\
				&\leq \sum_D (t_i - t_{i-1})
					\biggl(\int_{0}^{T} \Norm{x}{C^1}^q\ du \biggr)^{p/q}
				= \Norm{x}{C^1}^p T^p
			\end{align}
			が成立し,$\Norm{x}{p} < \infty$が従う.
	\end{description}
	以上より,$p \geq 1$ならば$\Norm{x}{p} \leq T\Norm{x}{C^1}\ (x \in C^1)$が成り立ち(2)の主張を得る.
	\QED
\end{prf}

	次節の考察対象は主に定理\ref{thm:T_s_t_continuous_w_r_t_C1_norm}と
	定理\ref{thm:C1_p_variation_topology}に関係する.
	定理\ref{thm:T_s_t_continuous_w_r_t_C1_norm}によれば,
	$C^1$に$\Norm{\cdot}{C^1}$でノルム位相を導入した場合,
	$f \in C(\R^d,L(\R^d \rightarrow \R^m))$に対して
	$C^1 \ni x \longmapsto I_{s,t}(x)$は連続である.
	一方で定理\ref{thm:T_s_t_continuous_w_r_t_C1_norm}によれば,
	0出発$C^1$-パス空間$\tilde{C}^1$に$\Norm{\cdot}{p}$でノルム位相を導入した場合,
	$\tilde{C}^1 \ni x \longmapsto I_{s,t}(x)$が連続であるという
	保証はない.しかし,次節以後の結果により,
	$1 \leq p < 3$かつ$f \in C^2(\R^d,L(\R^d \rightarrow \R^m))$が満たされているなら
	$\tilde{C}^1 \ni x \longmapsto I_{s,t}(x)$は或る意味での連続性を持つ.
