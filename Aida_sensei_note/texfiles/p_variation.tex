以下,$x \in \R^d$に対して,各成分を$x^i\ (1 \leq i \leq d)$とし
$x = (x^1,\cdots,x^d)$と表す.
また実数の$d \times m$行列$f$を$f=(f_{kj})$と表す.
$C^1 = C^1([0,T] \rightarrow \R^d)$とおく.
ただし端点においては片側微分を考える.
区間$[0,T]$の分割を
$D = \{0 = t_0 < t_1 < \cdots < t_N = T\}$で表現し
$|D| \coloneqq \max{1 \leq i \leq N}{\left| t_i - t_{i-1} \right|}$とおく.
また$\sup{D}{\cdots}$などと書く場合は,任意の分割数,分点で構成される全ての$D$にわたる上限を考える.

\begin{screen}
	\begin{thm}
		任意の$x \in C^1,\ f \in C_b^\infty(\R^d,L(\R^d \rightarrow \R^m))$に対し,
		次の極限が確定する:
		\begin{align}
			\lim_{|D| \to 0}
			\sum_{i=1}^{N} f(x_{s_{i-1}})(x_{t_i} - x_{t_{i-1}})
			\in \R^m.
		\end{align}
	\end{thm}
\end{screen}

\begin{prf}
	$x = (x^1,\cdots,x^d),
	f = (f_{kj})_{1 \leq k \leq m,1 \leq j \leq d}$と書けば,
	\begin{align}
		\sum_{i=1}^{N} f(x_{s_{i-1}})(x_{t_i} - x_{t_{i-1}})
	\end{align}
	の第$k\ (1 \leq k \leq m)$成分は
	\begin{align}
		\sum_{j=1}^{d} \sum_{i=1}^{N} f_{kj}(x_{s_{i-1}})(x^j_{t_i} - x^j_{t_{i-1}})
	\end{align}
	と表現できる.従って
	\begin{align}
		\lim_{|D| \to 0} \sum_{i=1}^{N} f_{kj}(x_{s_{i-1}})(x^j_{t_i} - x^j_{t_{i-1}})
	\end{align}
	が確定すれば各成分の極限が確定し主張を得る.
	$x$が$C^1$級なので$x^j$も$C^1$級である.また
	$f_{kj}:\R^d \longrightarrow \R$も$C_b^\infty$に属する.
	分割$D$に対し
	\begin{align}
		S_D \coloneqq \sum_{i=1}^{N} M_i(t_i - t_{i-1}), \\
		S_d \coloneqq \sum_{i=1}^{N} m_i(t_i - t_{i-1})
	\end{align}
	とおき,
	\begin{align}
		S \coloneqq \inf{D}{S_D},
		\quad s \coloneqq \sup{D}{s_D}
	\end{align}
	とおけば,
	\begin{align}
		s_D \leq s \leq S \leq S_D,
		\quad s_D \leq \Sigma_D \leq S_D
	\end{align}
	が成立する.$s = S$なら$\Sigma_D \longrightarrow S$が従う.
\end{prf}

\begin{screen}
	\begin{dfn}
		$I_{0,T}(x)$の定義.
	\end{dfn}
\end{screen}

$C^1$において次でノルム$\Norm{\cdot}{C^1}$を定める:
\begin{align}
	\Norm{x}{\infty} &\coloneqq \sup{t \in [0,T]}{|x(t)|},
	\quad \Norm{x'}{\infty} \coloneqq \sup{t \in [0,T]}{|x'(t)|}, \\
	\Norm{x}{C^1} &\coloneqq
	\Norm{x}{\infty} + \Norm{x}{\infty}.
\end{align}

\begin{screen}
	\begin{thm}
		$x \longmapsto I_{0,T}(x)$は連続である.
	\end{thm}
\end{screen}

\begin{prf}
	$C^1$は次のノルムでBanach空間になる.ゆえに
	$x \longmapsto I_{0,T}(x)$は距離空間から距離空間への対応であり
	点列連続性と連続性が一致するから,
	$x_n \longrightarrow x$なら$I_{0,T}(x_n) \longrightarrow I_{0,T}(x)$
	が従うことを示せばよい.実際,$f,x$の連続性より
	\begin{align}
		\sum_{i=1}^{N} f(x^{(n)}_{s_{i-1}})(x^{(n)}_{t_i} - x^{(n)}_{t_{i-1}})
		\longrightarrow \sum_{i=1}^{N} f(x_{s_{i-1}})(x_{t_i} - x_{t_{i-1}})
	\end{align}
	が成り立つ.
\end{prf}

\begin{screen}
	\begin{dfn}[$p$-variation]
		$[0,T]$上の$\R^d$値関数$x$に対し$p$-variationノルムを次で定める:
		\begin{align}
			\Norm{x}{p}
			\coloneqq \left\{ \sup{D}{\sum_{i=1}^{N} 
				\left| x_{t_i} - x_{t_{i-1}} \right|^p }\right\}^{1/p}.
		\end{align}
		また線形空間$B_{p,T}(\R^d)$を
		\begin{align}
			B_{p,T}(\R^d)
			\coloneqq \Set{x:[0,T] \longrightarrow \R^d}{x_0=0,\ x:\mbox{continuous},\ \Norm{x}{p} < \infty}
		\end{align}
		により定める.
	\end{dfn}
\end{screen}

\begin{screen}
	\begin{thm}
		$\tilde{C}^1 \coloneqq \Set{x \in C^1}{x_0 = 0}$とおくと,
		$\tilde{C}^1 \subset B_{p,T}(\R^d)$が成り立つ.
	\end{thm}
\end{screen}

\begin{prf}
	$x \in \tilde{C}^1$に対して
	\begin{align}
		M \coloneqq \sum_{j=1}^{d} \sup{x \in [0,T]}{|x^j(t)|}
	\end{align}
	とおけば,$x'$の連続性より$M < \infty$が定まる.
	平均値の定理より,$|D| < 1$を満たす分割$D$に対して
	\begin{align}
		\left\{ \sum_{i=1}^{N} |x_{t_i} - x_{t_{i-1}}|^p \right\}^{1/p}
		\leq \left\{ \sum_{i=1}^{N} \Norm{x}{C^1}^p (t_i - t_{i-1})^p \right\}^{1/p}
		\leq M T < \infty
	\end{align}
	が成立し$\Norm{x}{p} \leq MT < \infty$が従う
	\footnote{
		$S_D \geq 0$ならば$(\sup{D}S_D)^{1/p} = \sup{D}S_D^{1/p}$が成り立つ.
	}
	.
\end{prf}

\begin{screen}
	\begin{thm}
		$B_{p,T}(\R^d)$はBanach空間である.
	\end{thm}
\end{screen}

\begin{prf}
	$(x^n)_{n=1}^{\infty} \subset B_{p,T}(\R^d)$をCauchy列とする.
	つまり任意の$\epsilon > 0$に対して或る$n_\epsilon \in \N$が存在し
	\begin{align}
		\Norm{x^n - x^m}{p}
		= \left\{ \sup{D}{\sum_{i=1}^{N} 
		\left| \left( x^n_{t_i} - x^m_{t_i} \right) 
		- \left(x^n_{t_{i-1}} - x^m_{t_{i-1}} \right) \right|^p }\right\}^{1/p} < \epsilon,
		\quad (n,m > n_\epsilon)
	\end{align}
	を満たす.いま,任意の$t \in [0,T]$に対して$[0,T]$の分割$\{0=t_0 \leq t \leq T\}$
	を考えれば
	\begin{align}
		|x^n_t - x^m_t| < \epsilon,
		\quad (n,m > n_\epsilon)
	\end{align}
	が得られ,実数の完備性より或る$x_t \in \R^d$が存在して
	\begin{align}
		|x^n_t - x_t| < \epsilon
		\quad (n > n_\epsilon)
	\end{align}
	を満たす.実際,もし或る$n > n_\epsilon$で$|x^n_t - x_t| \eqqcolon \alpha \geq \epsilon$
	が成り立つと,任意の$m > n_\epsilon$に対して
	\begin{align}
		|x^m_t - x_t| \geq |x^n_t - x_t| - |x^n_t - x^m_t| > \alpha - \epsilon
	\end{align}
	が従い$x^m_t \longrightarrow x_t$に反する.
	ゆえに収束は$t$に関して一様であり,$t \longmapsto x_t$は0出発かつ連続である.
	あとは$\Norm{x^n - x}{p} \longrightarrow 0\ (n \longrightarrow \infty)$であればよい.
\end{prf}

\begin{screen}
	\begin{thm}
		$C^1$空間において,$\Norm{}{C^1}$で定まる位相は
		$\Norm{}{p}$で定まる位相より強い.
		特に,$B_{p,T}(\R^d)$上で考える写像$x \longmapsto I_{p,T}(x)$は
		$\Norm{\cdot}{p}$の定める位相により連続である.
	\end{thm}
\end{screen}

\begin{prf}
	任意の$x \in \tilde{C}^1$に対して
	\begin{align}
		\Norm{x}{p} \leq T \Norm{x}{C^1}
	\end{align}
	を満たすことを証明する.実際,任意の分割$D$に対して
	\begin{align}
		\sum_{i=1}^{N} |x_{t_i} - x_{t_{i-1}}|
		\leq \sum_{i=1}^{N} \Norm{x}{C^1} (t_i - t_{i-1})
		= T \Norm{x}{C^1}
	\end{align}
	が成り立つ.
	\QED
\end{prf}