\section{導入}
	以下,$x \in \R^d$について成分を込めて表現する場合は
	$x = (x^1,\cdots,x^d)$と書き,
	実$m \times d$行列$a$については$a=(a^i_j)_{1 \leq i \leq m,1 \leq j \leq d}$と表す.
	また$T > 0$を固定し$C^1 = C^1([0,T] \rightarrow \R^d)$とおく.
	ただし端点においては片側微分を考える.
	区間$[s,t] \subset [0,T]$の分割を
	$D = \{s = t_0 < t_1 < \cdots < t_N = t\}$で表現し
	$|D| \coloneqq \max{1 \leq i \leq N}{\left| t_i - t_{i-1} \right|}$とおく.
	また$[s,t]$の分割の全体を$\delta[s,t]$と書く.また
	\begin{align}
		\sum_D = \sum_{i=1}^{N}
	\end{align}
	と書く.

	\begin{screen}
		\begin{thm}[Riemann-Stieltjes積分]\label{thm:existence_of_Riemann_Stieltjes_integral}
			$[s,t] \subset [0,T]$とし,$D \in \delta[s,t]$についてのみ考えるとき,
			任意の$x \in C^1,\ f \in C(\R^d,L(\R^d \rightarrow \R^m))$
			に対して次の極限が存在する:\footnotemark
			\begin{align}
				\lim_{|D| \to 0}
				\sum_{D} f(x_{s_{i-1}})(x_{t_i} - x_{t_{i-1}})
				\in \R^m.
			\end{align}
			ここで$s_{i-1}$は区間$[t_{i-1},t_i]$に属する任意の点であり,
			極限は$s_{i-1}$の取り方に依らず確定する.
		\end{thm}
	\end{screen}
	\footnotetext{
		極限の存在を保証する条件としては,$f$の有界性と微分可能性は必要ない.
	}
	\begin{prf}
		各$x^j$は$C^1$-級であるから,平均値の定理より
		$\sum_{D} f(x_{s_{i-1}})(x_{t_i} - x_{t_{i-1}})$
		の第$k$成分を
		\begin{align}
			&\sum_{j=1}^{d} \sum_{D} f^k_j (x_{s_{i-1}})(x^j_{t_i} - x^j_{t_{i-1}}) 
				\label{eq:thm_existence_of_Riemann_Stieltjes_integral}\\
			&\qquad 
			= \sum_{j=1}^{d} \sum_{D} f^k_j (x_{s_{i-1}}) \tfrac{d}{dt}x^j(\xi_{i-1,j})(t_i - t_{i-1}),
			\quad ({}^\exists \xi_{i-1,j} \in [t_{i-1},t_i])
		\end{align}
		と表現できる.各$j,k$について
		\begin{align}
			\lim_{|D| \to 0} \sum_{D} f^k_j (x_{s_{i-1}}) \tfrac{d}{dt}x^j(\xi_{i-1,j})(t_i - t_{i-1})
		\end{align}
		が確定すれば,第$k$成分の極限が確定し定理の主張を得る.
		いま,$t \longrightarrow f^k_j(x_t)$及び$t \longmapsto (d/dt)x^j_t$は($[s,t]$上一様)連続であるから,
		分割$D$による各区間$[t_{i-1},t_i]$において次の最大最小値が定まる:
		\begin{align}
			M_{i-1} \coloneqq \sup{t_{i-1} \leq t \leq t_i} f^k_j(x_t)\tfrac{d}{dt}x^j_t,
			\quad m_{i-1} \coloneqq \inf{t_{i-1} \leq t \leq t_i} f^k_j(x_t)\tfrac{d}{dt}x^j_t.
		\end{align}
		ここで
		\begin{align}
			S_D \coloneqq \sum_{D} M_{i-1}(t_i - t_{i-1}),
			\quad s_D \coloneqq \sum_{D} m_{i-1}(t_i - t_{i-1}),
			\quad \Sigma_D \coloneqq \sum_{D} f^k_j (x_{s_{i-1}}) \tfrac{d}{dt}x^j(\xi_{i-1})(t_i - t_{i-1})
		\end{align}
		とおいて
		\begin{align}
			S \coloneqq \inf{D \in \delta[s,t]}{S_D},
			\quad s \coloneqq \sup{D \in \delta[s,t]}{s_D}
		\end{align}
		を定めれば
		\begin{align}
			s_D \leq s \leq S \leq S_D,
			\quad s_D \leq \Sigma_D \leq S_D
		\end{align}
		が満たされる.実際,任意の$D_1,D_2 \in \delta[s,t]$に対して,
		分割の合併を$D_3$とすれば
		\begin{align}
			s_{D_1} \leq s_{D_3} \leq S_{D_3} \leq S_{D_2}
		\end{align}
		が成立し$s \leq S_D\ (\forall D \in \delta[s,t])$すなわち$s \leq S$が出る.
		一方で一様連続性から
		\begin{align}
			0 \leq S - s \leq S_D - s_D = \sum_D (M_{i-1} - m_{i-1})(t_i - t_{i-1})
			\longrightarrow 0
			\quad (|D| \longrightarrow 0)
		\end{align}
		が従い$s = S$を得る.以上より
		\begin{align}
			|S - \Sigma_D| \leq |S - S_D| + |S_D - \Sigma_D|
			\leq |S - S_D| + |S_D - s_D|
			\longrightarrow 0
			\quad (|D| \longrightarrow 0)
		\end{align}
		が成り立つ.
		\QED
\end{prf}

上の証明において,各$k,j$ごとに定まる極限$S$を
\begin{align}
	S = \int_s^t f^k_j(x_u)\ dx^j_u
\end{align}
と書く.

\begin{screen}
	\begin{dfn}[$C^1$-級のパスに対する汎関数]
		$x \in C^1$と$f \in C(\R^d,L(\R^d \rightarrow \R^m))$に対して,
		$[s,t] \subset [0,T]$におけるRiemann-Stieltjes積分を次で表現する:
		\begin{align}
			I_{s,t}(x) = \int_s^t f(x_u)\ dx_u 
			&\coloneqq \lim_{|D| \to 0}
				\sum_{D} f(x_{t_{i-1}})(x_{t_i} - x_{t_{i-1}}), \\
			\left[ \int_s^t f(x_u)\ dx_u \right]^k
			&= \sum_{j=1}^d \int_s^t f^k_j(x_u)\ dx^j_u,
			\quad (k = 1,\cdots,m).
		\end{align}
		ただし$D \in \delta[s,t]$のみを考える.
	\end{dfn}
\end{screen}

\begin{screen}
	\begin{thm}[Riemann-Stieltjes積分の線型性]
	\label{thm:linearity_of_Riemann_Stieltjes_integral}
		$x \in C^1,\ f \in C(\R^d,L(\R^d \rightarrow \R^m))$とする.
		\begin{description}
			\item[(1)] 任意の$0 \leq s < u < t \leq T$に対し
				$I_{s,u}(x) + I_{u,t}(x) = I_{s,t}(x)$が成り立つ.
				
			\item[(2)] $\alpha,\beta \in \R$と$g \in C(\R^d,L(\R^d \rightarrow \R^m))$に対して
				\begin{align}
					\int_s^t \alpha f(x_u) + \beta g(x_u)\ dx_u
					= \alpha \int_s^t f(x_u)\ dx_u + \beta \int_s^t g(x_u)\ dx_u.
				\end{align}
				が成り立つ.
		\end{description}
	\end{thm}
\end{screen}

\begin{prf}\mbox{}
	\begin{description}
		\item[(1)] 
			各$k,j$に対して
			\begin{align}
				\int_s^u f^k_j(x_r)\ dx^j_r
				+ \int_u^t f^k_j(x_r)\ dx^j_r
				= \int_s^t f^k_j(x_r)\ dx^j_r
				\label{eq:thm_linearity_of_Riemann_Stieltjes_integral_1}
			\end{align}
			が成り立つことを示せばよい.
			以下,分割$D$に対するRiemann和$\sum_D f^k_j(x_{t_{i-1}})(x^j_{t_i} - x^j_{t_{i-1}})$を$\Sigma_D$と略記する.
			定理\ref{thm:existence_of_Riemann_Stieltjes_integral}より,
			任意の$\epsilon > 0$に対して或る$\delta > 0$が存在し,
			\begin{align}
				|D_1|,|D_2|,|D_3| < \delta,
				\quad \left(D_1 \in \delta[s,u],\ D_2 \in \delta[u,t],\ D_3 \in \delta[s,t] \right)
			\end{align}
			である限り
			\begin{align}
				\left| \int_s^u f^k_j(x_r)\ dx^j_r - \Sigma_{D_1} \right| < \epsilon,
				\quad \left| \int_u^t f^k_j(x_r)\ dx^j_r - \Sigma_{D_2} \right| < \epsilon,
				\quad \left| \int_s^t f^k_j(x_r)\ dx^j_r - \Sigma_{D_3} \right| < \epsilon
			\end{align}
			が成立する.$|D_1|,|D_2| < \delta/2$を満たす
			$D_1,D_2$を取り$D_3$をその合併とすれば,$|D_3| < \delta$かつ
			\begin{align}
				\Sigma_{D_1} + \Sigma_{D_2} = \Sigma_{D_3}
			\end{align}
			が成り立ち,
			\begin{align}
				&\left| \int_s^u f^k_j(x_r)\ dx^j_r + \int_u^t f^k_j(x_r)\ dx^j_r
					- \int_s^t f^k_j(x_r)\ dx^j_r \right| \\
				&\qquad \leq \left| \int_s^u f^k_j(x_r)\ dx^j_r - \Sigma_{D_1} \right|
				+ \left| \int_u^t f^k_j(x_r)\ dx^j_r - \Sigma_{D_2} \right|
				+ \left| \int_s^t f^k_j(x_r)\ dx^j_r - \Sigma_{D_3} \right| \\
				&\qquad < 3\epsilon
			\end{align}
			が従い(\refeq{eq:thm_linearity_of_Riemann_Stieltjes_integral_1})を得る.
		
		\item[(2)]
	\end{description}
\end{prf}
$C^1$において次でノルム$\Norm{\cdot}{C^1}$を定める:
\begin{align}
	\Norm{x}{\infty} \coloneqq \sup{t \in [0,T]}{|x(t)|},
	\quad \Norm{x'}{\infty} \coloneqq \sup{t \in [0,T]}{|x'(t)|},
	\quad \Norm{x}{C^1} \coloneqq
	\Norm{x}{\infty} + \Norm{x'}{\infty}.
\end{align}
以降,$C^1$には$\Norm{\cdot}{C^1}$によりノルム位相を導入する.

\begin{screen}
	\begin{thm}[有界な$f$のStieltjes積分は$x$に関し連続]
		$[s,t] \subset [0,T]$とする.$x \in C^1$と$f \in C_b(\R^d,L(\R^d \rightarrow \R^m))$により定める
		$I_{s,t}(x)$について,$C^1 \ni x \longmapsto I_{s,t}(x) \in \R^m$は連続である.
	\end{thm}
\end{screen}

\begin{prf}
	$C^1$の各点は可算な基本近傍系を持つから
	$x \longmapsto I_{0,T}(x)$の点列連続性と連続性は一致する.
	すなわち$x^{(n)} \longrightarrow x$なら$I_{0,T}(x^{(n)}) \longrightarrow I_{0,T}(x)$
	が従うことを示せばよい.今回も各$j,k$について
	\begin{align}
		\int_s^t f^k_j(x^{(n)}_u)\ dx^{(n),j}_u
		\longrightarrow \int_s^t f^k_j(x_u)\ dx^j_u,
		\quad (n \longrightarrow \infty)
	\end{align}
	が成り立つことを示せば十分である.
	\begin{description}
		\item[第一段]
	\end{description}
	任意の分割$D \in \delta[s,t]$に対し,平均値の定理を使うと以下のように式変形される:
	\begin{align}
		&\left| \sum_D f^k_j(x^{(n)}_{t_{i-1}})(x^{(n),j}_{t_i} - x^{(n),j}_{t_{i-1}})
			- \sum_D f^k_j(x_{t_{i-1}})(x^j_{t_i} - x^j_{t_{i-1}}) \right| \\
		\leq &\sum_D \left| f^k_j(x^{(n)}_{t_{i-1}}) \tfrac{d}{dt}x^{(n),j}_{\xi_{i-1},j} 
			- f^k_j(x_{t_{i-1}}) \tfrac{d}{dt}x^j_{\eta_{i-1},j} \right|(t_i - t_{i-1}) \\
		\leq &\sum_D \left| f^k_j(x^{(n)}_{t_{i-1}}) \tfrac{d}{dt}x^{(n),j}_{\xi_{i-1},j} 
			+ f^k_j(x_{t_{i-1}}) \tfrac{d}{dt}x^{(n),j}_{\eta_{i-1},j} \right.\\
			&\qquad \left.- f^k_j(x_{t_{i-1}}) \tfrac{d}{dt}x^{(n),j}_{\eta_{i-1},j}
			- f^k_j(x_{t_{i-1}}) \tfrac{d}{dt}x^j_{\eta_{i-1},j} \right|(t_i - t_{i-1}) \\
		\leq &\sum_D \left| f^k_j(x^{(n)}_{t_{i-1}}) \tfrac{d}{dt}x^{(n),j}_{\xi_{i-1},j} 
			- f^k_j(x_{t_{i-1}}) \tfrac{d}{dt}x^{(n),j}_{\eta_{i-1},j} \right|(t_i - t_{i-1}) \\
			&\qquad + \sum_D \left|f^k_j(x_{t_{i-1}}) \tfrac{d}{dt}x^{(n),j}_{\eta_{i-1},j}
			- f^k_j(x_{t_{i-1}}) \tfrac{d}{dt}x^j_{\eta_{i-1},j} \right|(t_i - t_{i-1}).
	\end{align}
	ここで最終式第二項については
	\begin{align}
		&\sum_D \left|f^k_j(x_{t_{i-1}}) \tfrac{d}{dt}x^{(n),j}_{\eta_{i-1},j}
			- f^k_j(x_{t_{i-1}}) \tfrac{d}{dt}x^j_{\eta_{i-1},j} \right|(t_i - t_{i-1}) \\
		&\qquad \leq \sum_D \Norm{f^k_j}{\infty} \Norm{x^{(n)} - x}{C^1} (t_i - t_{i-1})
		= \Norm{f^k_j}{\infty} \Norm{x^{(n)} - x}{C^1}(t - s)
	\end{align}
	が成り立ち,第一項については
	\begin{align}
		&\sum_D \left| f^k_j(x^{(n)}_{t_{i-1}}) \tfrac{d}{dt}x^{(n),j}_{\xi_{i-1},j} 
			- f^k_j(x_{t_{i-1}}) \tfrac{d}{dt}x^{(n),j}_{\eta_{i-1},j} \right|(t_i - t_{i-1}) \\
		\leq &\sum_D \left| f^k_j(x^{(n)}_{t_{i-1}}) \tfrac{d}{dt}x^{(n),j}_{\xi_{i-1},j} 
			- f^k_j(x^{(n)}_{t_{i-1}}) \tfrac{d}{dt}x^{j}_{\xi_{i-1},j} \right.\\
			&\qquad + f^k_j(x^{(n)}_{t_{i-1}}) \tfrac{d}{dt}x^{j}_{\xi_{i-1},j}
			- f^k_j(x^{(n)}_{t_{i-1}}) \tfrac{d}{dt}x^{j}_{\eta_{i-1},j} \\
			&\qquad + f^k_j(x^{(n)}_{t_{i-1}}) \tfrac{d}{dt}x^{j}_{\eta_{i-1},j}
			- f^k_j(x^{(n)}_{t_{i-1}}) \tfrac{d}{dt}x^{(n),j}_{\eta_{i-1},j} \\
			&\qquad + f^k_j(x^{(n)}_{t_{i-1}}) \tfrac{d}{dt}x^{(n),j}_{\eta_{i-1},j}
			\left. - f^k_j(x_{t_{i-1}}) \tfrac{d}{dt}x^{(n),j}_{\eta_{i-1},j} \right|(t_i - t_{i-1}) \\
		\leq &2\Norm{f^k_j}{\infty} \Norm{x^{(n)} - x}{C^1}(t - s)
			+ \Norm{f^k_j}{\infty}(t-s) \sup{|\xi - \eta| \leq |D|}{\left|x^j_\xi - x^j_\eta \right|} \\
			&\qquad + \Norm{x}{C^1}(t-s)\sup{t \in [0,T]}{\left|f^k_j(x^{(n)}_t) - f^k_j(x_t) \right|}
	\end{align}
	とできる.いま,
\end{prf}

\begin{screen}
	\begin{dfn}[$p$-variation]
		$[0,T]$上の$\R^d$値関数$x$に対し,$p$-variationを次で定める:
		\begin{align}
			\Norm{x}{p,[s,t]}
			\coloneqq \left\{ \sup{D \in \delta[s,t]}{\sum_{D} 
				\left| x_{t_i} - x_{t_{i-1}} \right|^p }\right\}^{1/p}.
		\end{align}
		特に,$\Norm{\cdot}{p,[0,T]}$を$\Norm{\cdot}{p}$と表記する.また$p \geq 1$として,
		線形空間$B_{p,T}(\R^d)$を
		\begin{align}
			B_{p,T}(\R^d)
			\coloneqq \Set{x:[0,T] \longrightarrow \R^d}{x_0=0,\ x:\mbox{continuous},\ \Norm{x}{p} < \infty}
		\end{align}
		により定める.
	\end{dfn}
\end{screen}

次の結果により,$0 < p < 1$に対し$B_{p,T}(\R^d)$を定めても
$0$の定数関数のみの空間でしかない.
\begin{screen}
	\begin{thm}[$0 < p < 1$に対して有界$p$-variationなら定数]
		$x:[0,T] \longrightarrow \R^d$を連続関数とする.
		このとき,$p \in (0, 1)$に対し$\Norm{x}{p} < \infty$が成り立つなら$x$は定数関数である.
	\end{thm}
\end{screen}

\begin{prf}
	$t \in [0,T]$を任意に取り固定する.このとき全ての$D \in \delta[0,t]$に対して,
	\begin{align}
		|x_t - x_0| \leq \sum_D \left| x_{t_i} - x_{t_{i-1}} \right|
		&\leq \max{D}{\left| x_{t_i} - x_{t_{i-1}} \right|^{1-p}} 
			\sum_D \left| x_{t_i} - x_{t_{i-1}} \right|^p \\
		&\leq \max{D}{\left| x_{t_i} - x_{t_{i-1}} \right|^{1-p}} \Norm{x}{p}
	\end{align}
	が成り立ち,$x$の一様連続性から右辺は$|D| \longrightarrow 0$で$0$に収束し,
	$x_t = x_0$が従う.
	\QED
\end{prf}

\begin{screen}
	\begin{thm}$[s,t] \subset [0,T]$とする.
		\begin{description}
			$x \in C^1$ならば$\Norm{x}{p,[s,t]} < \infty$が成り立つ.
			ただちに,$p$-variationは線形空間$C^1$においてノルムとなる.
		\end{description}
	\end{thm}
\end{screen}

\begin{prf}\mbox{}
	\begin{description}
		\item[$p = 1$の場合]
			平均値の定理より,任意の$D \in \delta[s,t]$に対し
			\begin{align}
				\sum_D \left| x_{t_i} - x_{t_{i-1}} \right|
				\leq \sum_D \sum_{j=1}^d \left| x^j_{t_i} - x^j_{t_{i-1}} \right|
				\leq \sum_D \sum_{j=1}^d \Norm{x}{C^1}(t_i - t_{i-1})
				= d\Norm{x}{C^1}(t - s) < \infty
			\end{align}
			が成り立ち$\Norm{x}{1} < \infty$が従う.
		
		\item[$p > 1$の場合] $q$を$p$の共役指数とする.
			任意の$D \in \delta[s,t]$に対し,H\Ddot{o}lderの不等式より
			\begin{align}
				\sum_D \left| x_{t_i} - x_{t_{i-1}} \right|^p
				&\leq \sum_D \sum_{j=1}^d \left| x^j_{t_i} - x^j_{t_{i-1}} \right|^p
				= \sum_D \sum_{j=1}^d \left| \int_{t_{i-1}}^{t_i} \tfrac{d}{dt} x^j_{u}\ du \right|^p \\
				&\qquad \leq \sum_D \sum_{j=1}^d (t_i - t_{i-1})
					\biggl( \int_{t_{i-1}}^{t_i} \left| \tfrac{d}{dt} x^j_{u} \right|^q\ du \biggr)^{p/q}
				\leq d\Norm{x}{C^1}^p(t - s)^p
			\end{align}
			が成立し,$\Norm{x}{p} < \infty$が従う.
			\QED
	\end{description}
\end{prf}

\begin{screen}
	\begin{thm}
		$B_{p,T}(\R^d)$は$\Norm{\cdot}{p}$をノルムとするBanach空間である.
	\end{thm}
\end{screen}

\begin{prf}\mbox{}
	\begin{description}
		\item[第一段]
			$\Norm{\cdot}{p}$がノルムであることについて,
			とくに三角不等式はMinkowskiの不等式より出る.
			
		\item[第二段] $(x^n)_{n=1}^{\infty} \subset B_{p,T}(\R^d)$をCauchy列とすれば,
			任意の$\epsilon > 0$に対して或る$n_\epsilon \in \N$が存在し
			\begin{align}
				\Norm{x^n - x^m}{p}
				= \left\{ \sup{D \in \delta[0,T]}{\sum_{D} 
				\left| \left( x^n_{t_i} - x^m_{t_i} \right) 
				- \left(x^n_{t_{i-1}} - x^m_{t_{i-1}} \right) \right|^p }\right\}^{1/p} < \epsilon,
				\quad (n,m > n_\epsilon)
			\end{align}
			を満たす.いま,任意の$t \in [0,T]$に対して$[0,T]$の分割$D = \{0 \leq t \leq T\}$
			を考えれば
			\begin{align}
				|x^n_t - x^m_t| < \epsilon,
				\quad (n,m > n_\epsilon)
			\end{align}
			が得られ,実数の完備性より或る$x_t \in \R^d$が存在して
			\begin{align}
				|x^n_t - x_t| < \epsilon
				\quad (n > n_\epsilon)
			\end{align}
			を満たす.
			\footnote{
				実際,もし或る$n > n_\epsilon$で$|x^n_t - x_t| \eqqcolon \alpha \geq \epsilon$
				が成り立つと,任意の$m > n_\epsilon$に対して
				\begin{align}
					|x^m_t - x_t| \geq |x^n_t - x_t| - |x^n_t - x^m_t| > \alpha - \epsilon
				\end{align}
				が従い$x^m_t \longrightarrow x_t$に反する.
			}
			この収束は$t$に関して一様であるから,$t \longmapsto x_t$は0出発かつ連続である.
			
		\item[第三段] $\Norm{x^n - x}{p} \longrightarrow 0\ (n \longrightarrow \infty)$を示す.
			前段によれば,任意の$D \in \delta[0,T]$に対し
			\begin{align}
				\sum_D \left| (x^m_{t_i} - x^n_{t_i}) - (x^m_{t_{i-1}} - x^n_{t_{i-1}}) \right|^p
				< \epsilon^p,
				\quad (n,m > n_\epsilon)
			\end{align}
			が成り立っている.$D$はせいぜい有限個の分割であるから,$m \longrightarrow \infty$として
			\begin{align}
				\sum_D \left| (x_{t_i} - x^n_{t_i}) - (x_{t_{i-1}} - x^n_{t_{i-1}}) \right|^p
				< \epsilon^p,
				\quad (n > n_\epsilon)
			\end{align}
			が従い,$D$の任意性より$\Norm{x^n - x}{p} < \epsilon\ (n > n_\epsilon)$を得る.
			\QED
	\end{description}
\end{prf}

\begin{screen}
	\begin{thm}
		$C^1$において,$\Norm{\cdot}{C^1}$で定まる位相は
		$\Norm{\cdot}{p}$で定まる位相より強い.
	\end{thm}
\end{screen}

\begin{prf}
	任意の$x \in C^1$に対し
	\begin{align}
		\Norm{x}{p} \leq T \Norm{x}{C^1}
	\end{align}
	が成り立てば,$\Norm{\cdot}{p}$による開集合は$\Norm{\cdot}{C^1}$による開集合である.
	実際,
	\begin{align}
		\sum_{i=1}^{N} |x_{t_i} - x_{t_{i-1}}|
		\leq \sum_{i=1}^{N} \Norm{x}{C^1} (t_i - t_{i-1})
		= T \Norm{x}{C^1}
	\end{align}
	が成り立つ.
	\QED
\end{prf}