\section{連続性定理の証明}
	\begin{screen}
		\begin{dfn}[control function]
			関数$\omega:\Delta_T \longrightarrow [0,\infty)$
			が任意の$0 \leq s \leq u \leq t \leq T$に対して
			\begin{align}
				\omega(s,u) + \omega(u,t) \leq \omega(s,t)
				\label{eq:control_function_subadditivity}
			\end{align}
			を満たすとき,$\omega$をcontrol functionと呼ぶ.
		\end{dfn}
	\end{screen}
	
	\begin{screen}
		\begin{dfn}[ノルム空間値写像の$p$-variation]
			$(V,\Norm{\cdot}{})$をノルム空間とするとき,$p \geq 1,\ [s,t] \subset [0,T]
			,\ \psi:\Delta_T \longrightarrow V$に対して
			\begin{align}
				\Norm{\psi}{p,[s,t]}
				\coloneqq \left\{ \sup{D \in \delta[s,t]}{ \sum_D \Norm{\psi_{t_{i-1},t_i}}{}^p} \right\}^{1/p}
			\end{align}
			と定める.特に$\Norm{\cdot}{p,[0,T]}$を$\Norm{\cdot}{p}$と書く.
		\end{dfn}
	\end{screen}
	
	\begin{screen}
		\begin{thm}[control functionの例]
			以下の関数$\omega:\Delta_T \longrightarrow [0,\infty)$はcontrol functionである.
			\begin{description}
				\item[(1)] $\omega \coloneqq \left( \omega_1^{1/p} + \omega_2^{1/p} \right)^p,
					\quad (p \geq 1,\ \omega_1,\omega_2:\mbox{control function}).$
				\item[(2)] $\omega:(s,t) \longmapsto \Norm{X^1}{p:[s,t]}^p,
					\quad (p \geq 1,\ x \in B_{p,T}(\R^d)).$
				\item[(3)] $\omega:(s,t) \longmapsto \Norm{X^2}{p:[s,t]}^p,
					\quad (p \geq 1,\ x \in C^1).$
			\end{description}
		\end{thm}
	\end{screen}
	
	行列$a = (a_j^i)$のノルムは$\Norm{a}{} = \sqrt{\sum_{i,j}|a_j^i|^2}$として考える.
	
	\begin{thm}\mbox{}
		\begin{description}
			\item[(1)]
			\item[(2)] 任意の$D_1 \in \delta[s,u],D_2 \in \delta[u,t]$に対して
				合併$D_1 \cup D_2$は$[s,t]$の分割であるから,
				\begin{align}
					\sum_{D_1}\left| x_{t_i} - x_{t_{i-1}} \right|^p
					+ \sum_{D_2}\left| x_{t_i} - x_{t_{i-1}} \right|^p
					\leq \sup{D \in \delta[s,t]}{\sum_{D} \left| x_{t_i} - x_{t_{i-1}} \right|^p}
					= \Norm{X^1}{p:[s,t]}^p
				\end{align}
				が成り立つ.左辺の$D_1,D_2$の取り方は独立であるから,それぞれに対し上限を取れば
				\begin{align}
					\Norm{X^1}{p:[s,u]}^p + \Norm{X^1}{p:[u,t]}^p
					\leq \Norm{X^1}{p:[s,t]}^p
				\end{align}
				が得られる.
				
			\item[(3)] 任意の$[s,t] \subset [0,T]$に対して
				$\Norm{X^2}{p:[s,t]}^p < \infty$が成立することを示せば,あとは
				(2)と同様の理由により(\refeq{eq:control_function_subadditivity})が従う.
				$p \geq 1$かつ$x \in C^1$の下では,$D = \{s=t_0 < \cdots < t_N = t\}$に対し
				\begin{align}
					\Norm{X^2_{t_{i-1},t_i}}{}
					&\leq \sum_{k,j=1}^d \left| \int_{t_{i-1}}^{t_i} (x^k_u - x^k_{t_{i-1}})\ dx^j_u \right|
					\leq \sum_{k,j=1}^d \int_{t_{i-1}}^{t_i} \Norm{x}{C^1}^2(u-t_{i-1})\ du \\
					&\leq d^2 \Norm{x}{C^1}^2 \left\{ \int_{t_{i-1}}^{t_i} (u-s)\ du \right\}^{1/p}
						\left\{ \int_{t_{i-1}}^{t_i} (u-s)\ du \right\}^{1-1/p} \\
					&\leq d^2 \Norm{x}{C^1}^2 \left\{ \int_{t_{i-1}}^{t_i} (u-s)\ du \right\}^{1/p}
						\left\{ \int_s^t (u-s)\ du \right\}^{1-1/p}
				\end{align}
				が成り立つから,
				\begin{align}
					&\sum_D \Norm{X^2_{t_{i-1},t_i}}{}^p
					\leq \sum_D d^{2p}\Norm{x}{C^1}^{2p} \left\{ \frac{1}{2}(t-s)^2 \right\}^{p-1}
						\int_{t_{i-1}}^{t_i} (u-s)\ du \\
					&\qquad= d^{2p}\Norm{x}{C^1}^{2p} \left\{ \frac{1}{2}(t-s)^2 \right\}^{p-1}
						\int_s^t (u-s)\ du
					= d^{2p}\Norm{x}{C^1}^{2p} \left\{ \frac{1}{2}(t-s)^2 \right\}^p
				\end{align}
				により$\Norm{X^2}{p:[s,t]}^p < \infty$を得る.
				\QED
		\end{description}
	\end{thm}
	
	\begin{screen}
		\begin{lem}
			
		\end{lem}
	\end{screen}
	
	\begin{prf}\mbox{}
		\begin{description}
			\item[(1)] $D = \{s=t_0 < \cdots < t_N = t\}\ (N \geq 2)$に対し,
				$D_{-1} \coloneqq D \backslash \{i\}$とおく.同様に
				$D_{-2} \coloneqq D_{-1} \backslash \{i'\}$と続けて
				$D_{-k}\ (k=1,2,\cdots,N-1)$を構成する.
		\end{description}
	\end{prf}
	
	\begin{screen}
		\begin{lem}
			\begin{description}
				\item[(1)]
				\item[(2)]
			\end{description}
		\end{lem}
	\end{screen}
	\begin{prf}\mbox{}
		\begin{description}
			\item[(1)] 
				\begin{align}
					&\left\{ \tilde{I}_{s,t}(x,D_{-k}) - \tilde{I}_{s,t}(x,D_{-k-1}) \right\}
					- \left\{ \tilde{I}_{s,t}(y,D_{-k}) - \tilde{I}_{s,t}(y,D_{-k-1}) \right\} \\
					&= \left\{ \int_0^1 (\nabla f)(x_{t_{i-1}} + \theta (x_{t_i} - x_{t_{i-1}}))\ d\theta \right\} X^1_{t_{i-1},t_i} \otimes X^1_{t_i,t_{i+1}} \\
						&\qquad - \left\{ \int_0^1 (\nabla f)(y_{t_{i-1}} + \theta (y_{t_i} - y_{t_{i-1}}))\ d\theta \right\} Y^1_{t_{i-1},t_i} \otimes Y^1_{t_i,t_{i+1}} \\
					&= \left\{ \int_0^1 (\nabla f)(x_{t_{i-1}} + \theta (x_{t_i} - x_{t_{i-1}}))\ d\theta \right\} X^1_{t_{i-1},t_i} \otimes \left( X^1_{t_i,t_{i+1}} -  Y^1_{t_i,t_{i+1}}\right) \\
						&\qquad + \left\{ \int_0^1 (\nabla f)(x_{t_{i-1}} + \theta (x_{t_i} - x_{t_{i-1}}))\ d\theta \right\} \left( X^1_{t_{i-1},t_i} - Y^1_{t_{i-1},t_i} \right) \otimes Y^1_{t_i,t_{i+1}} \\
						&\qquad + \left\{ \int_0^1 \int_0^\theta (\nabla^2 f)\left( x_{t_{i-1}} + r(x_{t_i} - x_{t_{i-1}})\right)\ dr\ d\theta \right\} X^1_{t_{i-1},t_i} \otimes Y^1_{t_{i-1},t_i} \otimes Y^1_{t_i,t_{i+1}} \\
						&\qquad + (\nabla f)(x_{t_{i-1}})Y^1_{t_{i-1},t_i} \otimes Y^1_{t_i,t_{i+1}}
							- (\nabla f)(y_{t_{i-1}})Y^1_{t_{i-1},t_i} \otimes Y^1_{t_i,t_{i+1}} \\
						&\qquad - \left\{ \int_0^1 \int_0^\theta (\nabla^2 f)\left( y_{t_{i-1}} + r(y_{t_i} - y_{t_{i-1}})\right)\ dr\ d\theta \right\} Y^1_{t_{i-1},t_i} \otimes Y^1_{t_{i-1},t_i} \otimes Y^1_{t_i,t_{i+1}} \\
					&= 
				\end{align}
				が成り立つ一方で
				\begin{align}
					\left| \tilde{I}_{s,t}(x) - \tilde{I}_{s,t}(y) \right|
					&= \left| f(x_s)X^1_{s,t} - f(y_s)Y^1_{s,t} \right| \\
					&\leq \left| f(x_s)X^1_{s,t} - f(x_s)Y^1_{s,t} \right|
						+ \left| f(x_s)Y^1_{s,t} - f(y_s)Y^1_{s,t} \right| \\
					&\leq M\left| X^1_{s,t} - Y^1_{s,t} \right|
						+ \left| \left\{ \int_0^1 (\nabla f)(y_s + \theta(x_s - y_s))\ d\theta \right\}
						(x_s - y_s) \otimes Y^1_{s,t} \right| \\
					&\leq M\left| X^1_{s,t} - Y^1_{s,t} \right|
						+ M' d \left| X^1_{0,s} - Y^1_{0,s}\right| \left| Y^1_{s,t} \right|
				\end{align}
				も成立する.
		\end{description}
	\end{prf}