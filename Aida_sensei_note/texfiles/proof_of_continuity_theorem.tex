\section{連続性定理}
	\begin{screen}
		\begin{dfn}[記号の定義]
			$x \in C^1,\ f \in C^2(\R^d,L(\R^d \rightarrow \R^m))$に対し次を定める.
			\begin{align}
				\Delta_T &\coloneqq \Set{(s,t)}{0 \leq s \leq t \leq T}, \\
				X^1 &:\Delta_T \longrightarrow \R^d\ \left( (s,t) \longmapsto X^1_{s,t} = x_t - x_s \right), \\
				X^2 &:\Delta_T \longrightarrow \R^d \otimes \R^d\ \left( (s,t) \longmapsto X^2_{s,t} = \int_s^t (x_u - x_s) \otimes dx_u \right), \\
				\tilde{I}_{s,t}(x) &\coloneqq f(x_s)X^1_{s,t} = f(x_s)(x_t - x_s), \\
				J_{s,t}(x) &\coloneqq f(x_s)X^1_{s,t} + (\nabla f)(x_s)X^2_{s,t}.
			\end{align}
		\end{dfn}
	\end{screen}
	
	以降,$a,b,c,d \in \R^d$に対して次の表現を使う:
	\begin{align}
		[a \otimes b]^i_j &= a^i b^j, \\
		\left[ (\nabla f)(x_s)X^2_{s,t} \right]^i &= \sum_{j,k=1}^d \partial_k f^i_j(x_s) \int_s^t \left(x^k_u - x^k_s \right)\ dx^j_u,\\
		\left[ (\nabla f)(x_s)(a \otimes b) \right]^i &= \sum_{j,k=1}^d \partial_k f^i_j(x_s) a^k b^j,\\
		\left[ (\nabla^2 f)(x_s)(a \otimes b \otimes c) \right]^i &= \sum_{j,k,v=1}^d \partial_v \partial_k f^i_j(x_s) a^v b^k c^j,\\
		\left[ (\nabla^3 f)(x_s)(a \otimes b \otimes c \otimes d) \right]^i &= \sum_{j,k,v,w=1}^d \partial_w \partial_v \partial_k f^i_j(x_s) a^w b^v c^k d^j.
	\end{align}
	
	\begin{screen}
		\begin{thm}\label{thm:Riemann_Stieltjes_approximation}
			$[s,t] \subset [0,T],\ x \in C^1,\ f \in C^2(\R^d,L(\R^d \rightarrow \R^m))$とする.$D \in \delta[s,t]$に対し
			\begin{align}
				\tilde{I}_{s,t}(x,D) \coloneqq \sum_D \tilde{I}_{t_{i-1},t_i}(x),
				\quad J_{s,t}(x,D) \coloneqq \sum_D J_{t_{i-1},t_i}(x)
			\end{align}
			を定めるとき,次が成立する:
			\begin{align}
				I_{s,t}(x) = \lim_{|D| \to 0} \tilde{I}_{s,t}(x,D)
				= \lim_{|D| \to 0} J_{s,t}(x,D).
			\end{align}
		\end{thm}
	\end{screen}
	
	\begin{prf}
		第一の等号は$I_{s,t}(x)$の定義によるから,第二の等号を証明する.まず,
		\begin{align}
			I_{s,t}(x)
			&= \int_s^t f(x_u)\ dx_u \\
			&= \int_s^t f(x_s) + f(x_u) - f(x_s)\ dx_u \\
			&= \int_s^t f(x_s)\ dx_u
				+ \int_s^t \int_0^1 (\nabla f)(x_s + \theta(x_u - x_s)) \left( X^1_{s,u} \otimes \dot{x}_u \right)\ d\theta\ du \\
			&= f(x_s)X^1_{s,t} + (\nabla f)(x_s) X^2_{s,t} \\
				&\quad+ \int_s^t \int_0^1 \left\{ (\nabla f)(x_s + \theta(x_u - x_s)) - (\nabla f)(x_s) \right\}\left( X^1_{s,u} \otimes \dot{x}_u \right)\ d\theta\ du \\
			&= J_{s,t}(x)
				+ \int_s^t 
				\int_0^1 \int_0^\theta (\nabla^2 f)(x_s + r(x_u - x_s))\left( X^1_{s,u} \otimes X^1_{s,u} \otimes \dot{x}_u \right)\ dr\ d\theta\ du
		\end{align}
		が成り立つ.$[0,T] \ni t \longmapsto x_t$の連続性より,
		最下段式中の$x_s + r(x_u - x_s)\ (0 \leq r \leq 1,\ s \leq u \leq t)$は或るコンパクト集合$K$に含まれ,
		$f$が$C^2$-級関数であるから
		\begin{align}
			M \coloneqq \sum_{i,j,k,v} \sup{x \in K}{\left|\partial_v \partial_k f_j^i(x) \right|}
		\end{align}
		として$M < \infty$を定めれば
		\begin{align}
			&\left| \int_s^t 
				\int_0^1 \int_0^\theta (\nabla^2 f)(x_s + r(x_u - x_s))\left( X^1_{s,u} \otimes X^1_{s,u} \otimes \dot{x}_u \right)\ dr\ d\theta\ du \right| \\
			&\qquad \leq \int_s^t 
				\int_0^1 \int_0^\theta \left| (\nabla^2 f)(x_s + r(x_u - x_s))\left( X^1_{s,u} \otimes X^1_{s,u} \otimes \dot{x}_u \right) \right|\ dr\ d\theta\ du \\
			&\qquad \leq M \int_s^t |X^1_{s,u}|^2 |\dot{x}_u|\ du \\
			&\qquad \leq M \Norm{x}{C^1}^3 \int_s^t (u - s)^2\ du
		\end{align}
		が出る.特に$D \in \delta[s,t]$に対して
		\begin{align}
			&\sum_D \int_{t_{i-1}}^{t_i} (u - t_{i-1})^2\ du
			\leq \sum_D |D| \int_{t_{i-1}}^{t_i} (u - t_{i-1})\ du \\
			&\qquad \leq \sum_D |D| \int_{t_{i-1}}^{t_i} (u - s)\ du
			\leq \frac{1}{2}(t-s)^2 |D|
			\longrightarrow 0 \quad (|D| \longrightarrow 0)
		\end{align}
		が成立するから,
		\begin{align}
			\left| I_{s,t}(x) - J_{s,t}(x,D) \right|
			\leq \sum_D \left| I_{t_{i-1},t_i}(x) - J_{t_{i-1},t_i}(x) \right| \longrightarrow 0 \quad (|D| \longrightarrow 0)
		\end{align}
		が従い定理の主張を得る.
		\QED
	\end{prf}
	
	\begin{screen}
		\begin{dfn}[コントロール関数]
			関数$\omega:\Delta_T \longrightarrow [0,\infty)$
			が連続かつ任意の$s \leq u \leq t$に対して
			\begin{align}
				\omega(s,u) + \omega(u,t) \leq \omega(s,t)
				\label{eq:control_function_subadditivity}
			\end{align}
			を満たすとき,$\omega$をコントロール関数(control function)と呼ぶ.
		\end{dfn}
	\end{screen}
	
	式(\refeq{eq:control_function_subadditivity})から$\omega(t,t)=0\ (\forall t \in [0,T])$が従う.
	つまりコントロール関数は対角線上で0になる.
	
	\begin{screen}
		\begin{dfn}[ノルム空間値写像の$p$-variation]
			$(V,\Norm{\cdot}{})$をノルム空間,$p > 0$とする.
			このとき連続写像$\psi:\Delta_T \longrightarrow V$に対する$p$-variationを
			\begin{align}
				\Norm{\psi}{p,[s,t]}
				\coloneqq \left\{ \sup{D \in \delta[s,t]}{ \sum_D \Norm{\psi_{t_{i-1},t_i}}{}^p} \right\}^{1/p},
				\quad ((s,t) \subset [0,T])
			\end{align}
			で定める.特に$\Norm{\cdot}{p,[0,T]}$を$\Norm{\cdot}{p}$と書く.
		\end{dfn}
	\end{screen}
	
	再定義した$p$-variationに対しても
	定理\ref{thm:p_variation_0_1},\ref{thm:p_variation_monotonously_decreasing},
	\ref{thm:B_p_T_Banach_space}が成立する.
	\begin{screen}
		\begin{thm}[定理\ref{thm:p_variation_0_1}のアナロジー]
			$V$をノルム空間,$X:\Delta_T \longrightarrow V$を連続写像とする.
			このとき,$X_{t,t} = 0\ (\forall t \in [0,T])$かつ
			$p \in (0, 1)$に対し$\Norm{X}{p} < \infty$が満たされれば$X \equiv 0$である.
		\end{thm}
	\end{screen}

	\begin{prf}
		$(s,t) \in \Delta_T$を任意に取り固定する.このとき全ての$D \in \delta[s,t]$に対して,
		\begin{align}
			\Norm{X_{s,t}}{} \leq \sum_D \Norm{X_{t_{i-1},t_i}}{}
			&\leq \max{D}{\Norm{X_{t_{i-1},t_i}}{}^{1-p}} 
				\sum_D \Norm{X_{t_{i-1},t_i}}{}^p \\
			&\leq \max{D}{\Norm{X_{t_{i-1},t_i}}{}^{1-p}} \Norm{X}{p}
		\end{align}
		が成り立ち,$X$の一様連続性から右辺は$|D| \longrightarrow 0$で$0$に収束し,
		$X_{s,t} = 0$が従う.
		\QED
	\end{prf}
	
	\begin{screen}
		\begin{thm}[定理\ref{thm:p_variation_monotonously_decreasing}のアナロジー]
		\label{thm:p_variation_monotonously_decreasing_2}
			$V$をノルム空間とするとき,$x:[0,T] \longrightarrow V$に対して
			$1 \leq p \leq q$なら$\Norm{X}{p} \geq \Norm{X}{q}$が成立する.
		\end{thm}
	\end{screen}
	
	ノルム空間$(V,\Norm{\cdot}{})$に対し
	\begin{align}
		\tilde{B}_{p,T}(V)
		\coloneqq \Set{X:\Delta_T \longrightarrow V}{\mbox{continuous},\ \Norm{X}{p} < \infty}
	\end{align}
	により線型空間を定めれば,定理\ref{thm:B_p_T_Banach_space}のアナロジーを得る.
	
	\begin{screen}
		\begin{thm}\label{thm:B_p_T_Banach_space_2}
			$V$がBanach空間のとき,$\tilde{B}_{p,T}(V)$は$\Norm{\cdot}{p}$をノルムとするBanach空間である.
		\end{thm}
	\end{screen}

\begin{prf}完備性を示す.
	\begin{description}
		\item[第一段] $(X^n)_{n=1}^{\infty} \subset \tilde{B}_{p,T}(V)$をCauchy列とすれば,
			任意の$\epsilon > 0$に対して或る$n_\epsilon \in \N$が存在し
			\begin{align}
				\Norm{X^n - X^m}{p}
				= \left\{ \sup{D \in \delta[0,T]}{\sum_{D} 
				\Norm{X^n_{t_{i-1},t_i} - X^m_{t_{i-1},t_i}}{}^p }\right\}^{1/p} < \epsilon,
				\quad (n,m > n_\epsilon)
			\end{align}
			を満たす.任意の$(s,t) \in \Delta_T$に対して分割$D = \{0 \leq s \leq t \leq T\}$
			を取れば
			\begin{align}
				\Norm{X^n_{s,t} - X^m_{s,t}}{} < \epsilon,
				\quad (n,m > n_\epsilon)
			\end{align}
			が成り立ち,$V$の完備性より或る$X_{s,t} \in \R^d$が存在して
			\begin{align}
				\Norm{X^n_{s,t} - X_{s,t}}{} < \epsilon
				\quad (n > n_\epsilon)
			\end{align}
			となる.この収束は$(s,t)$に関して一様であるから$X$は連続である.
			
		\item[第二段] $\Norm{X^n - X}{p} \longrightarrow 0\ (n \longrightarrow \infty)$を示す.
			前段より,任意の$D \in \delta[0,T]$に対し
			\begin{align}
				\sum_D \Norm{X^n_{t_{i-1},t_i} - X^m_{t_{i-1},t_i}}{}^p
				< \epsilon^p,
				\quad (n,m > n_\epsilon)
			\end{align}
			が満たされる.$D$は有限分割であるから,$m \longrightarrow \infty$として
			\begin{align}
				\sum_D \Norm{X^n_{t_{i-1},t_i} - X_{t_{i-1},t_i}}{}^p
				< \epsilon^p,
				\quad (n > n_\epsilon)
			\end{align}
			が従い,$D$の任意性より$\Norm{X^n - X}{p} < \epsilon\ (n > n_\epsilon)$を得る.
			\QED
	\end{description}
\end{prf}

	\begin{screen}
		\begin{thm}[$p$-variationが定めるcontrol function]
		\label{thm:control_function_defined_by_p_variation}
			$(V,\Norm{\cdot}{})$をノルム空間,$p > 0$とする.
			$\Norm{\psi}{p} < \infty$かつ$\psi_{t,t} = 0\ (\forall t \in [0,T])$を満たす連続写像$\psi:\Delta_T \longrightarrow V$に対して,
			\begin{align}
				\omega:\Delta_T \ni (s,t) \longmapsto \Norm{\psi}{p,[s,t]}^p
			\end{align}
			により定める$\omega$はcontrol functionである.
		\end{thm}
	\end{screen}
	
	\begin{prf}[参考:\cite{key3}(pp. 96-98)]
		$\Norm{\psi}{p} < \infty$の仮定より$\omega$は$[0,\infty)$値であるから,
		以下では式(\refeq{eq:control_function_subadditivity})と連続性を示す.
		\begin{description}
			\item[第一段]
				$\omega$が式(\refeq{eq:control_function_subadditivity})を満たすことを示す.実際,
				任意に$D_1 \in \delta[s,u],D_2 \in \delta[u,t]$を取れば
				\begin{align}
					\sum_{D_1}\Norm{\psi_{t_{i-1},t_i}}{}^p
					+ \sum_{D_2}\Norm{\psi_{t_{i-1},t_i}}{}^p
					= \sum_{D_1 \cup D_2}\Norm{\psi_{t_{i-1},t_i}}{}^p
					\leq \Norm{\psi}{p:[s,t]}^p
				\end{align}
				が成り立つ.左辺の$D_1,D_2$の取り方は独立であるから,それぞれに対し上限を取れば
				\begin{align}
					\Norm{\psi}{p:[s,u]}^p + \Norm{\psi}{p:[u,t]}^p
					\leq \Norm{\psi}{p:[s,t]}^p
				\end{align}
				が従う.
			\item[第二段]
				任意の$[s,t] \subset [0,T]$について
				\footnote{
					下段の二式については$s < t$と仮定する.また
					上段についても,$t=T$或は$s=0$の場合を除く必要がある.
				},
				\begin{align}
					\lim_{h \to +0} \omega(s,t+h) &= \inf{h>0}{\omega(s,t+h)},
					&\lim_{h \to +0} \omega(s-h,t) &= \inf{h>0}{\omega(s-h,t)}, \\
					\lim_{h \to +0} \omega(s,t-h) &= \sup{h>0}{\omega(s,t-h)},
					&\lim_{h \to +0} \omega(s+h,t) &= \sup{h>0}{\omega(s+h,t)}
				\end{align}
				が成立する.実際$\omega(s,t+h)$について見れば,これは下に有界かつ
				$h \to +0$に対し単調減少であるから極限が確定し下限に一致する.
				残りの三つも同様の理由で成立する.
				
			\item[第三段]
				任意の$s \in [0,T)$に対し,$(s,T] \ni t \longmapsto \omega(s,t)$
				の左連続性を示す.ここでは
				\begin{align}
					\tilde{\omega}(s,t) \coloneqq 
					\begin{cases}
						\lim_{h \to +0}\omega(s,t-h), & (s < t), \\
						0, & (s=t),
					\end{cases}
					\quad (\forall (s,t) \in \Delta_T)
				\end{align}
				で定める$\tilde{\omega}$が優加法性を持ち,かつ
				\begin{align}
					\Norm{\psi_{s,t}}{}^p \leq \tilde{\omega}(s,t),
					\quad (\forall (s,t) \in \Delta_T)
				\end{align}
				を満たすことを示す.
				実際これが示されれば,任意の$D \in \delta[s,t]$に対し
				\begin{align}
					\sum_D \Norm{\psi_{t_{i-1},t_i}}{}^p
					\leq \sum_D \tilde{\omega}(t_{i-1},t_i)
					\leq \tilde{\omega}(s,t)
				\end{align}
				が成立し$\omega(s,t) \leq \tilde{\omega}(s,t)$が従い,
				$\omega(s,t) \geq \omega(s,t-h)\ (\forall h > 0)$と併せて
				\begin{align}
					\omega(s,t) = \tilde{\omega}(s,t) = \lim_{h \to +0} \omega(s,t-h)
				\end{align}
				を得る.いま,任意に$s < u < t$を取れば,十分小さい$h_1,h_2 > 0$に対して
				\begin{align}	
					\omega(s,u-h_1) + \omega(u,t-h_2) \leq \omega(s,t-h_2)
				\end{align}
				が満たされ,$h_1 \longrightarrow +0,\ h_2 \longrightarrow +0$として
				\begin{align}
					\tilde{\omega}(s,u) + \tilde{\omega}(u,t) \leq \tilde{\omega}(s,t)
				\end{align}
				が成り立つから$\tilde{\omega}$は優加法性を持つ.また,もし
				或る$(u,v) \in \Delta_T$に対して
				\begin{align}
					\Norm{\psi_{u,v}}{}^p > \tilde{\omega}(u,v)
				\end{align}
				が成り立つと仮定すると($u = v$なら両辺0になるから$u < v$である)
				\begin{align}
					\Norm{\psi_{u,v}}{}^p > \tilde{\omega}(u,v) \geq \omega(u,v-h) \geq \Norm{\psi_{u,v-h}}{}^p,
					\quad (\forall h > 0)
				\end{align}
				となる.一方$\psi$の連続性より
				$\Norm{\psi_{u,v-h}}{}^p \longrightarrow \Norm{\psi_{u,v}}{}^p\ (h \longrightarrow +0)$が従い矛盾が生じる.
				同様にして,任意の$t \in (0,T]$に対し$[0,t) \ni s \longmapsto \omega(s,t)$
				の右連続性も出る.
				
			\item[第四段]
				任意の$t \in [0,T)$に対して次を示す:
				\begin{align}
					\lim_{h \to +0} \omega(t,t+h) = \inf{h>0}{\omega(t,t+h)} = 0.
				\end{align}
				第一の等号は前段より従うから,第二の等号を背理法により証明する.いま
				\begin{align}
					\inf{h>0}{\omega(t,t+h)} \eqqcolon \delta > 0
					\label{eq:thm_continuity_of_norm_val_p_variation_2}
				\end{align}
				と仮定すれば,$\psi$の連続性より或る$h_1$が存在して
				\begin{align}
					\Norm{\psi_{t,t+h}}{}^p
					 = \Norm{\psi_{t,t+h} - \psi_{t,t}}{}^p
					 < \frac{\delta}{8},
					\quad (\forall h < h_1)
					\label{eq:thm_continuity_of_norm_val_p_variation_1}
				\end{align}
				が成立する.ここで任意に$h_0 < h_1$を取り固定する.
				一方で$\omega(t,t+h_0) \geq \delta$より
				\begin{align}
					\sum_{i=1}^{N} \Norm{\psi_{\tau_{i-1},\tau_i}}{}^p > \frac{7\delta}{8}
				\end{align}
				を満たす$D = \{t = \tau_0 < \tau_1 < \cdots, \tau_N = t+h_0\} \in \delta[t,t+h_0]$が存在し,
				(\refeq{eq:thm_continuity_of_norm_val_p_variation_1})と併せて
				\begin{align}
					\sum_{i=2}^{N} \Norm{\psi_{\tau_{i-1},\tau_i}}{}^p
					> \frac{7\delta}{8} - \Norm{\psi_{t,\tau_1}}{}^p
					>\frac{7\delta}{8} - \frac{\delta}{8}
					= \frac{3 \delta}{4}
				\end{align}
				を得る.また,$\omega(t,\tau_1) \geq \delta$より或る$D' \in \delta[t,\tau_1]$が存在して
				\begin{align}
					\sum_{D'} \Norm{\psi_{t_{i-1},t_i}}{}^p > \frac{3 \delta}{4}
				\end{align}
				を満たすから,$D' \cup \{\tau_1 < \cdots, \tau_N = t+h_0\} \in \delta[t,t+h_0]$に対して
				\begin{align}
					\omega(t,t+h_0) > \sum_{D'} \Norm{\psi_{t_{i-1},t_i}}{}^p + \sum_{i=2}^{N} \Norm{\psi_{\tau_{i-1},\tau_i}}{}^p
					> \frac{3\delta}{2}
				\end{align}
				が従うが,$h_0 < h_1$の任意性と単調減少性により
				\begin{align}
					\delta = \inf{h>0}{\omega(t,t+h)} = \inf{h_1>h>0}{\omega(t,t+h)} \geq \frac{3\delta}{2}
				\end{align}
				となり矛盾が生じる.
				同様にして
				\begin{align}
					\lim_{h \to +0} \omega(t-h,t) = 0,
					\quad (\forall t \in (0,T])
				\end{align}
				も成立する.
				
			\item[第五段]
				任意に$s \in [0,T)$を取り固定し,
				$[s,T) \ni t \longmapsto \omega(s,t)$が右連続であることを示す.
				\begin{align}
					\lim_{h \to +0} \omega(s,t+h) \leq \omega(s,t)
					\label{eq:thm_continuity_of_norm_val_p_variation_3}
				\end{align}
				を示せば,第二段より逆向きの不等号も従い右連続性を得る.
				任意に$h,\epsilon > 0$を取れば,
				\begin{align}
					\omega(s,t + h) - \epsilon
					\leq \sum_D \Norm{\psi_{t_{i-1},t_i}}{}^p
				\end{align}
				を満たす$D \in \delta[s,t+h]$が存在し,
				\begin{align}
					D_1 \coloneqq \{t_0 < \cdots < t_k\} = [s,t] \cap D,
					\quad 
					D_2 \coloneqq D \backslash \left( D_1 \cup \{ t_{k+1} \} \right)
				\end{align}
				とおくと
				\begin{align}
					\omega(s,t + h) - \epsilon
					&\leq \sum_{i=1}^{k} \Norm{\psi_{t_{i-1},t_i}}{}^p 
						+ \Norm{\psi_{t_k,t_{k+1}}}{}^p + \sum_{D_2} \Norm{\psi_{t_{i-1},t_i}}{}^p \\
					&= \sum_{i=1}^{k} \Norm{\psi_{t_{i-1},t_i}}{}^p 
						+ \Norm{\psi_{t_k,t}}{}^p + \Norm{\psi_{t_k,t_{k+1}}}{}^p - \Norm{\psi_{t_k,t}}{}^p
						+ \sum_{D_2} \Norm{\psi_{t_{i-1},t_i}}{}^p \\
					&\leq \omega(s,t) + \omega(t,t+h) + \Norm{\psi_{t_k,t_{k+1}}}{}^p - \Norm{\psi_{t_k,t}}{}^p
				\end{align}
				が成り立つ.$\psi$の一様連続性より$\Norm{\psi_{t_k,t_{k+1}}}{}^p \longrightarrow \Norm{\psi_{t_k,t}}{}^p\ (h \longrightarrow +0)$が成り立つから
				\begin{align}
					\lim_{h \to +0} \omega(s,t+h) - \epsilon \leq \omega(s,t),
					\quad (\forall \epsilon > 0)
				\end{align}
				が従い(\refeq{eq:thm_continuity_of_norm_val_p_variation_3})が出る.
				同様に$(0,t] \ni s \longmapsto \omega(s,t)\ (\forall t \in (0,T])$
				の左連続性も成立する.
				
			\item[第六段]
				$\omega$の$(s,t) \in \Delta_T$における連続性を示す.$h,k \geq 0$とする.
				\begin{description}
					\item[(A)] $(s,t)$を基準に第一象限の点について
						\begin{align}
							&|\omega(s,t) - \omega(s+h,t+k)| \\
							&\qquad\leq |\omega(s,t) - \omega(s+h,t)| + |\omega(s+h,t) - \omega(s+h,t+k)| \\
							&\qquad= |\omega(s,t) - \omega(s+h,t)| + \omega(s+h,t+k) - \omega(s+h,t) \\
							&\qquad\leq |\omega(s,t) - \omega(s+h,t)| + \omega(s,t+k) - \omega(s+h,t) \\
							&\qquad\leq |\omega(s,t) - \omega(s+h,t)| + |\omega(s,t+k) - \omega(s,t)| + |\omega(s,t) - \omega(s+h,t)|
						\end{align}
						が成り立つ.前段までに示した左右の連続性より,
						近づけ方に依らず$h,k \longrightarrow +0$とすれば,
						左辺をいくらでも0に近づけることができる.
					
					\item[(B)] $(s,t)$を基準に第三象限の点について
						\begin{align}
							&|\omega(s,t) - \omega(s-h,t-k)| \\
							&\qquad\leq |\omega(s,t) - \omega(s-h,t)| + |\omega(s-h,t) - \omega(s-h,t-k)| \\
							&\qquad= |\omega(s,t) - \omega(s-h,t)| + \omega(s-h,t) - \omega(s-h,t-k) \\
							&\qquad\leq |\omega(s,t) - \omega(s-h,t)| + \omega(s-h,t) - \omega(s,t-k) \\
							&\qquad\leq |\omega(s,t) - \omega(s-h,t)| + |\omega(s-h,t) - \omega(s,t)| + |\omega(s,t) - \omega(s,t-k)|,
						\end{align}
						が成り立つ.(A)と同じく$h,k \longrightarrow +0$として左辺は0に収束する.
						
					\item[(C)] $((h_n,k_n))_{n=1}^{\infty}$を
						第一象限から$(0,0)$に近づく任意の点列とするとき,
						\begin{align}
							\lim_{n \to \infty} \omega(s-h_n, t+k_n) = \omega(s,t),
							\quad \lim_{n \to \infty} \omega(s+h_n, t-k_n) = \omega(s,t)
						\end{align}
						が成り立つことを示す.これが示されれば
						\begin{align}
							\lim_{h,k \to +0} \omega(s-h, t+k) = \omega(s,t),
							\quad \lim_{h,k \to +0} \omega(s+h, t-k) = \omega(s,t)
						\end{align}
						が従い,(A)(B)と併せて$\omega$の連続性が出る.背理法で証明する.いま,
						\begin{align}
							\alpha \coloneqq \lim_{n \to \infty} \omega(s-h_n, t+k_n)
							> \omega(s,t)
						\end{align}
						と仮定して$\epsilon \coloneqq \alpha - \omega(s,t)$とおく.
						$\lim_{t' \downarrow t}\omega(s,t') = \omega(s,t)$より
						\begin{align}
							0 \leq \omega(s,t') - \omega(s,t) < \frac{\epsilon}{3}
						\end{align}
						を満たす$t' > t$が存在し,また
						$\lim_{s' \uparrow s}\omega(s',t') = \omega(s,t')$より
						\begin{align}
							0 \leq \omega(s',t') - \omega(s,t') < \frac{\epsilon}{3}
						\end{align}
						を満たす$s' < s$も存在する.このとき或る$n$で
						$s' \leq s- h_n,\ t + k_n \leq t'$かつ
						\begin{align}
							|\omega(s-h_n, t+k_n) - \alpha| < \frac{\epsilon}{3}
						\end{align}
						が成立し,特に$(s-h_n, t+k_n) \subset (s',t')$より
						\begin{align}
							\omega(s-h_n, t+k_n) \leq \omega(s',t')
						\end{align}
						となるはずであるが,一方で
						\begin{align}
							\omega(s',t') < \frac{2}{3}\epsilon + \omega(s,t)
							 = \alpha - \frac{\epsilon}{3}
							 < \omega(s-h_n, t+k_n)
						\end{align}
						が従い矛盾が生じる.よって
						\begin{align}
							\lim_{n \to \infty} \omega(s-h_n, t+k_n) = \omega(s,t)
						\end{align}
						でなくてはならず,同様にして$\lim_{n \to \infty} \omega(s+h_n, t-k_n) = \omega(s,t)$
						も得られる.
						\QED
				\end{description}
		\end{description}
	\end{prf}
	
	\begin{screen}
		\begin{thm}[control functionの例]\label{thm:examples_of_control_functions}
			以下の関数$\omega:\Delta_T \longrightarrow [0,\infty)$はcontrol functionである.
			\begin{description}
				\item[(1)] $\omega:(s,t) \longmapsto \Norm{X^1}{p:[s,t]}^p,
					\quad (p \geq 1,\ x \in B_{p,T}(\R^d)).$
				\item[(2)] $\omega:(s,t) \longmapsto \Norm{X^2}{p:[s,t]}^p,
					\quad (p \geq 1,\ x \in C^1).$
			\end{description}
		\end{thm}
	\end{screen}
	
	行列$a = (a_j^i)$のノルムは$|a| = \sqrt{\sum_{i,j}|a_j^i|^2}$として考える.
	
	\begin{thm}\mbox{}
		\begin{description}
			\item[(1)] $\omega:(s,t) \longmapsto X^1_{s,t} = x_t - x_s$は連続であるから,
				前定理より$\omega$はcontrol functionである.
				
			\item[(2)] 任意の$[s,t] \subset [0,T]$に対して
				$\Norm{X^2}{p:[s,t]}^p < \infty$を示せば,あとは
				上と同じ理由により定理の主張が得られる.
				実際,任意の分割$D = \{s=t_0 < \cdots < t_N = t\}$に対し
				\begin{align}
					\Norm{X^2_{t_{i-1},t_i}}{}
					&\leq \left| \int_{t_{i-1}}^{t_i} (x_u - x_{t_{i-1}}) \otimes \dot{x}_u\ du \right| \\
					&\leq \int_{t_{i-1}}^{t_i} \left| (x_u - x_{t_{i-1}}) \otimes \dot{x}_u \right|\ du \\
					&\leq \Norm{x}{C^1}^2 \left\{ \int_{t_{i-1}}^{t_i} (u-s)\ du \right\}^{1/p}
						\left\{ \int_{t_{i-1}}^{t_i} (u-s)\ du \right\}^{1-1/p} \\
					&\leq \Norm{x}{C^1}^2 \left\{ \int_{t_{i-1}}^{t_i} (u-s)\ du \right\}^{1/p}
						\left\{ \int_s^t (u-s)\ du \right\}^{1-1/p}
				\end{align}
				が成り立つから,
				\begin{align}
					&\sum_D \Norm{X^2_{t_{i-1},t_i}}{}^p
					\leq \sum_D \Norm{x}{C^1}^{2p} \left\{ \frac{1}{2}(t-s)^2 \right\}^{p-1}
						\int_{t_{i-1}}^{t_i} (u-s)\ du \\
					&\qquad= \Norm{x}{C^1}^{2p} \left\{ \frac{1}{2}(t-s)^2 \right\}^{p-1}
						\int_s^t (u-s)\ du
					= \Norm{x}{C^1}^{2p} \left\{ \frac{1}{2}(t-s)^2 \right\}^p
				\end{align}
				により$\Norm{X^2}{p:[s,t]}^p < \infty$が従う.
				\QED
		\end{description}
	\end{thm}
	
	\begin{comment}
		\begin{align}
			&\omega_1(s,u)\omega_2(s,u) + \omega_1(u,t)\omega_2(u,t) \\
			&= \left\{ \omega_1(s,u) + \omega_1(u,t) \right\}\omega_2(s,u) + \omega_1(u,t)\left\{\omega_2(u,t) - \omega_2(s,u)\right\} \\
			&\leq \omega_1(s,t)\omega_2(s,u) + \omega_1(u,t)\left\{\omega_2(u,t) - \omega_2(s,u)\right\} \\
			&\leq \omega_1(s,t)\omega_2(s,u) + \omega_1(u,t)\omega_2(u,t) \\
			&\leq \omega_1(s,t)\left\{ \omega_2(s,u) + \omega_2(u,t)\right\} \\
			&\leq \omega_1(s,t)\omega_2(s,t)
		\end{align}
	\end{comment}
	
	\begin{screen}
		\begin{lem}\label{lem:control_function_min}
			$\omega$を$\Delta_T$上のcontrol functionとする.
			$D = \{s = t_0 < t_1 < \cdots < t_N= t\}$について,$N \geq 2$の場合
			或る$1 \leq i \leq N-1$が存在して次を満たす:
			\begin{align}
				\omega(t_{i-1},t_{i+1})
				\leq \frac{2 \omega(s,t)}{N-1}.
				\label{eq:lem_control_function_min}
			\end{align}
		\end{lem}
	\end{screen}
	
	\begin{prf}
		(会田先生のテキスト.)
		\QED
	\end{prf}
	
	\begin{screen}
		\begin{thm}[$1 \leq p < 2$の場合の連続性定理]\label{thm:continuity_theorem_1}
			$1 \leq p < 2$とし,
			$x_0 = y_0$を満たす$x,y \in C^1$と$f \in C^2_b(\R^d,L(\R^d \rightarrow \R^m)),\ 0 < \epsilon, R < \infty$を任意に取る.
			このとき,
			\begin{align}
				\Norm{X^1}{p},\Norm{Y^1}{p} \leq R,
				\quad \Norm{X^1 - Y^1}{p} \leq \epsilon
			\end{align}
			なら,或る定数$C = C(p,R,f)$が存在し,任意の$0 \leq s \leq t \leq T$に対して次が成立する:
			\begin{align}
				\left| I_{s,t}(x) - I_{s,t}(y) \right| \leq \epsilon C.
			\end{align}
		\end{thm}
	\end{screen}
	
	\begin{screen}
		\begin{cor}[$p$-variationによる閉球上のLipschitz連続性]\label{cor:continuity_theorem_1}
			$1 \leq p < 2$とし,
			$x_0 = y_0$を満たす$x,y \in C^1$と$f \in C^2_b(\R^d,L(\R^d \rightarrow \R^m)),\ 0 < R < \infty$を任意に取る.
			このとき,
			\begin{align}
				\Norm{X^1}{p},\Norm{Y^1}{p} \leq R
			\end{align}
			なら,或る定数$C = C(p,R,f)$が存在して次を満たす:
			\begin{align}
				\left| I_{0,T}(x) - I_{0,T}(y) \right| \leq C\Norm{X^1 - Y^1}{p}.
			\end{align}
		\end{cor}
	\end{screen}
	
	\begin{prf}[系\ref{cor:continuity_theorem_1}]
		定理\ref{thm:continuity_theorem_1}において,
		$\epsilon = \Norm{X^1 - Y^1}{p}\ (x \neq y)$
		\footnote{
			$x=y$なら$\Norm{X^1 - Y^1}{p} = 0$かつ$I_{s,t}(x) = I_{s,t}(y)$が成り立つ.
		}
		として証明が通る.
		\QED
	\end{prf}
	
	\begin{prf}[定理\ref{thm:continuity_theorem_1}]
		$[s,t] \subset [0,T]$とする.
		\begin{description}
			\item[第一段]
				$\omega:\Delta_T \longrightarrow [0,\infty)$を
				\begin{align}
					\omega(\alpha,\beta) = \Norm{X^1}{p,[\alpha,\beta]}^p + \Norm{Y^1}{p,[\alpha,\beta]}^p + \epsilon^{-p} \Norm{X^1 - Y^1}{p,[\alpha,\beta]}^p,
					\quad ((\alpha,\beta) \in \Delta_T)
				\end{align}
				で定めれば,定理\ref{thm:examples_of_control_functions}により$1 \leq p$の下で$\omega$はcontrol functionである.
				
			\item[第二段]
				任意に$[s,t]$の分割
				$D = \{s=t_0 < \cdots < t_N=t\}\ (N \geq 2)$を取れば,
				補題\ref{lem:control_function_min}より(\refeq{eq:lem_control_function_min})を満たす$t_{(0)}$が存在する.
				ここで,$D_{-0} \coloneqq D,\ D_{-1} \coloneqq D \backslash \{t_{(0)}\}$と定める.
				$N \geq 3$ならば$D_{-1}$についても(\refeq{eq:lem_control_function_min})を満たす$t_{(1)}$が存在するから,
				$D_{-2} \coloneqq D_{-1} \backslash \{t_{(1)}\}$と定める.この操作を繰り返せば
				$t_{(k)},D_{-k}\ (k=0,1,\cdots,N-1)$が得られ,
				\begin{align}
					&\tilde{I}_{s,t}(x,D) - \tilde{I}_{s,t}(y,D) \\
					&\qquad = \sum_{k=0}^{N-2} \left[ \left\{ \tilde{I}_{s,t}(x,D_{-k}) - \tilde{I}_{s,t}(x,D_{-k-1}) \right\} - 
						\left\{ \tilde{I}_{s,t}(y,D_{-k}) - \tilde{I}_{s,t}(y,D_{-k-1}) \right\} \right] \label{eq:continuity_theorem_1_1}\\
					&\quad \qquad + \left\{ \tilde{I}_{s,t}(x) - \tilde{I}_{s,t}(y) \right\}	\label{eq:continuity_theorem_1_2}
				\end{align}
				と表現できる.
			
			\item[第三段]
				式(\refeq{eq:continuity_theorem_1_1})について,次を満たす定数$C_1$が存在することを示す:
				\begin{align}
					|(\refeq{eq:continuity_theorem_1_1})| \leq \epsilon C_1
					\label{eq:continuity_theorem_1_3}
				\end{align}
				見やすくするために$t_k = t_{(k)}$と書き直せば,
				\begin{align}
					&\left\{ \tilde{I}_{s,t}(x,D_{-k}) - \tilde{I}_{s,t}(x,D_{-k-1}) \right\} - 
						\left\{ \tilde{I}_{s,t}(y,D_{-k}) - \tilde{I}_{s,t}(y,D_{-k-1}) \right\} \\
					&=	\left\{ f(x_{t_k}) - f(x_{t_{k-1}}) \right\} X^1_{t_k,t_{k+1}}
						- \left\{ f(y_{t_k}) - f(y_{t_{k-1}}) \right\} Y^1_{t_k,t_{k+1}} \\
					&= \left\{ f(x_{t_k}) - f(x_{t_{k-1}}) \right\} \left( X^1_{t_k,t_{k+1}} - Y^1_{t_k,t_{k+1}} \right) \\
						&\quad +\left\{ f(x_{t_k}) - f(x_{t_{k-1}}) \right\} Y^1_{t_k,t_{k+1}} - \left\{ f(y_{t_k}) - f(y_{t_{k-1}}) \right\} Y^1_{t_k,t_{k+1}} \\
					&= \int_0^1 (\nabla f)( x_{t_{k-1}}+\theta ( x_{t_k}-x_{t_{k-1}} ))
						X^1_{t_{k-1},t_k} \otimes \left( X^1_{t_k,t_{k+1}} - Y^1_{t_k,t_{k+1}} \right)\ d\theta \\
						&\quad + \int_0^1 (\nabla f)( x_{t_{k-1}}+\theta ( x_{t_k}-x_{t_{k-1}} ))
						X^1_{t_{i_k-1},t_k} \otimes Y^1_{t_k,t_{i_k+1}}\ d\theta \\
						&\quad - \int_0^1 (\nabla f)( y_{t_{k-1}}+\theta ( y_{t_k}-y_{t_{k-1}} ))
						Y^1_{t_{i_k-1},t_k} \otimes Y^1_{t_k,t_{i_k+1}}\ d\theta \\
					&= \int_0^1 (\nabla f)( x_{t_{k-1}}+\theta ( x_{t_k}-x_{t_{k-1}} ))
						X^1_{t_{i_k-1},t_k} \otimes \left( X^1_{t_k,t_{i_k+1}} - Y^1_{t_k,t_{i_k+1}} \right)\ d\theta \\
						&\quad + \int_0^1 (\nabla f)( x_{t_{k-1}}+\theta ( x_{t_k}-x_{t_{k-1}} ))
						\left( X^1_{t_{k-1},t_k} - Y^1_{t_{k-1},t_k} \right) \otimes Y^1_{t_k,t_{k+1}}\ d\theta \\
						&\quad + \int_0^1 (\nabla f)( x_{t_{k-1}}+\theta ( x_{t_k}-x_{t_{k-1}} ))
						Y^1_{t_{k-1},t_k} \otimes Y^1_{t_k,t_{k+1}}\ d\theta \\
						&\quad - \int_0^1 (\nabla f)( y_{t_{k-1}}+\theta ( y_{t_k}-y_{t_{k-1}} ))
						Y^1_{t_{k-1},t_k} \otimes Y^1_{t_k,t_{k+1}}\ d\theta \\
					&= \int_0^1 (\nabla f)( x_{t_{k-1}}+\theta ( x_{t_k}-x_{t_{k-1}} ))
						X^1_{t_{k-1},t_k} \otimes \left( X^1_{t_k,t_{k+1}} - Y^1_{t_k,t_{k+1}} \right)\ d\theta \\
						&\quad + \int_0^1 (\nabla f)( x_{t_{k-1}}+\theta ( x_{t_k}-x_{t_{k-1}} ))
						\left( X^1_{t_{k-1},t_k} - Y^1_{t_{k-1},t_k} \right) \otimes Y^1_{t_k,t_{k+1}}\ d\theta \\
						&\quad + \int_0^1 \int_0^1 (\nabla^2 f)( y_{t_{k-1}}+\theta ( y_{t_k}-y_{t_{k-1}} + r(x_{t_{k-1}}+\theta ( x_{t_k}-x_{t_{k-1}}) - y_{t_{k-1}}-\theta ( y_{t_k}-y_{t_{k-1}}))) \\
						&\qquad\qquad \left( X^1_{0,t_{k-1}} - Y^1_{0,t_{k-1}} \right) \otimes Y^1_{t_{k-1},t_k} \otimes Y^1_{t_k,t_{k+1}}\ dr\ d\theta\ \footnotemark \\
						&\quad + \int_0^1 \int_0^1 (\nabla^2 f)( y_{t_{k-1}}+\theta ( y_{t_k}-y_{t_{k-1}} + r(x_{t_{k-1}}+\theta ( x_{t_k}-x_{t_{k-1}}) - y_{t_{k-1}}-\theta ( y_{t_k}-y_{t_{k-1}}))) \\
						&\qquad\qquad \theta \left( X^1_{t_{k-1},t_k} - Y^1_{t_{k-1},t_k} \right) \otimes Y^1_{t_{k-1},t_k} \otimes Y^1_{t_k,t_{k+1}}\ dr\ d\theta
				\end{align}
				\footnotetext{
					$x_0 = y_0$の仮定より$x_{t_{k-1}} - y_{t_{k-1}} = X^1_{0,t_{k-1}} - Y^1_{0,t_{k-1}}$が成り立つ.
				}
				が成り立つ.補題\ref{lem:control_function_min}より
				\begin{align}
					&\left| X^1_{t_{k-1},t_k} \right|, \left| Y^1_{t_{k-1},t_k} \right|,\left| X^1_{t_k,t_{k+1}} \right|, \left| Y^1_{t_k,t_{k+1}} \right|
					\leq \omega(t_{k-1},t_{k+1})^{1/p} \leq \left( \frac{2\omega(s,t)}{N-k-1} \right)^{1/p}, \\
					&\left| X^1_{t_{k-1},t_k} - Y^1_{t_{k-1},t_k} \right|,\left| X^1_{t_k,t_{k+1}} - Y^1_{t_k,t_{k+1}} \right|
					\leq \epsilon \omega(t_{k-1},t_{k+1})^{1/p} \leq \epsilon \left( \frac{2\omega(s,t)}{N-k-1} \right)^{1/p}
				\end{align}
				が満たされ,また
				\begin{align}
					\left| X^1_{0,t_{k-1}} - Y^1_{0,t_{k-1}} \right|
					\leq \epsilon \omega(0,t_{k-1})^{1/p}
					\leq \epsilon \omega(0,T)^{1/p}
					\leq \epsilon \left( 2 R^p + 1 \right)^{1/p}
				\end{align}
				でもあるから,
				\begin{align}
					M &\coloneqq \sum_{i,j} \sup{x \in \R^d}{|f^i_j(x)|} + \sum_{i,j,k} \sup{x \in \R^d}{|\partial_k f^i_j(x)|}
						+ \sum_{i,j,k,v} \sup{x \in \R^d}{|\partial_v \partial_k f^i_j(x)|}
					\label{eq:continuity_theorem_1_4}
				\end{align}
				と定めて
				\begin{align}
					&\left| \left\{ \tilde{I}_{s,t}(x,D_{-k}) - \tilde{I}_{s,t}(x,D_{-k-1}) \right\} - 
						\left\{ \tilde{I}_{s,t}(y,D_{-k}) - \tilde{I}_{s,t}(y,D_{-k-1}) \right\} \right| \\
					&\leq M \left|X^1_{t_{k-1},t_k}\right|\left| X^1_{t_k,t_{k+1}} - Y^1_{t_k,t_{k+1}} \right| \\
						&\quad + M \left| X^1_{t_{k-1},t_k} - Y^1_{t_{k-1},t_k} \right| \left| Y^1_{t_k,t_{k+1}} \right| \\
						&\quad + M \left| X^1_{0,t_{k-1}} - Y^1_{0,t_{k-1}} \right| \left| Y^1_{t_{k-1},t_k} \right|\left| Y^1_{t_k,t_{k+1}} \right| \\
						&\quad + M \left| X^1_{t_{k-1},t_k} - Y^1_{t_{k-1},t_k} \right| \left| Y^1_{t_{k-1},t_k} \right|\left| Y^1_{t_k,t_{k+1}} \right| \\
					&\leq \epsilon M \left[2 + 2 \left( 2 R^p + 1 \right)^{1/p} \right] \left( \frac{2\omega(s,t)}{N-k-1} \right)^{2/p} \\
					&\leq \epsilon M \left[2 + 2 \left( 2 R^p + 1 \right)^{1/p} \right] 2^{2/p} \left( 2 R^p + 1 \right)^{2/p} \left( \frac{1}{N-k-1} \right)^{2/p}
				\end{align}
				を得る.
				\begin{align}
					C_1' \coloneqq  M \left[2 + 2 \left( 2 R^p + 1 \right)^{1/p} \right] 2^{2/p} \left( 2 R^p + 1 \right)^{2/p}
				\end{align}
				とおけば
				\begin{align}
					|(\refeq{eq:continuity_theorem_1_1})| \leq
					\sum_{k=0}^{N-2} \epsilon C_1' \left( \frac{1}{N-k-1} \right)^{2/p}
					< \epsilon C_1' \zeta \biggl( \frac{2}{p} \biggr)
				\end{align}
				が成立し,$p < 2$より$\zeta(2/p) < \infty$であるから
				$C_1 \coloneqq C_1' \zeta(2/p)$とおいて(\refeq{eq:continuity_theorem_1_3})が従う.
			
			\item[第四段]
				$x_0 = y_0$の仮定により$x_s-y_s = X^1_{0,s} - Y^1_{0,s}$が成り立ち
				\begin{align}
					\left| \tilde{I}_{s,t}(x) - \tilde{I}_{s,t}(y) \right|
					&= \left| f(x_s)X^1_{s,t} - f(y_s)Y^1_{s,t} \right| \\
					&\leq \left| f(x_s)X^1_{s,t} - f(x_s)Y^1_{s,t} \right|
						+ \left| f(x_s)Y^1_{s,t} - f(y_s)Y^1_{s,t} \right| \\
					&\leq M \left| X^1_{s,t} - Y^1_{s,t} \right|
						+ \left| \int_0^1 (\nabla f)(y_s + \theta(x_s - y_s)) \left[ \left( X^1_{0,s} - Y^1_{0,s}\right) \otimes Y^1_{s,t} \right]\ d\theta \right| \\
					&\leq M \left| X^1_{s,t} - Y^1_{s,t} \right|
						+ M \left| X^1_{0,s} - Y^1_{0,s}\right| \left| Y^1_{s,t} \right| \\
					&\leq M \epsilon \omega(s,t)^{1/p} + M \epsilon \omega(0,s)^{1/p} \omega(s,t)^{1/p} \\
					&\leq \epsilon M\left[\left( 2 R^p + 1 \right)^{1/p} + \left( 2 R^p + 1 \right)^{2/p} \right]
				\end{align}
				が従う.ここで$C_2 \coloneqq M\left[\left( 2 R^p + 1 \right)^{1/p} + \left( 2 R^p + 1 \right)^{2/p} \right]$とおく.
				
			\item[第五段]
				第二段と第三段より,任意の$D \in \delta[s,t]$に対し
				\begin{align}
					\left| \tilde{I}_{s,t}(x,D) - \tilde{I}_{s,t}(y,D) \right|
					\leq \epsilon (C_1 + C_2)
				\end{align}
				が成立し.定理\ref{thm:Riemann_Stieltjes_approximation}により
				$|D| \longrightarrow 0$として
				\begin{align}
					\left| I_{s,t}(x) - I_{s,t}(y) \right|
					\leq \epsilon (C_1 + C_2)
				\end{align}
				が出る.
				\QED
		\end{description}
	\end{prf}
	
	\begin{screen}
		\begin{thm}[$2 \leq p < 3$の場合の連続性定理]\label{thm:continuity_theorem_2}
			$2 \leq p < 3$とし,
			$x_0 = y_0$を満たす$x,y \in C^1$と$f \in C^2_b(\R^d,L(\R^d \rightarrow \R^m)),\ 0 < \epsilon, R < \infty$を任意に取る.
			このとき,
			\begin{align}
				&\Norm{X^1}{p},\Norm{Y^1}{p},\Norm{X^2}{p/2},\Norm{Y^2}{p/2} \leq R < \infty,\\
				&\Norm{X^1 - Y^1}{p},\Norm{X^2 - Y^2}{p/2} \leq \epsilon
			\end{align}
			なら,或る定数$C = C(p,R,f)$が存在し,任意の$0 \leq s \leq t \leq T$に対して次が成立する:
			\begin{align}
				\left| I_{s,t}(x) - I_{s,t}(y) \right| \leq \epsilon C.
			\end{align}
		\end{thm}
	\end{screen}
	
	\begin{screen}
		\begin{cor}\label{cor:continuity_theorem_2}
			$1 \leq p < 2$とし,
			$x_0 = y_0$を満たす$x,y \in C^1$と$f \in C^2_b(\R^d,L(\R^d \rightarrow \R^m)),\ 0 < R < \infty$を任意に取る.
			このとき,
			\begin{align}
				\Norm{X^1}{p},\Norm{Y^1}{p}, \Norm{X^2}{p/2},\Norm{Y^2}{p/2} \leq R
			\end{align}
			なら,或る定数$C = C(p,R,f)$が存在して次を満たす:
			\begin{align}
				\left| I_{0,T}(x) - I_{0,T}(y) \right| \leq C\left( \Norm{X^1 - Y^1}{p}+\Norm{X^2 - Y^2}{p/2} \right).
			\end{align}
		\end{cor}
	\end{screen}
	
	\begin{prf}[系\ref{cor:continuity_theorem_2}]
		定理\ref{thm:continuity_theorem_2}において,
		$\epsilon = \Norm{X^1 - Y^1}{p} + \Norm{X^2 - Y^2}{p/2}\ (x \neq y)$
		として証明が通る.
		\QED
	\end{prf}
	
	\begin{prf}[定理\ref{cor:continuity_theorem_2}]
		$[s,t] \subset [0,T]$とする.
		\begin{description}
			\item[第一段]
				$\omega:\Delta_T \longrightarrow [0,\infty)$を
				\begin{align}
					\omega(\alpha,\beta) &= \Norm{X^1}{p,[\alpha,\beta]}^p + \Norm{Y^1}{p,[\alpha,\beta]}^p 
						+ \Norm{X^2}{p/2,[\alpha,\beta]}^{p/2} + \Norm{Y^2}{p/2,[\alpha,\beta]}^{p/2} \\
						&\quad + \epsilon^{-p} \Norm{X^1 - Y^1}{p,[\alpha,\beta]}^p +  \epsilon^{-p/2} \Norm{X^2 - Y^2}{p/2,[\alpha,\beta]}^{p/2},
					\quad ((\alpha,\beta) \in \Delta_T)
				\end{align}
				で定めれば,定理\ref{thm:examples_of_control_functions}により$2 \leq p$の下で$\omega$はcontrol functionである.
				
			\item[第二段]
				$D \in \delta[s,t]$に対し,
				定理\ref{thm:continuity_theorem_1}の証明と同様にして
				$t_{(k)},D_{-k}$を構成すれば
				\begin{align}
					&J_{s,t}(x,D) - J_{s,t}(y,D) \\
					&\qquad= \sum_{k=0}^{N-2} \left[ \left\{ J_{s,t}(x,D_{-k}) - J_{s,t}(x,D_{-k-1}) \right\} - 
						\left\{ J_{s,t}(y,D_{-k}) - J_{s,t}(y,D_{-k-1}) \right\} \right] \label{eq:continuity_theorem_2_1}\\
					&\quad\qquad + \left\{ J_{s,t}(x) - J_{s,t}(y) \right\}	\label{eq:continuity_theorem_2_2}
				\end{align}
				と表現できる.
			
			\item[第三段]
				$J_{s,t}(x,D_{-k}) - J_{s,t}(x,D_{-k-1})$を変形する.
				以降$t_k = t_{(k)}$と書き直せば
				\begin{align}
					&J_{s,t}(x,D_{-k}) - J_{s,t}(x,D_{-k-1}) \\
					&\qquad = J_{t_{k-1},t_k}(x) + J_{t_k,t_{k+1}}(x) - J_{t_{k-1},t_{k+1}}(x) \\
					&\qquad = f(x_{t_{k-1}}) X^1_{t_{k-1},t_k} + f(x_{t_k}) X^1_{t_k,t_{k+1}} - f(x_{t_{k-1}}) X^1_{t_{k-1},t_{k+1}} \\
						&\qquad \qquad + (\nabla f)(x_{t_{k-1}})X^2_{t_{k-1},t_k} + (\nabla f)(x_{t_k})X^2_{t_k,t_{k+1}} - (\nabla f)(x_{t_{k-1}})X^2_{t_{k-1},t_{k+1}} \\
					&\qquad = \left\{ f(x_{t_k}) - f(x_{t_{k-1}}) \right\} X^1_{t_k,t_{k+1}} \\
						&\qquad \qquad + (\nabla f)(x_{t_{k-1}})X^2_{t_{k-1},t_k} + (\nabla f)(x_{t_k})X^2_{t_k,t_{k+1}} - (\nabla f)(x_{t_{k-1}})X^2_{t_{k-1},t_{k+1}} \\
					&\qquad = \int_0^1 \left\{ (\nabla f)(x_{t_{k-1}} + \theta(x_{t_k} - x_{t_{k-1}})) - (\nabla f)(x_{t_{k-1}}) \right\} X^1_{t_{k-1},t_k} \otimes X^1_{t_k,t_{k+1}}\ d\theta \\
						&\qquad \qquad + (\nabla f)(x_{t_{k-1}})X^1_{t_{k-1},t_k} \otimes X^1_{t_k,t_{k+1}} \\
						&\qquad \qquad + (\nabla f)(x_{t_{k-1}})X^2_{t_{k-1},t_k} + (\nabla f)(x_{t_k})X^2_{t_k,t_{k+1}} - (\nabla f)(x_{t_{k-1}})X^2_{t_{k-1},t_{k+1}} \\
					&\qquad = \int_0^1 \int_0^\theta (\nabla f)(x_{t_{k-1}} + r(x_{t_k} - x_{t_{k-1}}))  X^1_{t_{k-1},t_k} \otimes X^1_{t_{k-1},t_k} \otimes X^1_{t_k,t_{k+1}}\ dr\ d\theta \\
						&\qquad \qquad + (\nabla f)(x_{t_{k-1}})\left( X^1_{t_{k-1},t_k} \otimes X^1_{t_k,t_{k+1}} + X^2_{t_{k-1},t_k} - X^2_{t_{k-1},t_{k+1}} \right) \\
						&\qquad \qquad + (\nabla f)(x_{t_k})X^2_{t_k,t_{k+1}} \\
					&\qquad = \int_0^1 \int_0^\theta (\nabla f)(x_{t_{k-1}} + r(x_{t_k} - x_{t_{k-1}}))  X^1_{t_{k-1},t_k} \otimes X^1_{t_{k-1},t_k} \otimes X^1_{t_k,t_{k+1}}\ dr\ d\theta \\
						&\qquad \qquad + \left\{ (\nabla f)(x_{t_k}) - (\nabla f)(x_{t_{k-1}}) \right\}X^2_{t_k,t_{k+1}} \\
					&\qquad = \int_0^1 \int_0^\theta (\nabla f)(x_{t_{k-1}} + r(x_{t_k} - x_{t_{k-1}}))  X^1_{t_{k-1},t_k} \otimes X^1_{t_{k-1},t_k} \otimes X^1_{t_k,t_{k+1}}\ dr\ d\theta \\
						&\qquad \qquad + \int_0^1 (\nabla^2 f)(x_{t_{k-1}} + \theta(x_{t_k} - x_{t_{k-1}})) X^1_{t_{k-1},t_k} \otimes X^2_{t_k,t_{k+1}}\ d\theta
				\end{align}
				を得る.
				
			\item[第四段]
				式(\refeq{eq:continuity_theorem_2_1})について,次を満たす定数$C_1$が存在することを示す:
				\begin{align}
					|(\refeq{eq:continuity_theorem_2_1})| \leq \epsilon C_1.
					\label{eq:continuity_theorem_2_3}
				\end{align}
				実際,前段の結果より
				\begin{align}
					&\left\{ J_{s,t}(x,D_{-k}) - J_{s,t}(x,D_{-k-1}) \right\} - 
						\left\{ J_{s,t}(y,D_{-k}) - J_{s,t}(y,D_{-k-1}) \right\} \\
					&=	\int_0^1 \int_0^\theta (\nabla^2 f)(x_{t_{k-1}}+r(x_{t_k}-x_{t_{k-1}})) X^1_{t_{k-1},t_k} \otimes X^1_{t_{k-1},t_k} \otimes X^1_{t_k,t_{k+1}}\ dr\ d\theta \\
						&\quad + \int_0^1 (\nabla^2 f)(x_{t_{k-1}}+\theta(x_{t_k}-x_{t_{k-1}}))X^1_{t_{k-1},t_k} \otimes X^2_{t_k,t_{k+1}}\ d\theta \\
						&\quad - \int_0^1 \int_0^\theta (\nabla^2 f)(y_{t_{k-1}}+r(y_{t_k}-y_{t_{k-1}})) Y^1_{t_{k-1},t_k} \otimes Y^1_{t_{k-1},t_k} \otimes Y^1_{t_k,t_{k+1}}\ dr\ d\theta \\
						&\quad - \int_0^1 (\nabla^2 f)(y_{t_{k-1}}+\theta(y_{t_k}-y_{t_{k-1}}))Y^1_{t_{k-1},t_k} \otimes Y^2_{t_k,t_{k+1}}\ d\theta \\
					&= \int_0^1 \int_0^\theta (\nabla^2 f)(x_{t_{k-1}}+r(x_{t_k}-x_{t_{k-1}})) X^1_{t_{k-1},t_k} \otimes X^1_{t_{k-1},t_k} \otimes \left(X^1_{t_k,t_{k+1}} - Y^1_{t_k,t_{k+1}}\right)\ dr\ d\theta \\
						&\quad + \int_0^1 \int_0^\theta (\nabla^2 f)(x_{t_{k-1}}+r(x_{t_k}-x_{t_{k-1}})) X^1_{t_{k-1},t_k} \otimes \left(X^1_{t_{k-1},t_k} - Y^1_{t_{k-1},t_k} \right) \otimes Y^1_{t_k,t_{k+1}}\ dr\ d\theta \\
						&\quad +  \int_0^1 \int_0^\theta \left\{ (\nabla^2 f)(x_{t_{k-1}}+r(x_{t_k}-x_{t_{k-1}})) - (\nabla^2 f)(y_{t_{k-1}}+r(y_{t_k}-y_{t_{k-1}})) \right\} \\
						&\qquad\qquad\qquad X^1_{t_{k-1},t_k} \otimes Y^1_{t_{k-1},t_k} \otimes Y^1_{t_k,t_{k+1}}\ dr\ d\theta \\
						&\quad + \int_0^1 \int_0^\theta (\nabla^2 f)(y_{t_{k-1}}+r(y_{t_k}-y_{t_{k-1}})) \left( X^1_{t_{k-1},t_k} - Y^1_{t_{k-1},t_k} \right) \otimes Y^1_{t_{k-1},t_k} \otimes Y^1_{t_k,t_{k+1}}\ dr\ d\theta \\
						&\quad + \int_0^1 (\nabla^2 f)(x_{t_{k-1}}+\theta(x_{t_k}-x_{t_{k-1}}))X^1_{t_{k-1},t_k} \otimes \left(X^2_{t_k,t_{k+1}} -  Y^2_{t_k,t_{k+1}}\right)\ d\theta \\
						&\quad + \int_0^1 \left\{ (\nabla^2 f) (x_{t_{k-1}}+\theta(x_{t_k}-x_{t_{k-1}})) - (\nabla^2 f)(y_{t_{k-1}}+\theta(y_{t_k}-y_{t_{k-1}})) \right\} \\
						&\qquad\qquad\qquad X^1_{t_{k-1},t_k} \otimes Y^2_{t_k,t_{k+1}}\ d\theta \\
						&\quad + \int_0^1 (\nabla^2 f)(y_{t_{k-1}}+\theta(y_{t_k}-y_{t_{k-1}})) \left( X^1_{t_{k-1},t_k} - Y^1_{t_{k-1},t_k} \right) \otimes Y^2_{t_k,t_{k+1}}\ d\theta \\
					&= \int_0^1 \int_0^\theta (\nabla^2 f)(x_{t_{k-1}}+r(x_{t_k}-x_{t_{k-1}})) X^1_{t_{k-1},t_k} \otimes X^1_{t_{k-1},t_k} \otimes \left(X^1_{t_k,t_{k+1}} - Y^1_{t_k,t_{k+1}}\right)\ dr\ d\theta \\
						&\quad + \int_0^1 \int_0^\theta (\nabla^2 f)(x_{t_{k-1}}+r(x_{t_k}-x_{t_{k-1}})) X^1_{t_{k-1},t_k} \otimes \left(X^1_{t_{k-1},t_k} - Y^1_{t_{k-1},t_k} \right) \otimes Y^1_{t_k,t_{k+1}}\ dr\ d\theta \\
						&\quad +  \int_0^1 \int_0^\theta \int_0^1 (\nabla^3 f)(y_{t_{k-1}}+r(y_{t_k}-y_{t_{k-1}}) + u(x_{t_{k-1}}+r(x_{t_k}-x_{t_{k-1}}) - y_{t_{k-1}}-r(y_{t_k}-y_{t_{k-1}}))) \\
						&\qquad\qquad\qquad \left\{ \left( X^1_{0,t_{k-1}} - Y^1_{0,t_{k-1}} \right) + r\left( X^1_{t_{k-1},t_k} - Y^1_{t_{k-1},t_k} \right) \right\} \otimes X^1_{t_{k-1},t_k} \otimes Y^1_{t_{k-1},t_k} \otimes Y^1_{t_k,t_{k+1}}\ du\ dr\ d\theta \\
						&\quad + \int_0^1 \int_0^\theta (\nabla^2 f)(y_{t_{k-1}}+r(y_{t_k}-y_{t_{k-1}})) \left( X^1_{t_{k-1},t_k} - Y^1_{t_{k-1},t_k} \right) \otimes Y^1_{t_{k-1},t_k} \otimes Y^1_{t_k,t_{k+1}}\ dr\ d\theta \\
						&\quad + \int_0^1 (\nabla^2 f)(x_{t_{k-1}}+\theta(x_{t_k}-x_{t_{k-1}}))X^1_{t_{k-1},t_k} \otimes \left(X^2_{t_k,t_{k+1}} -  Y^2_{t_k,t_{k+1}}\right)\ d\theta \\
						&\quad + \int_0^1 \int_0^1 (\nabla^3 f) (y_{t_{k-1}}+\theta(y_{t_k}-y_{t_{k-1}}) + r(x_{t_{k-1}}+\theta(x_{t_k}-x_{t_{k-1}})-y_{t_{k-1}}-\theta(y_{t_k}-y_{t_{k-1}}))) \\
						&\qquad\qquad\qquad \left\{ \left( X^1_{0,t_{k-1}} - Y^1_{0,t_{k-1}} \right) + \theta\left( X^1_{t_{k-1},t_k} - Y^1_{t_{k-1},t_k} \right) \right\} \otimes X^1_{t_{k-1},t_k} \otimes Y^2_{t_k,t_{k+1}}\ dr\ d\theta \\
						&\quad + \int_0^1 (\nabla^2 f)(y_{t_{k-1}}+\theta(y_{t_k}-y_{t_{k-1}})) \left( X^1_{t_{k-1},t_k} - Y^1_{t_{k-1},t_k} \right) \otimes Y^2_{t_k,t_{k+1}}\ d\theta
				\end{align}
				が成り立つから,
				\begin{align}
					M &\coloneqq \sum_{i,j} \sup{x \in \R^d}{|f^i_j(x)|} + \sum_{i,j,k} \sup{x \in \R^d}{|\partial_k f^i_j(x)|} \\
						&\qquad + \sum_{i,j,k,v} \sup{x \in \R^d}{|\partial_v \partial_k f^i_j(x)|}
						+ \sum_{i,j,k,v,w} \sup{x \in \R^d}{|\partial_w \partial_v \partial_k f^i_j(x)|}
					\label{eq:continuity_theorem_2_4}
				\end{align}
				とおいて
				\begin{align}
					&\left| \left\{ J_{s,t}(x,D_{-k}) - J_{s,t}(x,D_{-k-1}) \right\} - 
						\left\{ J_{s,t}(y,D_{-k}) - J_{s,t}(y,D_{-k-1}) \right\} \right| \\
					&\leq M \left|X^1_{t_{k-1},t_k}\right| \left|X^1_{t_{k-1},t_k}\right| \left| X^1_{t_k,t_{k+1}} - Y^1_{t_k,t_{k+1}} \right| \\
						&\quad + M \left| X^1_{t_{k-1},t_k} \right| \left| X^1_{t_{k-1},t_k} - Y^1_{t_{k-1},t_k} \right| \left| Y^1_{t_k,t_{k+1}} \right| \\
						&\quad + M \left| X^1_{0,t_{k-1}} - Y^1_{0,t_{k-1}} \right| \left| X^1_{t_{k-1},t_k} \right| \left| Y^1_{t_{k-1},t_k} \right| \left| Y^1_{t_k,t_{k+1}} \right| \\
						&\quad + M \left| X^1_{t_{k-1},t_k} - Y^1_{t_{k-1},t_k} \right| \left| X^1_{t_{k-1},t_k} \right| \left| Y^1_{t_{k-1},t_k} \right| \left| Y^1_{t_k,t_{k+1}} \right| \\
						&\quad + M \left| X^1_{t_{k-1},t_k} - Y^1_{t_{k-1},t_k} \right| \left| Y^1_{t_{k-1},t_k} \right| \left| Y^1_{t_k,t_{k+1}} \right| \\
						&\quad + M \left| X^1_{t_{k-1},t_k} \right| \left| X^2_{t_k,t_{k+1}} - Y^2_{t_k,t_{k+1}} \right| \\
						&\quad + M \left| X^1_{0,t_{k-1}} - Y^1_{0,t_{k-1}} \right| \left| X^1_{t_{k-1},t_k} \right| \left| Y^2_{t_k,t_{k+1}} \right| \\
						&\quad + M \left| X^1_{t_{k-1},t_k} - Y^1_{t_{k-1},t_k} \right| \left| X^1_{t_{k-1},t_k} \right| \left| Y^2_{t_k,t_{k+1}} \right| \\
						&\quad + M \left| X^1_{t_{k-1},t_k} - Y^1_{t_{k-1},t_k} \right| \left| Y^2_{t_k,t_{k+1}} \right| \\
					&\leq \epsilon M \left[5 + 2\omega(0,t_{k-1})^{1/p} + 2\omega(t_{k-1},t_k)^{1/p} \right] \left( \frac{2\omega(s,t)}{N-k-1} \right)^{3/p} \\
					&\leq \epsilon M \left[2 + 4\left( 2 R^p + 2R^{p/2} + 2 \right)^{1/p} \right] 2^{3/p} \left( 2 R^p + 2R^{p/2} + 2 \right)^{3/p} \left( \frac{1}{N-k-1} \right)^{3/p}
				\end{align}
				を得る.ここで
				\begin{align}
					C_1' \coloneqq  M \left[2 + 4\left( 2 R^p + 2R^{p/2} + 2 \right)^{1/p} \right] 2^{3/p} \left( 2 R^p + 2R^{p/2} + 2 \right)^{3/p}
				\end{align}
				と定めれば
				\begin{align}
					|(\refeq{eq:continuity_theorem_2_1})| \leq
					\sum_{k=0}^{N-2} \epsilon C_1' \left( \frac{1}{N-k-1} \right)^{3/p}
					< \epsilon C_1' \zeta \biggl( \frac{3}{p} \biggr)
				\end{align}
				が成立し,$p < 3$より$\zeta(3/p) < \infty$であるから
				$C_1 \coloneqq C_1' \zeta(3/p)$とおいて(\refeq{eq:continuity_theorem_2_3})が出る.
			
			\item[第五段]
				$x_0 = y_0$の仮定により
				\begin{align}
					&\left| J_{s,t}(x) - J_{s,t}(y) \right| \\
					&\leq \left| f(x_s)X^1_{s,t} - f(y_s)Y^1_{s,t} \right| + \left| (\nabla f)(x_s)X^2_{s,t} - (\nabla f)(y_s)Y^2_{s,t} \right| \\
					&\leq \left| f(x_s)X^1_{s,t} - f(x_s)Y^1_{s,t} \right|
						+ \left| f(x_s)Y^1_{s,t} - f(y_s)Y^1_{s,t} \right| \\
						&\qquad + \left| (\nabla f)(x_s)X^2_{s,t} - (\nabla f)(x_s)Y^2_{s,t} \right|
						+ \left| (\nabla f)(x_s)Y^2_{s,t} - (\nabla f)(y_s)Y^2_{s,t} \right| \\
					&\leq M \left| X^1_{s,t} - Y^1_{s,t} \right|
						+ \left| \int_0^1 (\nabla f)(y_s + \theta(x_s - y_s))(x_s - y_s) \otimes Y^1_{s,t}\ d\theta \right| \\
						&\qquad + M \left| X^2_{s,t} - Y^2_{s,t} \right|
						+ \left| \int_0^1 (\nabla^2 f)(y_s + \theta(x_s - y_s))(x_s - y_s) \otimes Y^2_{s,t}\ d\theta \right| \\
					&\leq M \left| X^1_{s,t} - Y^1_{s,t} \right|
						+ M \left| X^1_{0,s} - Y^1_{0,s}\right| \left| Y^1_{s,t} \right| \\
						&\qquad + M \left| X^2_{s,t} - Y^2_{s,t} \right| 
						+ M \left| X^1_{0,s} - Y^1_{0,s}\right| \left| Y^2_{s,t} \right| \\
					&\leq \epsilon M \omega(s,t)^{1/p} + \epsilon M \omega(0,s)^{1/p} \omega(s,t)^{1/p} \\
						&\quad + \epsilon M \omega(s,t)^{2/p} + \epsilon M \omega(0,s)^{1/p} \omega(s,t)^{2/p} \\
					&\leq \epsilon M \left[ \omega(0,T)^{1/p} + 2\omega(0,T)^{2/p} + \omega(0,T)^{3/p} \right] \\
					&\leq \epsilon M \left[ \left( 2 R^p + 2R^{p/2} + 2 \right)^{1/p}
						+ 2\left( 2 R^p + 2R^{p/2} + 2 \right)^{2/p}
						+\left( 2 R^p + 2R^{p/2} + 2 \right)^{3/p} \right]
				\end{align}
				が従う.ここで最下段の$\epsilon$の係数を$C_2$とおく.
				
			\item[第六段]
				以上より,任意の$D \in \delta[s,t]$に対し
				\begin{align}
					\left| J_{s,t}(x,D) - J_{s,t}(y,D) \right|
					\leq \epsilon (C_1 + C_2)
				\end{align}
				が成り立ち,定理\ref{thm:Riemann_Stieltjes_approximation}により
				$|D| \longrightarrow 0$として
				\begin{align}
					\left| I_{s,t}(x) - I_{s,t}(y) \right|
					\leq \epsilon (C_1 + C_2)
				\end{align}
				が出る.
				\QED
		\end{description}
	\end{prf}
	
	\begin{screen}
		\begin{cor}[パスが0出発なら$f$の有界性は要らない]
			定理\ref{thm:continuity_theorem_1}と定理\ref{thm:continuity_theorem_2}について,
			$x,y \in \tilde{C}^1$ならば
			$f \in C^2(\R^d,L(\R^d \rightarrow \R^m))$として主張が成り立つ.
		\end{cor}
	\end{screen}
	
	\begin{prf}
		$x_0 = 0$なら
		\begin{align}
			\Norm{X^1}{p} \leq R \quad \Rightarrow \quad |x_t| \leq R \quad (\forall t \in [0,T])
		\end{align}
		が成り立つから,式(\refeq{eq:continuity_theorem_1_4})と(\refeq{eq:continuity_theorem_2_4})において
		$\sup{x\in\R^d}{}$を$\sup{|x| \leq 9R}$に替えればよい.
		\QED
	\end{prf}