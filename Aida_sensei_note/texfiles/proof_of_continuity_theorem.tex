\section{連続性定理の証明}
	\begin{screen}
		\begin{dfn}[control function]
			関数$\omega:\Delta_T \longrightarrow [0,\infty)$
			が任意の$0 \leq s \leq u \leq t \leq T$に対して
			\begin{align}
				\omega(s,u) + \omega(u,t) \leq \omega(s,t)
			\end{align}
			を満たすとき,$\omega$をcontrol functionと呼ぶ.
		\end{dfn}
	\end{screen}
	
	\begin{screen}
		\begin{thm}[control functionの例]
			以下の関数$\omega:\Delta_T \longrightarrow [0,\infty)$はcontrol functionである.
			\begin{description}
				\item[(1)] $\omega \coloneqq \left( \omega_1^{1/p} + \omega_2^{1/p} \right)^p,
					\quad (p \geq 1,\ \omega_1,\omega_2:\mbox{control function}).$
				\item[(2)] $\omega:(s,t) \longmapsto \Norm{X^1}{p:[s,t]}^p,
					\quad (p \geq 1,\ x \in B_{p,T}(\R^d)).$
				\item[(3)]
			\end{description}
		\end{thm}
	\end{screen}
	
	\begin{thm}\mbox{}
		\begin{description}
			\item[(1)]
			\item[(2)] 任意の$D_1 \in \delta[s,u],D_2 \in \delta[u,t]$に対して
				合併$D_1 \cup D_2$は$[s,t]$の分割であるから,
				\begin{align}
					\sum_{D_1}\left| x_{t_i} - x_{t_{i-1}} \right|^p
					+ \sum_{D_2}\left| x_{t_i} - x_{t_{i-1}} \right|^p
					\leq \sup{D \in \delta[s,t]}{\sum_{D} \left| x_{t_i} - x_{t_{i-1}} \right|^p}
					= \Norm{X^1}{p:[s,t]}^p
				\end{align}
				が成り立つ.左辺の$D_1,D_2$の取り方は独立であるから,それぞれに対し上限を取れば
				\begin{align}
					\Norm{X^1}{p:[s,u]}^p + \Norm{X^1}{p:[u,t]}^p
					\leq \Norm{X^1}{p:[s,t]}^p
				\end{align}
				が得られる.
		\end{description}
	\end{thm}
	
	\begin{screen}
		\begin{thm}[$1 \leq p < 2$の場合の連続性定理]
		\end{thm}
	\end{screen}
	
	\begin{screen}
		\begin{thm}[$2 \leq p < 3$の場合の連続性定理]
		\end{thm}
	\end{screen}