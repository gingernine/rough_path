\documentclass[11pt,a4paper]{jsreport}
%
\usepackage{amsmath,amssymb}
\usepackage{amsthm}
\usepackage{makeidx}
\usepackage{txfonts}
\usepackage{mathrsfs} %花文字
\usepackage{mathtools} %参照式のみ式番号表示
\usepackage{latexsym} %qed
\usepackage{ascmac}
\usepackage{color}
\usepackage{comment}

\newtheoremstyle{mystyle}% % Name
	{20pt}%                      % Space above
	{20pt}%                      % Space below
	{\rm}%           % Body font
	{}%                      % Indent amount
	{\gt}%             % Theorem head font
	{.}%                      % Punctuation after theorem head
	{10pt}%                     % Space after theorem head, ' ', or \newline
	{}%                      % Theorem head spec (can be left empty, meaning `normal')
\theoremstyle{mystyle}

\allowdisplaybreaks[1]

\newcommand{\bhline}[1]{\noalign {\hrule height #1}} %表の罫線を太くする.
\newcommand{\bvline}[1]{\vrule width #1} %表の罫線を太くする.
\newtheorem{Prop}{$Proposition.$}
\newtheorem{Proof}{$Proof.$}
\newcommand{\QED}{% %証明終了
	\relax\ifmmode
		\eqno{%
		\setlength{\fboxsep}{2pt}\setlength{\fboxrule}{0.3pt}
		\fcolorbox{black}{black}{\rule[2pt]{0pt}{1ex}}}
	\else
		\begingroup
		\setlength{\fboxsep}{2pt}\setlength{\fboxrule}{0.3pt}
		\hfill\fcolorbox{black}{black}{\rule[2pt]{0pt}{1ex}}
		\endgroup
	\fi}
\newtheorem{thm}{定理}[section]
\newtheorem{dfn}[thm]{定義}
\newtheorem{prp}[thm]{命題}
\newtheorem{lem}[thm]{補題}
\newtheorem*{prf}{証明}
\newtheorem{qst}{レポート問題}
\newtheorem*{bcs}{なぜならば}
\newtheorem{rem}[thm]{注意}
\newcommand{\defunc}{\mbox{1}\hspace{-0.25em}\mbox{l}} %定義関数
\newcommand{\wlim}{\mbox{w-}\lim}
\newcommand{\dx}{\ \operatorname{d}} %積分のd
\def\Set#1#2{\left\{\ #1\ \, ; \quad #2\ \right\}} %集合の書き方
\def\Box#1{$(\mbox{#1})$} %丸括弧つきコメント
\def\Hat#1{$\hat{\mathrm{#1}}$} %文中ハット
\def\Ddot#1{$\ddot{\mathrm{#1}}$} %文中ddot
\def\DEF{\overset{\mathrm{def}}{\Leftrightarrow}} %定義記号
\def\eqqcolon{=\mathrel{\mathop:}} %定義=:
\def\max#1#2{\operatorname*{max}_{#1} #2 } %最大
\def\min#1#2{\operatorname*{min}_{#1} #2 } %最小
\def\sin#1#2{\operatorname{sin}^{#2} #1} %sin
\def\cos#1#2{\operatorname{cos}^{#2} #1} %cos
\def\tan#1#2{\operatorname{tan}^{#2} #1} %tan
\def\inprod<#1>{\left\langle #1 \right\rangle} %内積
\def\sup#1#2{\operatorname*{sup}_{#1} #2 } %上限
\def\inf#1#2{\operatorname*{inf}_{#1} #2 } %下限
\def\Vector#1{\mbox{\boldmath $#1$}} %ベクトルを太字表示
\def\Norm#1#2{\left\|\, #1\, \right\|_{#2}} %ノルム
\def\Log#1{\operatorname{log} #1} %log
\def\Det#1{\operatorname{det} ( #1 )} %行列式
\def\Diag#1{\operatorname{diag} \left( #1 \right)} %行列の対角成分
\def\Tmat#1{#1^\mathrm{T}} %転置行列
\def\Exp#1{\operatorname{E} \left[ #1 \right]} %期待値
\def\Var#1{\operatorname{V} \left[ #1 \right]} %分散
\def\Cov#1#2{\operatorname{Cov} \left[ #1,\ #2 \right]} %共分散
\def\exp#1{e^{#1}} %指数関数
\def\N{\mathbb{N}} %自然数全体
\def\Z{\mathbb{Z}} %整数全体
\def\Q{\mathbb{Q}} %有理数全体
\def\R{\mathbb{R}} %実数全体
\def\C{\mathbb{C}} %複素数全体
\def\K{\mathbb{K}} %係数体K
\def\conj#1{\overline{#1}} %共役複素数
\def\Re#1{\mathfrak{Re}#1} %実部
\def\Im#1{\mathfrak{Im}#1} %虚部
\def\Conv#1{\mathrm{Conv}\left[\left\{\, #1 \, \right\}\right]} %凸包
\def\Span#1{\mathrm{Span}\left[ #1 \right]} %線型包
\def\borel#1{\mathfrak{B}(#1)} %Borel集合族
\def\open#1{\mathfrak{O}(#1)} %位相空間 #1 の位相
\def\close#1{\mathfrak{A}(#1)} %%位相空間 #1 の閉集合系
\def\closure#1{\overline{#1}}
%\def\equiv#1#2{\left[#1\right]_{#2}} %同値類
\def\rapid#1{\mathfrak{S}(#1)} %急減少空間
\def\c#1{C(#1)} %有界実連続関数
\def\cbound#1{C_{b} (#1)} %有界実連続関数
\def\Dim#1{\mathrm{dim}#1} %次元
\def\pmod#1{\mathrm{Mod}\ #1} %mod
\def\Map#1#2{\mathrm{Map} \left(#1,#2\right)} %写像全体
\def\Hom#1#2{\mathrm{Hom} \left(#1,#2\right)} %線形写像全体
\def\Ln#1#2#3{\mathrm{Hom}^{(#3)} \left(#1,#2\right)} %多重線形写像全体
\def\semiLp#1#2{\mathscr{L}^{#1} \left(#2\right)} %ノルム空間L^p
\def\Lp#1#2{\operatorname{L}^{#1} \left(#2\right)} %ノルム空間L^p
\def\cinf#1{C^{\infty} (#1)} %無限回連続微分可能関数
\def\sgmalg#1{\sigma \left[#1\right]} %#1が生成するσ加法族
\def\ball#1#2{\operatorname{B} \left(#1\, ;\, #2 \right)} %開球
\def\prob#1{\operatorname{P} \left(#1\right)} %確率
\def\cprob#1#2{\operatorname{P} \left(\left\{ #1 \ \middle|\ #2 \right\}\right)} %条件付確率
\def\cexp#1#2{\operatorname{E} \left[ #1 \ \middle|\ #2 \right]} %条件付期待値
\def\tExp#1{\tilde{\operatorname{E}} \left[ #1 \right]} %拡張期待値
\def\tcexp#1#2{\tilde{\operatorname{E}} \left[ #1 \ \middle|\ #2 \right]} %拡張条件付期待値
%\renewcommand{\contentsname}{\bm Index}
%
\makeindex
%
\setlength{\textwidth}{\fullwidth}
\setlength{\textheight}{40\baselineskip}
\addtolength{\textheight}{\topskip}
\setlength{\voffset}{-0.2in}
\setlength{\topmargin}{0pt}
\setlength{\headheight}{0pt}
\setlength{\headsep}{0pt}
%
\title{テンソル積のノルム\\平井さん講義まとめ}
\author{基礎工学研究科システム創成専攻\\学籍番号29C17095\\百合川尚学}
\date{\today}

\begin{document}
%
%

\mathtoolsset{showonlyrefs = true}
\maketitle

\newpage
\tableofcontents
	\section{テンソル積}
	$n \geq 2$として,体$\K$上の線形空間の族$(E_i)_{i=1}^n$に対して
	テンソル積を定義する.
	\begin{align}
		\Lambda\biggl( \bigoplus_{i=1}^n E_i \biggr)
		= \Set{b:\bigoplus_{i=1}^n E_i \longrightarrow \K}{\mbox{有限個の$e \in \bigoplus_{i=1}^n E_i$を除いて$b(e)=0$.}}
	\end{align}
	により$\K$-線形空間$\Lambda\biggl( \bigoplus_{i=1}^n E_i \biggr)$を定める.また$e=(e_1,\cdots,e_n) \in \bigoplus_{i=1}^n E_i$に対する定義関数を
	\begin{align}
		\defunc_{e_1,\cdots,e_n} (x) = 
		\begin{cases}
			1, & x = e, \\
			0, & x \neq e
		\end{cases}
	\end{align}
	で表す.$\Lambda\biggl( \bigoplus_{i=1}^n E_i \biggr)$の線型部分空間を
	\begin{align}
		&\Lambda_0\biggl( \bigoplus_{i=1}^n E_i \biggr) \\
		&\coloneqq
		\Span{\Set{ \substack{\defunc_{e_1,\cdots,e_i + e_i',\cdots,e_n}
			-\defunc_{e_1,\cdots,e_i,\cdots,e_n}
			-\defunc_{e_1,\cdots,e_i',\cdots,e_n},\\
			\defunc_{e_1,\cdots,\lambda e_i,\cdots,e_n}
			-\lambda\defunc_{e_1,\cdots,e_i,\cdots,e_n}}}{e_i,e_i' \in E_i,
			\lambda \in \K,
			1 \leq i \leq n}}
	\end{align}
	により定め,
	$b \in \Lambda\biggl( \bigoplus_{i=1}^n E_i \biggr)$の$\Lambda_0\biggl( \bigoplus_{i=1}^n E_i \biggr)$に関する同値類を$[b]$と書く.そして
	\begin{align}
		E_1 \otimes \cdots \otimes E_n = \bigotimes_{i=1}^n E_i 
		\coloneqq \Lambda\biggl( \bigoplus_{i=1}^n E_i \biggr)
		\left/ \Lambda_0\biggl( \bigoplus_{i=1}^n E_i \biggr) \right.
	\end{align}
	で定める商空間を$(E_i)_{i=1}^n$のテンソル積と定義する.
	また$(e_1,\cdots,e_n) \in \bigoplus_{i=1}^n E_i$に対し
	\begin{align}
		e_1 \otimes \cdots \otimes e_n \coloneqq \left[ \defunc_{e_1,\cdots,e_n} \right]
	\end{align}
	により定める$\otimes:\bigoplus_{i=1}^n E_i \longrightarrow \bigotimes_{i=1}^n E_i$を
	テンソル積の標準写像と呼ぶ.
	
	\begin{screen}
		\begin{thm}[標準写像の多重線型性]\label{thm:tensor_product_is_bilinear}
			$(E_i)_{i=1}^n$を$\K$-線形空間の族とするとき,
			\begin{align}
				\otimes : \bigoplus_{i=1}^n E_i \ni (e_1,\cdots,e_n) \longmapsto e_1 \otimes \cdots \otimes e_n \in \bigotimes_{i=1}^n E_i
			\end{align}
			は$n$重線型写像である.また次が成り立つ:
			\begin{align}
				\bigotimes_{i=1}^n E_i = \Span{\Set{e_1 \otimes \cdots \otimes e_n}{(e_1,\cdots,e_n) \in \bigoplus_{i=1}^n E_i}}.
				\label{eq:thm_tensor_product_is_bilinear}
			\end{align}
		\end{thm}
	\end{screen}
	
	\begin{prf}
		任意の$1 \leq i \leq n,\ e_1 \in E_1,\cdots,e_n \in E_n,
		\ e_i,e_i' \in E_i,\ \lambda \in \K$に対して
		\begin{align}
			e_1 \otimes \cdots \otimes (e_i + e_i') \otimes \cdots \otimes e_n 
			&= \left[ \defunc_{e_1,\cdots,e_i + e_i',\cdots,e_n} \right] \\
			&= \left[ \defunc_{e_1,\cdots,e_i,\cdots,e_n} 
				+ \defunc_{e_1,\cdots,e_i',\cdots,e_n} \right] \\
			&= \left[ \defunc_{e_1,\cdots,e_i,\cdots,e_n} \right]
				+ \left[ \defunc_{e_1,\cdots,e_i',\cdots,e_n} \right] \\
			&=e_1 \otimes \cdots \otimes e_i \otimes \cdots \otimes e_n 
			+ e_1 \otimes \cdots \otimes e_i' \otimes \cdots \otimes e_n, \\
			e_1 \otimes \cdots \otimes (\lambda e_i) \otimes \cdots \otimes e_n 
			&= \left[ \defunc_{e_1,\cdots,\lambda e_i,\cdots,e_n} \right] \\
			&= \left[ \lambda \defunc_{e_1,\cdots,e_i,\cdots,e_n} \right] \\
			&= \lambda \left[ \defunc_{e_1,\cdots,e_i,\cdots,e_n} \right] \\
			&= \lambda (e_1 \otimes \cdots \otimes e_i \otimes \cdots \otimes e_n) 
		\end{align}
		が成立するから$\otimes$は$n$重線型である.また
		任意に$u = [b] \in E \otimes F$を取れば
		\begin{align}
			b = \sum_{j=1}^m k_j \defunc_{e^j_i,\cdots,e^j_n},
			\quad \left( k_j = b(e^j_i,\cdots,e^j_n),\ j=1,\cdots,m \right)
		\end{align}
		と表せるから,
		\begin{align}
			u = \left[ \sum_{j=1}^m k_j \defunc_{e^j_i,\cdots,e^j_n} \right]
			= \left[ \sum_{j=1}^m \defunc_{k_j e^j_i,\cdots,e^j_n} \right]
			= \sum_{j=1}^m (k_j e^j_1) \otimes \cdots \otimes e^j_n
		\end{align}
		が従い(\refeq{eq:thm_tensor_product_is_bilinear})を得る.
		\QED
	\end{prf}
	
	\begin{screen}
		\begin{thm}[$\cdots \otimes 0 \otimes \cdots$は零ベクトル]
		\label{thm:when_tensor_product_zero}
			$(E_i)_{i=1}^n$を$\K$-線形空間の族とし,
			テンソル積$\bigotimes_{i=1}^n E_i$を定める.
			このとき,或る$i$で$e_i = 0$なら
			$e_1 \otimes \cdots \otimes e_n = 0$が成り立つ.
		\end{thm}
	\end{screen}
	
	\begin{prf}
		$e_i = 0$のとき,$\lambda = 0$とすれば
		\begin{align}
			e_1 \otimes \cdots \otimes e_n
			= \left[ \defunc_{e_1,\cdots,0,\cdots,e_n} \right]
			= \left[ \defunc_{e_1,\cdots,\lambda e_i,\cdots,e_n} - \lambda \defunc_{e_1,\cdots,e_i,\cdots,e_n}\right]
			= 0
		\end{align}
		が成立する.
		\QED
	\end{prf}
	
	\begin{screen}
		\begin{thm}[普遍性(universality of tensor products)]
		\label{thm:universality_of_tensor_product}
			$(E_i)_{i=1}^n$を$\K$-線形空間の族とする.このとき
			任意の$\K$-線型空間$V$に対して,$T \in \Hom{\bigotimes_{i=1}^n E_i}{V}$ならば
			$T \circ \otimes \in \Ln{\bigoplus_{i=1}^n E_i}{V}{n}$
			が満たされ,これで定める次の対応$\Phi_V$は線型同型である:
			\begin{align}
				\begin{array}{ccc}
					\Phi_V:\Hom{\bigotimes_{i=1}^n E_i}{V} & \longrightarrow & \Ln{\bigoplus_{i=1}^n E_i}{V}{n} \\
					\rotatebox{90}{$\in$} & & \rotatebox{90}{$\in$} \\
					T & \longmapsto & T \circ \otimes 
					\label{eq:thm_universality_of_tensor_product}
				\end{array}
			\end{align}
			\begin{align}
				\xymatrix{
					&\bigoplus_{i=1}^n E_i \ar[d]_-\otimes \ar[rd]^-{\Phi(T)} & \\
					&\bigotimes_{i=1}^n E_i \ar[r]^-T & V \ar@{}[lu]<2ex>|\circlearrowright
				}
			\end{align}
			また$\K$-線型空間$U_0$と$n$重線型写像$\iota:\bigoplus_{i=1}^n E_i \longrightarrow U_0$が,
			任意の$\K$-線型空間$V$に対し
			\begin{description}
				\item[$(\otimes)_1$] $U_0$は$\iota$の像で生成される.
				\item[$(\otimes)_2$] 任意の$\delta \in \Ln{\bigoplus_{i=1}^n E_i}{V}{n}$に対して
					$\delta = \tau \circ \iota$を満たす$\tau \in \Hom{U_0}{V}$が存在する.
			\end{description}
			を満たすなら,(\refeq{eq:thm_universality_of_tensor_product})において
			$V = U_0$とするとき$T = \Phi_{U_0}^{-1}(\iota):
			\bigotimes_{i=1}^n E_i \longrightarrow U_0$は線型同型である.
		\end{thm}
	\end{screen}
	後半の主張により,$(E_i)_i$のテンソル積を別の方法で導入しても,
	商空間を用いて導入した$\bigotimes_i E_i$と線型同型に結ばれる.
	このとき,別の方法で導入したテンソル積及び標準写像を$\bigotimes\tilde{ }_i E_i,\ \tilde{\otimes}$と表せば,
	或る線型同型$T:\bigotimes_i E_i \longrightarrow \bigotimes\tilde{ }_i E_i$がただ一つ存在して
	\begin{align}
		T(e_1 \otimes \cdots \otimes e_n) = e_1 \tilde{\otimes} \cdots \tilde{\otimes} e_n 
	\end{align}
	を満たす.特に任意の並べ替え$\varphi:\{1,\cdots,n\} \longrightarrow \{1,\cdots,n\}$に対し
	\begin{align}
		\begin{array}{ccc}
		\bigotimes_{i=1}^{n} E_i & \cong & \bigotimes_{i=1}^{n} E_{\varphi(i)} \\
		\rotatebox{90}{$\in$} & & \rotatebox{90}{$\in$} \\
		e_1 \otimes \cdots \otimes e_n & \longleftrightarrow & e_{\varphi(1)} \otimes \cdots \otimes e_{\varphi(n)}
		\end{array}
	\end{align}
	が成立する.
	
	\begin{prf}\mbox{}
		\begin{description}
			\item[第一段]
				$T \in \Hom{\bigotimes_{i=1}^n E_i}{V}$の線型性と
				$\otimes$の$n$重線型性より
				$T \circ \otimes$は$n$重線型である.
				
			\item[第二段]
				$\Phi_V(T_1) = \Phi_V(T_2)$ならば
				$T_1$と$T_2$は$\Set{e_1 \otimes \cdots \otimes e_n}{(e_1,\cdots,e_n) \in \bigoplus_{i=1}^n E_i}$の上で一致する.
				(\refeq{eq:thm_tensor_product_is_bilinear})より
				$T_1 = T_2$が成立し$\Phi_V$の単射性が従う.
			
			\item[第三段]
				次の二段で$\Phi_V$の全射性を示す.まず,$\varphi \in \Hom{\Lambda(\bigoplus_{i=1}^n E_i)}{V}$に対し
				\begin{align}
					g: \bigoplus_{i=1}^n E_i \ni (e_1,\cdots,e_n) \longmapsto \varphi(\defunc_{e_1,\cdots,e_n}) \in V
				\end{align}
				を対応させる次の写像が全単射であることを示す:
				\begin{align}
					\begin{array}{ccc}
						F:\Hom{\Lambda(\bigoplus_{i=1}^n E_i)}{V} & \longrightarrow & \Map{\bigoplus_{i=1}^n E_i}{V} \\
						\rotatebox{90}{$\in$} & & \rotatebox{90}{$\in$} \\
						\varphi & \longmapsto & g
					\end{array}
				\end{align}
				$F(\varphi_1) = F(\varphi_2)$のとき,
				任意の$e \in \bigoplus_{i=1}^n E_i$に対して
				$\varphi_1(\defunc_{e_1,\cdots,e_n}) = \varphi_2(\defunc_{e_1,\cdots,e_n})$が成り立ち,
				\begin{align}
					\Lambda\biggl( \bigoplus_{i=1}^n E_i \biggr) 
					= \Span{\Set{\defunc_{e_1,\cdots,e_n}}{(e_1,\cdots,e_n) \in \bigoplus_{i=1}^n E_i}}
				\end{align}
				より$\varphi_1 = \varphi_2$が従い$F$の単射性が得られる.また
				$g \in \Map{\bigoplus_{i=1}^n E_i}{V}$に対して
				\begin{align}
					\varphi(0) &\coloneqq 0, \\
					\varphi(a) &\coloneqq \sum_{\substack{e \in \bigoplus_{i=1}^n E_i \\ a(e) \neq 0}} a(e) g(e),
					\quad \biggl(\forall a \in \Lambda\biggl(\bigoplus_{i=1}^n E_i\biggr),\ a \neq 0\biggr)
				\end{align}
				により$\varphi$を定めれば,$\varphi \in \Hom{\Lambda(\bigoplus_{i=1}^n E_i)}{V}$より
				\footnote{
					$\left( \Set{e}{a(e) \neq 0} \cup \Set{e}{a'(e) \neq 0} \right) 
					\cap \Set{e}{(a+a')(e) \neq 0} = \Set{e}{(a+a')(e) \neq 0}$より
					\begin{align}
						\varphi(a) + \varphi(a')
						&= \sum_{a(e) \neq 0} a(e) g(e) + \sum_{a'(e) \neq 0} a'(e) g(e)
						= \sum_{\substack{a(e) \neq 0 \\ (a + a')(e) = 0}} a(e) g(e)
						+ \sum_{\substack{a(e) \neq 0 \\ (a + a')(e) \neq 0}} a(e) g(e)
						+ \sum_{\substack{a'(e) \neq 0 \\ (a + a')(e) = 0}} a'(e) g(e)
						+ \sum_{\substack{a'(e) \neq 0 \\ (a + a')(e) \neq 0}} a'(e) g(e) \\
						&= \sum_{\substack{a(e) \neq 0 \\ (a + a')(e) \neq 0}} a(e) g(e)
						+ \sum_{\substack{a'(e) \neq 0 \\ (a + a')(e) \neq 0}} a'(e) g(e) 
						= \sum_{\substack{(a + a')(e) \neq 0}} (a + a')(e) g(e) 
						= \varphi(a + a')
					\end{align}
					$\varphi$の加法性を得る.スカラ倍は$\varphi(\beta a) = \sum_{(\beta a)(e) \neq 0} (\beta a)(e)g(e) = \beta \sum_{a(e) \neq 0} a(e)g(e) = \beta \varphi(a)\ (\beta \neq 0)$及び$\varphi(0) = 0$より従う.
				}
				$F$の全射性が出る.
				
			\item[第四段]
				任意に$b \in \Ln{\bigoplus_{i=1}^n E_i}{V}{n}$を取り
				$h \coloneqq F^{-1}(b)$とおけば,$h$の線型性より
				\begin{align}
					&b(e_1,\cdots,e_i+e_i',\cdots,e_n) - b(e_1,\cdots,e_i,\cdots,e_n) - b(e_1,\cdots,e_i',\cdots,e_n) \\
					&\qquad = h(\defunc_{e_1,\cdots,e_i+e_i',\cdots,e_n} - \defunc_{e_1,\cdots,e_i,\cdots,e_n} - \defunc_{e_1,\cdots,e_i',\cdots,e_n}), \\
					&b(e_1,\cdots,\lambda e_i,\cdots,e_n) - \lambda b(e_1,\cdots,e_i,\cdots,e_n) \\
					&\qquad = h(\defunc_{e_1,\cdots,\lambda e_i,\cdots,e_n} - \lambda \defunc_{e_1,\cdots,e_i,\cdots,e_n})
				\end{align}
				が成り立ち,$b$の$n$重線型性により$h$は$\Lambda_0(\bigoplus_{i=1}^n E_i)$上で0である.
				従って
				\begin{align}
					T([b]) \coloneqq h(b),
					\quad (b \in \Lambda(\bigoplus_{i=1}^n E_i))
				\end{align}
				で定める$T$はwell-definedであり,$T \in \Hom{\bigotimes_{i=1}^n E_i}{V}$かつ
				\begin{align}	
					b(e_1,\cdots,e_n) = h(\defunc_{e_1,\cdots,e_n}) = (T \circ \otimes) (e_1,\cdots,e_n),
					\quad (\forall (e_1,\cdots,e_n) \in \bigoplus_{i=1}^n E_i)
				\end{align}
				が満たされ$\Phi_V$の全射性が得られる.
				
			\item[第五段]
				$(\otimes)_1,(\otimes)_2$の下で
				$\Hom{U_0}{\bigotimes_{i=1}^n E_i} \ni \tau \longmapsto \tau \circ \iota \in \Ln{\bigoplus_{i=1}^n E_i}{\bigotimes_{i=1}^n E_i}{n}$は全単射であるから,
				$\tau \circ \iota = \otimes$を満たす$\tau \in \Hom{U_0}{\bigotimes_{i=1}^n E_i}$がただ一つ存在する.
				同様にして$\iota = T \circ \otimes$を満たす
				$T \in \Hom{\bigotimes_{i=1}^n E_i}{U_0}$がただ一つ存在し,併せれば
				\begin{align}
					\otimes = \tau \circ \iota = (\tau \circ T) \circ \otimes,
					\quad \iota = T \circ \otimes = (T \circ \tau) \circ \iota
				\end{align}
				が成り立つ.$T \longmapsto T \circ \otimes,\ \tau \longmapsto \tau \circ \iota$
				が一対一であるから,$\tau \circ T,\ T \circ \tau$はそれぞれ恒等写像に一致して
				$T^{-1} = \tau$が従う.すなわち$T$は$\bigotimes_{i=1}^n E_i$から$U_0$への
				線型同型である.
				\QED
		\end{description}
	\end{prf}
	
	\begin{screen}
		\begin{thm}[スカラーとのテンソル積]\label{thm:tensor_product_with_scalar}
			$E$を$\K$-線型空間とするとき,
			$\K \otimes E$と$E$は
			$f(\alpha \otimes e) = \alpha e$を満たす
			線型写像$f:\K \otimes E \longmapsto E$により
			同型となる.同様に$E \otimes \K$と$E$は
			$g(e \otimes \alpha) = \alpha e$を満たす
			線型写像$g$により
			同型となる.
		\end{thm}
	\end{screen}
	
	\begin{prf}
		スカラ倍$\iota:(\alpha, e) \longmapsto \alpha e$は双線型である.
		また定理\ref{thm:universality_of_tensor_product}の
		$(\otimes)_1,(\otimes)_2$について,
		\begin{align}
			E = \Span{\Set{\alpha e}{\alpha \in \K,\ e \in E}}
		\end{align}
		より$(\otimes)_1$が従い,かつ
		任意の双線型写像$\delta:\K \times E \longrightarrow V$に対し
		\begin{align}
			\tau(e) \coloneqq \delta(1,e),
			\quad (\forall e \in E)
		\end{align}
		で線型写像$\tau:E \longrightarrow V$を定めれば,
		\begin{align}
			\tau \circ \iota (\alpha,e) 
			= \tau(\alpha e) 
			= \delta(1,\alpha e)
			= \alpha \delta(1,e)
			= \delta (\alpha ,e)
		\end{align}
		となり$(\otimes)_2$が満たされる.従って,定理\ref{thm:universality_of_tensor_product}より
		$f \circ \otimes = \iota$を満たす線型同型$f:\K \otimes E \longrightarrow E$が存在して
		\begin{align}
			f(\alpha \otimes e) = \iota(\alpha,e) = \alpha e,
			\quad (\forall \alpha \in \K,\ e \in E)
		\end{align}
		が成立する.
		\QED
	\end{prf}
	
	\begin{screen}
		\begin{dfn}[線型写像のテンソル積]
				$(E_i)_{i=1}^n$と$(F_i)_{i=1}^n$を$\K$-線型空間の族とする.
				$f_i:E_i \longrightarrow F_i\ (i=1,\cdots,n)$が線型写像であるとき,
				\begin{align}
					b: \bigoplus_{i=1}^n E_i \ni (e_1,\cdots,e_n)
					\longmapsto f_1(e_1)\otimes \cdots \otimes f_n(e_n)
					\in \bigotimes_{i=1}^n F_i
				\end{align}
				により定める$b$は$n$重線型であり,定理\ref{thm:universality_of_tensor_product}
				より$b = g \circ \otimes$を満たす
				$g \in \Hom{\bigotimes_{i=1}^{n} E_i}{\bigotimes_{i=1}^{n} F_i}$がただ一つ存在する.
				$g$を$f_1 \otimes \cdots \otimes f_n$と表記して線型写像のテンソル積と定義する.特に,
				\begin{align}
					f_1 \otimes \cdots \otimes f_n(e_1 \otimes \cdots \otimes e_n)
					= f_1(e_1)\otimes \cdots \otimes f_n(e_n),
					\quad (\forall (e_1,\cdots,e_n) \in \bigoplus_{i=1}^n E_i)
				\end{align}
				が成り立つ.
				%$F_i = \K\ (i=1,\cdots,n)$の場合は$\otimes$を$\K$の乗法と考える$(\bigotimes_{i=1}^n F_i = \K)$.
		\end{dfn}
	\end{screen}
	
	\begin{screen}
		\begin{thm}[零写像のテンソル積は零写像]
		\label{thm:tensor_product_contains_zero_mapping_is_zero}
			$\K$-線型空間の族$(E_i)_{i=1}^n$と$(F_i)_{i=1}^n$と
			線型写像$f_i:E_i \longrightarrow F_i\ (i=1,\cdots,n)$について,
			或る$f_i$が零写像なら
			$f_1 \otimes \cdots \otimes f_n = 0$となる.
		\end{thm}
	\end{screen}
	
	\begin{prf}
		$f_i = 0$とすると,定理\ref{thm:when_tensor_product_zero}より
		$f_1 \otimes \cdots \otimes f_n$は
		$\Set{e_1 \otimes \cdots \otimes e_n}{e_i \in E_i}$上で0となる.
		この空間は$\bigotimes_{i=1}^{n} E_i$を生成するから
		$f_1 \otimes \cdots \otimes f_n = 0$が従う.
		\QED
	\end{prf}
	
	\begin{screen}
		\begin{thm}[テンソル積の基底]
			$(E_i)_{i=1}^n$を$\K$-線型空間の族とし,$E_i$の基底を
			$\left\{ u^i_{\lambda_i} \right\}_{\lambda_i \in \Lambda_i}$
			とする$(i=1,\cdots,n)$.このとき$\left\{ u^1_{\lambda_1} \otimes 
			\cdots \otimes u^n_{\lambda_n} \right\}_{\lambda_1,\cdots,\lambda_n}$
			は$\bigotimes_{i=1}^n E_i$の基底となる.
		\end{thm}
	\end{screen}
	
	\begin{prf}\mbox{}
		\begin{description}
			\item[第一段]
				任意の$e_1 \otimes \cdots \otimes e_n \in \bigotimes_{i=1}^n E_i$は
				$\left\{ u^1_{\lambda_1} \otimes \cdots \otimes u^n_{\lambda_n} 
				\right\}_{\lambda_1,\cdots,\lambda_n}$
				の線型結合で表現されるから,式(\refeq{eq:thm_tensor_product_is_bilinear})より
				\begin{align}
					\bigotimes_{i=1}^n E_i = \Span{\Set{u^1_{\lambda_1} \otimes 
				\cdots \otimes u^n_{\lambda_n}}{\lambda_i \in \Lambda_i,\ i=1,\cdots,n}}
				\end{align}
				が成立する.
				
			\item[第二段]
				$\left\{ u^1_{\lambda_1} \otimes 
				\cdots \otimes u^n_{\lambda_n} \right\}_{\lambda_1,\cdots,\lambda_n}$
				の一次独立性を示す.
				$\left\{ u^i_{\lambda_i} \right\}_{\lambda_i \in \Lambda_i}$に対する
				双対基底を$\left\{ f^i_{\lambda_i} \right\}_{\lambda_i \in \Lambda_i}$
				と書けば,各$f^i_{\lambda_i}$は
				\begin{align}
					f^i_{\lambda_i}(u^i_\lambda) =
					\begin{cases}
						1, & (\lambda = \lambda_i), \\
						0, & (\lambda \neq \lambda_i),
					\end{cases}
					\quad \forall \lambda \in \Lambda_i
				\end{align}
				を満たし,双対基底により構成する写像のテンソル積
				$f^1_{\lambda_1} \otimes \cdots \otimes f^n_{\lambda_n}$について
				\begin{align}
					f^1_{\lambda_1} \otimes \cdots \otimes f^n_{\lambda_n}(u^1_{\nu_1} \otimes 
					\cdots \otimes u^n_{\nu_n}) = 
					\begin{cases}
						1, & (\nu_1,\cdots,\nu_n) = (\lambda_1,\cdots,\lambda_n), \\
						0, & (\nu_1,\cdots,\nu_n) \neq (\lambda_1,\cdots,\lambda_n),
					\end{cases}
					\quad \forall (\nu_1,\cdots,\nu_n) \in \prod_{i=1}^n \Lambda_i
				\end{align}
				が成立する.従って$u^1_{\lambda_1} \otimes \cdots \otimes u^n_{\lambda_n}$
				は全て零ではなく,かつ
				\begin{align}
					0 = \sum_{j=1}^k \alpha_j \left(u^1_{\lambda^{(j)}_1} \otimes 
					\cdots \otimes u^n_{\lambda^{(j)}_n} \right),
					\quad (\alpha_j \in \K,\ j=1,\cdots,k)
				\end{align}
				を満たすような任意の線型結合に対し(ただし$i \neq j$なら$(\lambda_1^{(i)},\cdots,\lambda_n^{(i)}) \neq (\lambda_1^{(j)},\cdots,\lambda_n^{(j)})$)
				\begin{align}
					\alpha_j = f^1_{\lambda^{(j)}_1} \otimes \cdots \otimes f^n_{\lambda^{(j)}_n}
					\biggl( \sum_{j=1}^k \alpha_j \left(u^1_{\lambda^{(j)}_1} \otimes 
					\cdots \otimes u^n_{\lambda^{(j)}_n} \right) \biggr)
					= 0,
					\quad (j=1,\cdots,k)
				\end{align}
				が従い$\left\{ u^1_{\lambda_1} \otimes 
				\cdots \otimes u^n_{\lambda_n} \right\}_{\lambda_1,\cdots,\lambda_n}$
				の一次独立性を得る.
				\QED
		\end{description}
	\end{prf}
	
	\begin{screen}
		\begin{thm}[結合律]
		\label{thm:associativity_of_tensor_products}
			$(E_i)_{i=1}^n$を$\K$-線型空間の族とし,
			$k \in \{ 1,\cdots,n-1 \}$を任意に取る.このとき,次の対応関係を満たす
			$F$は線型同型である:
			\begin{align}
				\begin{array}{ccc}
					F:\bigotimes_{i=1}^n E_i & \longrightarrow & \biggl( \bigotimes_{i=1}^k E_i \biggr) \bigotimes \biggl( \bigotimes_{i=k+1}^n E_i \biggr) \\
					\rotatebox{90}{$\in$} & & \rotatebox{90}{$\in$} \\
					e_1 \otimes \cdots \otimes e_n & \longmapsto & (e_1 \otimes \cdots \otimes e_k) \otimes (e_{k+1} \otimes \cdots \otimes e_n)
				\end{array}
			\end{align}
		\end{thm}
	\end{screen}
	
	\begin{prf}\mbox{}
		\begin{description}
			\item[第一段]
				$n$重線型写像$f:\bigoplus_{i=1}^n E_i \longrightarrow \biggl( \bigotimes_{i=1}^k E_i \biggr) \bigotimes \biggl( \bigotimes_{i=k+1}^n E_i \biggr)$を
				\begin{align}
					f(e_1,\cdots,e_n) = (e_1 \otimes \cdots \otimes e_k) \otimes 
					(e_{k+1} \otimes \cdots \otimes e_n),
					\quad (\forall (e_1,\cdots,e_n) \in \bigoplus_{i=1}^n E_i)
				\end{align}
				により定めれば,定理\ref{thm:universality_of_tensor_product}より
				\begin{align}
					F(e_1 \otimes \cdots \otimes e_n)
					= (e_1 \otimes \cdots \otimes e_k) \otimes 
					(e_{k+1} \otimes \cdots \otimes e_n),
					\quad (\forall (e_1,\cdots,e_n) \in \bigoplus_{i=1}^n E_i)
				\end{align}
				を満たす線型写像$F:\bigotimes_{i=1}^n E_i \longrightarrow \biggl( \bigotimes_{i=1}^k E_i \biggr) \bigotimes \biggl( \bigotimes_{i=k+1}^n E_i \biggr)$
				が存在する:
				\begin{align}
					\xymatrix{
						&\bigoplus_{i=1}^n E_i \ar[d]_-\otimes \ar[rd]^-f & \\
						&\bigotimes_{i=1}^n E_i \ar[r]^-F & \biggl( \bigotimes_{i=1}^k E_i \biggr) \bigotimes \biggl( \bigotimes_{i=k+1}^n E_i \biggr)
					}
				\end{align}
				以降は$F$の逆写像を構成し$F$が全単射であることを示す.
				
			\item[第二段]
				$u_{k+1} \in E_{k+1},\cdots,u_n \in E_n$を固定し
				\begin{align}
					\Phi_{u_{k+1},\cdots,u_n}(e_1,\cdots,e_n)
					\coloneqq e_1 \otimes \cdots \otimes e_k \otimes u_{k+1} \otimes \cdots \otimes u_n
				\end{align}
				によって$k$重線型$\Phi_{u_{k+1},\cdots,u_n}:\bigoplus_{i=1}^k E_i \longrightarrow 
				\bigotimes_{i=1}^n E_i$を定めれば,定理\ref{thm:universality_of_tensor_product}より
				\begin{align}
					G_{u_{k+1},\cdots,u_n}(e_1 \otimes \cdots \otimes e_k)
					= e_1 \otimes \cdots \otimes e_k \otimes u_{k+1} \otimes \cdots \otimes u_n
				\end{align}
				を満たす線型写像$G_{u_{k+1},\cdots,u_n}:\bigotimes_{i=1}^k E_i \longrightarrow 
				\bigotimes_{i=1}^n E_i$が存在する.
				\begin{align}
					\xymatrix{
						&\bigoplus_{i=1}^k E_i \ar[d]_-\otimes \ar[rd]^-{\Phi_{u_{k+1},\cdots,u_n}} & \\
						&\bigotimes_{i=1}^k E_i \ar[r]^-{G_{u_{k+1},\cdots,u_n}} & \bigotimes_{i=1}^n E_i
					}
				\end{align}
			
			\item[第三段]
				任意の$v \in \bigotimes_{i=1}^k E_i$に対して
				\begin{align}
					\Psi_v: \bigoplus_{i=k+1}^n E_i
					\ni (u_{k+1},\cdots,u_n)
					\longmapsto G_{u_{k+1},\cdots,u_n}(v)
				\end{align}
				を定めれば,$\Psi_v$は$n-k$重線型であるから,定理\ref{thm:universality_of_tensor_product}より
				\begin{align}
					H_v(u_{k+1} \otimes \cdots \otimes u_n)
					= \Psi_v(u_{k+1},\cdots,u_n)
				\end{align}
				を満たす線型写像$H_v:\bigotimes_{i=k+1}^n E_i \longrightarrow 
				\bigotimes_{i=1}^n E_i$が存在する.
				\begin{align}
					\xymatrix{
						&\bigoplus_{i=k+1}^n E_i \ar[d]_-\otimes \ar[rd]^-{\Psi_v} & \\
						&\bigotimes_{i=k+1}^n E_i \ar[r]^-{H_v} & \bigotimes_{i=1}^n E_i
					}
				\end{align}
				いま,$v \longmapsto \Psi_v$は線型であり,かつ
				$\Psi_v$と$H_v$は一対一対応であるから
				$v \longmapsto H_v$の線型性が従う.
				
			\item[第四段]
				$H_v$の線型性と$v \longmapsto H_v$の線型性より
				\begin{align}
					\Gamma:\biggl( \bigotimes_{i=1}^k E_i \biggr) \times \biggl( \bigotimes_{i=k+1}^n E_i \biggr) \ni (v,w) \longmapsto H_v(w)
				\end{align}
				により定める$\Gamma$は
				\begin{align}
					\Gamma(e_1 \otimes \cdots \otimes e_k, e_{k+1} \otimes \cdots \otimes e_n) 
					&= H_{e_1 \otimes \cdots \otimes e_k}\left(e_{k+1} \otimes \cdots \otimes e_n \right) \\
					&= \Psi_{e_1 \otimes \cdots \otimes e_k}\left(e_{k+1},\cdots,e_n \right) \\
					&= G_{e_{k+1},\cdots,e_n}\left(e_1 \otimes \cdots \otimes e_k \right) \\
					&= \Phi_{e_{k+1},\cdots,e_n}\left(e_1, \cdots, e_k \right) \\
					&= e_1 \otimes \cdots \otimes e_n
					\label{eq:thm_associativity_of_tensor_products}
				\end{align}
				を満たす双線型であり,定理\ref{thm:universality_of_tensor_product}より
				\begin{align}
					\xymatrix{
						&\biggl( \bigotimes_{i=1}^k E_i \biggr) \times \biggl( \bigotimes_{i=k+1}^n E_i \biggr) \ar[d]_-\otimes \ar[rd]^-{\Gamma} & \\
						&\biggl( \bigotimes_{i=1}^k E_i \biggr) \bigotimes \biggl( \bigotimes_{i=k+1}^n E_i \biggr) \ar[r]^-{G} & \bigotimes_{i=1}^n E_i
					}
				\end{align}
				を可換にする線型写像$G$が存在する.この$G$は$F$の逆写像である.実際,
				(\refeq{eq:thm_associativity_of_tensor_products})より
				\begin{align}
					F \circ G \left( (e_1 \otimes \cdots \otimes e_k) \otimes (e_{k+1} \otimes \cdots \otimes e_n) \right)
					&= F \left(\Gamma(e_1 \otimes \cdots \otimes e_k, e_{k+1} \otimes \cdots \otimes e_n) \right) \\
					&= F (e_1 \otimes \cdots \otimes e_n) \\
					&= (e_1 \otimes \cdots \otimes e_k) \otimes (e_{k+1} \otimes \cdots \otimes e_n)
				\end{align}
				かつ
				\begin{align}
					G \circ F \left( e_1 \otimes \cdots \otimes e_n \right)
					&= G \left( (e_1 \otimes \cdots \otimes e_k) \otimes (e_{k+1} \otimes \cdots \otimes e_n) \right) \\
					&= \Gamma\left(e_1 \otimes \cdots \otimes e_k,e_{k+1} \otimes \cdots \otimes e_n \right) \\
					&= e_1 \otimes \cdots \otimes e_n
				\end{align}
				が満たされ$F^{-1} = G$が従う.
				\QED
		\end{description}
	\end{prf}
\end{document}