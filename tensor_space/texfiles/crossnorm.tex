\section{クロスノルム}
	\begin{screen}
		\begin{dfn}[クロスノルム]
			$\K$-Banach空間$X,Y$のテンソル積$X \otimes Y$において
			\begin{align}
				&\alpha(x \otimes y) \leq \Norm{x}{X} \Norm{y}{Y}, && (x \otimes y \in X \otimes Y), \\
				&\Norm{x^* \otimes y^*}{(X \otimes Y, \alpha)^*} \leq \Norm{x^*}{X^*} \Norm{y^*}{Y^*},
				&& (x^* \in X^*, y^* \in Y^*)
			\end{align}
			を満たすようなノルム$\alpha:X \otimes Y \longrightarrow \R$をクロスノルム(cross norm)と呼ぶ.
			$(X \otimes Y, \alpha)$の完備化を$X \otimes_{\alpha} Y$と書く.
		\end{dfn}
	\end{screen}
	
	\begin{screen}
		\begin{thm}
			$X,Y$を$\K$-Banach空間とする.クロスノルム$\alpha:X \otimes Y \longrightarrow \R$に対し次が成り立つ:
		\end{thm}
	\end{screen}
	
	\begin{screen}
		\begin{dfn}[インジェクティブノルム]
			$\K$-Banach空間$X,Y$に対し
			\begin{align}
				\epsilon(v) \coloneqq
				\sup{\Norm{x^*}{X^*}\leq 1,\Norm{y^*}{Y^*} \leq 1}{\left| x^* \otimes y^* (v) \right|},
				\quad (v \in X \otimes Y)
			\end{align}
			により$\epsilon$を定めれば,これは$X \otimes Y$においてノルムとなる.
			この$\epsilon$をインジェクティブノルム(injective norm)と呼ぶ.
			また$(X \otimes Y, \epsilon)$の完備化を$X \otimes_{\epsilon} Y$と書く.
		\end{dfn}
	\end{screen}
	
	\begin{screen}
		\begin{thm}[インジェクティブノルムは最小のクロスノルム]
			$\K$-Banach空間$X,Y$のテンソル積$X \otimes Y$において,
			インジェクティブノルムは最小のクロスノルムである.
		\end{thm}
	\end{screen}
	
	\begin{prf}
		始めに$\epsilon$がクロスノルムであることを示す.
		先ずHahn-Banachの定理より
		\begin{align}
			\epsilon(x \otimes y) 
			&= \sup{\Norm{x^*}{X^*}\leq 1,\Norm{y^*}{Y^*} \leq 1}{\left| x^* \otimes y^* (x \otimes y) \right|} \\
			&= \sup{\Norm{x^*}{X^*} \leq 1}{\left| x^*(x) \right|}
			\sup{\Norm{y^*}{Y^*} \leq 1}{\left| y^*(y) \right|} \\
			&= \Norm{x}{X} \Norm{y}{Y},
			\quad (\forall (x,y) \in X \times Y)
		\end{align}
		が成り立つ.また0でない$x^* \in X^*,\ y^* \in Y^*$に対しては
		\begin{align}
			\left| x^* \otimes y^* (v) \right|
			\leq \Norm{x^*}{X^*} \Norm{y^*}{Y^*} \left( \frac{x^*}{\Norm{x^*}{X^*}} \otimes \frac{y^*}{\Norm{y^*}{Y^*}} \right) (v)
			\leq \Norm{x^*}{X^*} \Norm{y^*}{Y^*} \epsilon(v)
		\end{align}
		が成立し,$x^*=0$或は$y^*=0$のときは
		定理\ref{thm:tensor_product_contains_zero_mapping_is_zero}より
		$x^* \otimes y^* = 0$が満たされ,
		\begin{align}
			\Norm{x^* \otimes y^*}{(X \otimes Y,\epsilon)} \leq \Norm{x^*}{X^*} \Norm{y^*}{Y^*}
		\end{align}
		を得る.いま,$\alpha$を任意のクロスノルムとすれば
		\begin{align}
			\left| x^* \otimes y^* (v) \right| \leq \Norm{x^*}{X^*} \Norm{y^*}{Y^*} \alpha(v),
			\quad (\forall v \in X \otimes Y)
		\end{align}
		が成り立つから,特に$\Norm{x^*}{X^*} \leq 1,\ \Norm{y^*}{Y^*} \leq 1$のsupを取れば
		\begin{align}
			\epsilon(v) \leq \alpha(v),
			\quad (\forall v \in X \otimes Y)
		\end{align}
		が従い$\epsilon$の最小性が出る.
		\QED
	\end{prf}