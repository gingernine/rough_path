\section{クロスノルム}
	$\K = \R$または$\K=\C$と考える.以下では$n (\geq 2)$個のBanach空間で構成する
	テンソル積におけるクロスノルムを考察する.
	\begin{screen}
		\begin{dfn}[クロスノルム]\label{def:cross_norm}
			$\K$-Banach空間の族$(X_i)_{i=1}^n$のテンソル積$\bigotimes_{i=1}^n X_i$において
			\begin{align}
				\alpha(x_1 \otimes \cdots \otimes x_n) &\leq \Norm{x_1}{X_1} \Norm{x_2}{X_2} \cdots \Norm{x_n}{X_n}, && (x_i \in X_i), \label{eq:cross_norm_def_1}\\
				\sup{\substack{v \in \bigotimes_{i=1}^n X_i \\ v \neq 0}}{\left| x_1^* \otimes \cdots \otimes x_n^* (v) \right|} &\leq \Norm{x_1^*}{X_1^*} \Norm{x_2^*}{X_2^*} \cdots \Norm{x_n^*}{X_n^*}\alpha(v),
				&& (x_i^* \in X_i^*) \label{eq:cross_norm_def_2}
			\end{align}
			を満たすようなノルム$\alpha:\bigotimes_{i=1}^n X_i \longrightarrow [0,\infty)$を
			クロスノルム(cross norm)と呼ぶ.
		\end{dfn}
	\end{screen}
	
	\begin{screen}
		\begin{thm}
			$\K$-Banach空間の族$(X_i)_{i=1}^{n}$に対するテンソル積
			$\bigotimes_{i=1}^n X_i$上のクロスノルム$\alpha$は
			次を満たす:
			\begin{align}
				&\alpha(x_1 \otimes \cdots \otimes x_n) 
					= \Norm{x_1}{X_1} \cdots \Norm{x_n}{X_n}, && (x_i \in X_i,\ i=1,\cdots,n), \\
				&\Norm{x_1^* \otimes \cdots \otimes x_n^*}{(\bigotimes_{i=1}^n X_i, \alpha)^*} 
					= \Norm{x_1^*}{X_1^*} \cdots \Norm{x_n^*}{X_n^*},
				&& (x_i^* \in X_i^*,\ i=1,\cdots,n).
			\end{align}
		\end{thm}
	\end{screen}
	
	\begin{prf}
		先ず,Hahn-Banachの定理と式(\refeq{eq:cross_norm_def_2})より
		\begin{align}
			\Norm{x_1}{X_1} \cdots \Norm{x_n}{X_n} 
			&= \sup{\Norm{x_1^*}{X_1^*} \leq 1}{\left| \inprod<x_1,x_1^*> \right|} 
				\cdots \sup{\Norm{x_n^*}{X_n^*} \leq 1}{\left| \inprod<x_n,x_n^*> \right|} \\
			&= \sup{\substack{\Norm{x_i^*}{X_i^*} \leq 1 \\ i=1,\cdots,n}}{\left| x_1^* \otimes \cdots \otimes x_n^* (x_1 \otimes \cdots \otimes x_n) \right|} \\
			&\leq \sup{\substack{\Norm{x_i^*}{X_i^*} \leq 1 \\ i=1,\cdots,n}}{\Norm{x_1^*}{X_1^*} \cdots \Norm{x_n^*}{X_n^*}}\alpha(x_1 \otimes \cdots \otimes x_n) \\
			&= \alpha(x_1 \otimes \cdots \otimes x_n)
		\end{align}
		が成り立ち定理の主張の第一式を得る.またこの結果より
		\begin{align}
			\Norm{x_1^*}{X_1^*} \cdots \Norm{x_n^*}{X_n^*} 
			&= \sup{\Norm{x_1}{X_1} \leq 1}{\left| \inprod<x_1,x_1^*> \right|}
				\cdots \sup{\Norm{x_n}{X_n} \leq 1}{\left| \inprod<x_n,x_n^*> \right|} \\
			&= \sup{\substack{\Norm{x_i}{X_i} \leq 1 \\ i=1,\cdots,n}}{\left| x_1^* \otimes \cdots \otimes x_n^* (x_1 \otimes \cdots \otimes x_n) \right|} \\
			&\leq \sup{\alpha(x_1 \otimes \cdots \otimes x_n) \leq 1}{\left| x_1^* \otimes \cdots \otimes x_n^* (x_1 \otimes \cdots \otimes x_n) \right|} \\
			&\leq \sup{\alpha(v) \leq 1}{\left| x_1^* \otimes \cdots \otimes x_n^* (v) \right|} \\
			&= \Norm{x_1^* \otimes \cdots \otimes x_n^*}{(\bigotimes_{i=1}^n X_i, \alpha)^*}
		\end{align}
		が成立し主張の第二式も得られる.
		\QED
	\end{prf}
	
	以下,実際クロスノルムが存在することを示す.
	\begin{screen}
		\begin{dfn}[インジェクティブノルム]
			$\K$-Banach空間の族$(X_i)_{i=1}^n$に対し
			\begin{align}
				\epsilon(v) \coloneqq
				\sup{\substack{\Norm{x_i^*}{X_i^*}\leq 1 \\ i=1,\cdots,n}}{\left| x_1^* \otimes \cdots \otimes x_n^* (v) \right|},
				\quad (v \in \bigotimes_{i=1}^n X_i)
			\end{align}
			により定める$\epsilon$をインジェクティブノルム(injective norm)と呼ぶ.
		\end{dfn}
	\end{screen}
	
	\begin{screen}
		\begin{thm}[インジェクティブノルムは最小のクロスノルム]
		\label{thm:injective_norm_is_the_minimum_cross_norm}
			$\K$-Banach空間の族$(X_i)_{i=1}^n$のテンソル積$\bigotimes_{i=1}^{n} X_i$において,
			インジェクティブノルムは最小のクロスノルムである.
		\end{thm}
	\end{screen}
	
	\begin{prf}\mbox{}
		\begin{description}
			\item[第一段]
				$\epsilon$が$\bigotimes_{i=1}^{n} X_i$上のノルムであることを示す.
				劣加法性と同次性は$x_1^* \otimes \cdots \otimes x_n^*$の線型性より従う.
				$v = 0 \Leftrightarrow \epsilon(v) = 0$については,
				$v = 0$なら任意の$x_1^* \otimes \cdots \otimes x_n^*$について
				$x_1^* \otimes \cdots \otimes x_n^* (v) = 0$
				が成り立ち$\epsilon(v) = 0$が出る.
				逆に$v \neq 0$とするとき,定理\ref{thm:tensor_product_is_bilinear}より
				\begin{align}
					v = \sum_{j=1}^{m} x^j_1 \otimes \cdots \otimes x^j_n,
					\quad (x^j_i \in X_i,\ j=1,\cdots,m,\ i=1,\cdots,n)
				\end{align}
				と表現できるが,定理\ref{thm:when_tensor_product_zero}より
				$x^1_i \neq 0\ (i=1,\cdots,n)$と仮定できる.
				$x^1_1$について,
				もし全ての$2 \leq j \leq m$に対し$x^j_1 = x^1_1$が満たされているなら,
				$\hat{x}_1^* \in X_1^*$を
				\begin{align}
					\inprod<x^1_1, \hat{x}_1^*> = \Norm{x^1_1}{X_1},
					\quad \Norm{\hat{x}_1^*}{X^*_1} = 1
				\end{align}
				を満たすように選ぶ(Hahn-Banachの定理).$x^j_1 \neq x^1_1$を満たす$j$がある場合,
				\begin{align}
					L_1 \coloneqq \Span{\Set{x^j_1}{2 \leq j \leq m,\ x^1_1 \neq x^j_1}}
				\end{align}
				により閉部分空間を定めれば$x^1_1$と$L_1$との距離$d_1$は正であり,Hahn-Banachの定理より
				\begin{align}
					\inprod<x_1,\hat{x}_1^*> = 0\ (\forall x_1 \in L_1),
					\quad \inprod<x^1_1,\hat{x}_1^*> = d_1 > 0,
					\quad \Norm{\hat{x}_1^*}{X_1^*} = 1
				\end{align}
				を満たす$\hat{x}_1^* \in X_1^*$を取ることができる.
				同様に$\hat{x}_i^* \in X_i^*\ (i=2,\cdots,n)$を選べば
				\begin{align}
					\hat{x}_1^* \otimes \cdots \otimes \hat{x}_n^*(x^j_1 \otimes \cdots \otimes x^j_n) =
					\begin{cases}
						\hat{x}_1^* \otimes \cdots \otimes \hat{x}_n^*(x^1_1 \otimes \cdots \otimes x^1_n), & (x^j_i = x^1_i,\ i=1,\cdots,n), \\
						0, & (\mbox{o.w.}),
					\end{cases}
				\end{align}
				$(j=2,\cdots,m)$が満たされるから
				\begin{align}
					0 < \hat{x}_1^* \otimes \cdots \otimes \hat{x}_n^*(x^1_1 \otimes \cdots \otimes x^1_n) 
					\leq \left| \hat{x}_1^* \otimes \cdots \otimes \hat{x}_n^* (v) \right|
					\leq \epsilon(v)
					\label{eq:thm_injective_norm_is_the_minimum_cross_norm_1}
				\end{align}
				が成立し,対偶により$\epsilon(v) = 0 \Rightarrow v = 0$が従う.
				
			\item[第二段]
				$\epsilon$がクロスノルムであることを示す.
				先ずHahn-Banachの定理より
				\begin{align}
					\epsilon(x_1 \otimes \cdots \otimes x_n) 
					&= \sup{\substack{\Norm{x_i^*}{X_i^*}\leq 1 \\ i=1,\cdots,n}}{\left| x_1^* \otimes \cdots \otimes x_n^* (x_1 \otimes \cdots \otimes x_n) \right|} \\
					&= \sup{\Norm{x_1^*}{X_1^*} \leq 1}{\left| \inprod<x_1,x_1^*> \right|}
					\cdots \sup{\Norm{x_n^*}{X_n^*} \leq 1}{\left| \inprod<x_n,x_n^*> \right|} \\
					&= \Norm{x_1}{X_1} \cdots \Norm{x_n}{X_n},
					\quad (\forall x_i \in X_i,\ i=1,\cdots,n)
				\end{align}
				が成り立つ.また0でない$x_i^* \in X_i^*,\ (i=1,\cdots,n)$に対しては
				\begin{align}
					\left| x_1^* \otimes \cdots \otimes x_n^* (v) \right|
					&\leq \Norm{x_1^*}{X_1^*} \cdots \Norm{x_n^*}{X_n^*} \left[ \frac{x_1^*}{\Norm{x_1^*}{X_1^*}} \otimes \cdots \otimes \frac{x_n^*}{\Norm{x_n^*}{X_n^*}} \right] (v) \\
					&\leq \Norm{x_1^*}{X_1^*} \cdots \Norm{x_n^*}{X_n^*} \epsilon(v)
				\end{align}
				が成立し,或る$i$で$x_i^*$が零写像のときは
				定理\ref{thm:tensor_product_contains_zero_mapping_is_zero}より
				$x_1^* \otimes \cdots \otimes x_n^* = 0$が満たされ,
				\begin{align}
					\Norm{x_1^* \otimes \cdots \otimes x_n^*}{(\bigotimes_{i=1}^n X_i,\epsilon)} \leq \Norm{x_1^*}{X_1^*} \cdots \Norm{x_n^*}{X_n^*}
				\end{align}
				を得る.
			
			\item[第三段]
				$\epsilon$が最小のクロスノルムであることを示す.$\alpha$を任意のクロスノルムとすれば
				\begin{align}
					\left| x_1^* \otimes \cdots \otimes x_n^* (v) \right| 
					\leq \Norm{x_1^*}{X_1^*} \cdots \Norm{x_n^*}{X_n^*} \alpha(v),
					\quad (\forall v \in \bigotimes_{i=1}^n X_i)
				\end{align}
				が成り立つから,特に$\Norm{x_i^*}{X_i^*} \leq 1,\ (i=1,\cdots,n)$の範囲でsupを取れば
				\begin{align}
					\epsilon(v) \leq \alpha(v),
					\quad (\forall v \in \bigotimes_{i=1}^n X_i)
				\end{align}
				が従い$\epsilon$の最小性が出る.
				\QED
		\end{description}
	\end{prf}
	
	\begin{screen}
		\begin{dfn}[プロジェクティブノルム]
			$\K$-Banach空間の族$(X_i)_{i=1}^n$に対し,定理
			\ref{thm:universality_of_tensor_product}により
			\begin{align}
				\begin{array}{ccc}
					\Phi:\Hom{\bigotimes_{i=1}^n X_i}{\K} & \longrightarrow & \Ln{\bigoplus_{i=1}^n X_i}{\K}{n} \\
					\rotatebox{90}{$\in$} & & \rotatebox{90}{$\in$} \\
					T & \longmapsto & T \circ \otimes
					\label{eq:dfn_projective_norm_isomorphism}
				\end{array}
			\end{align}
			により線型同型$\Phi$が定まる.これを用いて
			\begin{align}
				\pi(v) \coloneqq
				\sup{\substack{b \in \ContLn{\bigoplus_{i=1}^n X_i}{\K}{n} \\ \Norm{b}{\ContLn{\bigoplus_{i=1}^n X_i}{\K}{n}} \leq 1}}{\left| \Phi^{-1}(b)(v) \right|},
				\quad (v \in \bigotimes_{i=1}^n X_i)
			\end{align}
			により定める$\pi$をプロジェクティブノルム(projective norm)と呼ぶ.
		\end{dfn}
	\end{screen}
	
	\begin{screen}
		\begin{thm}[プロジェクティブノルムは最大のクロスノルム]
		\label{thm:projective_norm_is_maximum_cross_norm}
			$\K$-Banach空間の族$(X_i)_{i=1}^n$のテンソル積$\bigotimes_{i=1}^n X_i$上に
			プロジェクティブノルム$\pi$を導入する.このとき式(\refeq{eq:cross_norm_def_1})を満たす任意のセミノルム
			$p$に対し$p \leq \pi$が成立する.特に$\pi$は最大のクロスノルムである.
		\end{thm}
	\end{screen}
	
	\begin{prf}\mbox{}
		\begin{description}
			\item[第一段]
				$\pi$がノルムであることを示す.$v \neq 0$とすれば,
				定理\ref{thm:injective_norm_is_the_minimum_cross_norm}
				の証明と同様にして
				\begin{align}
					0 < \left| \hat{x}_1^* \otimes \cdots \otimes \hat{x}_n^* (v) \right|,
					\quad \Norm{\hat{x}^*_i}{X^*_i} = 1,
					\quad (i=1,\cdots,n)
				\end{align}
				を満たす$\hat{x}^*_i \in X^*_i\ (i=1,\cdots,n)$
				が存在する.
				\begin{align}
					b(x_1,\cdots,x_n) 
					\coloneqq \inprod<x_1,\hat{x}_1^*> \cdots \inprod<x_n,\hat{x}_n^*>,
					\quad (x_i \in X_i,\ i=1,\cdots,n)
				\end{align}
				により$n$重線型写像$b$を定めれば,$\Norm{b}{\ContLn{\bigoplus_{i=1}^n X_i}{\K}{n}}
				\leq \Norm{x_1^*}{X_1^*} \cdots \Norm{x_n^*}{X_n^*} = 1$かつ
				\begin{align}
					0 < \left| \hat{x}_1^* \otimes \cdots \otimes \hat{x}_n^* (v) \right| 
					= |\Phi^{-1}(b)(v)| \leq \pi(v)
				\end{align}
				が成立する.$\pi(0) = 0$と
				劣加法性及び同次性は$\Phi^{-1}(b)$の線型性より従う.
			
			\item[第二段]
				$\pi$がクロスノルムであることを示す.
				先ず,任意の$x_i \in X_i,\ (i=1,\cdots,n)$に対して
				\begin{align}
					\pi(x_1 \otimes \cdots \otimes x_n) 
					&= \sup{\substack{b \in \ContLn{\bigoplus_{i=1}^n X_i}{\K}{n} \\ \Norm{b}{\ContLn{\bigoplus_{i=1}^n X_i}{\K}{n}} \leq 1}}{\left| \Phi^{-1}(b)(x_1 \otimes \cdots \otimes x_n) \right|} \\
					&\leq \sup{\substack{b \in \ContLn{\bigoplus_{i=1}^n X_i}{\K}{n} \\ \Norm{b}{\ContLn{\bigoplus_{i=1}^n X_i}{\K}{n}} \leq 1}}{\Norm{b}{\ContLn{\bigoplus_{i=1}^n X_i}{\K}{n}}\Norm{x_1}{X_1} \cdots \Norm{x_n}{X_n}} \\
					&= \Norm{x_1}{X_1} \cdots \Norm{x_n}{X_n}
				\end{align}
				が成立する.また0でない$x_i^* \in X_i^*,\ (i=1,\cdots,n)$に対し
				\begin{align}
					b(x_1,\cdots,x_n) 
					\coloneqq \frac{x_1^*}{\Norm{x_1^*}{X_1^*}}(x_1) \cdots \frac{x_n^*}{\Norm{x_n^*}{X_n^*}}(x_n),
					\quad (x_i \in X_i,\ i=1,\cdots,n)
				\end{align}
				により$\Norm{b}{\ContLn{\bigoplus_{i=1}^n X_i}{\K}{n}} \leq 1$
				を満たす有界$n$重線型$b$を定めれば,
				$\pi$の定義より
				\begin{align}
					\left| \Phi^{-1}(b)(v) \right| \leq \pi(v),
					\quad (\forall v \in \bigotimes_{i=1}^n X_i)
				\end{align}
				が成り立つ.一方で写像のテンソル積の定義より
				\begin{align}
					\Phi^{-1}(b) 
					= \frac{x_1^*}{\Norm{x_1^*}{X_1^*}} 
						\otimes \cdots \otimes \frac{x_n^*}{\Norm{x_n^*}{X_n^*}}
					= \frac{1}{\Norm{x_1^*}{X_1^*} \cdots \Norm{x_n^*}{X_n^*}} 
						x_1^* \otimes \cdots \otimes x_n^*
				\end{align}
				が満たされるから
				\begin{align}
					\left| x_1^* \otimes \cdots \otimes x_n^*(v) \right| 
						\leq \Norm{x_1^*}{X_1^*} \cdots \Norm{x_n^*}{X_n^*} \pi(v),
					\quad (\forall v \in \bigotimes_{i=1}^n X_i)
				\end{align}
				が従う.定理\ref{thm:tensor_product_contains_zero_mapping_is_zero}より
				上式は$x_i^* = 0$の場合も込めて成立するから
				\begin{align}
					\Norm{x_1^* \otimes \cdots \otimes x_n^*}{(\bigotimes_{i=1}^n X_i,\pi)^*} 
					\leq \Norm{x_1^*}{X_1^*} \cdots \Norm{x_n^*}{X_n^*}
				\end{align}
				が得られる.
				
			\item[第三段]
				$p$を(\refeq{eq:cross_norm_def_1})を満たすセミノルムとし,
				$v \in \bigotimes_{i=1}^n X_i$を任意に取れば
				\begin{align}
					p(v) = \phi_v(v),
					\quad |\phi_v(u)| \leq p(u)
					\quad (\forall u \in \bigotimes_{i=1}^n X_i)
				\end{align}
				を満たす$\phi_v \in (\bigotimes_{i=1}^n X_i,\pi)^*$が存在する(Hahn-Banachの定理).
				\begin{align}
					\left| (\phi_v \circ \otimes)(x_1,\cdots,x_n) \right|
					&= \left|\phi_v(x_1 \otimes \cdots \otimes x_n)\right| \\ 
					&\leq p(x_1 \otimes \cdots \otimes x_n) \\
					&\leq \Norm{x_1}{X_1} \cdots \Norm{x_n}{X_n},
					\quad (\forall x_i \in X_i,\ i=1,\cdots,n)
				\end{align}
				が成り立つから$\Norm{\phi_v \circ \otimes}{\ContLn{\bigoplus_{i=1}^n X_i}{\K}{n}} \leq 1$が従い,
				$\pi$の定義より
				\begin{align}
					p(v) = \phi_v(v) = \Phi^{-1}(\phi_v \circ \otimes)(v)
					\leq \pi(v)
				\end{align}
				が得られる.
				\QED
		\end{description}
	\end{prf}
	
	\begin{screen}
		\begin{thm}[プロジェクティブノルムの表現]\label{thm:expression_of_projective_norm}
			$\K$-Banach空間の族$(X_i)_{i=1}^n$のテンソル積$\bigotimes_{i=1}^n X_i$
			にプロジェクトノルム$\pi$を導入する.このとき次が成り立つ:
			\begin{align}
				\pi(v) = \inf{}{\Set{\sum_{i=1}^n \Norm{x_1}{X_1} \cdots \Norm{x_n}{X_n}}{
					v = \sum_{j=1}^m x^j_1 \otimes \cdots \otimes x^j_n}}.
			\end{align}
		\end{thm}
	\end{screen}
	
	\begin{prf}\mbox{}
		\begin{description}
			\item[第一段]
				$x_1 \otimes \cdots \otimes x_n$上のセミノルム$\lambda$を次で定める:
				\begin{align}
					\lambda(v) \coloneqq \inf{}{\Set{\sum_{i=1}^n \Norm{x_1}{X_1} \cdots \Norm{x_n}{X_n}}{
					v = \sum_{j=1}^m x^j_1 \otimes \cdots \otimes x^j_n}},
					\quad (\forall v \in \bigotimes_{i=1}^n X_i).
				\end{align}
				このとき$\lambda$が式(\refeq{eq:cross_norm_def_1})かつ
				$\lambda \geq \pi$を満たせば,
				定理\ref{thm:projective_norm_is_maximum_cross_norm}より$\lambda = \pi$が従う.
				
			\item[第二段]
				$\lambda$がセミノルムであることを示す.実際,任意に
				$u,v \in \bigotimes_{i=1}^n X_i$を取り,
				\begin{align}
					u = \sum_{j=1}^{m} x^j_1 \otimes \cdots \otimes x^j_n,
					\quad v = \sum_{k=1}^r a^k_1 \otimes \cdots \otimes a^k_n
				\end{align}
				を一つの表現とすれば,$\lambda$の定め方より
				\begin{align}
					\lambda(u+v) \leq \sum_{j=1}^{m} x^j_1 \otimes \cdots \otimes x^j_n 
						+ \sum_{k=1}^r a^k_1 \otimes \cdots \otimes a^k_n
				\end{align}
				が成り立つ.右辺を移項して
				\begin{align}
					\lambda(u+v) - \sum_{k=1}^r a^k_1 \otimes \cdots \otimes a^k_n 
					\leq \lambda(u) 
					\leq \sum_{j=1}^{m} x^j_1 \otimes \cdots \otimes x^j_n
				\end{align}
				かつ
				\begin{align}
					\lambda(u+v) - \lambda(u) \leq \lambda(v) \leq \sum_{k=1}^r a^k_1 \otimes \cdots \otimes a^k_n 
				\end{align}
				が従い$\lambda$の劣加法性を得る.
				また任意の$0 \neq \alpha \in \K,\ v \in \bigotimes_{i=1}^n X_i$に対し
				\begin{align}
					v = \sum_{j=1}^{m} x^j_1 \otimes \cdots \otimes x^j_n
				\end{align}
				を一つの分割とすれば
				\begin{align}
					\alpha v = \sum_{j=1}^{m} \left( \alpha x^j_1 \right) \otimes \cdots \otimes x^j_n
				\end{align}
				は$\alpha v$の一つの分割となるから
				\begin{align}
					\lambda(\alpha v) \leq \sum_{j=1}^m \Norm{\alpha x^j_1}{X_1} \cdots \Norm{x^j_n}{X_n}
					= |\alpha| \sum_{j=1}^m \Norm{x^j_1}{X_1} \cdots \Norm{x^j_n}{X_n}
				\end{align}
				が成立し,$v$の分割について下限を取れば$\lambda(\alpha v) \leq |\alpha| \lambda(v)$
				が従う.逆に
				\begin{align}
					\alpha v = \sum_{k=1}^r a^k_1 \otimes \cdots \otimes a^k_n
				\end{align}
				に対しては
				\begin{align}
					\lambda(v) \leq \sum_{k=1}^r \Norm{\frac{1}{\alpha} a^k_1}{X_1} \cdots \Norm{a^k_n}{X_n}
					= \frac{1}{|\alpha|} \sum_{k=1}^r \Norm{a^k_1}{X_1} \cdots \Norm{a^k_n}{X_n}
				\end{align}
				が成り立ち$|\alpha| \lambda(v) \leq \lambda(\alpha v)$
				が従う.$v=0$なら$v = 0 \otimes \cdots \otimes 0$より$\lambda(v) = 0$が満たされ
				\begin{align}
					\lambda(\alpha v) = |\alpha| \lambda(v),
					\quad (\forall \alpha \in \K,\ v \in \bigotimes_{i=1}^n X_i)
				\end{align}
				が得られる.
				
			\item[第三段]
				$\lambda$が式(\refeq{eq:cross_norm_def_1})を満たすことを示す.実際$\lambda$の定め方より
				\begin{align}
					\lambda(x_1 \otimes \cdots \otimes x_n) 
					\leq \Norm{x_1}{X_1} \cdots \Norm{x_n}{X_n},
					\quad (\forall x_i \in X_i,\ i=1,\cdots,n)
				\end{align}
				が成り立つ.
				
			\item[第四段]
				$\lambda \geq \pi$を示す.いま,任意に$v \in \bigotimes_{i=1}^n X_i$を取り,
				次の分割を持つとする:
				\begin{align}
					v = \sum_{j=1}^{m} x^j_1 \otimes \cdots \otimes x^j_n.
				\end{align}
				$\Norm{b}{\ContLn{\bigoplus_{i=1}^n X_i}{\K}{n}} \leq 1$を満たす
				$b \in \ContLn{\bigoplus_{i=1}^n X_i}{\K}{n}$と
				式(\refeq{eq:dfn_projective_norm_isomorphism})の
				$\Phi$に対し
				\begin{align}
					&\left| \Phi^{-1}(b)(v) \right|
					\leq \sum_{j=1}^{m} \left| \Phi^{-1}(b)(x^j_1 \otimes \cdots \otimes x^j_n) \right| \\
					&\qquad = \sum_{j=1}^{m} \left| b(x^j_1,\cdots,x^j_n) \right|
					\leq \sum_{j=1}^{m} \Norm{x^j_1}{X_1} \cdots \Norm{x^j_n}{X_n}
				\end{align}
				が成り立つから,$b$に無関係に
				\begin{align}
					\left| \Phi^{-1}(b)(v) \right| \leq \lambda(v)
				\end{align}
				が満たされ
				\begin{align}
					\pi(v) = \sup{\substack{b \in \ContLn{\bigoplus_{i=1}^n X_i}{\K}{n} \\ \Norm{b}{\ContLn{\bigoplus_{i=1}^n X_i}{\K}{n}} \leq 1}}{\left| \Phi^{-1}(b)(v) \right|}
					\leq \lambda(v)
				\end{align}
				が従う.
				\QED
		\end{description}
	\end{prf}
	
	\begin{screen}
		\begin{thm}
			$\bigotimes_{i=1}^n X_i$を$\K$-Banach空間の族$(X_i)_{i=1}^n$のテンソル積とする.このとき
			$\bigotimes_{i=1}^n X_i$上の任意のノルム$\alpha$に対し次が成立する:
			\begin{align}
				\mbox{$\alpha$がクロスノルム}
				\quad \Leftrightarrow \quad 
				\epsilon \leq \alpha \leq \pi.
			\end{align}
		\end{thm}
	\end{screen}
	
	\begin{prf}
		$(\Rightarrow)$はすでに示したから
		$(\Leftarrow)$を示す.実際,任意の$x_i \in X_i,\ (i=1,\cdots,n)$に対して
		\begin{align}
			\alpha(x_1 \otimes \cdots \otimes x_n) 
			\leq \pi(x_1 \otimes \cdots \otimes x_n) 
			\leq \Norm{x_1}{X_1} \cdots \Norm{x_n}{X_n}
		\end{align}
		が成立し,また任意の$x_i^* \in X_i^*,\ (i=1,\cdots,n)$に対して
		\begin{align}
			\left| x_1^* \otimes \cdots \otimes x_n^*(v) \right|
			&\leq \Norm{x_1^* \otimes \cdots \otimes x_n^*}{(\bigotimes_{i=1}^n X_i, \epsilon)^*} \epsilon(v) \\
			&\leq \Norm{x_1^*}{X_1^*} \cdots \Norm{x_n^*}{X_n^*} \alpha(v),
			\quad (\forall v \in \bigotimes_{i=1}^n X_i)
		\end{align}
		が満たされ$\Norm{x_1^* \otimes \cdots \otimes x_n^*}{(\bigotimes_{i=1}^n X_i, \alpha)^*} 
		\leq \Norm{x_1^*}{X_1^*} \cdots \Norm{x_n^*}{X_n^*}$が得られる.
		\QED
	\end{prf}