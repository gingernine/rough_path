\section{クロスノルム}
	$\K = \R$または$\K=\C$と考える.
	\begin{screen}
		\begin{dfn}[クロスノルム]
			$\K$-Banach空間$X,Y$のテンソル積$X \otimes Y$において
			\begin{align}
				\alpha(x \otimes y) &\leq \Norm{x}{X} \Norm{y}{Y}, && (x \otimes y \in X \otimes Y), \label{eq:cross_norm_def_1}\\
				\sup{\substack{v \in X \otimes Y \\ v \neq 0}}{\left| x^* \otimes y^* (v) \right|} &\leq \Norm{x^*}{X^*} \Norm{y^*}{Y^*}\alpha(v),
				&& (x^* \in X^*, y^* \in Y^*)
			\end{align}
			を満たすようなノルム$\alpha:X \otimes Y \longrightarrow [0,\infty)$を
			クロスノルム(cross norm)と呼ぶ.
		\end{dfn}
	\end{screen}
	
	\begin{screen}
		\begin{thm}
			$\K$-Banach空間のテンソル積$X \otimes Y$上のクロスノルム$\alpha$は
			次を満たす:
			\begin{align}
				&\alpha(x \otimes y) = \Norm{x}{X} \Norm{y}{Y}, && (x \otimes y \in X \otimes Y), \\
				&\Norm{x^* \otimes y^*}{(X \otimes Y, \alpha)^*} = \Norm{x^*}{X^*} \Norm{y^*}{Y^*},
				&& (x^* \in X^*, y^* \in Y^*).
			\end{align}
		\end{thm}
	\end{screen}
	
	\begin{prf}
		先ず,Hahn-Banachの定理より
		\begin{align}
			\Norm{x}{X} \Norm{y}{Y} 
			&= \sup{\Norm{x^*}{X^*} \leq 1}{\left| \inprod<x,x^*> \right|} 
				\sup{\Norm{y^*}{Y^*} \leq 1}{\left| \inprod<y,y^*> \right|} \\
			&= \sup{\Norm{x^*}{X^*} \leq 1,\Norm{y^*}{Y^*} \leq 1}{\left| x^* \otimes y^* (x \otimes y) \right|} \\
			&\leq \sup{\Norm{x^*}{X^*} \leq 1,\Norm{y^*}{Y^*} \leq 1}{\Norm{x^*}{X^*} \Norm{y^*}{Y^*}}\alpha(x \otimes y) \\
			&= \alpha(x \otimes y)
		\end{align}
		が成り立ち$\alpha(x \otimes y) = \Norm{x}{X} \Norm{y}{Y}$を得る.
		また$\alpha(x \otimes y) \leq \Norm{x}{X} \Norm{y}{Y}$であるから
		\begin{align}
			\Norm{x^*}{X^*} \Norm{y^*}{Y^*} 
			&= \sup{\Norm{x}{X} \leq 1}{\left| \inprod<x,x^*> \right|}
				\sup{\Norm{y}{Y} \leq 1}{\left| \inprod<y,y^*> \right|} \\
			&= \sup{\Norm{x}{X} \leq 1,\Norm{y}{Y} \leq 1}{\left| x^* \otimes y^* (x \otimes y) \right|} \\
			&\leq \sup{\alpha(x \otimes y) \leq 1}{\left| x^* \otimes y^* (x \otimes y) \right|} \\
			&\leq \sup{\alpha(v) \leq 1}{\left| x^* \otimes y^* (v) \right|} \\
			&= \Norm{x^* \otimes y^*}{(X \otimes Y, \alpha)^*}
		\end{align}
		が成立し$\Norm{x^* \otimes y^*}{(X \otimes Y, \alpha)^*} = \Norm{x^*}{X^*} \Norm{y^*}{Y^*}$
		が出る.
		\QED
	\end{prf}
	
	$(X \otimes Y, \alpha)$の完備化を$X \hat{\otimes}_{\alpha} Y$と書く.
	以下,実際クロスノルムが存在することを示す.
	\begin{screen}
		\begin{dfn}[インジェクティブノルム]
			$\K$-Banach空間$X,Y$に対し
			\begin{align}
				\epsilon(v) \coloneqq
				\sup{\Norm{x^*}{X^*}\leq 1,\Norm{y^*}{Y^*} \leq 1}{\left| x^* \otimes y^* (v) \right|},
				\quad (v \in X \otimes Y)
			\end{align}
			により定める$\epsilon$をインジェクティブノルム(injective norm)と呼ぶ.
		\end{dfn}
	\end{screen}
	
	\begin{screen}
		\begin{thm}[インジェクティブノルムは最小のクロスノルム]
		\label{thm:injective_norm_is_the_minimum_cross_norm}
			$\K$-Banach空間$X,Y$のテンソル積$X \otimes Y$において,
			インジェクティブノルムは最小のクロスノルムである.
		\end{thm}
	\end{screen}
	
	\begin{prf}\mbox{}
		\begin{description}
			\item[第一段]
				$\epsilon$が$X \otimes Y$上のノルムであることを示す.実際,
				\begin{align}
					\left| x^* \otimes y^* (u+v) \right|
					\leq \left| x^* \otimes y^* (u) \right|
					+ \left| x^* \otimes y^* (v) \right|,
					\quad (u,v \in X \otimes Y)
				\end{align}
				より$\epsilon(u + v) \leq \epsilon(u) + \epsilon(v)$が従い,また
				スカラー$\alpha$に対し
				\begin{align}
					\epsilon(\alpha v)
					= \sup{\Norm{x^*}{X^*}\leq 1,\Norm{y^*}{Y^*} \leq 1}{\left| x^* \otimes y^* (\alpha v) \right|}
					= |\alpha| \sup{\Norm{x^*}{X^*}\leq 1,\Norm{y^*}{Y^*} \leq 1}{\left| x^* \otimes y^* (v) \right|}
					= |\alpha| \epsilon(v)
				\end{align}
				も成立する.次に$v = 0 \Leftrightarrow \epsilon(v) = 0$を示す.
				$v = 0$ならば任意の$x^* \otimes y^*$に対し$x^* \otimes y^* (v) = 0$
				が成り立ち$\epsilon(v) = 0$が出る.
				逆に$v \neq 0$とする.定理\ref{thm:tensor_product_is_bilinear}より
				\begin{align}
					v = \sum_{i=1}^{n} x_i \otimes y_i,
					\quad (x_i \in X,\ y_i \in Y,\  i=1,\cdots,n)
				\end{align}
				と表現できるが,このとき$(x_i \otimes y_i)_{i=1}^{n}$を構成し直して
				\begin{align}
					v = \sum_{k=1}^{\ell} x_{i_k} \otimes \tilde{y}_{i_k},
					\quad (x_{i_k} \neq x_{i_j}\ (k \neq j))
					\label{eq:thm_injective_norm_is_the_minimum_cross_norm_1}
				\end{align}
				と書ける.実際,$i_1 \coloneqq 1$として
				\begin{align}
					\tilde{y}_{i_1} \coloneqq \sum_{x_1 = x_i} y_i
				\end{align}
				により$\tilde{y}_{i_1}$を定め,$x_i \neq x_1$を満たす最小の$i$
				を$i_2$として再び
				\begin{align}
					\tilde{y}_{i_2} \coloneqq \sum_{x_{i_2} = x_i} y_i
				\end{align}
				により$\tilde{y}_{i_2}$を定め,この操作を有限回繰り返して
				(\refeq{eq:thm_injective_norm_is_the_minimum_cross_norm_1})を得る.
				いま,$v \neq 0$の仮定と定理\ref{thm:when_tensor_product_zero}により,
				或る$k$に対し$x_{i_k}, \tilde{y}_{i_k} \neq 0$が満たされている.
				\begin{align}
					L \coloneqq \Span{\Set{x_{i_j}}{1 \leq j \leq \ell,\ j \neq k}}
				\end{align}
				により$X$の有限次元部分空間,すなわち閉部分空間を定めれば
				$x_{i_k}$と$L$との距離$d$は正であり,Hahn-Banachの定理より
				或る$x_k^* \in X^*$が存在して,$\Norm{x_k^*}{X^*} = 1$かつ
				\begin{align}
					&\inprod<x,x_k^*> = 0, \quad (\forall x \in L), \\
					&\inprod<x_{i_k},x_k^*> = d > 0
				\end{align}
				を満たす.一方$\tilde{y}_{i_k}$に対しても,
				Hahn-Banachの定理より或る$y_k^* \in Y^*$が存在して
				$\inprod<\tilde{y}_{i_k},y_k^*> = \Norm{\tilde{y}_{i_k}}{Y}$かつ
				$\Norm{y_k^*}{Y^*} = 1$を満たすから,
				\begin{align}
					0< d \Norm{\tilde{y}_{i_k}}{Y} = \left| x_k^* \otimes y_k^* (v) \right|
					\leq \epsilon(v)
				\end{align}
				が成立する.対偶により$\epsilon(v) = 0$ならば$v = 0$が従う.
				
			\item[第二段]
				$\epsilon$がクロスノルムであることを示す.
				先ずHahn-Banachの定理より
				\begin{align}
					\epsilon(x \otimes y) 
					&= \sup{\Norm{x^*}{X^*}\leq 1,\Norm{y^*}{Y^*} \leq 1}{\left| x^* \otimes y^* (x \otimes y) \right|} \\
					&= \sup{\Norm{x^*}{X^*} \leq 1}{\left| \inprod<x,x^*> \right|}
					\sup{\Norm{y^*}{Y^*} \leq 1}{\left| \inprod<y,y^*> \right|} \\
					&= \Norm{x}{X} \Norm{y}{Y},
					\quad (\forall (x,y) \in X \times Y)
				\end{align}
				が成り立つ.また0でない$x^* \in X^*,\ y^* \in Y^*$に対しては
				\begin{align}
					\left| x^* \otimes y^* (v) \right|
					\leq \Norm{x^*}{X^*} \Norm{y^*}{Y^*} \left( \frac{x^*}{\Norm{x^*}{X^*}} \otimes \frac{y^*}{\Norm{y^*}{Y^*}} \right) (v)
					\leq \Norm{x^*}{X^*} \Norm{y^*}{Y^*} \epsilon(v)
				\end{align}
				が成立し,$x^*=0$或は$y^*=0$のときは
				定理\ref{thm:tensor_product_contains_zero_mapping_is_zero}より
				$x^* \otimes y^* = 0$が満たされ,
				\begin{align}
					\Norm{x^* \otimes y^*}{(X \otimes Y,\epsilon)} \leq \Norm{x^*}{X^*} \Norm{y^*}{Y^*}
				\end{align}
				を得る.
			
			\item[第三段]
				$\epsilon$が最小のクロスノルムであることを示す.$\alpha$を任意のクロスノルムとすれば
				\begin{align}
					\left| x^* \otimes y^* (v) \right| \leq \Norm{x^*}{X^*} \Norm{y^*}{Y^*} \alpha(v),
					\quad (\forall v \in X \otimes Y)
				\end{align}
				が成り立つから,特に$\Norm{x^*}{X^*} \leq 1,\ \Norm{y^*}{Y^*} \leq 1$のsupを取れば
				\begin{align}
					\epsilon(v) \leq \alpha(v),
					\quad (\forall v \in X \otimes Y)
				\end{align}
				が従い$\epsilon$の最小性が出る.
				\QED
		\end{description}
	\end{prf}
	
	\begin{screen}
		\begin{dfn}[プロジェクティブノルム]
			$\K$-Banach空間$X,Y$に対し,定理
			\ref{thm:universality_of_tensor_product}により
			\begin{align}
				\begin{array}{ccc}
					\Phi:\Hom{X \otimes Y}{\K} & \longrightarrow & \Ln{X \times Y}{\K}{2} \\
					\rotatebox{90}{$\in$} & & \rotatebox{90}{$\in$} \\
					T & \longmapsto & T \circ \otimes
					\label{eq:dfn_projective_norm_isomorphism}
				\end{array}
			\end{align}
			により定まる線型同型$\Phi$が存在する.これを用いて
			\begin{align}
				\pi(v) \coloneqq
				\sup{\substack{b \in \ContLn{X \times Y}{\K}{2} \\ \Norm{b}{\ContLn{X \times Y}{\K}{2}} \leq 1}}{\left| \Phi^{-1}(b)(v) \right|},
				\quad (v \in X \otimes Y)
			\end{align}
			により定める$\pi$をプロジェクティブノルム(projective norm)と呼ぶ.
		\end{dfn}
	\end{screen}
	
	\begin{screen}
		\begin{thm}[プロジェクティブノルムは最大のクロスノルム]
		\label{thm:projective_norm_is_maximum_cross_norm}
			$\K$-Banach空間$X,Y$のテンソル積$X \otimes Y$上に
			プロジェクティブノルム$\pi$を導入する.このときクロスノルムの定義
			の(\refeq{eq:cross_norm_def_1})を満たす任意のセミノルム
			$p$に対し$p \leq \pi$が成立し,特に$\pi$は最大のクロスノルムである.
		\end{thm}
	\end{screen}
	
	\begin{prf}\mbox{}
		\begin{description}
			\item[第一段]
				$\pi$がノルムであることを示す.$v \neq 0$とすれば,
				定理\ref{thm:injective_norm_is_the_minimum_cross_norm}の証明
				と同様にして
				\begin{align}
					v = \sum_{i=1}^n x_i \otimes y_i,
					\quad (x_i \in X,\ y_i \in Y,\ x_i \neq x_j\ (i \neq j))
				\end{align}
				と表すことができ,或る$i$で$x_i,y_i \neq 0$が満たされる.
				Hahn-Banachの定理より
				\begin{align}
					&\Norm{x_i^*}{X^*} = \Norm{y_i^*}{Y^*} = 1, \\
					&\inprod<x_i,x_i^*> > 0,
						\quad \inprod<x_j,x_i^*> = 0,\quad (i \neq j) \\
					&\inprod<y_i,y_i^*> = \Norm{y_i}{Y}
				\end{align}
				を満たす$x_i^* \in X^*$と$y^* \in Y^*$が存在するから,
				\begin{align}
					b(x,y) \coloneqq \inprod<x,x_i^*> \inprod<y,y_i^*>,
					\quad (x \in X,\ y \in Y)
				\end{align}
				により双線型写像$b$を定めれば,$\Norm{b}{\ContLn{X \times Y}{\K}{2}}
				\leq \Norm{x_i^*}{X^*} \Norm{y_i^*}{Y^*} = 1$かつ
				\begin{align}
					0 < b(x_i,y_i) = |\Phi^{-1}(b)(v)| \leq \pi(v)
				\end{align}
				が成立する.$\pi(0) = 0$と
				劣加法性及び同次性は$\Phi^{-1}(b)$の線型性より従う.
			
			\item[第二段]
				$\pi$がクロスノルムであることを示す.
				先ず,任意の$x \in X,\ y \in Y$に対して
				\begin{align}
					\pi(x \otimes y) 
					&= \sup{\substack{b \in \ContLn{X \times Y}{\K}{2} \\ \Norm{b}{\ContLn{X \times Y}{\K}{2}} \leq 1}}{\left| \Phi^{-1}(b)(x \otimes y) \right|} \\
					& \leq \sup{\substack{b \in \ContLn{X \times Y}{\K}{2} \\ \Norm{b}{\ContLn{X \times Y}{\K}{2}} \leq 1}}{\Norm{b}{\ContLn{X \times Y}{\K}{2}}\Norm{x}{X} \Norm{y}{Y}} \\
					&= \Norm{x}{X} \Norm{y}{Y}
				\end{align}
				が成立する.また0でない$x^* \in X^*,\ y^* \in Y^*$に対し
				\begin{align}
					b(x,y) \coloneqq \frac{x^*}{\Norm{x^*}{X^*}}(x)\frac{y^*}{\Norm{y^*}{Y^*}}(y),
					\quad (x \in X,\ y \in Y)
				\end{align}
				により$\Norm{b}{\ContLn{X \times Y}{\K}{2}} \leq 1$を満たす有界双線型$b$を定めれば,
				$\pi$の定義より
				\begin{align}
					\left| \Phi^{-1}(b)(v) \right| \leq \pi(v),
					\quad (\forall v \in X \otimes Y)
				\end{align}
				が成り立つ.一方で写像のテンソル積の定義より
				\begin{align}
					\Phi^{-1}(b) = \frac{x^*}{\Norm{x^*}{X^*}} \otimes \frac{y^*}{\Norm{y^*}{Y^*}}
					= \frac{1}{\Norm{x^*}{X^*}\Norm{y^*}{Y^*}} x^* \otimes y^*
				\end{align}
				が満たされるから
				\begin{align}
					\left| x^* \otimes y^*(v) \right| \leq \Norm{x^*}{X^*}\Norm{y^*}{Y^*} \pi(v),
					\quad (\forall v \in X \otimes Y)
				\end{align}
				が従う.定理\ref{thm:tensor_product_contains_zero_mapping_is_zero}より
				上式は$x^* = 0$或は$y^* = 0$の場合も込めて成立するから
				\begin{align}
					\Norm{x^* \otimes y^*}{(X \otimes Y,\pi)^*} \leq \Norm{x^*}{X^*}\Norm{y^*}{Y^*}
				\end{align}
				が得られる.
				
			\item[第三段]
				$p$を(\refeq{eq:cross_norm_def_1})を満たすセミノルムとし,
				$v \in X \otimes Y$を任意に取る.Hahn-Banachの定理より
				\begin{align}
					p(v) = \phi_v(v),
					\quad |\phi_v(u)| \leq p(u)
					\quad (\forall u \in X \otimes Y)
				\end{align}
				を満たす$\phi_v \in (X \otimes Y,\pi)^*$が存在する.
				\begin{align}
					\left| (\phi_v \circ \otimes)(x,y) \right|
					= \left|\phi_v(x \otimes y)\right| 
					\leq p(x \otimes y)
					\leq \Norm{x}{X} \Norm{y}{Y},
					\quad (\forall x \in X,\ y \in Y)
				\end{align}
				が成り立つから$\Norm{\phi_v \circ \otimes}{\ContLn{X \times Y}{\K}{2}} \leq 1$が従い,
				$\pi$の定義より
				\begin{align}
					p(v) = \phi_v(v) = \Phi^{-1}(\phi_v \circ \otimes)(v)
					\leq \pi(v)
				\end{align}
				が得られる.
				\QED
		\end{description}
	\end{prf}
	
	\begin{screen}
		\begin{thm}[プロジェクティブノルムの表現]
			$\K$-Banach空間$X,Y$のテンソル積$X \otimes Y$にプロジェクトノルム
			$\pi$を導入する.このとき次が成り立つ:
			\begin{align}
				\pi(v) = \inf{}{\Set{\sum_{i=1}^n \Norm{x_i}{X}\Norm{y_i}{Y}}{
					v = \sum_{i=1}^n x_i \otimes y_i}}.
			\end{align}
		\end{thm}
	\end{screen}
	
	\begin{prf}\mbox{}
		\begin{description}
			\item[第一段]
				$X \otimes Y$上のセミノルム$\lambda$を次で定める:
				\begin{align}
					\lambda(v) \coloneqq \inf{}{\Set{\sum_{i=1}^n \Norm{x_i}{X}\Norm{y_i}{Y}}{
					v = \sum_{i=1}^n x_i \otimes y_i}},
					\quad (\forall v \in X \otimes Y)
				\end{align}
				このとき$\lambda$が式(\refeq{eq:cross_norm_def_1})かつ
				$\lambda \geq \pi$を満たせば,
				定理\ref{thm:projective_norm_is_maximum_cross_norm}より$\lambda = \pi$が従う.
				
			\item[第二段]
				$\lambda$がセミノルムであることを示す.実際,任意に
				$u,v \in X \otimes Y$を取り,
				\begin{align}
					u = \sum_{i=1}^{n} x_i \otimes y_i,
					\quad v = \sum_{j=1}^m a_j \otimes b_j
				\end{align}
				を一つの表現とすれば,$\lambda$の定め方より
				\begin{align}
					\lambda(u+v) \leq \sum_{i=1}^{n} x_i \otimes y_i + \sum_{j=1}^{m} a_j \otimes b_j
				\end{align}
				が成り立つ.右辺を移項して
				\begin{align}
					\lambda(u+v) - \sum_{j=1}^{m} a_j \otimes b_j \leq \lambda(u) \leq \sum_{i=1}^{n} x_i \otimes y_i
				\end{align}
				かつ
				\begin{align}
					\lambda(u+v) - \lambda(u) \leq \lambda(v) \leq \sum_{j=1}^{m} a_j \otimes b_j
				\end{align}
				が従い$\lambda$の劣加法性を得る.また任意の$0 \neq \alpha \in \K,\ v \in X \otimes Y$に対し
				\begin{align}
					v = \sum_{i=1}^n x_i \otimes y_i
				\end{align}
				を一つの分割とすれば
				\begin{align}
					\alpha v = \sum_{i=1}^n (\alpha x_i) \otimes y_i
				\end{align}
				は$\alpha v$の一つの分割となるから
				\begin{align}
					\lambda(\alpha v) \leq \sum_{i=1}^n \Norm{\alpha x_i}{X} \Norm{y_i}{Y}
					= |\alpha| \sum_{i=1}^n \Norm{x_i}{X} \Norm{y_i}{Y}
				\end{align}
				が成立し,$v$の分割について下限を取れば$\lambda(\alpha v) \leq |\alpha| \lambda(v)$
				が従う.逆に
				\begin{align}
					\alpha v = \sum_{j=1}^m a_j \otimes b_j
				\end{align}
				とすれば
				\begin{align}
					\lambda(v) \leq \sum_{j=1}^m \Norm{\frac{1}{\alpha}a_j}{X} \Norm{b_j}{Y}
					= \frac{1}{|\alpha|} \sum_{j=1}^m \Norm{a_j}{X} \Norm{b_j}{Y}
				\end{align}
				が成り立ち$|\alpha| \lambda(v) \leq \lambda(\alpha v)$
				が従う.$v=0$なら$v = 0 \otimes y$より$\lambda(v) = 0$が満たされ
				\begin{align}
					\lambda(\alpha v) = |\alpha| \lambda(v),
					\quad (\forall \alpha \in \K,\ v \in X \otimes Y)
				\end{align}
				が出る.
				
			\item[第三段]
				$\lambda$は式(\refeq{eq:cross_norm_def_1})を満たすことを示す.実際$\lambda$の定め方より
				\begin{align}
					\lambda(x \otimes y) \leq \Norm{x}{X} \Norm{y}{Y},
					\quad (\forall x \in X,\ y \in Y)
				\end{align}
				が成り立つ.
				
			\item[第四段]
				$\lambda \geq \pi$を示す.いま,任意に$v \in X \otimes Y$を取り
				\begin{align}
					v = \sum_{i=1}^{n} x_i \otimes y_i
				\end{align}
				を一つの分割とする.
				$\Norm{b}{\ContLn{X \times Y}{\K}{2}} \leq 1$を満たす
				$b \in \ContLn{X \times Y}{\K}{2}$と
				式(\refeq{eq:dfn_projective_norm_isomorphism})の
				$\Phi$に対し
				\begin{align}
					\left| \Phi^{-1}(b)(v) \right|
					\leq \sum_{i=1}^{n} \left| \Phi^{-1}(b)(x_i \otimes y_i) \right|
					= \sum_{i=1}^{n} \left| b(x_i,y_i) \right|
					\leq \sum_{i=1}^{n} \Norm{x_i}{X} \Norm{y_i}{Y}
				\end{align}
				が成り立つから,$b$に無関係に
				\begin{align}
					\left| \Phi^{-1}(b)(v) \right| \leq \lambda(v)
				\end{align}
				が満たされ
				\begin{align}
					\pi(v) = \sup{\substack{b \in \ContLn{X \times Y}{\K}{2} \\ \Norm{b}{\ContLn{X \times Y}{\K}{2}} \leq 1}}{\left| \Phi^{-1}(b)(v) \right|}
					\leq \lambda(v)
				\end{align}
				が従う.
				\QED
		\end{description}
	\end{prf}
	
	\begin{screen}
		\begin{thm}
			$X \otimes Y$を$\K$-Banach空間$X,Y$のテンソル積とし,
			$\epsilon, \pi$をそれぞれインジェクティブノルムとクロスノルムとする.このとき
			$X \otimes Y$上の任意のノルム$\alpha$に対し次が成立する:
			\begin{align}
				\mbox{$\alpha$がクロスノルム}
				\quad \Leftrightarrow \quad 
				\epsilon \leq \alpha \leq \pi.
			\end{align}
		\end{thm}
	\end{screen}
	
	\begin{prf}
		$(\Rightarrow)$はすでに示したから
		$(\Leftarrow)$を示す.実際,任意の$x \in X,\ y \in Y$に対して
		\begin{align}
			\alpha(x \otimes y) \leq \pi(x \otimes y) \leq \Norm{x}{X} \Norm{y}{Y}
		\end{align}
		が成立し,また任意の$x^* \in X^*,\ y^* \in Y^*$に対して
		\begin{align}
			\left| x^* \otimes y^*(v) \right|
			\leq \Norm{x^* \otimes y^*}{(X \otimes Y, \epsilon)^*} \epsilon(v)
			\leq \Norm{x^*}{X^*} \Norm{y^*}{Y^*} \alpha(v),
			\quad (\forall v \in X \otimes Y)
		\end{align}
		が満たされ$\Norm{x^* \otimes y^*}{(X \otimes Y, \alpha)^*} 
		\leq \Norm{x^*}{X^*} \Norm{y^*}{Y^*}$が得られる.
		\QED
	\end{prf}