\section{ノルム空間上の有界多重線型写像}
	[参照元:平井\cite{key7}]
	$\K$を$\R$または$\C$と考える.また$n \geq 1$とする.
	\begin{screen}
		\begin{dfn}[有界な多重線型写像]
			$(X_i)_{i=1}^n$及び$Y$を全て$\K$上のノルム空間とするとき,
			有界な$n$重線型写像の全体を$\ContLn{\bigoplus_{i=1}^n X_i}{Y}{n}$で表す.
			つまり任意の$f \in \ContLn{\bigoplus_{i=1}^n X_i}{Y}{n}$に対して
			次を満たす定数$C \geq 0$が存在する:
			\begin{align}
				\Norm{f(x_1,\cdots,x_n)}{Y} \leq C \Norm{x_1}{X_1}\cdots\Norm{x_n}{X_n},
				\quad (\forall x_i \in X_i,\ i=1,\cdots,n).
				\label{eq:def_bounced_multilinear_mapping}
			\end{align}
		\end{dfn}
	\end{screen}
	
	\begin{screen}
		\begin{thm}[有界$\Leftrightarrow$連続]
			$(X_i)_{i=1}^n$及び$Y$を全て$\K$上のノルム空間とする.
			任意の$f \in \Ln{\bigoplus_{i=1}^n X_i}{Y}{n}$に対して,
			$f$が連続であることと$f$が有界であることは一致する.
		\end{thm}
	\end{screen}
	
	\begin{prf}\mbox{}
		\begin{description}
			\item[第一段]
				$f \in \Ln{\bigoplus_{i=1}^n X_i}{Y}{n}$が連続であるとする.このとき
				$f$は$0 \in \bigoplus_{i=1}^n X_i$で連続であるから,
				或る$\delta_1,\cdots,\delta_n > 0$が存在して
				$\Norm{x_1}{X_1} \leq \delta_1,\cdots,\Norm{x_n}{X_n} \leq \delta_n$が
				満たされている限り
				\begin{align}
					\Norm{f(x_1,\cdots,x_n)}{Y} \leq 1
				\end{align}
				が成立する.よって任意の$x_i \in X_i\ (x_i \neq 0,\ i=1,\cdots,n)$に対して
				\begin{align}
					\frac{\delta_1 \cdots \delta_n}{\Norm{x_1}{X_1} \cdots \Norm{x_n}{X_n}}
					\Norm{f(x_1,\cdots,x_n)}{Y}
					= \Norm{f\biggl(\delta_1 \frac{x_1}{\Norm{x_1}{X_1}}, \cdots, \delta_n \frac{x_n}{\Norm{x_n}{X_n}} \biggr)}{Y}
					\leq 1
				\end{align}
				が従い
				\begin{align}
					\Norm{f(x_1,\cdots,x_n)}{Y} \leq \frac{1}{\delta_1 \cdots \delta_n} \Norm{x_1}{X_1} \cdots \Norm{x_n}{X_n}
				\end{align}
				を得る.或る$i$で$x_i = 0$であっても上の不等式は満たされるから$f$は有界である.
			
			\item[第二段]
				$f$が有界であるとする.このとき或る定数$C \geq 0$が存在して
				(\refeq{eq:def_bounced_multilinear_mapping})を満たし,
				\begin{align}
					\Norm{f(x_1,\cdots,x_n) - f(y_1,\cdots,y_n)}{Y}
						&\leq \|\, f(x_1,x_2,\cdots,x_n) - f(y_1,x_2,\cdots,x_n) \\
						&\quad + f(y_1,x_2,\cdots,x_n) - f(y_1,y_2,\cdots,x_n) \\
						&\quad \cdots \\
						&\quad + f(y_1,\cdots,y_{n-1},x_n) - f(y_1,\cdots,y_n)\, \|_Y \\
					&\leq C \Norm{x_1 - y_1}{X_1} \Norm{x_2}{X_2} \cdots \Norm{x_n}{X_n} \\
						&\quad + C \Norm{y_1}{X_1} \Norm{x_2 - y_2}{X_2} \cdots \Norm{x_n}{X_n} \\
						&\quad + C \Norm{y_1}{X_1} \cdots \Norm{y_{n-1}}{X_{n-1}} \Norm{x_n - y_n}{X_n} \\
					&\longrightarrow 0
					\quad \left( \Norm{(x_1,\cdots,x_n) - (y_1,\cdots,y_n)}{\bigoplus_{i=1}^n X_i} \longrightarrow 0 \right)
				\end{align}
				が成り立つから$f$の連続性が出る.
				\QED
		\end{description}
	\end{prf}
	
	$(X_i)_{i=1}^n$及び$Y$を全て$\K$上のノルム空間とする.
	このとき$f \in \ContLn{\bigoplus_{i=1}^n X_i}{Y}{n}$の作用素ノルムは次で定まる:
	\begin{align}
		\Norm{f}{\ContLn{\bigoplus_{i=1}^n X_i}{Y}{n}}
		\coloneqq \inf{}{\Set{C \geq 0}{\Norm{f(x_1,\cdots,x_n)}{Y} \leq C \Norm{x_1}{X_1}\cdots\Norm{x_n}{X_n},\ (\forall x_i \in X_i,\ i=1,\cdots,n)}}.
	\end{align}
	下限の定義より次が成立する:
	\begin{align}
		\Norm{f(x_1,\cdots,x_n)}{Y} \leq \Norm{f}{\ContLn{\bigoplus_{i=1}^n X_i}{Y}{n}} 
		\Norm{x_1}{X_1}\cdots\Norm{x_n}{X_n},
		\quad (\forall x_i \in X_i,\ i=1,\cdots,n).
		\label{eq:operator_norm_of_multilinear_mapping}
	\end{align}
	実際,(\refeq{eq:operator_norm_of_multilinear_mapping})が満たされない場合,
	或る$(u_1,\cdots,u_n) \in \bigoplus_{i=1}^n X_i\
	 (u_i \neq 0,\ i=1,\cdots,n)$が存在して
	\begin{align}
		\frac{\Norm{f(u_1,\cdots,u_n)}{Y}}{\Norm{u_1}{X_1}\cdots\Norm{u_n}{X_n}} 
		> \Norm{f}{\ContLn{\bigoplus_{i=1}^n X_i}{Y}{n}}
	\end{align}
	が成立するが,実数の連続性より
	\begin{align}
		\frac{\Norm{f(u_1,\cdots,u_n)}{Y}}{\Norm{u_1}{X_1}\cdots\Norm{u_n}{X_n}} 
		> \delta > \Norm{f}{\ContLn{\bigoplus_{i=1}^n X_i}{Y}{n}}
	\end{align}
	を満たす$\delta$が存在し,
	\begin{align}
		\Norm{f}{\ContLn{\bigoplus_{i=1}^n X_i}{Y}{n}} 
		&< \delta \\
		&\leq \inf{}{\Set{C \geq 0}{\Norm{f(x_1,\cdots,x_n)}{Y} \leq C \Norm{x_1}{X_1}\cdots\Norm{x_n}{X_n},\ (\forall x_i \in X_i,\ i=1,\cdots,n)}} \\
		&= \Norm{f}{\ContLn{\bigoplus_{i=1}^n X_i}{Y}{n}}
	\end{align}
	が従い矛盾が生じる.
	
	\begin{screen}
		\begin{thm}[多重線型写像の作用素ノルム]
			$(X_i)_{i=1}^n$及び$Y$を全て$\K$上のノルム空間とする.
			このとき,任意の$f \in \ContLn{\bigoplus_{i=1}^n X_i}{Y}{n}$に対して次が成立する:
			\begin{align}
				\Norm{f}{\ContLn{\bigoplus_{i=1}^n X_i}{Y}{n}}
				= \sup{\substack{\Norm{x_i}{X_i} = 1 \\ i=1,\cdots,n}}{\Norm{f(x_1,\cdots,x_n)}{Y}}
				= \sup{\substack{\Norm{x_i}{X_i} \leq 1 \\ i=1,\cdots,n}}{\Norm{f(x_1,\cdots,x_n)}{Y}}
				= \sup{\substack{\Norm{x_i}{X_i} \neq 0 \\ i=1,\cdots,n}}{
					\frac{\Norm{f(x_1,\cdots,x_n)}{Y}}{\Norm{x_1}{X_1} \cdots \Norm{x_n}{X_n}}}.
			\end{align}
		\end{thm}
	\end{screen}
	
	\begin{prf}$(\mbox{第四式}) \leq (\mbox{第一式}) \leq (\mbox{第二式}) \leq (\mbox{第三式}) \leq (\mbox{第四式})$を示す.
		\begin{description}
			\item[第一段]
				式(\refeq{eq:operator_norm_of_multilinear_mapping})より次を得る:
				\begin{align}
					\sup{\substack{\Norm{x_i}{X_i} \neq 0 \\ i=1,\cdots,n}}{
					\frac{\Norm{f(x_1,\cdots,x_n)}{Y}}{\Norm{x_1}{X_1} \cdots \Norm{x_n}{X_n}}}
					\leq \Norm{f}{\ContLn{\bigoplus_{i=1}^n X_i}{Y}{n}}.
				\end{align}
			
			\item[第二段]
				任意の$0 \neq x_i \in X_i\ (i=1,\cdots,n)$に対して
				\begin{align}
					\Norm{f(x_1,\cdots,x_n)}{Y}
					&= \Norm{x_1}{X_1}\cdots\Norm{x_n}{X_n}\Norm{f \biggl( \frac{x_1}{\Norm{x_1}{X_1}},\cdots,\frac{x_n}{\Norm{x_n}{X_n}} \biggr)}{Y} \\
					&\leq \Norm{x_1}{X_1}\cdots\Norm{x_n}{X_n} \sup{\substack{\Norm{x_i}{X_i} = 1 \\ i=1,\cdots,n}}{\Norm{f(x_1,\cdots,x_n)}{Y}}
				\end{align}
				が成立するから
				\begin{align}
					\Norm{f}{\ContLn{\bigoplus_{i=1}^n X_i}{Y}{n}}
					\leq 
					\sup{\substack{\Norm{x_i}{X_i} = 1 \\ i=1,\cdots,n}}{\Norm{f(x_1,\cdots,x_n)}{Y}}
				\end{align}
				が従う.
			
			\item[第三段]
				上限を取る範囲の大小より
				\begin{align}
					\sup{\substack{\Norm{x_i}{X_i} = 1 \\ i=1,\cdots,n}}{\Norm{f(x_1,\cdots,x_n)}{Y}}
					\leq \sup{\substack{\Norm{x_i}{X_i} \leq 1 \\ i=1,\cdots,n}}{\Norm{f(x_1,\cdots,x_n)}{Y}}
				\end{align}
				が出る.
				
			\item[第四段]
				$0 < \Norm{x_i}{X_i} \leq 1\ (i=1,\cdots,n)$ならば
				\begin{align}
					\Norm{f(x_1,\cdots,x_n)}{Y}
					&= \Norm{x_1}{X_1} \cdots \Norm{x_n}{X_n} \frac{\Norm{f(x_1,\cdots,x_n)}{Y}}{\Norm{x_1}{X_1} \cdots \Norm{x_n}{X_n}} \\
					&\leq \frac{\Norm{f(x_1,\cdots,x_n)}{Y}}{\Norm{x_1}{X_1} \cdots \Norm{x_n}{X_n}} \\
					&\leq \sup{\substack{\Norm{x_i}{X_i} \neq 0 \\ i=1,\cdots,n}}{
					\frac{\Norm{f(x_1,\cdots,x_n)}{Y}}{\Norm{x_1}{X_1} \cdots \Norm{x_n}{X_n}}}
				\end{align}
				が成立するから
				\begin{align}
					\sup{\substack{\Norm{x_i}{X_i} \leq 1 \\ i=1,\cdots,n}}{\Norm{f(x_1,\cdots,x_n)}{Y}}
					\leq \sup{\substack{\Norm{x_i}{X_i} \neq 0 \\ i=1,\cdots,n}}{
					\frac{\Norm{f(x_1,\cdots,x_n)}{Y}}{\Norm{x_1}{X_1} \cdots \Norm{x_n}{X_n}}}
				\end{align}
				が得られる.
				\QED
		\end{description}
	\end{prf}
	
	\begin{screen}
		\begin{thm}[有界な多重線型写像の一意拡張]\label{thm:expansion_of_multilinear_mapping}
			$n \geq 1$とする.$(X_i)_{i=1}^{n}$をノルム空間,$Z$をBanach空間,
			$Y_i$を$X_i$の稠密な部分空間とする$(i=1,\cdots,n)$.
			このとき,有界$n$重線型写像
			$b:\bigoplus_{i=1}^n Y_i \longrightarrow Z$は
			$(X_i)_{i=1}^{n}$上の$Z$値$n$重線型写像$\tilde{b}$に一意に拡張され,
			$b$と$\tilde{b}$の作用素ノルムは一致する.
		\end{thm}
	\end{screen}
	
	\begin{prf}
		$\bigoplus_{i=1}^n Y_i$は$\bigoplus_{i=1}^n X_i$で稠密であるから,
		任意の$x = (x_1,\cdots,x_n) \in \bigoplus_{i=1}^n X_i$に対して
		\begin{align}
			\Norm{x - x^k}{\bigoplus_{i=1}^n X_i}
			= \sum_{i=1}^n \Norm{x_i - x^k_i}{X_i} \longrightarrow 0,
			\quad (k \longrightarrow \infty)
		\end{align}
		を満たす点列$x^k = (x^k_1,\cdots,x^k_n) \in \bigoplus_{i=1}^n Y_i\ (k=1,2,\cdots)$
		が存在する.
		\begin{align}
			M_i \coloneqq \sup{k \geq 1}{\Norm{x^k_i}{X_i}},
			\quad (i=1,\cdots,n)
		\end{align}
		とおけば,$M_i < \infty\ (i=1,\cdots,n)$より
		\begin{align}
			\Norm{b(x^k) - b(x^\ell)}{Z}
			&= \left\|\, b(x^k_1,x^k_2,\cdots,x^k_n) - b(x^\ell_1,x^k_2,\cdots,x^k_n) \right. \\
				&\quad + b(x^\ell_1,x^k_2,\cdots,x^k_n) - b(x^\ell_1,x^\ell_2,\cdots,x^k_n) \\
				&\quad \cdots \\
				&\quad \left. + b(x^\ell_1,\cdots,x^\ell_{n-1},x^k_n) - b(x^\ell_1,\cdots,x^\ell_n)\,  \right\|_Z \\
			&\leq \Norm{b}{\ContLn{\bigoplus_{i=1}^n Y_i}{Z}{n}} \Norm{x^k_1 - x^\ell_1}{X_1} M_2 \cdots M_n \\
			&\quad + \Norm{b}{\ContLn{\bigoplus_{i=1}^n Y_i}{Z}{n}} M_1 \Norm{x^k_2 - x^\ell_2}{X_2} M_3 \cdots M_n \\
			&\quad \cdots \\
			&\quad + \Norm{b}{\ContLn{\bigoplus_{i=1}^n Y_i}{Z}{n}} M_1 \cdots M_{n-1} \Norm{x^k_n - x^\ell_n}{X_n} \\
			&\longrightarrow 0,
			\quad (k,\ell \longrightarrow \infty)
		\end{align}
		が成り立ち,$Z$の完備性より$\lim_{k \to \infty}b(x^k)$が存在する.
		別の収束列
		$\bigoplus_{i=1}^n Y_i \ni y^m \longrightarrow x$を取れば
		\begin{align}
			\Norm{x^k_i - y^m_i}{X_i} \leq \Norm{x^k_i - x_i}{X_i} + \Norm{x_i - y^m_i}{X_i}
			\longrightarrow 0,
			\quad (k,m \longrightarrow \infty,\ i=1,\cdots,n)
		\end{align}
		より$\Norm{b(x^k) - b(y^m)}{Z} \longrightarrow 0\ (k,m \longrightarrow \infty)$が従い
		\begin{align}
			\lim_{k \to \infty} b(x^k) = \lim_{m \to \infty} b(y^m)
		\end{align}
		が得られ,これにより
		写像$\tilde{b}:x \longmapsto \lim_{k \to \infty} b(x^k)$が定まる.
		この$\tilde{b}$は$b$の拡張であり,有界かつ$n$重線型性を持つ.先ず$n$重線型性を示す.
		$x = (x_1,x_2,\cdots,x_n)$と$y = (y_1,x_2,\cdots,x_n)$に対し
		\begin{align}
			\Norm{x - x^k}{\bigoplus_{i=1}^n X_i} 
			\longrightarrow 0,
			\quad \Norm{y - y^k}{\bigoplus_{i=1}^n X_i} 
			\longrightarrow 0
		\end{align}
		を満たす点列$(x^k)_{k=1}^\infty,(y^k)_{m=1}^\infty \subset \bigoplus_{i=1}^n Y_i$を取れば
		\begin{align}
			&\Norm{\tilde{b}(\alpha x_1 + \beta y_1,x_2,\cdots,x_n) 
				- \alpha \tilde{b}(x_1,\cdots,x_n)
				- \beta \tilde{b}(y_1,\cdots,x_n)}{Z} \\
			&\leq \Norm{\tilde{b}(\alpha x_1 + \beta y_1,x_2,\cdots,x_n)
				- b(\alpha x^k_1 + \beta y^k_1,x^k_2,\cdots,x^k_n)}{Z} \\
				&\quad + |\alpha| \Norm{\tilde{b}(x_1,\cdots,x_n)
				- b(x^k_1,\cdots,x^k_n)}{Z} \\
				&\quad + |\beta| \Norm{\tilde{b}(y_1,\cdots,x_n)
				- b(y^k_1,\cdots,x^k_n)}{Z} \\
			&\longrightarrow 0,
			\quad (k \longrightarrow \infty)
		\end{align}
		が成り立ち,$\tilde{b}$の第一成分に関する線型性を得る.他の成分も同じである.
		また任意の$x \in \bigoplus_{i=1}^\infty X_i$に対して
		収束列$(x^k)_{k=1}^\infty \subset \bigoplus_{i=1}^n Y_i$を取れば,
		任意の$\epsilon > 0$に対し或る$k$が存在して
		\begin{align}
			\Norm{\tilde{b}(x)}{Z} \leq \Norm{b(x^k)}{Z} + \epsilon
		\end{align}
		かつ
		\begin{align}
			\Norm{x^k_i}{X_i} \leq \Norm{x_i}{X_i} + \epsilon,
			\quad (i=1,\cdots,n)
		\end{align}
		が満たされ
		\begin{align}
			\Norm{\tilde{b}(x)}{Z} \leq \Norm{b(x^k)}{Z} + \epsilon
			\leq \Norm{b}{\ContLn{\bigoplus_{i=1}^n Y_i}{Z}{n}} \prod_{i=1}^n \left( \Norm{x_i}{X_i} + \epsilon \right) + \epsilon
		\end{align}
		が従う.$x$及び$\epsilon$の任意性より$\Norm{\tilde{b}}{\ContLn{\bigoplus_{i=1}^n X_i}{Z}{n}}
		\leq \Norm{b}{\ContLn{\bigoplus_{i=1}^n Y_i}{Z}{n}}$が成り立ち,
		$\tilde{b}$は$b$の拡張だから
		\begin{align}
			\Norm{\tilde{b}}{\ContLn{\bigoplus_{i=1}^n X_i}{Z}{n}}
			= \Norm{b}{\ContLn{\bigoplus_{i=1}^n Y_i}{Z}{n}}
		\end{align}
		が出る.拡張の一意性は$\bigoplus_{i=1}^n Y_i$の稠密性と$\tilde{b}$の連続性による.
		\QED
	\end{prf}