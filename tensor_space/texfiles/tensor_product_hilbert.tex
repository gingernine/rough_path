\section{テンソル積の内積}
	$(H_i)_{i=1}^n$を$\R$上の内積空間の族として,
	そのテンソル積$\bigotimes_{i=1}^n H_i$上に内積を導入する.
	ここで各$H_i$の内積を$\inprod<\cdot,\cdot>_i$と書く.
	いま,任意に$y_i \in H_i,\ i=1,\cdots,n$を取り
	\begin{align}
		\Phi_{y_1,\cdots,y_n}:
		\bigoplus_{i=1}^n H_i \ni (x_1,\cdots,x_n)
		\longmapsto \inprod<x_1,y_1>_1 \cdots \inprod<x_n,y_n>_n
	\end{align}
	とおけば,$\Phi_{y_1,\cdots,y_n}$は$n$重線型であるから
	或る
	
	\begin{screen}
		\begin{thm}
		\end{thm}
	\end{screen}
	
	\begin{prf} $s$は双線型性を持つ写像として定めたから,後は対称性と正定値性及び$s(u,u)=0 \Leftrightarrow u=0$
		を示せばよい.
		\begin{description}
			\item[第一段]
				任意の$u \in \bigotimes_{i=1}^n H_i$に対し$s(u,u) \geq 0$が成り立つことを示す.
				
			\item[第一段]
				$s(u,u)=0 \Leftrightarrow u=0,\ (u \in \bigotimes_{i=1}^n H_i)$が成り立つことを示す.
				定理\ref{thm:basis_of_tensor_product}より,
				基底$\left\{ e^i_{\lambda_i} \right\}_{\lambda_i \in \Lambda_i} \subset H_i$
				に対し$\left\{ e^1_{\lambda_1} \otimes \cdots \otimes e^n_{\lambda_n} \right\}_{\lambda_1,\cdots,\lambda_n}$
				は$\bigotimes_{i=1}^n H_i$の基底となるから,任意の
				$u \in \bigotimes_{i=1}^n H_i$は
				\begin{align}
					u = \sum_{j=1}^k \alpha_j \left( e^1_{\lambda^{(j)}_1} \otimes \cdots \otimes e^n_{\lambda^{(j)}_n} \right)
					= \sum_{j=1}^k e^1_{\lambda^{(j)}_1} \otimes \cdots \otimes \left( \alpha_j e^n_{\lambda^{(j)}_n} \right),
					\quad (\alpha_j \neq 0,\ j=1,\cdots,k)
				\end{align}
				と表現できる.
			
			\item[第三段]
				$s$の対称性を示す.
		\end{description}
	\end{prf}
	
	\begin{screen}
		\begin{dfn}[テンソル積上の内積]
		\end{dfn}
	\end{screen}
	
	\begin{screen}
		\begin{thm}[$\sigma$はクロスノルム]
		\end{thm}
	\end{screen}
	
	\begin{screen}
		\begin{thm}[$H_i$が有限次元なら$\sigma$と$\pi$は同値]
		\end{thm}
	\end{screen}