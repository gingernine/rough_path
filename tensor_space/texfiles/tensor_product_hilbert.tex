\section{テンソル積の内積}
	[参考:\cite{key8}(pp. 1-24), \cite{key7}] 
	$(H_i)_{i=1}^n$を$\R$-Hilbert空間の族として,$\bigotimes_{i=1}^n H_i$に内積を導入する.
	$H_i$の内積を$\inprod<\cdot,\cdot>_{H_i}$と書く.
	\begin{description}
		\item[第一段] 
			任意に$y_i \in H_i,\ i=1,\cdots,n$を取り
			\begin{align}
				\Phi_{y_1,\cdots,y_n}:
				\bigoplus_{i=1}^n H_i \ni (x_1,\cdots,x_n)
				\longmapsto \inprod<x_1,y_1>_{H_1} \cdots \inprod<x_n,y_n>_{H_n}
			\end{align}
			とおけば,$\Phi_{y_1,\cdots,y_n}$は$n$重線型であるから
			或る$\Psi_{y_1,\cdots,y_n} \in \Hom{\bigotimes_{i=1}^n H_i}{\R}$がただ一つ存在して
			\begin{align}
				\Psi_{y_1,\cdots,y_n} \circ \otimes = \Phi_{y_1,\cdots,y_n}
			\end{align}
			を満たす.
			
		\item[第二段]
			任意の$u \in \bigotimes_{i=1}^n H_i$に対し
			$\bigoplus_{i=1}^n H_i \ni (y_1,\cdots,y_n) \longmapsto \Psi_{y_1,\cdots,y_n}(u)$
			は$n$重線型である.実際,
			\begin{align}
				u = \sum_{j=1}^k x^j_1 \otimes \cdots \otimes x^j_n
			\end{align}
			と表せるとき,任意の$\alpha,\beta \in \R$と$y_i,z_i \in H_i$に対して
			\begin{align}
				\Psi_{y_1,\cdots,\alpha y_i + \beta z_i,\cdots,y_n}(u)
				&= \sum_{j=1}^k \Psi_{y_1,\cdots,\alpha y_i + \beta z_i,\cdots,y_n}(x^j_1 \otimes \cdots \otimes x^j_n) \\
				&= \sum_{j=1}^k \inprod<x^j_1,y_1>_{H_1} \cdots \inprod<x^j_i,\alpha y_i + \beta z_i>_{H_i} \cdots \inprod<x^j_n,y_n>_{H_n} \\
				&= \alpha \sum_{j=1}^k \inprod<x^j_1,y_1>_{H_1} \cdots \inprod<x^j_i,y_i>_{H_i} \cdots \inprod<x^j_n,y_n>_{H_n} \\
				&\quad + \beta \sum_{j=1}^k \inprod<x^j_1,y_1>_{H_1} \cdots \inprod<x^j_i,z_i>_{H_i} \cdots \inprod<x^j_n,y_n>_{H_n} \\
				&= \alpha \sum_{j=1}^k \Psi_{y_1,\cdots,y_i,\cdots,y_n}(x^j_1 \otimes \cdots \otimes x^j_n)
				+ \beta \sum_{j=1}^k \Psi_{y_1,\cdots,z_i,\cdots,y_n}(x^j_1 \otimes \cdots \otimes x^j_n) \\
				&= \alpha \Psi_{y_1,\cdots,y_i,\cdots,y_n}(u)
				+ \beta \Psi_{y_1,\cdots,z_i,\cdots,y_n}(u)
			\end{align}
			が成立する.従って或る$F_u \in \Hom{\bigotimes_{i=1}^n H_i}{\R}$がただ一つ存在して
			\begin{align}
				F_u(y_1 \otimes \cdots \otimes y_n) = \Psi_{y_1,\cdots,y_n}(u),
				\quad (\forall y_1 \otimes \cdots \otimes y_n \in \bigotimes_{i=1}^n H_i)
			\end{align}
			を満たす.
			
		\item[第三段]
			任意の$v \in \bigotimes_{i=1}^n H_i$に対し
			$\bigotimes_{i=1}^n H_i \ni u \longmapsto F_u(v)$は線型性を持つ.実際,
			\begin{align}
				v = \sum_{j=1}^k x^j_1 \otimes \cdots \otimes x^j_n
			\end{align}
			と表せるとき,任意の$\alpha,\beta \in \R$と$u,w \in \bigotimes_{i=1}^n H_i$に対して
			\begin{align}
				F_{\alpha u + \beta w}(v)
				&= \sum_{j=1}^k F_{\alpha u + \beta w}(x^j_1 \otimes \cdots \otimes x^j_n) \\
				&= \sum_{j=1}^k \Psi_{x^j_1, \cdots, x^j_n}(\alpha u + \beta w) \\
				&= \alpha \sum_{j=1}^k \Psi_{x^j_1, \cdots, x^j_n}(u)
					+ \beta \sum_{j=1}^k \Psi_{x^j_1, \cdots, x^j_n}(w) \\
				&= \alpha F_{u}(v) + \beta F_{w}(v)
			\end{align}
			が成立する.従って
			\begin{align}
				s(u,v) \coloneqq F_u(v),
				\quad (\forall u,v \in \bigotimes_{i=1}^n H_i)
				\label{eq:def_inner_product_of_tensor_product}
			\end{align}
			により定める$s:\bigotimes_{i=1}^n H_i \times \bigotimes_{i=1}^n H_i \longrightarrow \R$
			は双線型形式である.
	\end{description}
	
	\begin{screen}
		\begin{thm}
			$s$は$\bigotimes_{i=1}^n H_i$の内積となり,任意の
			$x_1 \otimes \cdots \otimes x_n,\ y_1 \otimes \cdots \otimes y_n 
			\in \bigotimes_{i=1}^n H_i$に対し次を満たす:
			\begin{align}
				s(x_1 \otimes \cdots \otimes x_n,y_1 \otimes \cdots \otimes y_n)
				= \inprod<x_1,y_1>_{H_1} \cdots \inprod<x_n,y_n>_{H_n}.
				\label{eq:def_inner_product_of_tensor_product_2}
			\end{align}
		\end{thm}
	\end{screen}
	
	\begin{prf} $s$は双線型性を持つように定めたから,後は対称性と正定値性及び$s(u,u)=0 \Leftrightarrow u=0$
		を示せばよい.
		\begin{description}
			\item[第一段]
				任意の$x_1 \otimes \cdots \otimes x_n,\ y_1 \otimes \cdots \otimes y_n 
				\in \bigotimes_{i=1}^n H_i$に対し
				\begin{align}
					s(x_1 \otimes \cdots \otimes x_n,y_1 \otimes \cdots \otimes y_n)
					&= F_{x_1 \otimes \cdots \otimes x_n}(y_1 \otimes \cdots \otimes y_n)
					= \Psi_{y_1,\cdots,y_n}(x_1 \otimes \cdots \otimes x_n)
					= \Phi_{y_1,\cdots,y_n}(x_1, \cdots, x_n) \\
					&= \inprod<x_1,y_1>_{H_1} \cdots \inprod<x_n,y_n>_{H_n}
				\end{align}
				が成り立ち(\refeq{eq:def_inner_product_of_tensor_product_2})を得る.
				また$s(x_1 \otimes \cdots \otimes x_n,y_1 \otimes \cdots \otimes y_n)
				= s(y_1 \otimes \cdots \otimes y_n,x_1 \otimes \cdots \otimes x_n)$
				も従う.
				
			\item[第二段]
				$s$の対称性を示す.任意の$u,v \in \bigotimes_{i=1}^n H_i$に対し分割を
				\begin{align}
					u = \sum_{j=1}^k x^j_1 \otimes \cdots \otimes x^j_n,
					\quad v = \sum_{r=1}^m y^r_1 \otimes \cdots \otimes y^r_n
				\end{align}
				と表せば,前段の結果より
				\begin{align}
					s(u,v) = \sum_{j=1}^k \sum_{r=1}^m s(x^j_1 \otimes \cdots \otimes x^j_n,y^r_1 \otimes \cdots \otimes y^r_n)
					= \sum_{r=1}^m \sum_{j=1}^k s(y^r_1 \otimes \cdots \otimes y^r_n,x^j_1 \otimes \cdots \otimes x^j_n)
					= s(v,u)
				\end{align}
				となる.
				
			\item[第三段]
				任意の$u \in \bigotimes_{i=1}^n H_i$に対し$s(u,u) \geq 0$が成り立つことを示す.
				実際,
				
			\item[第一段]
				$s(u,u)=0 \Leftrightarrow u=0,\ (u \in \bigotimes_{i=1}^n H_i)$が成り立つことを示す.
				定理\ref{thm:basis_of_tensor_product}より,
				基底$\left\{ e^i_{\lambda_i} \right\}_{\lambda_i \in \Lambda_i} \subset H_i,\ (i=1,\cdots,n)$
				に対し$\left\{ e^1_{\lambda_1} \otimes \cdots \otimes e^n_{\lambda_n} \right\}_{\lambda_1,\cdots,\lambda_n}$
				は$\bigotimes_{i=1}^n H_i$の基底となるから,任意の
				$u \in \bigotimes_{i=1}^n H_i$は
				\begin{align}
					u = \sum_{j=1}^k \alpha_j \left( e^1_{\lambda^{(j)}_1} \otimes \cdots \otimes e^n_{\lambda^{(j)}_n} \right)
					= \sum_{j=1}^k e^1_{\lambda^{(j)}_1} \otimes \cdots \otimes \left( \alpha_j e^n_{\lambda^{(j)}_n} \right),
					\quad (\alpha_j \neq 0,\ j=1,\cdots,k)
				\end{align}
				と表現できる.
		\end{description}
	\end{prf}
	
	\begin{screen}
		\begin{dfn}[テンソル積上の内積]
			式(\refeq{eq:def_inner_product_of_tensor_product})で定めた双線型形式$s$を
			$\inprod<\cdot,\cdot>$と書き直して$\bigotimes_{i=1}^n H_i$の内積とする.
			また$\sigma(u) \coloneqq \sqrt{\inprod<u,u>},\ (u \in \bigotimes_{i=1}^n H_i)$
			によりノルム$\sigma$を導入する.
		\end{dfn}
	\end{screen}
	
	\begin{screen}
		\begin{thm}[$\sigma$はクロスノルム]
			$\sigma$はクロスノルムである.
		\end{thm}
	\end{screen}
	
	\begin{prf} 定義\ref{def:cross_norm}の(\refeq{eq:cross_norm_def_1})と
		(\refeq{eq:cross_norm_def_2})を満たすことを示せばよい.
		\begin{description}
			\item[第一段]
				式(\refeq{eq:def_inner_product_of_tensor_product_2})より
				(\refeq{eq:cross_norm_def_1})が従う.
				
			\item[第二段]
				任意に$x^*_i \in H^*_i$を取れば,RieszのHilbert空間の表現定理より
				或る$a_i \in H_i$がただ一つ存在して
				\begin{align}
					x^*_i(\cdot) = \inprod<\cdot,a_i>_{H_i},
					\quad \Norm{x^*_i}{H^*_i} = \Norm{a_i}{H_i},
					\quad i=1,\cdots,n
				\end{align}
				を満たす.Cauchy-Schwartzの不等式と併せれば,任意の
				\begin{align}
					x = \sum_{j=1}^{k} x^j_1 \otimes \cdots \otimes x^j_n \in \bigotimes_{i=1}^n H_i
				\end{align}
				に対し
				\begin{align}
					(x^*_1 \otimes \cdots \otimes x^*_n)(x)
					&= \sum_{j=1}^{k} (x^*_1 \otimes \cdots \otimes x^*_n)(x^j_1 \otimes \cdots \otimes x^j_n)
					= \sum_{j=1}^{k} x^*_1(x^j_1) \cdots x^*_n(x^j_n) \\
					&= \sum_{j=1}^{k} \inprod<x^j_1,a_1>_{H_1} \cdots \inprod<x^j_n,a_n>_{H_n}
					= \sum_{j=1}^{k} \inprod<x^j_1 \otimes \cdots \otimes x^j_n,a_1 \otimes \cdots \otimes a_n>
					= \inprod<x,a_1 \otimes \cdots \otimes a_n> \\
					&\leq \sigma(x) \sigma(a_1 \otimes \cdots \otimes a_n)
					= \sigma(x) \Norm{a_1}{H_1} \cdots \Norm{a_n}{H_n} \\
					&= \sigma(x) \Norm{x^*_1}{H^*_1} \cdots \Norm{x^*_n}{H^*_n}
				\end{align}
				が成立し(\refeq{eq:cross_norm_def_2})を得る.
				\QED
		\end{description}
	\end{prf}
	
	\begin{screen}
		\begin{thm}[$H_i$が有限次元なら$\sigma$と$\pi$は同値]
		\end{thm}
	\end{screen}