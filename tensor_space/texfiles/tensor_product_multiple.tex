\section{notation}
	以下,零元のみの線型空間は考えない.
	$E,E_i,F$を体$\K$上の線形空間とするとき,
	$\Hom{E}{F}$で$E$から$F$への$\K$-線型写像の全体を表し,
	特に$F = \K$のとき$E^\#$と書く.
	また$\Ln{E_1 \times \cdots \times E_n}{F}{n}$
	で$E_1 \times \cdots \times E_n$から$F$への$\K$-$n$重線型写像の全体を表す.
	
\section{テンソル積}
	$n \geq 2$とする.体$\K$上の線形空間の族$(E_i)_{i=1}^n$に対して
	テンソル積を定めたい.
	\begin{align}
		\Lambda\biggl( \bigoplus_{i=1}^n E_i \biggr)
		= \Set{b:\bigoplus_{i=1}^n E_i \longrightarrow \K}{\mbox{有限個の$e \in \bigoplus_{i=1}^n E_i$を除いて$b(e)=0$.}}
	\end{align}
	により$\K$-線形空間$\Lambda\biggl( \bigoplus_{i=1}^n E_i \biggr)$を定める.また$e=(e_1,\cdots,e_n) \in \bigoplus_{i=1}^n E_i$に対する定義関数を
	\begin{align}
		\defunc_{e_1,\cdots,e_n} (x) = 
		\begin{cases}
			1, & x = e, \\
			0, & x \neq e
		\end{cases}
	\end{align}
	で表す.$\Lambda\biggl( \bigoplus_{i=1}^n E_i \biggr)$の線型部分空間を
	\begin{align}
		&\Lambda_0\biggl( \bigoplus_{i=1}^n E_i \biggr) \\
		&\coloneqq
		\Span{\Set{ \substack{\defunc_{e_1,\cdots,e_i + e_i',\cdots,e_n}
			-\defunc_{e_1,\cdots,e_i,\cdots,e_n}
			-\defunc{e_1,\cdots,e_i',\cdots,e_n},\\
			\defunc_{e_1,\cdots,\lambda e_i,\cdots,e_n}
			-\lambda\defunc_{e_1,\cdots,e_i,\cdots,e_n}}}{e_i,e_i' \in E_i,
			\lambda \in \K,
			1 \leq i \leq n}}
	\end{align}
	により定め,
	$b \in \Lambda\biggl( \bigoplus_{i=1}^n E_i \biggr)$の$\Lambda_0\biggl( \bigoplus_{i=1}^n E_i \biggr)$に関する同値類を$[b]$と書く.そして
	\begin{align}
		E_1 \otimes \cdots \otimes E_n = \bigotimes_{i=1}^n E_i 
		\coloneqq \Lambda\biggl( \bigoplus_{i=1}^n E_i \biggr)/\Lambda_0\biggl( \bigoplus_{i=1}^n E_i \biggr)
	\end{align}
	で定める商空間を$(E_i)_{i=1}^n$のテンソル積と定義する.
	また$(e_1,\cdots,e_n) \in \bigoplus_{i=1}^n E_i$に対し
	\begin{align}
		e_1 \otimes \cdots \otimes e_n \coloneqq \left[ \defunc_{e_1,\cdots,e_n} \right]
	\end{align}
	により定める$\otimes:\bigoplus_{i=1}^n E_i \longrightarrow \bigotimes_{i=1}^n E_i$を
	テンソル積の標準写像と呼ぶ.
	
	\begin{screen}
		\begin{thm}[標準写像の多重線型性]\label{thm:tensor_product_is_bilinear}
			$(E_i)_{i=1}^n$を$\K$-線形空間の族とするとき,
			\begin{align}
				\otimes : \bigoplus_{i=1}^n E_i \ni (e_1,\cdots,e_n) \longmapsto e_1 \otimes \cdots \otimes e_n \in \bigotimes_{i=1}^n E_i
			\end{align}
			は$n$重線型写像である.また次が成り立つ:
			\begin{align}
				\bigotimes_{i=1}^n E_i = \Span{\Set{e_1 \otimes \cdots \otimes e_n}{(e_1,\cdots,e_n) \in \bigoplus_{i=1}^n E_i}}.
				\label{eq:thm_tensor_product_is_bilinear}
			\end{align}
		\end{thm}
	\end{screen}
	
	\begin{prf}
		任意の$1 \leq i \leq n,\ e_1 \in E_1,\cdots,e_n \in E_n,
		\ e_i,e_i' \in E_i,\ \lambda \in \K$に対して
		\begin{align}
			e_1 \otimes \cdots \otimes (e_i + e_i') \otimes \cdots \otimes e_n 
			&= \left[ \defunc_{e_1,\cdots,e_i + e_i',\cdots,e_n} \right] \\
			&= \left[ \defunc_{e_1,\cdots,e_i,\cdots,e_n} 
				+ \defunc_{e_1,\cdots,e_i',\cdots,e_n} \right] \\
			&= \left[ \defunc_{e_1,\cdots,e_i,\cdots,e_n} \right]
				+ \left[ \defunc_{e_1,\cdots,e_i',\cdots,e_n} \right] \\
			&=e_1 \otimes \cdots \otimes e_i \otimes \cdots \otimes e_n 
			+ e_1 \otimes \cdots \otimes e_i' \otimes \cdots \otimes e_n, \\
			e_1 \otimes \cdots \otimes (\lambda e_i) \otimes \cdots \otimes e_n 
			&= \left[ \defunc_{e_1,\cdots,\lambda e_i,\cdots,e_n} \right] \\
			&= \left[ \lambda \defunc_{e_1,\cdots,e_i,\cdots,e_n} \right] \\
			&= \lambda \left[ \defunc_{e_1,\cdots,e_i,\cdots,e_n} \right] \\
			&= \lambda (e_1 \otimes \cdots \otimes e_i \otimes \cdots \otimes e_n) 
		\end{align}
		が成立するから$\otimes$は多重線型である.また
		任意に$u = [b] \in E \otimes F$を取れば
		\begin{align}
			b = \sum_{j=1}^m k_j \defunc_{e^j_i,\cdots,e^j_n},
			\quad \left( k_j = b(e^j_i,\cdots,e^j_n),\ j=1,\cdots,m \right)
		\end{align}
		と表せるから,
		\begin{align}
			u = \left[ \sum_{j=1}^m k_j \defunc_{e^j_i,\cdots,e^j_n} \right]
			= \left[ \sum_{j=1}^m \defunc_{k_j e^j_i,\cdots,e^j_n} \right]
			= \sum_{j=1}^m (k_j e^j_1) \otimes \cdots \otimes e^j_n
		\end{align}
		が従い(\refeq{eq:thm_tensor_product_is_bilinear})を得る.
		\QED
	\end{prf}
	
	\begin{screen}
		\begin{thm}[$\cdots \otimes 0 \otimes \cdots$は零ベクトル]
			$(E_i)_{i=1}^n$を$\K$-線形空間の族とし,
			テンソル積$\bigotimes_{i=1}^n E_i$を定める.
			このとき,或る$i$で$e_i = 0$なら
			$e_1 \otimes \cdots \otimes e_n = 0$が成り立つ.
		\end{thm}
	\end{screen}
	
	\begin{prf}
		$e_i = 0$のとき,$\lambda = 0$とすれば
		\begin{align}
			e_1 \otimes \cdots \otimes e_n
			= \left[ \defunc_{e_1,\cdots,0,\cdots,e_n} \right]
			= \left[ \defunc_{e_1,\cdots,\lambda e_i,\cdots,e_n} - \lambda \defunc_{e_1,\cdots,e_i,\cdots,e_n}\right]
			= 0
		\end{align}
		が成立する.
		\QED
	\end{prf}
	
	\begin{screen}
		\begin{thm}[普遍性(universality of tensor products)]
		\label{thm:universality_of_tensor_product}
			$(E_i)_{i=1}^n$を$\K$-線形空間の族とする.このとき
			任意の$\K$-線型空間$V$に対して,$T \in \Hom{\bigotimes_{i=1}^n E_i}{V}$ならば
			$T \circ \otimes \in \Ln{\bigoplus_{i=1}^n E_i}{V}{n}$
			が満たされ,これで定める次の対応$\Phi$は線型同型である:
			\begin{align}
				\begin{array}{ccc}
					\Phi:\Hom{\bigotimes_{i=1}^n E_i}{V} & \longrightarrow & \Ln{\bigoplus_{i=1}^n E_i}{V}{n} \\
					\rotatebox{90}{$\in$} & & \rotatebox{90}{$\in$} \\
					T & \longmapsto & T \circ \otimes 
					\label{eq:thm_universality_of_tensor_product}
				\end{array}
			\end{align}
			また$\K$-線型空間$U_0$と多重線型写像$\iota:\bigoplus_{i=1}^n E_i \longrightarrow U_0$が,
			任意の$\K$-線型空間$V$に対し
			\begin{description}
				\item[$(\otimes)_1$] $U_0$は$\iota$の像で生成される.
				\item[$(\otimes)_2$] 任意の$\delta \in \Ln{\bigoplus_{i=1}^n E_i}{V}{n}$に対して
					或る$\tau \in \Hom{U_0}{V}$が$\delta = \tau \circ \iota$を満たす.
			\end{description}
			を満たすなら,(\refeq{eq:thm_universality_of_tensor_product})において
			$V = U_0$とするとき$T = \Phi^{-1}(\iota)$は線型同型である.
		\end{thm}
	\end{screen}
	後半の主張により,$(E_i)_i$のテンソル積を別の方法で導入しても,
	商空間を用いて導入した$\bigotimes_i E_i$と線型同型に結ばれる.
	このとき,別の方法で導入したテンソル積及び標準写像を$\bigotimes\tilde{ }_i E_i,\ \tilde{\otimes}$と表せば,
	或る線型同型$T:\bigotimes_i E_i \longrightarrow \bigotimes\tilde{ }_i E_i$がただ一つ存在して
	\begin{align}
		T(e_1 \otimes \cdots \otimes e_n) = e_1 \tilde{\otimes} \cdots \tilde{\otimes} e_n 
	\end{align}
	を満たす.
	\begin{prf}\mbox{}
		\begin{description}
			\item[第一段]
				$T \in \Hom{\bigotimes_{i=1}^n E_i}{V}$の線型性と
				$\otimes$の多重線型性より
				$T \circ \otimes$は多重線型である.
				
			\item[第二段]
				$\Phi(T_1) = \Phi(T_2)$ならば
				$T_1$と$T_2$は$\Set{e_1 \otimes \cdots \otimes e_n}{(e_1,\cdots,e_n) \in \bigoplus_{i=1}^n E_i}$の上で一致する.
				(\refeq{eq:thm_tensor_product_is_bilinear})より
				$T_1 = T_2$が成立し$\Phi$の単射性が従う.
			
			\item[第三段]
				次の二段で$\Phi$の全射性を示す.まず,$\varphi \in \Hom{\Lambda(\bigoplus_{i=1}^n E_i)}{V}$に対し
				\begin{align}
					g: \bigoplus_{i=1}^n E_i \ni (e_1,\cdots,e_n) \longmapsto \varphi(\defunc_{e_1,\cdots,e_n}) \in V
				\end{align}
				を対応させる次の写像が全単射であることを示す:
				\begin{align}
					\begin{array}{ccc}
						F:\Hom{\Lambda(\bigoplus_{i=1}^n E_i)}{V} & \longrightarrow & \Map{\bigoplus_{i=1}^n E_i}{V} \\
						\rotatebox{90}{$\in$} & & \rotatebox{90}{$\in$} \\
						\varphi & \longmapsto & g
					\end{array}
				\end{align}
				$F(\varphi_1) = F(\varphi_2)$のとき,
				任意の$e \in \bigoplus_{i=1}^n E_i$に対して
				$\varphi_1(\defunc_{e_1,\cdots,e_n}) = \varphi_2(\defunc_{e_1,\cdots,e_n})$が成り立ち,
				\begin{align}
					\Lambda\biggl( \bigoplus_{i=1}^n E_i \biggr) 
					= \Span{\Set{\defunc_{e_1,\cdots,e_n}}{(e_1,\cdots,e_n) \in \bigoplus_{i=1}^n E_i}}
				\end{align}
				であるから$\varphi_1 = \varphi_2$が従い$F$の単射性を得る.また
				$g \in \Map{\bigoplus_{i=1}^n E_i}{V}$に対して
				\begin{align}
					\varphi(a) \coloneqq \sum_{\substack{e \in \bigoplus_{i=1}^n E_i \\ a(e) \neq 0}} a(e) g(e),
					\quad (a \in \Lambda(\bigoplus_{i=1}^n E_i))
				\end{align}
				により$\varphi$を定めれば,$\varphi \in \Hom{\Lambda(\bigoplus_{i=1}^n E_i)}{V}$が満たされ
				$F$の全射性が従う.
				
			\item[第四段]
				任意に$b \in \Ln{\bigoplus_{i=1}^n E_i}{V}{n}$を取り
				$h \coloneqq F^{-1}(b)$とおけば,$h$の線型性より
				\begin{align}
					&b(e_1,\cdots,e_i+e_i',\cdots,e_n) - b(e_1,\cdots,e_i,\cdots,e_n) - b(e_1,\cdots,e_i',\cdots,e_n) \\
					&\qquad = h(\defunc_{e_1,\cdots,e_i+e_i',\cdots,e_n} - \defunc_{e_1,\cdots,e_i,\cdots,e_n} - \defunc_{e_1,\cdots,e_i',\cdots,e_n}), \\
					&b(e_1,\cdots,\lambda e_i,\cdots,e_n) - \lambda b(e_1,\cdots,e_i,\cdots,e_n) \\
					&\qquad = h(\defunc_{e_1,\cdots,\lambda e_i,\cdots,e_n} - \lambda \defunc_{e_1,\cdots,e_i,\cdots,e_n})
				\end{align}
				が成り立ち,$b$の双線型性により$h$は$\Lambda_0(\bigoplus_{i=1}^n E_i)$上で0である.従って
				\begin{align}
					T([b]) \coloneqq h(b),
					\quad (b \in \Lambda(\bigoplus_{i=1}^n E_i))
				\end{align}
				で定める$T$はwell-definedであり,$T \in \Hom{\bigoplus_{i=1}^n E_i}{V}$かつ
				\begin{align}	
					b(e_1,\cdots,e_n) = h(\defunc_{e_1,\cdots,e_n}) = (T \circ \otimes) (e_1,\cdots,e_n),
					\quad (\forall (e_1,\cdots,e_n) \in \bigoplus_{i=1}^n E_i)
				\end{align}
				が満たされ$\Phi$の全射性が得られる.
				
			\item[第五段]
				$(\otimes)_1,(\otimes)_2$の下で
				$\Hom{U_0}{\bigotimes_{i=1}^n E_i} \ni \tau \longmapsto \tau \circ \iota \in \Ln{\bigoplus_{i=1}^n E_i}{\bigotimes_{i=1}^n E_i}{n}$は全単射であるから,
				$\tau \circ \iota = \otimes$を満たす$\tau \in \Hom{U_0}{\bigotimes_{i=1}^n E_i}$がただ一つ存在する.
				同様にして$\iota = T \circ \otimes$を満たす
				$T \in \Hom{\bigotimes_{i=1}^n E_i}{U_0}$がただ一つ存在し,併せれば
				\begin{align}
					\otimes = \tau \circ \iota = (\tau \circ T) \circ \otimes,
					\quad \iota = T \circ \otimes = (T \circ \tau) \circ \iota
				\end{align}
				が成り立ち,$T \longmapsto T \circ \otimes,\ \tau \longmapsto \tau \circ \iota$
				が一対一であるから$\tau \circ T,\ T \circ \tau$はそれぞれ恒等写像に一致して
				$T^{-1} = \tau$が従う.すなわち$T$は$\bigotimes_{i=1}^n E_i$から$U_0$への
				線型同型である.
				\QED
		\end{description}
	\end{prf}
	
	\begin{screen}
		\begin{dfn}[線型写像のテンソル積]
				$(E_i)_{i=1}^n$と$(F_i)_{i=1}^n$を$\K$-線型空間の族とする.
				$f_i:E_i \longrightarrow F_i\ (i=1,\cdots,n)$が線型写像であるとき,
				\begin{align}
					b: \bigoplus_{i=1}^n E_i \ni (e_1,\cdots,e_n)
					\longmapsto f_1(e_1)\otimes \cdots \otimes f_n(e_n)
					\in \bigotimes_{i=1}^n F_i
				\end{align}
				により定める$b$は$n$重線型であり,定理\ref{thm:universality_of_tensor_product}
				より$b = g \circ \otimes$を満たす
				$g:\bigotimes_{i=1}^{n} E_i \longrightarrow \bigotimes_{i=1}^{n} F_i$がただ一つ存在する.
				$g$を$f_1 \otimes \cdots \otimes f_n$と表記して線型写像のテンソル積と定義する.いま,
				\begin{align}
					f_1 \otimes \cdots \otimes f_n(e_1 \otimes \cdots \otimes e_n)
					= f_1(e_1)\otimes \cdots \otimes f_n(e_n),
					\quad (\forall (e_1,\cdots,e_n) \in \bigoplus_{i=1}^n E_i)
				\end{align}
				が成り立つ.
		\end{dfn}
	\end{screen}
	
	\begin{screen}
		\begin{thm}[テンソル積の基底]
			$(E_i)_{i=1}^n$を$\K$-線型空間の族とし,各々の基底を
			$\left( u^i_{\lambda_i} \right)_{\lambda_i \in \Lambda_i}\ (i=1,\cdots,n)$
			とする.このとき$\left( u^1_{\lambda_1} \otimes 
			\cdots \otimes u^n_{\lambda_n} \right)_{\lambda_1,\cdots,\lambda_n}$
			は$\bigotimes_{i=1}^n E_i$の基底となる.
		\end{thm}
	\end{screen}
	
	\begin{prf}\mbox{}
		\begin{description}
			\item[第一段]
				各$u^i_{\lambda_i}$の生成する一次元空間を$W^i_{\lambda_i} \coloneqq \K u^i_{\lambda_i}$と表し
				\begin{align}
					V_i \coloneqq \bigoplus_{\lambda_i \in \Lambda_i} W^i_{\lambda_i},
					\quad (i=1,\cdots,n)
				\end{align}
				とおく.$\left( u^i_{\lambda_i} \right)_{\lambda_i \in \Lambda_i}$
				は$E_i$の基底であるから,任意の$e_i \in E_i$に対し$v_i \in V_i$がただ一つ定まり,
				\begin{align}
					f_i:E_i \ni e_i \longmapsto v_i \in V_i
				\end{align}
				により定める線型写像$f_i$は同型写像である.
				このとき,写像のテンソル積
				\begin{align}
					f_1 \otimes \cdots \otimes f_n:
					\bigotimes_{i=1}^n E_i \longrightarrow \bigotimes_{i=1}^n V_i
				\end{align}
				は線型同型となる.実際,$f_i$の逆写像$f^{-1}_i$のテンソル積
				\begin{align}
					f^{-1}_1 \otimes \cdots \otimes f^{-1}_n:
					\bigotimes_{i=1}^n V_i \longrightarrow \bigotimes_{i=1}^n E_i
				\end{align}
				によって,全ての$(e_1 \otimes\cdots\otimes e_n) \in \bigotimes_{i=1}^n E_i$及び
				$(v_1 \otimes\cdots\otimes v_n) \in \bigotimes_{i=1}^n V_i$に対し
				\begin{align}
					&f^{-1}_1 \otimes \cdots \otimes f^{-1}_n \circ 
						f_1 \otimes \cdots \otimes f_n
						(e_1 \otimes \cdots \otimes e_n) \\
					&\qquad = f^{-1}_1 \otimes \cdots \otimes f^{-1}_n
						\left( f_1(e_1) \otimes \cdots \otimes f_n(e_n) \right)
						= (e_1 \otimes \cdots \otimes e_n), \\
					&f_1 \otimes \cdots \otimes f_n \circ 
						f^{-1}_1 \otimes \cdots \otimes f^{-1}_n
						(v_1 \otimes \cdots \otimes v_n) \\
					&\qquad = f_1 \otimes \cdots \otimes f_n
						\left( f^{-1}_1(v_1) \otimes \cdots \otimes f^{-1}_n(v_n) \right)
						= (v_1 \otimes \cdots \otimes v_n)
				\end{align}
				が成立し,それぞれ$\bigotimes_{i=1}^n E_i$と$\bigotimes_{i=1}^n V_i$
				を生成するから
				\begin{align}
					(f_1 \otimes \cdots \otimes f_n)^{-1}
					= f^{-1}_1 \otimes \cdots \otimes f^{-1}_n
				\end{align}
				の関係を得る.
				
			\item[第二段]
				$\bigotimes_{i=1}^n V_i$と$\bigoplus_{\lambda_1,\cdots,\lambda_n} 
				W^1_{\lambda_1} \otimes \cdots \otimes W^n_{\lambda_n}$が線型同型であることを示す.先ず
				\begin{align}
					g:\sum_j (v^j_1 \otimes \cdots \otimes v^j_n) \longmapsto 
					\sum_j \left( v^j_1(\lambda_1)\otimes\cdots\otimes v^j_n(\lambda_n) \right)_{\lambda_1,\cdots,\lambda_n}
				\end{align}
				により線型写像$g:\bigotimes_{i=1}^n V_i \longrightarrow \bigoplus_{\lambda_1,\cdots,\lambda_n} W^1_{\lambda_1} \otimes \cdots \otimes W^n_{\lambda_n}$を定める.また
				\begin{align}
					\iota_{\lambda_i}:W^i_{\lambda_i}
					\longrightarrow V_i,
					\quad (\lambda_i \in \Lambda_i,\ i=1,\cdots,n)
				\end{align}
				を次の標準単射として定める:
				\begin{align}
					\iota_{\lambda_i}(u)(\lambda)
					= \begin{cases}
						u, & (\lambda = \lambda_i), \\
						0, & (\lambda \neq \lambda_i),
					\end{cases}
					\quad (\lambda \in \Lambda_i,\ u \in W^i_{\lambda_i}).
				\end{align}
				$\iota_{\lambda_i}$は線型であるから
				$\iota_{\lambda_1} \otimes \cdots \otimes \iota_{\lambda_n}:
				W^1_{\lambda_1} \otimes \cdots \otimes W^n_{\lambda_n} \longrightarrow
				\bigotimes_{i=1}^n V_i$
				を定義出来て,
				\begin{align}
					h:w \longmapsto \sum_{\lambda_1,\cdots,\lambda_n} \iota_{\lambda_1} \otimes \cdots \otimes \iota_{\lambda_n}\left( w(\lambda_1,\cdots,\lambda_n) \right)
				\end{align}
				により線型写像$h:W^1_{\lambda_1} \otimes \cdots \otimes W^n_{\lambda_n} \longrightarrow
				\bigotimes_{i=1}^n V_i$が定めれば$g^{-1} = h$が成り立つ.実際,
				\begin{align}
					g \circ h(w)
					&= g \Biggl( \sum_{\lambda_1,\cdots,\lambda_n} \iota_{\lambda_1} \otimes \cdots \otimes \iota_{\lambda_n}\left( w(\lambda_1,\cdots,\lambda_n) \right) \Biggr) \\
					&= \sum_{\lambda_1,\cdots,\lambda_n} g \left( \iota_{\lambda_1} \otimes \cdots \otimes \iota_{\lambda_n}\left( w(\lambda_1,\cdots,\lambda_n) \right) \right) \\
					&= w
				\end{align}
				が任意の$w \in \bigoplus_{\lambda_1,\cdots,\lambda_n} 
				W^1_{\lambda_1} \otimes \cdots \otimes W^n_{\lambda_n}$に対して成立し,
				かつ任意の$v_1 \otimes \cdots \otimes v_n$に対し
				\begin{align}
					h \circ g (v_1 \otimes \cdots \otimes v_n)
					&= \sum_{\lambda_1,\cdots,\lambda_n} \iota_{\lambda_1} \otimes \cdots \otimes \iota_{\lambda_n}\left( v_1(\lambda_1)\otimes \cdots \otimes v_n(\lambda_n) \right) \\
					&= \sum_{\lambda_1,\cdots,\lambda_n}
						\iota_{\lambda_1}(v_1(\lambda_1)) \otimes \cdots \otimes \iota_{\lambda_n}(v_n(\lambda_n)) \\
					&= \Biggl( \sum_{\lambda_1 \in \Lambda_1} \iota_{\lambda_1}(v_1(\lambda_1)) \Biggr) \otimes \cdots \otimes \Biggl( \sum_{\lambda_n \in \Lambda_n} \iota_{\lambda_n}(v_n(\lambda_n)) \Biggr) \\
					&= v_1 \otimes \cdots \otimes v_n
				\end{align}
				が成り立つから$g^{-1} = h$が従う.よって$g$は線型同型である.
				
			\item[第三段]
				いま,$g \circ f_1 \otimes \cdots \otimes f_n$によって
				$\bigotimes_{i=1}^n E_i$と$\bigoplus_{\lambda_1,\cdots,\lambda_n} 
				W^1_{\lambda_1} \otimes \cdots \otimes W^n_{\lambda_n}$
				は同型に対応し,
				\begin{align}
					w_{\lambda_1,\cdots,\lambda_n}(\nu_1,\cdots,\nu_n)
					\coloneqq 
					\begin{cases}
						u^1_{\lambda_1} \otimes \cdots \otimes u^n_{\lambda_n}, & (\lambda_1,\cdots,\lambda_n) = (\nu_1,\cdots,\nu_n), \\
						0, & (\lambda_1,\cdots,\lambda_n) \neq (\nu_1,\cdots,\nu_n)
					\end{cases}
				\end{align}
				として$w_{\lambda_1,\cdots,\lambda_n}$を定めれば
				\begin{align}
					u^1_{\lambda_1} \otimes \cdots \otimes u^n_{\lambda_n}
					\xmapsto{g \circ f_1 \otimes \cdots \otimes f_n}
					w_{\lambda_1,\cdots,\lambda_n}
				\end{align}
				が成り立つ.$\left( w_{\lambda_1,\cdots,\lambda_n} \right)$
				の一次独立性から$\left( u^1_{\lambda_1} \otimes 
			\cdots \otimes u^n_{\lambda_n} \right)_{\lambda_1,\cdots,\lambda_n}$
			の一次独立性が従う.
		\end{description}
	\end{prf}