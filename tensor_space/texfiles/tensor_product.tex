\section{テンソル積の導入}
	$E,F$を体$\K$上の線形空間とするとき,
	テンソル積$E \otimes F$を定めたい.
	\begin{align}
		\Lambda(E)
		= \Set{f:E\longrightarrow \K}{\mbox{有限個の$e \in E$を除いて$f(e)=0$.}}
	\end{align}
	により$\K$-線形空間$\Lambda(E)$を定める.また$e \in E$に対する定義関数を
	\begin{align}
		\defunc_e (x) = 
		\begin{cases}
			1, & x = e, \\
			0, & x \neq e
		\end{cases}
	\end{align}
	で表し,同様にして$\Lambda(E \times F)$及び$(e,f) \in E \times F$に対し
	$\defunc_{e,f}$を定める.$\Lambda(E \times F)$に対し
	\begin{align}
		\Lambda_0(E \times F) \coloneqq
		\Span{\Set{ \substack{\defunc_{(e+e',f)}-\defunc_{(e,f)}-\defunc_{(e',f)},\\
			\defunc_{(e,f+f')}-\defunc_{(e,f)}-\defunc_{(e,f')}, \\
			\defunc_{(\lambda e,f)}-\lambda\defunc_{(e,f)}, \\
			\defunc_{(e,\lambda f)}-\lambda\defunc_{(e,f)} }}{e,e' \in E,f,f' \in F,\lambda \in \K}}
	\end{align}
	により線型部分空間を定め,
	$b \in \Lambda(E \times F)$の$\Lambda_0(E \times F)$に関する同値類を$[b]$と書く.
	\begin{screen}
		\begin{dfn}[テンソル積]
			$\K$-線形空間$E,F$に対し
			\begin{align}
				E \otimes F \coloneqq \Lambda(E \times F)/\Lambda_0(E \times F)
			\end{align}
			で定める商空間を$E$と$F$のテンソル積という.また$(e,f) \in E \times F$に対するテンソル積を
			\begin{align}
				e \otimes f \coloneqq \left[ \defunc_{(e,f)} \right]
			\end{align}
			により定める.
		\end{dfn}
	\end{screen}
	
	\begin{screen}
		\begin{thm}[テンソル積は双線型]\label{thm:tensor_product_is_bilinear}
			$E,F$を$\K$-線形空間とするとき,
			\begin{align}
				\otimes : E \times F \ni (e,f) \longmapsto e \otimes f \in E \otimes F
			\end{align}
			は双線型である.また次が成り立つ:
			\begin{align}
				E \otimes F = \Span{\Set{e \otimes f}{(e,f) \in E \times F}}.
				\label{eq:thm_tensor_product_is_bilinear}
			\end{align}
		\end{thm}
	\end{screen}
	
	\begin{prf}
		$e,e' \in E,\ f,f' \in F,\ \lambda \in \K$とする.
		$\Lambda_0(E \times F)$の定義より
		\begin{align}
			\defunc_{(e+e',f)} \equiv \defunc_{(e,f)}+\defunc_{(e',f)},
			\quad \pmod{\Lambda_0(E \times F)}
		\end{align}
		が満たされ$(e + e') \otimes f = e \otimes f + e' \otimes f$が成立する.
		同様にして
		\begin{align}
			e \otimes (f+f') &= e \otimes f + e \otimes f', \\
			(\lambda e) \otimes f &= \lambda e \otimes f, \\
			e \otimes (\lambda f) &= \lambda e \otimes f
		\end{align}
		が従うので$\otimes$は双線型である.また
		任意に$u = [b] \in E \otimes F$を取れば
		\begin{align}
			b = \sum_{i=1}^n k_i \defunc_{(e_i,f_i)},
			\quad \left( k_i = b(e_i,f_i),\ i=1,2,\cdots,n \right)
		\end{align}
		と表せるから,
		\begin{align}
			u = \left[ \sum_{i=1}^n k_i \defunc_{(e_i,f_i)} \right]
			= \left[ \sum_{i=1}^n \defunc_{(k_i e_i,f_i)} \right]
			= \sum_{i=1}^n (k_i e_i) \otimes f_i
		\end{align}
		が従い(\refeq{eq:thm_tensor_product_is_bilinear})を得る.
		\QED
	\end{prf}
	
	\begin{screen}
		\begin{thm}[テンソル積の普遍性]
			$E,F,V$を$\K$-線形空間とするとき,
			$\Hom{E \otimes F}{V}$と
			$\Bilin{E \times F}{V}$は
			次の写像$\Phi$により線型同型である:
			\begin{align}
				\begin{array}{ccc}
					\Phi:\Hom{E \otimes F}{V} & \longrightarrow & \Bilin{E \times F}{V} \\
					\rotatebox{90}{$\in$} & & \rotatebox{90}{$\in$} \\
					T & \longmapsto & T \circ \otimes
				\end{array}
			\end{align}
		\end{thm}
	\end{screen}
	
	\begin{prf}全射性と単射性を示す.
		\begin{description}
			\item[第一段]
				始めの二段で$\Phi$の全射性を示す.まず,$f \in \Hom{\Lambda(E \times F)}{V}$に対し
				\begin{align}
					g: E \times F \ni (e,f) \longmapsto f(\defunc_{e,f}) \in V
				\end{align}
				を対応させる次の写像が線形同型であることを示す:
				\begin{align}
					\begin{array}{ccc}
						F:\Hom{\Lambda(E \times F)}{V} & \longrightarrow & \Map{E \times F}{V} \\
						\rotatebox{90}{$\in$} & & \rotatebox{90}{$\in$} \\
						f & \longmapsto & g
					\end{array}
				\end{align}
				実際$g = F(f)$の定め方より
				$F$は線型である.また$F(f_1) = F(f_2)$なら
				\begin{align}
					f_1(\defunc_{e,f}) = f_2(\defunc_{e,f}),
					\quad (\forall (e,f) \in E \times F)
				\end{align}
				が成り立ち,$\Span{\Set{\defunc_{e,f}}{(e,f) \in E \times F}}=\Lambda(E \times F)$より
				$F$の単射性を得る.いま,
				\begin{align}
					\sum_{(e,f) \in E \times F} \psi(e,f) g(e,f)
					\coloneqq \sum_{\substack{(e,f) \in E \times F \\ \psi(e,f) \neq 0}} \psi(e,f) g(e,f),
					\quad (\psi \in \Lambda(E \times F),\ g \in \Map{E \times F}{V})
				\end{align}
				により総和記号を定め,
				$g \in \Map{E \times F}{V}$に対して
				\begin{align}
					f(\psi) \coloneqq \sum_{(e,f) \in E \times F} \psi(e,f) g(e,f),
					\quad (\psi \in \Hom{\Lambda(E \times F)}{V})
				\end{align}
				で$f$を作れば,$f \in \Hom{\Lambda(E \times F)}{V}$が満たされ
				$F$の全射性が従う.
				
			\item[第二段]
				任意に$b \in \Bilin{E \times F}{V}$を取り
				$h \coloneqq F^{-1}(b)$とおけば,$h$の線型性より
				\begin{align}
					b(e+e',f) - b(e,f) - b(e',f)
					&= h(\defunc_{e+e',f} - \defunc_{e,f} - \defunc_{e',f}), \\
					b(e,f+f') - b(e,f) - b(e,f')
					&= h(\defunc_{e,f+f'} - \defunc_{e,f} - \defunc_{e,f'}), \\
					b(\lambda e,f) - \lambda b(e,f)
					&= h(\defunc_{\lambda e,f} - \lambda \defunc_{e,f}), \\
					b(e,\lambda f) - \lambda b(e,f)
					&= h(\defunc_{e,\lambda f} - \lambda \defunc_{e,f})
				\end{align}
				が成り立ち,$b$の双線型性により$h$は$\Lambda_0(E \times F)$上で0である.従って
				\begin{align}
					T([b]) \coloneqq h(b),
					\quad (b \in \Lambda(E \times F))
				\end{align}
				で定める$T$はwell-definedであり,$T \in \Hom{E \otimes F}{V}$かつ
				\begin{align}	
					b(e,f) = h(\defunc_{e,f}) = (T \circ \otimes) (e,f),
					\quad (\forall (e,f) \in E \times F)
				\end{align}
				が満たされ$\Phi$の全射性が得られる.
				
			\item[第三段]
				$\Phi(T_1) = \Phi(T_2)$ならば
				$T_1$と$T_2$は$\Set{e \otimes f}{e \in E,\ f \in F}$の上で一致する.
				定理\ref{thm:tensor_product_is_bilinear}より
				$\Set{e \otimes f}{e \in E,\ f \in F}$は
				$E \otimes F$を生成するから$\Phi$の単射性が従う.
				\QED
		\end{description}
	\end{prf}
	
	$\K=\R,\C$のとき,$E$の弱位相を
	$\sigma(E,E^*)$と書き,$E^*$の汎弱位相を$\sigma(E^*,E)$と書く.
	また
	\begin{align}
		\mathcal{F}(E,F)
		\coloneqq &\Set{T:E \longrightarrow F}{\Dim{T E} < \infty}, \\
		\mathcal{F}_{w^*}(E^*,F)
		\coloneqq &\Set{T \in \mathcal{F}(E^*,F)}{T^{-1}(A) \in \sigma(F^*,F),\quad \forall A \in \sigma(E,E^*)} \\
		= &\Set{T \in \mathcal{F}(E^*,F)}{T : \mbox{$w^*$-$w$-continuous}}
	\end{align}
	とおく.
	
	\begin{screen}
		\begin{thm}
			$\K$を$\R$または$\C$と考える.$E,F$が$\K$-Banach空間なら,
			\begin{align}
				\begin{array}{ccc}
					i:E \otimes F & \longrightarrow & \mathcal{F}_{w^*}(E^*,F), \\
					i(e \otimes f): E^* \ni e^* & \longmapsto & \inprod<e^*,e>f 
				\end{array}
			\end{align}
			で定める$i$は線型同型である.
		\end{thm}
	\end{screen}