\section{テンソル積の導入}
	$\K = \C$または$\R$と考える.$E,F$を$\K$上の線形空間とするとき,
	テンソル積$E \otimes F$を定めたい.
	\begin{align}
		\Lambda(E \times F)
		= \Set{b:E \times F \longrightarrow \K}{\mbox{有限個の$(e,f) \in E \times F$を除いて$b(e,f)=0$.}}
	\end{align}
	により$\Lambda(E \times F)$を定める.また$(e,f) \in E \times F$に対する定義関数
	\begin{align}
		\defunc_{(e,f)} (g,h) = 
		\begin{cases}
			1, & (e,f) = (g,h), \\
			0, & (e,f) \neq (g,h)
		\end{cases}
	\end{align}
	は$\defunc_{(e,f)} \in \Lambda(E \times F)$を満たす.
	$\Lambda(E \times F)$が$\K$-線形空間をなすことにより
	\begin{align}
		\Lambda_0(E \times F) \coloneqq
		\Span{\Set{ \substack{\defunc_{(e+e',f)}-\defunc_{(e,f)}-\defunc_{(e',f)},\\
			\defunc_{(e,f+f')}-\defunc_{(e,f)}-\defunc_{(e,f')}, \\
			\defunc_{(\lambda e,f)}-\lambda\defunc_{(e,f)}, \\
			\defunc_{(e,\lambda f)}-\lambda\defunc_{(e,f)} }}{e,e' \in E,f,f' \in F,\lambda \in \K}}
	\end{align}
	は$\Lambda(E \times F)$の線型部分空間である.
	$b \in \Lambda(E \times F)$の$\Lambda_0(E \times F)$に関する同値類を$[b]$で表す.
	\begin{screen}
		\begin{dfn}[テンソル積]
			$\K$-線形空間$E,F$のテンソル積を
			\begin{align}
				E \otimes F \coloneqq \Lambda(E \times F)/\Lambda_0(E \times F)
			\end{align}
			で定める.また$(e,f) \in E \times F$に対するテンソル積を
			\begin{align}
				e \otimes f \coloneqq \left[ \defunc_{(e,f)} \right]
			\end{align}
			により定める.
		\end{dfn}
	\end{screen}
	
	\begin{screen}
		\begin{thm}[テンソル積は双線型・元の表現]
			$E,F$を$\K$-線形空間とするとき,
			\begin{align}
				\otimes : E \times F \ni (e,f) \longmapsto e \otimes f \in E \otimes F
			\end{align}
			は双線型である.また任意の$u \in E \otimes F$は,或る有限個の
			$e_i \otimes f_i \in E \otimes F$により
			\begin{align}
				u = \sum_{i=1}^{n} e_i \otimes f_i
			\end{align}
			と書ける.
		\end{thm}
	\end{screen}
	
	\begin{prf}
		$e,e' \in E,\ f,f' \in F,\ \lambda \in \K$とする.
		$\Lambda_0(E \times F)$の定義より
		\begin{align}
			\defunc_{(e+e',f)} \equiv \defunc_{(e,f)}+\defunc_{(e',f)},
			\quad \pmod{\Lambda_0(E \times F)}
		\end{align}
		が満たされ$(e + e') \otimes f = e \otimes f + e' \otimes f$が成立する.
		同様にして
		\begin{align}
			e \otimes (f+f') &= e \otimes f + e \otimes f', \\
			(\lambda e) \otimes f &= \lambda e \otimes f, \\
			e \otimes (\lambda f) &= \lambda e \otimes f
		\end{align}
		が従うので$\otimes$は双線型である.また
		任意の$u \in E \otimes F$に対し,その代表を$b \in \Lambda(E \times F)$とすれば
		\begin{align}
			b = \sum_{i=1}^n k_i \defunc_{(e_i,f_i)},
			\quad \left( k_i = b(e_i,f_i),\ i=1,2,\cdots,n \right)
		\end{align}
		と表せるから
		\begin{align}
			u = \left[ \sum_{i=1}^n k_i \defunc_{(e_i,f_i)} \right]
			= \left[ \sum_{i=1}^n \defunc_{(k_i e_i,f_i)} \right]
			= \sum_{i=1}^n (k_i e_i) \otimes f_i
		\end{align}
		を得る.
		\QED
	\end{prf}
	
	\begin{screen}
		\begin{thm}
			$E,F,V$を$\K$-線形空間とするとき,
			$\Hom{E \otimes F}{V}$と
			$\Bilin{E \times F}{V}$は線型同型である.
			ただしHomは線形写像全体,$\mathrm{Hom}^{(2)}$は双線型写像全体を表す.
		\end{thm}
	\end{screen}
	
	\begin{prf}
		同型写像は
		\begin{align}
			\Phi: \Hom{E \otimes F}{V} \ni T \longmapsto T \circ \otimes \in \Bilin{E \times F}{V}
		\end{align}
		により与えられる.実際,
	\end{prf}