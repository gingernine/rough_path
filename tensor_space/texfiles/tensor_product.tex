\section{テンソル積の導入}
	$\K = \C$または$\R$と考える.$E,F$を$\K$上の線形空間とするとき,
	テンソル積$E \otimes F$を定めたい.
	
	\begin{align}
		\Lambda(E \times F) 
		\coloneqq \K^{\oplus(E \times F)}
		&= \Set{b:E \times F \longrightarrow \K}{\mbox{有限個の$(e,f) \in E \times F$を除いて$b(e,f)=0$.}} \\
		&= \bigoplus_{i \in E \times F} X_i,
		\quad (X_i = \K,\ \forall i \in E \times F)
	\end{align}
	により$\Lambda(E \times F)$を定める.また$(e,f) \in E \times F$に対する定義関数
	\begin{align}
		\defunc_{\{(e,f)\}} (g,h) = 
		\begin{cases}
			1, & (e,f) = (g,h), \\
			0, & (e,f) \neq (g,h)
		\end{cases}
	\end{align}
	は$\defunc_{\{(e,f)\}} \in \Lambda(E \times F)$を満たす.
	$\Lambda(E \times F)$が$\K$-線形空間をなすことにより
	\begin{align}
		\Lambda_0(E \times F) \coloneqq
		%\span{\Set{\defunc_{\{(e+e',f)\}}}{}}
	\end{align}